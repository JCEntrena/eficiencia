%%%
% Modificación de una plantilla de Latex para adaptarla al castellano.
%%%

%%%%%%%%%%%%%%%%%%%%%%%%%%%%%%%%%%%%%%%%%
% Thin Sectioned Essay
% LaTeX Template
% Version 1.0 (3/8/13)
%
% This template has been downloaded from:
% http://www.LaTeXTemplates.com
%
% Original Author:
% Nicolas Diaz (nsdiaz@uc.cl) with extensive modifications by:
% Vel (vel@latextemplates.com)
%
% License:
% CC BY-NC-SA 3.0 (http://creativecommons.org/licenses/by-nc-sa/3.0/)
%
%%%%%%%%%%%%%%%%%%%%%%%%%%%%%%%%%%%%%%%%%

%----------------------------------------------------------------------------------------
%	PACKAGES AND OTHER DOCUMENT CONFIGURATIONS
%----------------------------------------------------------------------------------------

\documentclass[a4paper, 11pt]{article} % Font size (can be 10pt, 11pt or 12pt) and paper size (remove a4paper for US letter paper)

\usepackage[protrusion=true,expansion=true]{microtype} % Better typography
\usepackage{graphicx} % Required for including pictures
\usepackage{wrapfig} % Allows in-line images
% Sorry, no me compila el LaTeX con estas opciones, descomentadlas para compilarlo vosotros si queréis -David
%\usepackage[spanish]{babel} % English language/hyphenation
%\selectlanguage{spanish}
\usepackage[utf8]{inputenc}
\usepackage{dcolumn} % Para el alineamiento del punto decimal
\newcolumntype{d}[1]{D{.}{.}{#1} } % Idem
\usepackage{longtable}
\usepackage{tabu}

\usepackage{mathpazo} % Use the Palatino font
\usepackage[T1]{fontenc} % Required for accented characters
\linespread{1.05} % Change line spacing here, Palatino benefits from a slight increase by default

\makeatletter
\renewcommand\@biblabel[1]{\textbf{#1.}} % Change the square brackets for each bibliography item from '[1]' to '1.'
\renewcommand{\@listI}{\itemsep=0pt} % Reduce the space between items in the itemize and enumerate environments and the bibliography

\renewcommand{\maketitle}{ % Customize the title - do not edit title and author name here, see the TITLE block below
\begin{flushright} % Right align
{\LARGE\@title} % Increase the font size of the title

\vspace{50pt} % Some vertical space between the title and author name

{\large\@author} % Author name
\\\@date % Date

\vspace{40pt} % Some vertical space between the author block and abstract
\end{flushright}
}

%----------------------------------------------------------------------------------------
%	TITLE
%----------------------------------------------------------------------------------------

\title{\textbf{Práctica 1}\\ % Title
Análisis de eficiencia} % Subtitle

\author{\textsc{Óscar Bermúdez,\\Francisco David Charte,\\Ignacio Cordón,\\José Carlos Entrena,\\Mario Román} % Author
\\{\textit{Universidad de Granada}}} % Institution

\date{\today} % Date

%----------------------------------------------------------------------------------------

\begin{document}

\maketitle % Print the title section

\renewcommand{\abstractname}{Resumen} % Uncomment to change the name of the abstract to something else
\begin{abstract}
La eficiencia de los algoritmos se puede medir de forma teórica 
y empírica. En esta memoria se recogen estudios acerca de la 
eficiencia de algunos algoritmos muy conocidos, y se exponen 
implementaciones de los mismos y datos de su ejecución en 
distintas máquinas y con diferentes niveles de optimización.
Se representan estos resultados en una serie de gráficas, y por 
último, se resume la información obtenida en unas conclusiones 
finales.
\end{abstract}
\tableofcontents

\pagebreak

\section {Preparación del código}
%Hay un problema con el espacio entre estos párrafos, que no deja separación enre ambos. Vale asi?
Con motivo de facilitarnos un poco el trabajo a la hora de realizar esta práctica,
hemos añadido algunos archivos y realizado ciertas modificaciones en los códigos, que
detallamos a continuación: 

\medskip
Inicialmente, hemos hecho algunos cambios léxicos en el código de los programas, con el
objetivo de que sean más legibles para nosotros y poder comprenderlos mejor. También hemos
cambiado la función que se utilizaba para medir el tiempo por una con más precisión, lo 
que nos permite ejecutar los algoritmos con un volumen de entrada menor. 

Para facilitarnos la compilación y ejecución de los programas, hemos creado un archivo 
makefile que compila los códigos y genera los datos y gráficas necesarios para la práctica.

Además, para la generación de dichas gráficas, hemos escrito un programa en bash que usa gnuplot 
para obtenerlas, al igual que los ajustes de la eficiencia. 

\section {Análisis teórico}
\subsection {Ordenación de la burbuja}
El algoritmo lo forman dos bucles anidados ejecutando un conjunto de sentencias elementales, $\mathcal{O}(1)$.
El bucle externo se repite $n$ veces y el interno depende del índice del bucle externo. Ejecutándose en total:
\begin{equation}
 T(n) = \sum_{i=0}^n i = \frac{n(n+1)}{2} = \frac{n^2}{2} + \frac{n}{2}
\end{equation}
por lo que el algoritmo es $\mathcal{O}(n^2)$.

\subsection{Ordenación por inserción}
De nuevo, el algoritmo está compuesto por dos bucles anidados, de los cuales el primero ejecuta $n-1$ iteraciones y el segundo realiza como máximo $i$ (el índice del primer bucle). La función que nos da el tiempo de ejecución en el peor caso es, por tanto:
\begin{equation}
 T(n) = \sum_{i=0}^{n-1} i = \frac{(n-1)n}{2} = \frac{n^2}{2} - \frac{n}{2}
\end{equation}
y el algoritmo es $\mathcal{O}(n^2)$ en el caso peor.
\subsection{Ordenación por selección}
Al igual que los anteriores, consta de dos bucles anidados. El exterior realiza $n-1$ iteraciones, y el interior tantas como indica el índice del primero, $i$, lo que nos da un tiempo de ejecución:
\begin{equation}
 T(n) = \sum_{i=0}^{n-1} i = \frac{(n-1)n}{2} = \frac{n^2}{2} - \frac{n}{2}
\end{equation}
por lo que este algoritmo también es $\mathcal{O}(n^2)$ en el caso peor.

\subsection{Ordenación \textit{heapsort}}
En este algoritmo insertaremos los datos en un árbol tipo heap y los extraeremos de nuevo ya ordenados.
La inserción en una estructura heap es del orden $\mathcal{O}(log n)$ y la extracción del mínimo, que involucra
reajustar el árbol, es también $\mathcal{O}(log n)$. En conjunto, y sabiendo que insertaremos y extraeremos $n$ elementos,
el orden del algoritmo es:
\begin{equation}
 T(n) = 2 n log(n)
\end{equation}
por lo que el algoritmo tiene complejidad $\mathcal{O}(nlog(n))$.


\subsection{Ordenación por mezcla (\textit{mergesort})}
El algoritmo mergesort es un algoritmo divide y vencerás para ordenación, basado en la ordenación de las dos mitades del
array independientemente. En nuestro caso usaremos \textit{insertion sort} para los problemas que quedan por debajo de una talla umbral.
Sabiendo que la ordenación en un caso suficientemente pequeño se acota por una constante y que el algoritmo recurre a la ordenación
de dos casos de tamaño mitad y una mezcla de ambos lineal:
\begin{equation}
 T(n) = 2T\left(\frac{n}{2}\right) + n
\end{equation}
Que tiene soluciones del tipo $an + bnlog(n)$, el algoritmo es de complejidad $\mathcal{O}(nlog(n))$.


\subsection{Algoritmo de Hoare (\textit{quicksort})}
El algoritmo quicksort toma un pivote para partir los elementos a ordenar en aquellos mayores y aquellos menores que el pivote.
En su caso medio, promediaremos las elecciones del pivote. Siempre se tendrá una fase de complejidad lineal previa en la que los
elementos se particionan por el pivote, y, posteriormente, habrá una fase en la que se analizarán las otras dos mitades. Tenemos:
\begin{equation}
 T(n) = n - 1 + \frac{1}{n}\left(\sum_{i=0}^{n-1} T(i) + T(n-1-i) \right)
\end{equation}
Que tiene una solución de complejidad $\mathcal{O}(nlog(n))$.


\subsection {Sucesión de Fibonacci}
El algoritmo está escrito recursivamente y sin memoización. La llamada a la función con tamaño $n$ produce 
dos llamadas recursivas con tamaños $n-1$ y $n-2$. Tendremos la ecuación en diferencias:
\begin{equation}
 T(n) = T(n-1) + T(n-2)
\end{equation}
con la solución conocida de complejidad $\mathcal{O}(\phi^n)$.

\subsection{Algoritmo de Floyd}
El algoritmo de Floyd guarda en una matriz los actuales caminos mínimos entre dos vértices de un grafo de $n$ vértices
y realiza un intercambio entre el camino de dos vértices si este puede mejorarse pasando por el vértice $n$. En resumen,
tendremos tres bucles anidados realizando una operación constante. Cada uno de ellos realizando $n$ iteraciones.

La complejidad del algoritmo es $\mathcal{O}(n^3)$.


\subsection{Torres de Hanoi}
El algoritmo de las torres de Hanoi de talla $n$ se llama a sí mismo recursivamente dos veces y cada una de ellas con
talla $n-1$. Resuelve la torre para los $n-1$ discos superiores, mueve la base, y la resuelve de nuevo para colocarlos sobre ella.
Tendremos entonces como ecuación principal:
\begin{equation}
 T(n) = 2T(n-1) + 1
\end{equation}
Con soluciones de complejidad $\mathcal{O}(n^2)$.

\section {Análisis práctico}

\section {Conclusiones}
\subsection {Gráficas}

\section{Apéndice: Tablas de datos}
% TABLA DE PRUEBA (Cambiar por datos de verdad...)
% generada mediante `./gengraf.sh "burbuja seleccion insercion" 3`
\begin{center}
    \begin{longtabu} to \linewidth{ l | *{3}{d{10}}}  % máx 10 decimales
\rowfont\bfseries Tamaño & \multicolumn{1}{l}{burbuja} & \multicolumn{1}{l}{seleccion} & \multicolumn{1}{l}{insercion} \\ \hline
    \endhead
    \endfoot
    \\ \hline
    \endlastfoot
10 & 0.0000011 & 0.0000010714 & 0.0000009218 \\
210 & 0.0001376 & 0.00009386 & 0.00006644 \\
410 & 0.000482 & 0.0002978 & 0.00019898 \\
610 & 0.0007078 & 0.0005778 & 0.000509 \\
810 & 0.0013222 & 0.000961 & 0.0009356 \\
1010 & 0.00246 & 0.0014 & 0.001422 \\
1210 & 0.003506 & 0.002008 & 0.002058 \\
1410 & 0.004836 & 0.00259 & 0.0025162 \\
1610 & 0.006208 & 0.003332 & 0.002768 \\
1810 & 0.00731 & 0.00404 & 0.003692 \\
2010 & 0.006584 & 0.004386 & 0.005536 \\
2210 & 0.00891 & 0.005908 & 0.006758 \\
2410 & 0.008302 & 0.006312 & 0.005614 \\
2610 & 0.008564 & 0.008096 & 0.007778 \\
2810 & 0.011416 & 0.00395 & 0.007554 \\
3010 & 0.011628 & 0.004918 & 0.008848 \\
3210 & 0.014066 & 0.009786 & 0.008754 \\
3410 & 0.01622 & 0.008532 & 0.009828 \\
3610 & 0.01578 & 0.010272 & 0.010922 \\
3810 & 0.01582 & 0.009466 & 0.011364 \\
4010 & 0.0197 & 0.00736 & 0.00873 \\
4210 & 0.02228 & 0.007608 & 0.012756 \\
4410 & 0.02414 & 0.00998 & 0.014812 \\
4610 & 0.0274 & 0.011016 & 0.01607 \\
4810 & 0.03172 & 0.01207 & 0.01814 \\
5010 & 0.0323 & 0.014922 & 0.01634 \\
5210 & 0.03652 & 0.014686 & 0.01834 \\
5410 & 0.03694 & 0.01282 & 0.0157 \\
5610 & 0.04406 & 0.01696 & 0.01948 \\
5810 & 0.05026 & 0.01514 & 0.02026 \\
6010 & 0.04842 & 0.01808 & 0.0217 \\
6210 & 0.05126 & 0.02174 & 0.02422 \\
6410 & 0.05456 & 0.02024 & 0.02578 \\
6610 & 0.0633 & 0.023 & 0.02776 \\
6810 & 0.06438 & 0.01962 & 0.02876 \\
7010 & 0.0664 & 0.02212 & 0.02902 \\
7210 & 0.07484 & 0.02212 & 0.02952 \\
7410 & 0.0778 & 0.02792 & 0.03208 \\
7610 & 0.0812 & 0.02454 & 0.03266 \\
7810 & 0.08816 & 0.029 & 0.03508 \\
8010 & 0.09084 & 0.0265 & 0.0358 \\
8210 & 0.09832 & 0.03044 & 0.04556 \\
8410 & 0.10698 & 0.03102 & 0.03926 \\
8610 & 0.1066 & 0.02998 & 0.04168 \\
8810 & 0.1135 & 0.03358 & 0.0441 \\
9010 & 0.12048 & 0.03942 & 0.04584 \\
9210 & 0.1262 & 0.03502 & 0.05034 \\
9410 & 0.1318 & 0.03802 & 0.04996 \\
9610 & 0.1382 & 0.04034 & 0.0519 \\
9810 & 0.1424 & 0.04118 & 0.0545 \\

    \end{longtabu}
\end{center}



%----------------------------------------------------------------------------------------
%	ABSTRACT AND KEYWORDS
%----------------------------------------------------------------------------------------

%\renewcommand{\abstractname}{Summary} % Uncomment to change the name of the abstract to something else

%\begin{abstract}
%Morbi tempor congue porta. Proin semper, leo vitae faucibus dictum, metus mauris lacinia lorem, ac congue leo felis eu turpis. Sed nec nunc pellentesque, gravida eros at, porttitor ipsum. Praesent consequat urna a lacus lobortis ultrices eget ac metus. In tempus hendrerit rhoncus. Mauris dignissim turpis id sollicitudin lacinia. Praesent libero tellus, fringilla nec ullamcorper at, ultrices id nulla. Phasellus placerat a tellus a malesuada.
%\end{abstract}

%\hspace*{3,6mm}\textit{Keywords:} lorem , ipsum , dolor , sit amet , lectus % Keywords

%\vspace{30pt} % Some vertical space between the abstract and first section

%----------------------------------------------------------------------------------------
%	ESSAY BODY
%----------------------------------------------------------------------------------------

% \section*{Introduction}
% 
% This statement requires citation \cite{Smith:2012qr}; this one does too \cite{Smith:2013jd}. Lorem ipsum dolor sit amet, consectetur adipiscing elit. Aenean dictum lacus sem, ut varius ante dignissim ac. Sed a mi quis lectus feugiat aliquam. Nunc sed vulputate velit. Sed commodo metus vel felis semper, quis rutrum odio vulputate. Donec a elit porttitor, facilisis nisl sit amet, dignissim arcu. Vivamus accumsan pellentesque nulla at euismod. Duis porta rutrum sem, eu facilisis mi varius sed. Suspendisse potenti. Mauris rhoncus neque nisi, ut laoreet augue pretium luctus. Vestibulum sit amet luctus sem, luctus ultrices leo. Aenean vitae sem leo.
% 
% Nullam semper quam at ante convallis posuere. Ut faucibus tellus ac massa luctus consectetur. Nulla pellentesque tortor et aliquam vehicula. Maecenas imperdiet euismod enim ut pharetra. Suspendisse pulvinar sapien vitae placerat pellentesque. Nulla facilisi. Aenean vitae nunc venenatis, vehicula neque in, congue ligula.
% 
% Pellentesque quis neque fringilla, varius ligula quis, malesuada dolor. Aenean malesuada urna porta, condimentum nisl sed, scelerisque nisi. Suspendisse ac orci quis massa porta dignissim. Morbi sollicitudin, felis eget tristique laoreet, ante lacus pretium lacus, nec ornare sem lorem a velit. Pellentesque eu erat congue, ullamcorper ante ut, tristique turpis. Nam sodales mi sed nisl tincidunt vestibulum. Interdum et malesuada fames ac ante ipsum primis in faucibus.
% 
% %------------------------------------------------
% 
% \section*{Section Name}
% 
% Cras gravida, est vel interdum euismod, tortor mi lobortis mi, quis adipiscing elit lacus ut orci. Phasellus nec fringilla nisi, ut vestibulum neque. Aenean non risus eu nunc accumsan condimentum at sed ipsum.
% \begin{wrapfigure}{l}{0.4\textwidth} % Inline image example
% \begin{center}
% \includegraphics[width=0.38\textwidth]{fish.png}
% \end{center}
% \caption{Fish}
% \end{wrapfigure}
% Aliquam fringilla non diam sed varius. Suspendisse tellus felis, hendrerit non bibendum ut, adipiscing vitae diam. Lorem ipsum dolor sit amet, consectetur adipiscing elit. Nulla lobortis purus eget nisl scelerisque, commodo rhoncus lacus porta. Vestibulum vitae turpis tincidunt, varius dolor in, dictum lectus. Aenean ac ornare augue, ac facilisis purus. Sed leo lorem, molestie sit amet fermentum id, suscipit ut sem. Vestibulum orci arcu, vehicula sed tortor id, ornare dapibus lorem. Praesent aliquet iaculis lacus nec fermentum. Morbi eleifend blandit dolor, pharetra hendrerit neque ornare vel. Nulla ornare, nisl eget imperdiet ornare, libero enim interdum mi, ut lobortis quam velit bibendum nibh.
% 
% Morbi tempor congue porta. Proin semper, leo vitae faucibus dictum, metus mauris lacinia lorem, ac congue leo felis eu turpis. Sed nec nunc pellentesque, gravida eros at, porttitor ipsum. Praesent consequat urna a lacus lobortis ultrices eget ac metus. In tempus hendrerit rhoncus. Mauris dignissim turpis id sollicitudin lacinia. Praesent libero tellus, fringilla nec ullamcorper at, ultrices id nulla. Phasellus placerat a tellus a malesuada.
% 
% %------------------------------------------------
% 
% \section*{Conclusion}
% 
% Fusce in nibh augue. Cum sociis natoque penatibus et magnis dis parturient montes, nascetur ridiculus mus. In dictum accumsan sapien, ut hendrerit nisi. Phasellus ut nulla mauris. Phasellus sagittis nec odio sed posuere. Vestibulum porttitor dolor quis suscipit bibendum. Mauris risus lectus, cursus vitae hendrerit posuere, congue ac est. Suspendisse commodo eu eros non cursus. Mauris ultrices venenatis dolor, sed aliquet odio tempor pellentesque. Duis ultricies, mauris id lobortis vulputate, tellus turpis eleifend elit, in gravida leo tortor ultricies est. Maecenas vitae ipsum at dui sodales condimentum a quis dui. Nam mi sapien, lobortis ac blandit eget, dignissim quis nunc.
% 
% \begin{enumerate}
% \item First numbered list item
% \item Second numbered list item
% \end{enumerate}
% 
% Donec luctus tincidunt mauris, non ultrices ligula aliquam id. Sed varius, magna a faucibus congue, arcu tellus pellentesque nisl, vel laoreet magna eros et magna. Vivamus lobortis elit eu dignissim ultrices. Fusce erat nulla, ornare at dolor quis, rhoncus venenatis velit. Donec sed elit mi. Sed semper tellus a convallis viverra. Maecenas mi lorem, placerat sit amet sem quis, adipiscing tincidunt turpis. Cras a urna et tellus dictum eleifend. Fusce dignissim lectus risus, in bibendum tortor lacinia interdum.

%----------------------------------------------------------------------------------------
%	BIBLIOGRAPHY
%----------------------------------------------------------------------------------------

% \bibliographystyle{unsrt}
% 
% \bibliography{sample}

%----------------------------------------------------------------------------------------



\end{document}