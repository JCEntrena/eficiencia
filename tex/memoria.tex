%%%
% Modificación de una plantilla de Latex para adaptarla al castellano.
%%%

%%%%%%%%%%%%%%%%%%%%%%%%%%%%%%%%%%%%%%%%%
% Thin Sectioned Essay
% LaTeX Template
% Version 1.0 (3/8/13)
%
% This template has been downloaded from:
% http://www.LaTeXTemplates.com
%
% Original Author:
% Nicolas Diaz (nsdiaz@uc.cl) with extensive modifications by:
% Vel (vel@latextemplates.com)
%
% License:
% CC BY-NC-SA 3.0 (http://creativecommons.org/licenses/by-nc-sa/3.0/)
%
%%%%%%%%%%%%%%%%%%%%%%%%%%%%%%%%%%%%%%%%%

%----------------------------------------------------------------------------------------
%	PACKAGES AND OTHER DOCUMENT CONFIGURATIONS
%----------------------------------------------------------------------------------------

\documentclass[a4paper, 11pt]{article} % Font size (can be 10pt, 11pt or 12pt) and paper size (remove a4paper for US letter paper)

\usepackage[protrusion=true,expansion=true]{microtype} % Better typography
\usepackage{graphicx} % Required for including pictures
\usepackage[usenames,dvipsnames]{color} % Coloring code
\usepackage{wrapfig} % Allows in-line images
% sudo apt-get install texlive-lang-spanish
%\usepackage[spanish]{babel} % English language/hyphenation
%\selectlanguage{spanish}
\usepackage[utf8]{inputenc}
\usepackage{dcolumn} % Para el alineamiento del punto decimal
\newcolumntype{d}[1]{D{.}{.}{#1} } % Idem
\usepackage{longtable}
\usepackage{tabu}

\usepackage{multicol}

\usepackage[section]{placeins} % Para gráficas en su sección.
\usepackage{mathpazo} % Use the Palatino font
\usepackage[T1]{fontenc} % Required for accented characters
\newenvironment{allintypewriter}{\ttfamily}{\par}
\setlength{\parindent}{0pt}
\parskip=8pt
\linespread{1.05} % Change line spacing here, Palatino benefits from a slight increase by default

\makeatletter
\renewcommand\@biblabel[1]{\textbf{#1.}} % Change the square brackets for each bibliography item from '[1]' to '1.'
\renewcommand{\@listI}{\itemsep=0pt} % Reduce the space between items in the itemize and enumerate environments and the bibliography
\newcommand{\imagen}[2]{\begin{figure*}[ht!] \centering  \includegraphics[width=90mm]{#1} \\ #2 \end{figure*}}

\renewcommand{\maketitle}{ % Customize the title - do not edit title and author name here, see the TITLE block below
\begin{flushright} % Right align
{\LARGE\@title} % Increase the font size of the title

\vspace{50pt} % Some vertical space between the title and author name

{\large\@author} % Author name
\\\@date % Date

\vspace{40pt} % Some vertical space between the author block and abstract
\end{flushright}
}

%----------------------------------------------------------------------------------------
%	TITLE
%----------------------------------------------------------------------------------------

\title{\textbf{Práctica 1}\\ % Title
Análisis de eficiencia} % Subtitle

\author{\textsc{Óscar Bermúdez,\\Francisco David Charte,\\Ignacio Cordón,\\José Carlos Entrena,\\Mario Román} % Author
\\{\textit{Universidad de Granada}}} % Institution

\date{\today} % Date

%----------------------------------------------------------------------------------------

\begin{document}

\maketitle % Print the title section

\renewcommand{\abstractname}{Resumen} % Uncomment to change the name of the abstract to something else
\begin{abstract}
La eficiencia de los algoritmos se puede medir de forma teórica 
y empírica. En esta memoria se recogen estudios acerca de la 
eficiencia de algunos algoritmos muy conocidos, y se exponen 
implementaciones de los mismos y datos de su ejecución en 
distintas máquinas y con diferentes niveles de optimización.
Se representan estos resultados en una serie de gráficas, y por 
último, se resume la información obtenida en unas conclusiones 
finales.
\end{abstract}
{\parskip=2pt
\tableofcontents
}
\pagebreak

\section {Introducción}
%Hay un problema con el espacio entre estos párrafos, que no deja separación enre ambos. Vale asi?
Hemos llevado a cabo análisis teóricos y estudios empíricos de la eficiencia de los algoritmos propuestos. Con motivo de facilitar el trabajo a la hora de realizar esta práctica,
hemos añadido algunos archivos y realizado ciertas modificaciones en los códigos, que
introducimos a continuación y detallaremos en las siguientes secciones: 

Inicialmente, hemos hecho algunos cambios léxicos en el código de los programas, con el
objetivo de que sean más legibles y facilitar su comprensión. También hemos
cambiado la función que se utilizaba para medir el tiempo por una con más precisión, lo 
que nos permite ejecutar los algoritmos con un volumen de entrada menor. 

Para facilitarnos la compilación y ejecución de los programas, hemos creado un archivo 
Makefile que compila los códigos y genera los datos y gráficas necesarios para la práctica.

Además, para la generación de dichas gráficas, hemos escrito un guion en Bash que usa 
\texttt{gnuplot} para obtenerlas, de la misma forma que se obtienen las funciones de ajuste 
de la eficiencia. 

\section {Análisis teórico}
\subsection {Ordenación de la burbuja}
El algoritmo lo forman dos bucles anidados ejecutando un conjunto de sentencias elementales, $\mathcal{O}(1)$.
El bucle externo se repite $n$ veces y el interno depende del índice del bucle externo. Ejecutándose en total:
\begin{equation}
 T(n) = \sum_{i=0}^n i = \frac{n(n+1)}{2} = \frac{n^2}{2} + \frac{n}{2}
\end{equation}
por lo que el algoritmo es $\mathcal{O}(n^2)$.

\subsection{Ordenación por inserción}
De nuevo, el algoritmo está compuesto por dos bucles anidados, de los cuales el primero ejecuta $n-1$ iteraciones y el segundo realiza como máximo $i$ (el índice del primer bucle). La función que nos da el tiempo de ejecución en el peor caso es, por tanto:
\begin{equation}
 T(n) = \sum_{i=0}^{n-1} i = \frac{(n-1)n}{2} = \frac{n^2}{2} - \frac{n}{2}
\end{equation}
y el algoritmo es $\mathcal{O}(n^2)$ en el caso peor.
\subsection{Ordenación por selección}
Al igual que los anteriores, consta de dos bucles anidados. El exterior realiza $n-1$ iteraciones, y el interior tantas como indica el índice del primero, $i$, lo que nos da un tiempo de ejecución:
\begin{equation}
 T(n) = \sum_{i=0}^{n-1} i = \frac{(n-1)n}{2} = \frac{n^2}{2} - \frac{n}{2}
\end{equation}
por lo que este algoritmo también es $\mathcal{O}(n^2)$ en el caso peor.

\subsection{Ordenación \textit{heapsort}}
En este algoritmo insertaremos los datos en un árbol tipo heap y los extraeremos de nuevo ya ordenados.
La inserción en una estructura heap es del orden $\mathcal{O}(log n)$ y la extracción del mínimo, que involucra
reajustar el árbol, es también $\mathcal{O}(log n)$. En conjunto, y sabiendo que insertaremos y extraeremos $n$ elementos,
el orden del algoritmo es:
\begin{equation}
 T(n) = 2 n log(n)
\end{equation}
por lo que el algoritmo tiene complejidad $\mathcal{O}(nlog(n))$.


\subsection{Ordenación por mezcla (\textit{mergesort})}
El algoritmo mergesort es un algoritmo divide y vencerás para ordenación, basado en la ordenación de las dos mitades del
array independientemente. En nuestro caso usaremos \textit{insertion sort} para los problemas que quedan por debajo de una talla umbral.
Sabiendo que la ordenación en un caso suficientemente pequeño se acota por una constante y que el algoritmo recurre a la ordenación
de dos casos de tamaño mitad y una mezcla de ambos lineal:
\begin{equation}
 T(n) = 2T\left(\frac{n}{2}\right) + n
\end{equation}
Que tiene soluciones del tipo $an + bnlog(n)$, el algoritmo es de complejidad $\mathcal{O}(nlog(n))$.


\subsection{Algoritmo de Hoare (\textit{quicksort})}
El algoritmo quicksort toma un pivote para partir los elementos a ordenar en aquellos mayores y aquellos menores que el pivote.
En su caso medio, promediaremos las elecciones del pivote. Siempre se tendrá una fase de complejidad lineal previa en la que los
elementos se particionan por el pivote, y, posteriormente, habrá una fase en la que se analizarán las otras dos mitades. Tenemos:
\begin{equation}
 T(n) = n - 1 + \frac{1}{n}\left(\sum_{i=0}^{n-1} T(i) + T(n-1-i) \right)
\end{equation}
Que tiene una solución de complejidad $\mathcal{O}(nlog(n))$.


\subsection {Sucesión de Fibonacci}
El algoritmo está escrito recursivamente y sin memoización. La llamada a la función con tamaño $n$ produce 
dos llamadas recursivas con tamaños $n-1$ y $n-2$. Tendremos la ecuación en diferencias:
\begin{equation}
 T(n) = T(n-1) + T(n-2)
\end{equation}
con la solución conocida de complejidad $\mathcal{O}(\phi^n)$.

\subsection{Algoritmo de Floyd}
El algoritmo de Floyd guarda en una matriz los actuales caminos mínimos entre dos vértices de un grafo de $n$ vértices
y realiza un intercambio entre el camino de dos vértices si este puede mejorarse pasando por el vértice $n$. En resumen,
tendremos tres bucles anidados realizando una operación constante. Cada uno de ellos realizando $n$ iteraciones.

La complejidad del algoritmo es $\mathcal{O}(n^3)$.


\subsection{Torres de Hanoi}
El algoritmo de las torres de Hanoi de talla $n$ se llama a sí mismo recursivamente dos veces y cada una de ellas con
talla $n-1$. Resuelve la torre para los $n-1$ discos superiores, mueve la base, y la resuelve de nuevo para colocarlos sobre ella.
Tendremos entonces como ecuación principal:
\begin{equation}
 T(n) = 2T(n-1) + 1
\end{equation}
Con soluciones de complejidad $\mathcal{O}(2^n)$.

\section {Análisis práctico}
\subsection{Modificaciones del código}
A la hora de realizar un análisis empírico de la eficiencia de estos algoritmos, se ha estudiado el tiempo de ejecución de los programas que llevan a cabo cada tarea.
Para ello, se ha utilizado la función \texttt{clock\_gettime()} de la biblioteca \texttt{time.h} de C. Esta función proporciona una resolución de $0.001\ \mu s$, mejor que otras utilidades como \texttt{clock()} o \texttt{time()}.
El código añadido para monitorear el tiempo de ejecución ha sido el siguiente:

\texttt{\mbox{}\ \ \ \ \textbf{\textcolor{Blue}{struct}}\ \textcolor{TealBlue}{timespec}\ t$\_$antes\textcolor{BrickRed}{,}\ t$\_$despues\textcolor{BrickRed}{;} \\
\mbox{}\ \ \ \ \textbf{\textcolor{Black}{clock$\_$gettime}}\textcolor{BrickRed}{(}CLOCK$\_$REALTIME\textcolor{BrickRed}{,\&}t$\_$antes\textcolor{BrickRed}{);} \\
\mbox{}\ \ \ \ \textbf{\textcolor{Brown}{/*** Ejecución del programa ***/}} \\
\mbox{}\ \ \ \ \textbf{\textcolor{Black}{clock$\_$gettime}}\textcolor{BrickRed}{(}CLOCK$\_$REALTIME\textcolor{BrickRed}{,\&}t$\_$despues\textcolor{BrickRed}{);} \\}

Una vez obtenidos los tiempos al inicio y finalización de la ejecución, se realiza el cálculo conveniente para hallar, mediante las estructuras \texttt{timespec}, los segundos y nanosegundos transcurridos. Se han modificado los programas para que muestren en la salida únicamente este valor numérico.

\subsection{Archivos auxiliares}

Para el tratamiento de los algoritmos con distintos tamaños de entrada, se ha utilizado un guion en Bash (sección \ref{gengraf}) que ejecuta los programas las veces necesarias y con los parámetros convenientes, para obtener todos los tiempos de ejecución requeridos. Permite además ejecutar más de una vez con el mismo tamaño de problema para realizar una media de los tiempos obtenidos (en nuestro caso se ha hecho la media de 5 repeticiones de cada tamaño).

Además, se ha añadido a este guion la funcionalidad necesaria para comunicarse con \texttt{gnuplot} y crear una gráfica a partir del archivo de datos generado por uno de los programas, y crear también la gráfica de la función de ajuste. En cada caso, el Makefile se encarga de proporcionar los parámetros necesarios al guion y de redirigir la salida al archivo correcto, de forma que el usuario solo debe ejecutar órdenes del tipo \texttt{make plot} o \texttt{make fit} para generar gráficas de datos y de ajuste, respectivamente, de todos los programas.


\normalsize

\section {Conclusiones}
\subsection {Gráficas}
\imagen{../regressionPlots/burbuja_fit.jpg}{Ordenación por burbuja}

\imagen{../regressionPlots/fibonacci_fit.jpg}{Serie de Fibonacci}

\imagen{../regressionPlots/floyd_fit.jpg}{Algoritmo de Floyd}

\imagen{../regressionPlots/hanoi_fit.jpg}{Torres de Hanoi}

\imagen{../regressionPlots/heapsort_fit.jpg}{Ordenación por heapsort}

\imagen{../regressionPlots/insercion_fit.jpg}{Ordenación por inserción}

\imagen{../regressionPlots/mergesort_fit.jpg}{Ordenación por mergesort}

\imagen{../regressionPlots/quicksort_fit.jpg}{Ordenación por quicksort}

\imagen{../regressionPlots/seleccion_fit.jpg}{Ordenación por seleccion}

\subsection{Análisis práctico de la eficiencia}
Para realizar un análisis práctico de la eficiencia, el script adjunto en el
apéndice, para una lista de funciones (lineales, cúbicas, exponenciales,...) 
calcula el ajuste llamando a \texttt{gnuplot} de cada función, y la media
de los residuos cuadráticos correspondientes. El mejor ajuste posible sería
aquel que generase unos residuos menores. Hay que tener en cuenta que la 
eficiencia es un concepto asintótico, y puesto que se han evaluado los algoritmos
para los datos que permitían un tiempo de ejecución de los mismos asumible, 
es posible que la eficiencia práctica no se haya obtenido como debiera.
\subsection{Algoritmos teóricamente $\mathcal{O}(n^2)$}
\begin{itemize}
\item Ordenación por burbuja\\
El mejor ajuste práctico ha resultado ser: $f(n)=-6.82525\cdot 10^{-12} \cdot n^3+1.26402\cdot 10^{7}\cdot n^2-0.000661563 \cdot n+0.999054$
El ajuste que deberíamos haber obtenido: $f(n)=a0 \cdot n^2+a1 \cdot n+a0$

\begin{tabular}{|c|c|}
\hline
Ajuste & Media de los residuos cuadráticos\\
\hline
$f(n)=3.55198\cdot 10^{-5}\cdot n -0.0635027$ & $0.0318751$ \\
$f(n)=4.20281\cdot 10^{-9}\cdot n^2-6.17206\cdot 10^{-6}\cdot n+0.00487322$ & $0.00391704$ \\
$f(n)=4.0846\cdot 10^{-5}\cdot n \cdot log(n)-0.000353425\cdot n+0.0204474 \cdot log(n)-0.0654402$ & $0.00767502$ \\
\hline
\end{tabular}

\item Ordenación por inserción\\
El mejor ajuste práctico ha resultado ser: $f(n)=1.03091\cdot 10^{-14} \cdot n^3+1.32572\cdot 10^{-9}\cdot n^2+7.21102 \cdot 10^{-7}\cdot n-0.000351074$
con una media de residuos cuadráticos de $0.000877862$
El ajuste que deberíamos haber obtenido: $f(n)=a0 \cdot n^2+a1 \cdot n+a0$

\begin{tabular}{|c|c|}
\hline
Ajuste & Media de los residuos cuadráticos\\
\hline
$f(n)=1.47877\cdot 10^{-5}\cdot n -0.0239239$ & $0.0111694$ \\
$f(n)=1.47912\cdot 10^{-9}\cdot n^2-1.1484\cdot 10^{-7}\cdot n+0.000140066$ & $0.000895417$ \\
$f(n)=1.40522\cdot 10^{-5}\cdot n \cdot log(n)-0.000118807\cdot n+0.00635583 \cdot log(n)-0.0200974$ & $0.00230142$ \\
\hline
\end{tabular}

\item Ordeanción por selección\\
El mejor ajuste práctico ha resultado ser: 

% rellenar
Es lógico que el análisis práctico en los algoritmos $\mathcal{O}(n^2)$
hayamos obtenido un polinomio cúbico, ya que $\mathbb{P}^2 \subseteq \mathbb{P}^3$

\subsection{Algoritmos teóricamente $\mathcal{O}(nlog(n)$}
\begin{itemize}
\item Ordenación por mergesort\\
El mejor ajuste práctico ha resultado ser su recta de regresión: $f(n)=2.38969\cdot 10^{-7} -4.94954\cdot 10 ^{-5}$
con una media de residuos cuadráticos de $0.000118196$
El ajuste que deberíamos haber obtenido: $f(n)=a0 \cdot nlog(n) + a1\cdot n+a2 \cdot log(n) + a3$

\hline
Ajuste & Media de los residuos cuadráticos\\
\hline
$f(n)=1.10858\cdot 10^{-12}\cdot n^2+2.27971\cdot 10^{-7}\cdot n-3.14598\cdot 10^{-5}$ & $0.000118507$ \\
$f(n)=4.74472\cdot 10^{-16} \cdot n^3-5.95156\cdot 10^{-12}\cdot n^2+2.55874 \cdot 10^{-7}\cdot n-5.40642 \cdot 10^{-5}$  & $0.000118771$ \\
$f(n)=3.03488 \cdot 10^{-9} \cdot nlog(n) + 2.13985 \cdot 10^{-7}\cdot n-1.09431\cdot 10^{-5} \cdot log(n) + 3.28662\cdot 10^{-5}$ & $0.000119057$\\
\hline

\item Ordenación por quicksort\\
El mejor ajuste práctico ha resultado ser: $f(n)=-4.52927\cdot 10^{-16} \cdot n^3+8.28261\cdot 10^{-12}\cdot n^2+1.13041 \cdot 10^{-7}\cdot n-
1.72007\cdot 10^{-6}$ con una media de residuos cuadráticos de $8.29712\cdot 10^{-6}$
El ajuste que deberíamos haber obtenido: $f(n)=a0 \cdot nlog(n) + a1\cdot n+a2 \cdot log(n) + a3$
\hline
Ajuste & Media de los residuos cuadráticos\\
\hline
$f(n)=1.54984\cdot 10^{-7}\cdot n -4.84022 \cdot 10^{-5}$ & $1.66434\cdot 10^{-5}$ \\
$f(n)=1.54306\cdot 10^{-12}\cdot n^2+1.39677\cdot 10^{-7}\cdot n-2.32981\cdot 10^{-5}$ & $1.19816\cdot 10^{-5}$ \\
$f(n)=1.07463\cdot 10^{-8}\cdot n \cdot log(n)+5.59666\cdot 10^{-8}\cdot n-5.16002\cdot 10^{-6} \cdot log(n)+2.0864\cdot 10^{-5}$ & $1.00822\cdot 10^{-5}$ \\
\hline
\end{tabular}

\newpage
\section{Apéndice}

\subsection{Listados de código}
A continuación se incluyen los códigos modificados que se han usado en la práctica.
\scriptsize{
\begin{allintypewriter}

\subsubsection{Ordenación de la burbuja}
% Generator: GNU source-highlight, by Lorenzo Bettini, http://www.gnu.org/software/src-highlite
\noindent
\mbox{}\textit{\textcolor{Brown}{/**}} \\
\mbox{}\textit{\textcolor{Brown}{\ *\ }}\textcolor{ForestGreen}{@file}\textit{\textcolor{Brown}{\ Ordenación\ por\ burbuja}} \\
\mbox{}\textit{\textcolor{Brown}{\ */}} \\
\mbox{} \\
\mbox{}\textbf{\textcolor{RoyalBlue}{\#include}}\ \texttt{\textcolor{Red}{\textless{}iostream\textgreater{}}} \\
\mbox{}\textbf{\textcolor{RoyalBlue}{\#include}}\ \texttt{\textcolor{Red}{\textless{}ctime\textgreater{}}} \\
\mbox{}\textbf{\textcolor{RoyalBlue}{\#include}}\ \texttt{\textcolor{Red}{\textless{}cstdlib\textgreater{}}} \\
\mbox{}\textbf{\textcolor{RoyalBlue}{\#include}}\ \texttt{\textcolor{Red}{\textless{}ctime\textgreater{}}} \\
\mbox{}\textbf{\textcolor{Blue}{using}}\ \textbf{\textcolor{Blue}{namespace}}\ std\textcolor{BrickRed}{;} \\
\mbox{} \\
\mbox{}\textit{\textcolor{Brown}{/**}} \\
\mbox{}\textit{\textcolor{Brown}{\ *\ }}\textcolor{ForestGreen}{@brief}\textit{\textcolor{Brown}{\ Ordena\ un\ vector\ por\ el\ método\ de\ la\ burbuja.}} \\
\mbox{}\textit{\textcolor{Brown}{\ *\ }}\textcolor{ForestGreen}{@param}\textit{\textcolor{Brown}{\ T:\ vector\ de\ elementos.\ Debe\ tener\ num$\_$elem\ elementos.}} \\
\mbox{}\textit{\textcolor{Brown}{\ *\ Es\ modificado.}} \\
\mbox{}\textit{\textcolor{Brown}{\ *\ }}\textcolor{ForestGreen}{@param}\textit{\textcolor{Brown}{\ num$\_$elem:\ número\ de\ elementos.\ num$\_$elem\ \textgreater{}\ 0.}} \\
\mbox{}\textit{\textcolor{Brown}{\ *\ }} \\
\mbox{}\textit{\textcolor{Brown}{\ *\ Cambia\ el\ orden\ de\ los\ elementos\ de\ T\ de\ forma\ que\ los\ dispone}} \\
\mbox{}\textit{\textcolor{Brown}{\ *\ en\ sentido\ creciente\ de\ menor\ a\ mayor.}} \\
\mbox{}\textit{\textcolor{Brown}{\ *\ Aplica\ el\ algoritmo\ de\ la\ burbuja.}} \\
\mbox{}\textit{\textcolor{Brown}{\ */}} \\
\mbox{} \\
\mbox{}\textbf{\textcolor{Blue}{inline}}\ \textbf{\textcolor{Blue}{static}}\ \textcolor{ForestGreen}{void}\ \textbf{\textcolor{Black}{burbuja}}\textcolor{BrickRed}{(}\textcolor{ForestGreen}{int}\ T\textcolor{BrickRed}{[],}\ \textcolor{ForestGreen}{int}\ num$\_$elem\textcolor{BrickRed}{);} \\
\mbox{} \\
\mbox{}\textit{\textcolor{Brown}{/**}} \\
\mbox{}\textit{\textcolor{Brown}{\ *\ }}\textcolor{ForestGreen}{@brief}\textit{\textcolor{Brown}{\ Ordena\ parte\ de\ un\ vector\ por\ el\ método\ de\ la\ burbuja.}} \\
\mbox{}\textit{\textcolor{Brown}{\ *\ }}\textcolor{ForestGreen}{@param}\textit{\textcolor{Brown}{\ T:\ vector\ de\ elementos.\ Tiene\ un\ número\ de\ elementos}} \\
\mbox{}\textit{\textcolor{Brown}{\ *\ mayor\ o\ igual\ a\ final.Es\ MODIFICADO.}} \\
\mbox{}\textit{\textcolor{Brown}{\ *\ }}\textcolor{ForestGreen}{@param}\textit{\textcolor{Brown}{\ inicial:\ Posición\ que\ marca\ el\ incio\ de\ la\ parte\ del}} \\
\mbox{}\textit{\textcolor{Brown}{\ *\ vector\ a\ ordenar.}} \\
\mbox{}\textit{\textcolor{Brown}{\ *\ }}\textcolor{ForestGreen}{@param}\textit{\textcolor{Brown}{\ final:\ Posición\ detrás\ de\ la\ última\ de\ la\ parte\ del}} \\
\mbox{}\textit{\textcolor{Brown}{\ *\ vector\ a\ ordenar.}} \\
\mbox{}\textit{\textcolor{Brown}{\ *\ inicial\ \textless{}\ final.}} \\
\mbox{}\textit{\textcolor{Brown}{\ *\ }} \\
\mbox{}\textit{\textcolor{Brown}{\ *\ Cambia\ el\ orden\ de\ los\ elementos\ de\ T\ entre\ las\ posiciones}} \\
\mbox{}\textit{\textcolor{Brown}{\ *\ inicial\ y\ final\ -\ 1de\ forma\ que\ los\ dispone\ en\ sentido\ creciente}} \\
\mbox{}\textit{\textcolor{Brown}{\ *\ de\ menor\ a\ mayor.}} \\
\mbox{}\textit{\textcolor{Brown}{\ *\ Aplica\ el\ algoritmo\ de\ la\ burbuja.}} \\
\mbox{}\textit{\textcolor{Brown}{\ */}} \\
\mbox{} \\
\mbox{}\textbf{\textcolor{Blue}{static}}\ \textcolor{ForestGreen}{void}\ \textbf{\textcolor{Black}{burbuja$\_$lims}}\textcolor{BrickRed}{(}\textcolor{ForestGreen}{int}\ T\textcolor{BrickRed}{[],}\ \textcolor{ForestGreen}{int}\ inicial\textcolor{BrickRed}{,}\ \textcolor{ForestGreen}{int}\ final\textcolor{BrickRed}{);} \\
\mbox{} \\
\mbox{}\textit{\textcolor{Brown}{//\ Implementación}} \\
\mbox{} \\
\mbox{}\textbf{\textcolor{Blue}{inline}}\ \textcolor{ForestGreen}{void}\ \textbf{\textcolor{Black}{burbuja}}\textcolor{BrickRed}{(}\textcolor{ForestGreen}{int}\ T\textcolor{BrickRed}{[],}\ \textcolor{ForestGreen}{int}\ num$\_$elem\textcolor{BrickRed}{)}\textcolor{Red}{\{} \\
\mbox{}\ \ \ \ \textbf{\textcolor{Black}{burbuja$\_$lims}}\textcolor{BrickRed}{(}T\textcolor{BrickRed}{,}\ \textcolor{Purple}{0}\textcolor{BrickRed}{,}\ num$\_$elem\textcolor{BrickRed}{);} \\
\mbox{}\textcolor{Red}{\}} \\
\mbox{} \\
\mbox{}\textbf{\textcolor{Blue}{static}}\ \textcolor{ForestGreen}{void}\ \textbf{\textcolor{Black}{burbuja$\_$lims}}\textcolor{BrickRed}{(}\textcolor{ForestGreen}{int}\ T\textcolor{BrickRed}{[],}\ \textcolor{ForestGreen}{int}\ inicial\textcolor{BrickRed}{,}\ \textcolor{ForestGreen}{int}\ final\textcolor{BrickRed}{)}\textcolor{Red}{\{} \\
\mbox{}\ \ \ \ \textcolor{ForestGreen}{int}\ i\textcolor{BrickRed}{,}\ j\textcolor{BrickRed}{;} \\
\mbox{}\ \ \ \ \textcolor{ForestGreen}{int}\ aux\textcolor{BrickRed}{;} \\
\mbox{}\ \ \ \ \textbf{\textcolor{Blue}{for}}\ \textcolor{BrickRed}{(}i\ \textcolor{BrickRed}{=}\ inicial\textcolor{BrickRed}{;}\ i\ \textcolor{BrickRed}{\textless{}}\ final\ \textcolor{BrickRed}{-}\ \textcolor{Purple}{1}\textcolor{BrickRed}{;}\ i\textcolor{BrickRed}{++)} \\
\mbox{}\ \ \ \ \ \ \ \ \textbf{\textcolor{Blue}{for}}\ \textcolor{BrickRed}{(}j\ \textcolor{BrickRed}{=}\ final\ \textcolor{BrickRed}{-}\ \textcolor{Purple}{1}\textcolor{BrickRed}{;}\ j\ \textcolor{BrickRed}{\textgreater{}}\ i\textcolor{BrickRed}{;}\ j\textcolor{BrickRed}{-\/-)} \\
\mbox{}\ \ \ \ \ \ \ \ \ \ \ \ \textbf{\textcolor{Blue}{if}}\ \textcolor{BrickRed}{(}T\textcolor{BrickRed}{[}j\textcolor{BrickRed}{]}\ \textcolor{BrickRed}{\textless{}}\ T\textcolor{BrickRed}{[}j\textcolor{BrickRed}{-}\textcolor{Purple}{1}\textcolor{BrickRed}{])} \\
\mbox{}\ \ \ \ \ \ \ \ \ \ \ \ \textcolor{Red}{\{} \\
\mbox{}\ \ \ \ \ \ \ \ \ \ \ \ \ \ \ \ aux\ \textcolor{BrickRed}{=}\ T\textcolor{BrickRed}{[}j\textcolor{BrickRed}{];} \\
\mbox{}\ \ \ \ \ \ \ \ \ \ \ \ \ \ \ \ T\textcolor{BrickRed}{[}j\textcolor{BrickRed}{]}\ \textcolor{BrickRed}{=}\ T\textcolor{BrickRed}{[}j\textcolor{BrickRed}{-}\textcolor{Purple}{1}\textcolor{BrickRed}{];} \\
\mbox{}\ \ \ \ \ \ \ \ \ \ \ \ \ \ \ \ T\textcolor{BrickRed}{[}j\textcolor{BrickRed}{-}\textcolor{Purple}{1}\textcolor{BrickRed}{]}\ \textcolor{BrickRed}{=}\ aux\textcolor{BrickRed}{;} \\
\mbox{}\ \ \ \ \ \ \ \ \ \ \ \ \textcolor{Red}{\}} \\
\mbox{}\textcolor{Red}{\}} \\
\mbox{} \\
\mbox{}\textcolor{ForestGreen}{int}\ \textbf{\textcolor{Black}{main}}\textcolor{BrickRed}{(}\textcolor{ForestGreen}{int}\ argc\textcolor{BrickRed}{,}\ \textcolor{ForestGreen}{char}\textcolor{BrickRed}{*}\ argv\textcolor{BrickRed}{[])}\textcolor{Red}{\{} \\
\mbox{}\ \ \ \ \textbf{\textcolor{Blue}{if}}\ \textcolor{BrickRed}{(}argc\ \textcolor{BrickRed}{!=}\textcolor{Purple}{2}\textcolor{BrickRed}{)}\textcolor{Red}{\{} \\
\mbox{}\ \ \ \ \ \ \ \ cerr\ \textcolor{BrickRed}{\textless{}\textless{}}\ \texttt{\textcolor{Red}{"{}Uso\ del\ programa:\ "{}}}\ \textcolor{BrickRed}{+}\ \textcolor{BrickRed}{(}string\textcolor{BrickRed}{)(}argv\textcolor{BrickRed}{[}\textcolor{Purple}{0}\textcolor{BrickRed}{])}\ \textcolor{BrickRed}{+}\ \texttt{\textcolor{Red}{"{}\ \textless{}número\ positivo\textgreater{}"{}}}\ \textcolor{BrickRed}{\textless{}\textless{}}\ endl\textcolor{BrickRed}{;}\ \  \\
\mbox{}\ \ \ \ \ \ \ \ \textbf{\textcolor{Blue}{return}}\ \textcolor{BrickRed}{-}\textcolor{Purple}{1}\textcolor{BrickRed}{;} \\
\mbox{}\ \ \ \ \textcolor{Red}{\}} \\
\mbox{}\ \ \ \ \textcolor{ForestGreen}{int}\ n\ \textcolor{BrickRed}{=}\ \textbf{\textcolor{Black}{atoi}}\textcolor{BrickRed}{(}argv\textcolor{BrickRed}{[}\textcolor{Purple}{1}\textcolor{BrickRed}{]);}\ \ \ \  \\
\mbox{}\ \ \ \ \textbf{\textcolor{Blue}{if}}\ \textcolor{BrickRed}{(}n\textcolor{BrickRed}{\textless{}}\textcolor{Purple}{0}\textcolor{BrickRed}{)}\ \textbf{\textcolor{Blue}{return}}\ \textcolor{BrickRed}{-}\textcolor{Purple}{1}\textcolor{BrickRed}{;} \\
\mbox{}\ \ \ \  \\
\mbox{}\ \ \ \ \textcolor{ForestGreen}{int}\ \textcolor{BrickRed}{*}\ T\ \textcolor{BrickRed}{=}\ \textbf{\textcolor{Blue}{new}}\ \textcolor{ForestGreen}{int}\textcolor{BrickRed}{[}n\textcolor{BrickRed}{];} \\
\mbox{}\ \ \ \ \textbf{\textcolor{Blue}{struct}}\ \textcolor{TealBlue}{timespec}\ t$\_$antes\textcolor{BrickRed}{,}\ t$\_$despues\textcolor{BrickRed}{;} \\
\mbox{}\ \ \ \  \\
\mbox{}\ \ \ \ \textbf{\textcolor{Black}{srandom}}\textcolor{BrickRed}{(}\textbf{\textcolor{Black}{time}}\textcolor{BrickRed}{(}\textcolor{Purple}{0}\textcolor{BrickRed}{));} \\
\mbox{}\ \ \ \  \\
\mbox{}\ \ \ \ \textbf{\textcolor{Blue}{for}}\ \textcolor{BrickRed}{(}\textcolor{ForestGreen}{int}\ i\textcolor{BrickRed}{=}\textcolor{Purple}{0}\textcolor{BrickRed}{;}\ i\textcolor{BrickRed}{\textless{}}n\textcolor{BrickRed}{;}\ \textcolor{BrickRed}{++}i\textcolor{BrickRed}{)}\textcolor{Red}{\{} \\
\mbox{}\ \ \ \ \ \ \ \ T\textcolor{BrickRed}{[}i\textcolor{BrickRed}{]}\ \textcolor{BrickRed}{=}\ \textbf{\textcolor{Black}{random}}\textcolor{BrickRed}{();} \\
\mbox{}\ \ \ \ \textcolor{Red}{\}} \\
\mbox{}\ \ \ \  \\
\mbox{}\ \ \ \ \textbf{\textcolor{Black}{clock$\_$gettime}}\textcolor{BrickRed}{(}CLOCK$\_$REALTIME\textcolor{BrickRed}{,\&}t$\_$antes\textcolor{BrickRed}{);} \\
\mbox{}\ \ \ \ \textbf{\textcolor{Black}{burbuja}}\ \textcolor{BrickRed}{(}T\textcolor{BrickRed}{,}n\textcolor{BrickRed}{);} \\
\mbox{}\ \ \ \ \textbf{\textcolor{Black}{clock$\_$gettime}}\textcolor{BrickRed}{(}CLOCK$\_$REALTIME\textcolor{BrickRed}{,\&}t$\_$despues\textcolor{BrickRed}{);} \\
\mbox{}\ \ \ \  \\
\mbox{}\ \ \ \ cout\textcolor{BrickRed}{.}\textbf{\textcolor{Black}{precision}}\textcolor{BrickRed}{(}\textcolor{Purple}{3}\textcolor{BrickRed}{);} \\
\mbox{}\ \ \ \ cout\ \textcolor{BrickRed}{\textless{}\textless{}}\ \textcolor{BrickRed}{(}\textcolor{ForestGreen}{double}\textcolor{BrickRed}{)}\ \textcolor{BrickRed}{(}t$\_$despues\textcolor{BrickRed}{.}tv$\_$sec\textcolor{BrickRed}{-}t$\_$antes\textcolor{BrickRed}{.}tv$\_$sec\textcolor{BrickRed}{)+} \\
\mbox{}\ \ \ \ \ \ \ \ \textcolor{BrickRed}{(}\textcolor{ForestGreen}{double}\textcolor{BrickRed}{)}\ \textcolor{BrickRed}{((}t$\_$despues\textcolor{BrickRed}{.}tv$\_$nsec\textcolor{BrickRed}{-}t$\_$antes\textcolor{BrickRed}{.}tv$\_$nsec\textcolor{BrickRed}{)/(}\textcolor{Purple}{1}\textcolor{BrickRed}{.}e\textcolor{BrickRed}{+}\textcolor{Purple}{9}\textcolor{BrickRed}{))}\ \textcolor{BrickRed}{\textless{}\textless{}}\ endl\textcolor{BrickRed}{;} \\
\mbox{} \\
\mbox{}\ \ \ \  \\
\mbox{}\ \ \ \ \textbf{\textcolor{Blue}{delete}}\ \textcolor{BrickRed}{[]}\ T\textcolor{BrickRed}{;} \\
\mbox{}\ \ \ \  \\
\mbox{}\ \ \ \ \textbf{\textcolor{Blue}{return}}\ \textcolor{Purple}{0}\textcolor{BrickRed}{;} \\
\mbox{}\textcolor{Red}{\}}


\subsubsection{Ordenación por inserción}
% Generator: GNU source-highlight, by Lorenzo Bettini, http://www.gnu.org/software/src-highlite
\noindent
\mbox{}\textit{\textcolor{Brown}{/**}} \\
\mbox{}\textit{\textcolor{Brown}{\ *\ }}\textcolor{ForestGreen}{@file}\textit{\textcolor{Brown}{\ Ordenación\ por\ inserción}} \\
\mbox{}\textit{\textcolor{Brown}{\ */}} \\
\mbox{} \\
\mbox{}\textbf{\textcolor{RoyalBlue}{\#include}}\ \texttt{\textcolor{Red}{\textless{}iostream\textgreater{}}} \\
\mbox{}\textbf{\textcolor{RoyalBlue}{\#include}}\ \texttt{\textcolor{Red}{\textless{}ctime\textgreater{}}} \\
\mbox{}\textbf{\textcolor{RoyalBlue}{\#include}}\ \texttt{\textcolor{Red}{\textless{}cstdlib\textgreater{}}} \\
\mbox{}\textbf{\textcolor{RoyalBlue}{\#include}}\ \texttt{\textcolor{Red}{\textless{}ctime\textgreater{}}} \\
\mbox{}\textbf{\textcolor{Blue}{using}}\ \textbf{\textcolor{Blue}{namespace}}\ std\textcolor{BrickRed}{;} \\
\mbox{} \\
\mbox{}\textbf{\textcolor{RoyalBlue}{\#define}}\ NUM$\_$VECES\ \textcolor{Purple}{500} \\
\mbox{}\textit{\textcolor{Brown}{/**}} \\
\mbox{}\textit{\textcolor{Brown}{\ *\ }}\textcolor{ForestGreen}{@brief}\textit{\textcolor{Brown}{\ Ordena\ un\ vector\ por\ el\ método\ de\ inserción.}} \\
\mbox{}\textit{\textcolor{Brown}{\ *\ }}\textcolor{ForestGreen}{@param}\textit{\textcolor{Brown}{\ T:\ vector\ de\ elementos.\ Debe\ tener\ num$\_$elem\ elementos.}} \\
\mbox{}\textit{\textcolor{Brown}{\ *\ Es\ modificado.}} \\
\mbox{}\textit{\textcolor{Brown}{\ *\ }}\textcolor{ForestGreen}{@param}\textit{\textcolor{Brown}{\ num$\_$elem:\ número\ de\ elementos.\ num$\_$elem\ \textgreater{}\ 0.}} \\
\mbox{}\textit{\textcolor{Brown}{\ *\ }} \\
\mbox{}\textit{\textcolor{Brown}{\ *\ Cambia\ el\ orden\ de\ los\ elementos\ de\ T\ de\ forma\ que\ los\ dispone}} \\
\mbox{}\textit{\textcolor{Brown}{\ *\ en\ sentido\ creciente\ de\ menor\ a\ mayor.}} \\
\mbox{}\textit{\textcolor{Brown}{\ *\ Aplica\ el\ algoritmo\ de\ inserción.}} \\
\mbox{}\textit{\textcolor{Brown}{\ */}} \\
\mbox{} \\
\mbox{}\textbf{\textcolor{Blue}{inline}}\ \textbf{\textcolor{Blue}{static}}\ \textcolor{ForestGreen}{void}\ \textbf{\textcolor{Black}{insercion}}\textcolor{BrickRed}{(}\textcolor{ForestGreen}{int}\ T\textcolor{BrickRed}{[],}\ \textcolor{ForestGreen}{int}\ num$\_$elem\textcolor{BrickRed}{);} \\
\mbox{} \\
\mbox{}\textit{\textcolor{Brown}{/**}} \\
\mbox{}\textit{\textcolor{Brown}{\ *\ }}\textcolor{ForestGreen}{@brief}\textit{\textcolor{Brown}{\ Ordena\ parte\ de\ un\ vector\ por\ el\ método\ de\ inserción.}} \\
\mbox{}\textit{\textcolor{Brown}{\ *\ }}\textcolor{ForestGreen}{@param}\textit{\textcolor{Brown}{\ T:\ vector\ de\ elementos.\ Tiene\ un\ número\ de\ elementos\ }} \\
\mbox{}\textit{\textcolor{Brown}{\ *\ mayor\ o\ igual\ a\ final.\ Es\ MODIFICADO.}} \\
\mbox{}\textit{\textcolor{Brown}{\ *\ }}\textcolor{ForestGreen}{@param}\textit{\textcolor{Brown}{\ inicial:\ Posición\ que\ marca\ el\ incio\ de\ la\ parte\ del}} \\
\mbox{}\textit{\textcolor{Brown}{\ *\ vector\ a\ ordenar.}} \\
\mbox{}\textit{\textcolor{Brown}{\ *\ }}\textcolor{ForestGreen}{@param}\textit{\textcolor{Brown}{\ final:\ Posición\ detrás\ de\ la\ última\ de\ la\ parte\ del}} \\
\mbox{}\textit{\textcolor{Brown}{\ *\ vector\ a\ ordenar.\ }} \\
\mbox{}\textit{\textcolor{Brown}{\ *\ }}\textcolor{ForestGreen}{@pre}\textit{\textcolor{Brown}{\ inicial\ \textless{}\ final.}} \\
\mbox{}\textit{\textcolor{Brown}{\ *\ }} \\
\mbox{}\textit{\textcolor{Brown}{\ *\ Cambia\ el\ orden\ de\ los\ elementos\ de\ T\ entre\ las\ posiciones}} \\
\mbox{}\textit{\textcolor{Brown}{\ *\ inicial\ y\ final\ -\ 1de\ forma\ que\ los\ dispone\ en\ sentido\ creciente}} \\
\mbox{}\textit{\textcolor{Brown}{\ *\ de\ menor\ a\ mayor.}} \\
\mbox{}\textit{\textcolor{Brown}{\ *\ Aplica\ el\ algoritmo\ de\ inserción.}} \\
\mbox{}\textit{\textcolor{Brown}{\ */}} \\
\mbox{} \\
\mbox{}\textbf{\textcolor{Blue}{static}}\ \textcolor{ForestGreen}{void}\ \textbf{\textcolor{Black}{insercion$\_$lims}}\textcolor{BrickRed}{(}\textcolor{ForestGreen}{int}\ T\textcolor{BrickRed}{[],}\ \textcolor{ForestGreen}{int}\ inicial\textcolor{BrickRed}{,}\ \textcolor{ForestGreen}{int}\ final\textcolor{BrickRed}{);} \\
\mbox{} \\
\mbox{}\textit{\textcolor{Brown}{//\ Implementación\ de\ las\ funciones}} \\
\mbox{} \\
\mbox{}\textbf{\textcolor{Blue}{inline}}\ \textbf{\textcolor{Blue}{static}}\ \textcolor{ForestGreen}{void}\ \textbf{\textcolor{Black}{insercion}}\textcolor{BrickRed}{(}\textcolor{ForestGreen}{int}\ T\textcolor{BrickRed}{[],}\ \textcolor{ForestGreen}{int}\ num$\_$elem\textcolor{BrickRed}{)}\textcolor{Red}{\{} \\
\mbox{}\ \ \ \ \textbf{\textcolor{Black}{insercion$\_$lims}}\textcolor{BrickRed}{(}T\textcolor{BrickRed}{,}\ \textcolor{Purple}{0}\textcolor{BrickRed}{,}\ num$\_$elem\textcolor{BrickRed}{);} \\
\mbox{}\textcolor{Red}{\}} \\
\mbox{} \\
\mbox{}\textbf{\textcolor{Blue}{static}}\ \textcolor{ForestGreen}{void}\ \textbf{\textcolor{Black}{insercion$\_$lims}}\textcolor{BrickRed}{(}\textcolor{ForestGreen}{int}\ T\textcolor{BrickRed}{[],}\ \textcolor{ForestGreen}{int}\ inicial\textcolor{BrickRed}{,}\ \textcolor{ForestGreen}{int}\ final\textcolor{BrickRed}{)}\textcolor{Red}{\{} \\
\mbox{}\ \ \ \ \textcolor{ForestGreen}{int}\ i\textcolor{BrickRed}{,}\ j\textcolor{BrickRed}{;} \\
\mbox{}\ \ \ \ \textcolor{ForestGreen}{int}\ aux\textcolor{BrickRed}{;} \\
\mbox{}\ \ \ \ \textbf{\textcolor{Blue}{for}}\ \textcolor{BrickRed}{(}i\ \textcolor{BrickRed}{=}\ inicial\ \textcolor{BrickRed}{+}\ \textcolor{Purple}{1}\textcolor{BrickRed}{;}\ i\ \textcolor{BrickRed}{\textless{}}\ final\textcolor{BrickRed}{;}\ i\textcolor{BrickRed}{++)}\textcolor{Red}{\{} \\
\mbox{}\ \ \ \ \ \ \ \ j\ \textcolor{BrickRed}{=}\ i\textcolor{BrickRed}{;} \\
\mbox{}\ \ \ \ \ \ \ \ \textbf{\textcolor{Blue}{while}}\ \textcolor{BrickRed}{((}T\textcolor{BrickRed}{[}j\textcolor{BrickRed}{]}\ \textcolor{BrickRed}{\textless{}}\ T\textcolor{BrickRed}{[}j\textcolor{BrickRed}{-}\textcolor{Purple}{1}\textcolor{BrickRed}{])}\ \textcolor{BrickRed}{\&\&}\ \textcolor{BrickRed}{(}j\ \textcolor{BrickRed}{\textgreater{}}\ \textcolor{Purple}{0}\textcolor{BrickRed}{))}\textcolor{Red}{\{} \\
\mbox{}\ \ \ \ \ \ \ \ \ \ \ \ aux\ \textcolor{BrickRed}{=}\ T\textcolor{BrickRed}{[}j\textcolor{BrickRed}{];} \\
\mbox{}\ \ \ \ \ \ \ \ \ \ \ \ T\textcolor{BrickRed}{[}j\textcolor{BrickRed}{]}\ \textcolor{BrickRed}{=}\ T\textcolor{BrickRed}{[}j\textcolor{BrickRed}{-}\textcolor{Purple}{1}\textcolor{BrickRed}{];} \\
\mbox{}\ \ \ \ \ \ \ \ \ \ \ \ T\textcolor{BrickRed}{[}j\textcolor{BrickRed}{-}\textcolor{Purple}{1}\textcolor{BrickRed}{]}\ \textcolor{BrickRed}{=}\ aux\textcolor{BrickRed}{;} \\
\mbox{}\ \ \ \ \ \ \ \ \ \ \ \ j\textcolor{BrickRed}{-\/-;} \\
\mbox{}\ \ \ \ \ \ \ \ \textcolor{Red}{\}} \\
\mbox{}\ \ \ \ \textcolor{Red}{\}} \\
\mbox{}\textcolor{Red}{\}} \\
\mbox{} \\
\mbox{}\textcolor{ForestGreen}{int}\ \textbf{\textcolor{Black}{main}}\textcolor{BrickRed}{(}\textcolor{ForestGreen}{int}\ argc\textcolor{BrickRed}{,}\ \textcolor{ForestGreen}{char}\textcolor{BrickRed}{*}\ argv\textcolor{BrickRed}{[])}\textcolor{Red}{\{} \\
\mbox{}\ \ \ \ \textbf{\textcolor{Blue}{if}}\ \textcolor{BrickRed}{(}argc\ \textcolor{BrickRed}{!=}\textcolor{Purple}{2}\textcolor{BrickRed}{)}\textcolor{Red}{\{} \\
\mbox{}\ \ \ \ \ \ \ \ cerr\ \textcolor{BrickRed}{\textless{}\textless{}}\ \texttt{\textcolor{Red}{"{}Uso\ del\ programa:\ "{}}}\ \textcolor{BrickRed}{+}\ \textcolor{BrickRed}{(}string\textcolor{BrickRed}{)(}argv\textcolor{BrickRed}{[}\textcolor{Purple}{0}\textcolor{BrickRed}{])}\ \textcolor{BrickRed}{+}\ \texttt{\textcolor{Red}{"{}\ \textless{}número\ positivo\textgreater{}"{}}}\ \textcolor{BrickRed}{\textless{}\textless{}}\ endl\textcolor{BrickRed}{;}\ \  \\
\mbox{}\ \ \ \ \ \ \ \ \textbf{\textcolor{Blue}{return}}\ \textcolor{BrickRed}{-}\textcolor{Purple}{1}\textcolor{BrickRed}{;} \\
\mbox{}\ \ \ \ \textcolor{Red}{\}} \\
\mbox{}\ \ \ \ \textcolor{ForestGreen}{int}\ n\ \textcolor{BrickRed}{=}\ \textbf{\textcolor{Black}{atoi}}\textcolor{BrickRed}{(}argv\textcolor{BrickRed}{[}\textcolor{Purple}{1}\textcolor{BrickRed}{]);}\ \ \ \  \\
\mbox{}\ \ \ \ \textbf{\textcolor{Blue}{if}}\ \textcolor{BrickRed}{(}n\textcolor{BrickRed}{\textless{}}\textcolor{Purple}{0}\textcolor{BrickRed}{)}\ \textbf{\textcolor{Blue}{return}}\ \textcolor{BrickRed}{-}\textcolor{Purple}{1}\textcolor{BrickRed}{;} \\
\mbox{}\ \ \ \  \\
\mbox{}\ \ \ \ \textcolor{ForestGreen}{int}\ \textcolor{BrickRed}{*}\ T\ \textcolor{BrickRed}{=}\ \textbf{\textcolor{Blue}{new}}\ \textcolor{ForestGreen}{int}\textcolor{BrickRed}{[}n\textcolor{BrickRed}{];} \\
\mbox{}\ \ \ \ \textbf{\textcolor{Blue}{struct}}\ \textcolor{TealBlue}{timespec}\ t$\_$antes\textcolor{BrickRed}{,}\ t$\_$despues\textcolor{BrickRed}{;} \\
\mbox{}\ \ \ \  \\
\mbox{}\ \ \ \ \textbf{\textcolor{Black}{srandom}}\textcolor{BrickRed}{(}\textbf{\textcolor{Black}{time}}\textcolor{BrickRed}{(}\textcolor{Purple}{0}\textcolor{BrickRed}{));} \\
\mbox{}\ \ \ \  \\
\mbox{}\ \ \ \ \textbf{\textcolor{Blue}{for}}\ \textcolor{BrickRed}{(}\textcolor{ForestGreen}{int}\ i\textcolor{BrickRed}{=}\textcolor{Purple}{0}\textcolor{BrickRed}{;}\ i\textcolor{BrickRed}{\textless{}}n\textcolor{BrickRed}{;}\ \textcolor{BrickRed}{++}i\textcolor{BrickRed}{)}\textcolor{Red}{\{} \\
\mbox{}\ \ \ \ \ \ \ \ T\textcolor{BrickRed}{[}i\textcolor{BrickRed}{]}\ \textcolor{BrickRed}{=}\ \textbf{\textcolor{Black}{random}}\textcolor{BrickRed}{();} \\
\mbox{}\ \ \ \ \textcolor{Red}{\}} \\
\mbox{}\ \ \ \  \\
\mbox{}\ \ \ \ \textbf{\textcolor{Black}{clock$\_$gettime}}\textcolor{BrickRed}{(}CLOCK$\_$REALTIME\textcolor{BrickRed}{,\&}t$\_$antes\textcolor{BrickRed}{);} \\
\mbox{}\ \ \ \ \textbf{\textcolor{Black}{insercion}}\ \textcolor{BrickRed}{(}T\textcolor{BrickRed}{,}n\textcolor{BrickRed}{);} \\
\mbox{}\ \ \ \ \textbf{\textcolor{Black}{clock$\_$gettime}}\textcolor{BrickRed}{(}CLOCK$\_$REALTIME\textcolor{BrickRed}{,\&}t$\_$despues\textcolor{BrickRed}{);} \\
\mbox{}\ \ \ \  \\
\mbox{}\ \ \ \ cout\textcolor{BrickRed}{.}\textbf{\textcolor{Black}{precision}}\textcolor{BrickRed}{(}\textcolor{Purple}{3}\textcolor{BrickRed}{);} \\
\mbox{}\ \ \ \ cout\ \textcolor{BrickRed}{\textless{}\textless{}}\ \textcolor{BrickRed}{(}\textcolor{ForestGreen}{double}\textcolor{BrickRed}{)}\ \textcolor{BrickRed}{(}t$\_$despues\textcolor{BrickRed}{.}tv$\_$sec\textcolor{BrickRed}{-}t$\_$antes\textcolor{BrickRed}{.}tv$\_$sec\textcolor{BrickRed}{)+} \\
\mbox{}\ \ \ \ \ \ \ \ \textcolor{BrickRed}{(}\textcolor{ForestGreen}{double}\textcolor{BrickRed}{)}\ \textcolor{BrickRed}{((}t$\_$despues\textcolor{BrickRed}{.}tv$\_$nsec\textcolor{BrickRed}{-}t$\_$antes\textcolor{BrickRed}{.}tv$\_$nsec\textcolor{BrickRed}{)/(}\textcolor{Purple}{1}\textcolor{BrickRed}{.}e\textcolor{BrickRed}{+}\textcolor{Purple}{9}\textcolor{BrickRed}{))}\ \textcolor{BrickRed}{\textless{}\textless{}}\ endl\textcolor{BrickRed}{;} \\
\mbox{} \\
\mbox{}\ \ \ \  \\
\mbox{}\ \ \ \ \textbf{\textcolor{Blue}{delete}}\ \textcolor{BrickRed}{[]}\ T\textcolor{BrickRed}{;} \\
\mbox{}\ \ \ \  \\
\mbox{}\ \ \ \ \textbf{\textcolor{Blue}{return}}\ \textcolor{Purple}{0}\textcolor{BrickRed}{;} \\
\mbox{}\textcolor{Red}{\}} \\
\mbox{}


\subsubsection{Ordenación por selección}
% Generator: GNU source-highlight, by Lorenzo Bettini, http://www.gnu.org/software/src-highlite
\noindent
\mbox{}\textit{\textcolor{Brown}{/**}} \\
\mbox{}\textit{\textcolor{Brown}{\ *\ }}\textcolor{ForestGreen}{@file}\textit{\textcolor{Brown}{\ Ordenación\ por\ selección}} \\
\mbox{}\textit{\textcolor{Brown}{\ */}} \\
\mbox{} \\
\mbox{}\textbf{\textcolor{RoyalBlue}{\#include}}\ \texttt{\textcolor{Red}{\textless{}iostream\textgreater{}}} \\
\mbox{}\textbf{\textcolor{RoyalBlue}{\#include}}\ \texttt{\textcolor{Red}{\textless{}ctime\textgreater{}}} \\
\mbox{}\textbf{\textcolor{RoyalBlue}{\#include}}\ \texttt{\textcolor{Red}{\textless{}cstdlib\textgreater{}}} \\
\mbox{}\textbf{\textcolor{RoyalBlue}{\#include}}\ \texttt{\textcolor{Red}{\textless{}ctime\textgreater{}}} \\
\mbox{}\textbf{\textcolor{RoyalBlue}{\#include}}\ \texttt{\textcolor{Red}{\textless{}cassert\textgreater{}}} \\
\mbox{}\textbf{\textcolor{RoyalBlue}{\#include}}\ \texttt{\textcolor{Red}{\textless{}climits\textgreater{}}} \\
\mbox{}\textbf{\textcolor{Blue}{using}}\ \textbf{\textcolor{Blue}{namespace}}\ std\textcolor{BrickRed}{;} \\
\mbox{} \\
\mbox{}\textit{\textcolor{Brown}{/**}} \\
\mbox{}\textit{\textcolor{Brown}{\ *\ }}\textcolor{ForestGreen}{@brief}\textit{\textcolor{Brown}{\ Ordena\ un\ vector\ por\ el\ método\ de\ selección.}} \\
\mbox{}\textit{\textcolor{Brown}{\ *\ }}\textcolor{ForestGreen}{@param}\textit{\textcolor{Brown}{\ T:\ vector\ de\ elementos.\ Debe\ tener\ num$\_$elem\ elementos.}} \\
\mbox{}\textit{\textcolor{Brown}{\ *\ Es\ modificado.}} \\
\mbox{}\textit{\textcolor{Brown}{\ *\ }}\textcolor{ForestGreen}{@param}\textit{\textcolor{Brown}{\ num$\_$elem:\ número\ de\ elementos.\ num$\_$elem\ \textgreater{}\ 0.}} \\
\mbox{}\textit{\textcolor{Brown}{\ *\ }} \\
\mbox{}\textit{\textcolor{Brown}{\ *\ Cambia\ el\ orden\ de\ los\ elementos\ de\ T\ de\ forma\ que\ los\ dispone}} \\
\mbox{}\textit{\textcolor{Brown}{\ *\ en\ sentido\ creciente\ de\ menor\ a\ mayor.}} \\
\mbox{}\textit{\textcolor{Brown}{\ *\ Aplica\ el\ algoritmo\ de\ selección.}} \\
\mbox{}\textit{\textcolor{Brown}{\ */}} \\
\mbox{} \\
\mbox{}\textbf{\textcolor{Blue}{inline}}\ \textbf{\textcolor{Blue}{static}}\ \textcolor{ForestGreen}{void}\ \textbf{\textcolor{Black}{seleccion}}\textcolor{BrickRed}{(}\textcolor{ForestGreen}{int}\ T\textcolor{BrickRed}{[],}\ \textcolor{ForestGreen}{int}\ num$\_$elem\textcolor{BrickRed}{);} \\
\mbox{} \\
\mbox{}\textit{\textcolor{Brown}{/**}} \\
\mbox{}\textit{\textcolor{Brown}{\ *\ }}\textcolor{ForestGreen}{@brief}\textit{\textcolor{Brown}{\ Ordena\ parte\ de\ un\ vector\ por\ el\ método\ de\ selección.}} \\
\mbox{}\textit{\textcolor{Brown}{\ *\ }}\textcolor{ForestGreen}{@param}\textit{\textcolor{Brown}{\ T:\ vector\ de\ elementos.\ Tiene\ un\ número\ de\ elementos\ }} \\
\mbox{}\textit{\textcolor{Brown}{\ *\ mayor\ o\ igual\ a\ final.\ Es\ MODIFICADO.}} \\
\mbox{}\textit{\textcolor{Brown}{\ *\ }}\textcolor{ForestGreen}{@param}\textit{\textcolor{Brown}{\ inicial:\ Posición\ que\ marca\ el\ incio\ de\ la\ parte\ del}} \\
\mbox{}\textit{\textcolor{Brown}{\ *\ vector\ a\ ordenar.}} \\
\mbox{}\textit{\textcolor{Brown}{\ *\ }}\textcolor{ForestGreen}{@param}\textit{\textcolor{Brown}{\ final:\ Posición\ detrás\ de\ la\ última\ de\ la\ parte\ del}} \\
\mbox{}\textit{\textcolor{Brown}{\ *\ vector\ a\ ordenar.\ }} \\
\mbox{}\textit{\textcolor{Brown}{\ *\ }}\textcolor{ForestGreen}{@pre}\textit{\textcolor{Brown}{\ inicial\ \textless{}\ final.}} \\
\mbox{}\textit{\textcolor{Brown}{\ *\ }} \\
\mbox{}\textit{\textcolor{Brown}{\ *\ Cambia\ el\ orden\ de\ los\ elementos\ de\ T\ entre\ las\ posiciones}} \\
\mbox{}\textit{\textcolor{Brown}{\ *\ inicial\ y\ final\ -\ 1de\ forma\ que\ los\ dispone\ en\ sentido\ creciente}} \\
\mbox{}\textit{\textcolor{Brown}{\ *\ de\ menor\ a\ mayor.}} \\
\mbox{}\textit{\textcolor{Brown}{\ *\ Aplica\ el\ algoritmo\ de\ selección.}} \\
\mbox{}\textit{\textcolor{Brown}{\ */}} \\
\mbox{} \\
\mbox{}\textbf{\textcolor{Blue}{static}}\ \textcolor{ForestGreen}{void}\ \textbf{\textcolor{Black}{seleccion$\_$lims}}\textcolor{BrickRed}{(}\textcolor{ForestGreen}{int}\ T\textcolor{BrickRed}{[],}\ \textcolor{ForestGreen}{int}\ inicial\textcolor{BrickRed}{,}\ \textcolor{ForestGreen}{int}\ final\textcolor{BrickRed}{);} \\
\mbox{} \\
\mbox{}\textit{\textcolor{Brown}{//\ Implementación\ de\ las\ funciones}} \\
\mbox{} \\
\mbox{}\textcolor{ForestGreen}{void}\ \textbf{\textcolor{Black}{seleccion}}\textcolor{BrickRed}{(}\textcolor{ForestGreen}{int}\ T\textcolor{BrickRed}{[],}\ \textcolor{ForestGreen}{int}\ num$\_$elem\textcolor{BrickRed}{)}\textcolor{Red}{\{} \\
\mbox{}\ \ \ \ \textbf{\textcolor{Black}{seleccion$\_$lims}}\textcolor{BrickRed}{(}T\textcolor{BrickRed}{,}\ \textcolor{Purple}{0}\textcolor{BrickRed}{,}\ num$\_$elem\textcolor{BrickRed}{);} \\
\mbox{}\textcolor{Red}{\}} \\
\mbox{} \\
\mbox{}\textbf{\textcolor{Blue}{static}}\ \textcolor{ForestGreen}{void}\ \textbf{\textcolor{Black}{seleccion$\_$lims}}\textcolor{BrickRed}{(}\textcolor{ForestGreen}{int}\ T\textcolor{BrickRed}{[],}\ \textcolor{ForestGreen}{int}\ inicial\textcolor{BrickRed}{,}\ \textcolor{ForestGreen}{int}\ final\textcolor{BrickRed}{)}\textcolor{Red}{\{} \\
\mbox{}\ \ \ \ \textcolor{ForestGreen}{int}\ i\textcolor{BrickRed}{,}\ j\textcolor{BrickRed}{,}\ indice$\_$menor\textcolor{BrickRed}{;} \\
\mbox{}\ \ \ \ \textcolor{ForestGreen}{int}\ menor\textcolor{BrickRed}{,}\ aux\textcolor{BrickRed}{;} \\
\mbox{}\ \ \ \ \textbf{\textcolor{Blue}{for}}\ \textcolor{BrickRed}{(}i\ \textcolor{BrickRed}{=}\ inicial\textcolor{BrickRed}{;}\ i\ \textcolor{BrickRed}{\textless{}}\ final\ \textcolor{BrickRed}{-}\ \textcolor{Purple}{1}\textcolor{BrickRed}{;}\ i\textcolor{BrickRed}{++)}\ \textcolor{Red}{\{} \\
\mbox{}\ \ \ \ \ \ \ \ indice$\_$menor\ \textcolor{BrickRed}{=}\ i\textcolor{BrickRed}{;} \\
\mbox{}\ \ \ \ \ \ \ \ menor\ \textcolor{BrickRed}{=}\ T\textcolor{BrickRed}{[}i\textcolor{BrickRed}{];} \\
\mbox{}\ \ \ \ \ \ \ \ \textbf{\textcolor{Blue}{for}}\ \textcolor{BrickRed}{(}j\ \textcolor{BrickRed}{=}\ i\textcolor{BrickRed}{;}\ j\ \textcolor{BrickRed}{\textless{}}\ final\textcolor{BrickRed}{;}\ j\textcolor{BrickRed}{++)} \\
\mbox{}\ \ \ \ \ \ \ \ \ \ \ \ \textbf{\textcolor{Blue}{if}}\ \textcolor{BrickRed}{(}T\textcolor{BrickRed}{[}j\textcolor{BrickRed}{]}\ \textcolor{BrickRed}{\textless{}}\ menor\textcolor{BrickRed}{)}\ \textcolor{Red}{\{} \\
\mbox{}\ \ \ \ \ \ \ \ \ \ \ \ \ \ \ \ indice$\_$menor\ \textcolor{BrickRed}{=}\ j\textcolor{BrickRed}{;} \\
\mbox{}\ \ \ \ \ \ \ \ \ \ \ \ \ \ \ \ menor\ \textcolor{BrickRed}{=}\ T\textcolor{BrickRed}{[}j\textcolor{BrickRed}{];} \\
\mbox{}\ \ \ \ \ \ \ \ \ \ \ \ \textcolor{Red}{\}} \\
\mbox{}\ \ \ \ \ \ \ \ \ \ \ \ aux\ \textcolor{BrickRed}{=}\ T\textcolor{BrickRed}{[}i\textcolor{BrickRed}{];} \\
\mbox{}\ \ \ \ \ \ \ \ T\textcolor{BrickRed}{[}i\textcolor{BrickRed}{]}\ \textcolor{BrickRed}{=}\ T\textcolor{BrickRed}{[}indice$\_$menor\textcolor{BrickRed}{];} \\
\mbox{}\ \ \ \ \ \ \ \ T\textcolor{BrickRed}{[}indice$\_$menor\textcolor{BrickRed}{]}\ \textcolor{BrickRed}{=}\ aux\textcolor{BrickRed}{;} \\
\mbox{}\ \ \ \ \textcolor{Red}{\}} \\
\mbox{}\textcolor{Red}{\}} \\
\mbox{}\  \\
\mbox{}\textit{\textcolor{Brown}{/**}} \\
\mbox{}\textit{\textcolor{Brown}{\ *\ }}\textcolor{ForestGreen}{@brief}\textit{\textcolor{Brown}{\ Permite\ duplicar\ un\ vector\ de\ enteros}} \\
\mbox{}\textit{\textcolor{Brown}{\ *\ }}\textcolor{ForestGreen}{@param}\textit{\textcolor{Brown}{\ T\ puntero\ a\ un\ vector\ de\ enteros}} \\
\mbox{}\textit{\textcolor{Brown}{\ *\ }}\textcolor{ForestGreen}{@param}\textit{\textcolor{Brown}{\ U\ puntero\ a\ otro\ vector\ de\ enteros}} \\
\mbox{}\textit{\textcolor{Brown}{\ *\ }}\textcolor{ForestGreen}{@param}\textit{\textcolor{Brown}{\ n\ tamanio\ de\ ambos\ vectores}} \\
\mbox{}\textit{\textcolor{Brown}{\ *\ }}\textcolor{ForestGreen}{@pre}\textit{\textcolor{Brown}{\ Han\ de\ tener\ el\ mismo\ tamanio}} \\
\mbox{}\textit{\textcolor{Brown}{\ */}} \\
\mbox{} \\
\mbox{}\textcolor{ForestGreen}{void}\ \textbf{\textcolor{Black}{duplicaVector}}\textcolor{BrickRed}{(}\textcolor{ForestGreen}{int}\textcolor{BrickRed}{*}\ T\textcolor{BrickRed}{,}\textcolor{ForestGreen}{int}\textcolor{BrickRed}{*}\ U\textcolor{BrickRed}{,}\textcolor{ForestGreen}{int}\ tam\textcolor{BrickRed}{)}\textcolor{Red}{\{} \\
\mbox{}\ \ \ \ \textbf{\textcolor{Blue}{for}}\ \textcolor{BrickRed}{(}\textcolor{ForestGreen}{int}\ i\textcolor{BrickRed}{=}\textcolor{Purple}{0}\textcolor{BrickRed}{;}\ i\textcolor{BrickRed}{\textless{}}tam\textcolor{BrickRed}{;}\ \textcolor{BrickRed}{++}i\textcolor{BrickRed}{)}\textcolor{Red}{\{} \\
\mbox{}\ \ \ \ \ \ \ \ U\textcolor{BrickRed}{[}i\textcolor{BrickRed}{]=}T\textcolor{BrickRed}{[}i\textcolor{BrickRed}{];} \\
\mbox{}\ \ \ \ \textcolor{Red}{\}} \\
\mbox{}\textcolor{Red}{\}} \\
\mbox{} \\
\mbox{}\textcolor{ForestGreen}{int}\ \textbf{\textcolor{Black}{main}}\textcolor{BrickRed}{(}\textcolor{ForestGreen}{int}\ argc\textcolor{BrickRed}{,}\ \textcolor{ForestGreen}{char}\textcolor{BrickRed}{*}\ argv\textcolor{BrickRed}{[])}\textcolor{Red}{\{} \\
\mbox{}\ \ \ \ \textbf{\textcolor{Blue}{if}}\ \textcolor{BrickRed}{(}argc\ \textcolor{BrickRed}{!=}\textcolor{Purple}{2}\textcolor{BrickRed}{)}\textcolor{Red}{\{} \\
\mbox{}\ \ \ \ \ \ \ \ cerr\ \textcolor{BrickRed}{\textless{}\textless{}}\ \texttt{\textcolor{Red}{"{}Uso\ del\ programa:\ "{}}}\ \textcolor{BrickRed}{+}\ \textcolor{BrickRed}{(}string\textcolor{BrickRed}{)(}argv\textcolor{BrickRed}{[}\textcolor{Purple}{0}\textcolor{BrickRed}{])}\ \textcolor{BrickRed}{+}\ \texttt{\textcolor{Red}{"{}\ \textless{}número\ positivo\textgreater{}"{}}}\ \textcolor{BrickRed}{\textless{}\textless{}}\ endl\textcolor{BrickRed}{;}\ \  \\
\mbox{}\ \ \ \ \ \ \ \ \textbf{\textcolor{Blue}{return}}\ \textcolor{BrickRed}{-}\textcolor{Purple}{1}\textcolor{BrickRed}{;} \\
\mbox{}\ \ \ \ \textcolor{Red}{\}} \\
\mbox{}\ \ \ \ \textcolor{ForestGreen}{int}\ n\ \textcolor{BrickRed}{=}\ \textbf{\textcolor{Black}{atoi}}\textcolor{BrickRed}{(}argv\textcolor{BrickRed}{[}\textcolor{Purple}{1}\textcolor{BrickRed}{]);}\ \ \ \  \\
\mbox{}\ \ \ \ \textbf{\textcolor{Blue}{if}}\ \textcolor{BrickRed}{(}n\textcolor{BrickRed}{\textless{}}\textcolor{Purple}{0}\textcolor{BrickRed}{)}\ \textbf{\textcolor{Blue}{return}}\ \textcolor{BrickRed}{-}\textcolor{Purple}{1}\textcolor{BrickRed}{;} \\
\mbox{}\ \ \ \  \\
\mbox{}\ \ \ \ \textcolor{ForestGreen}{int}\ \textcolor{BrickRed}{*}\ T\ \textcolor{BrickRed}{=}\ \textbf{\textcolor{Blue}{new}}\ \textcolor{ForestGreen}{int}\textcolor{BrickRed}{[}n\textcolor{BrickRed}{];} \\
\mbox{}\ \ \ \ \textbf{\textcolor{Blue}{struct}}\ \textcolor{TealBlue}{timespec}\ t$\_$antes\textcolor{BrickRed}{,}\ t$\_$despues\textcolor{BrickRed}{;} \\
\mbox{}\ \ \ \  \\
\mbox{}\ \ \ \ \textbf{\textcolor{Black}{srandom}}\textcolor{BrickRed}{(}\textbf{\textcolor{Black}{time}}\textcolor{BrickRed}{(}\textcolor{Purple}{0}\textcolor{BrickRed}{));} \\
\mbox{}\ \ \ \  \\
\mbox{}\ \ \ \ \textbf{\textcolor{Blue}{for}}\ \textcolor{BrickRed}{(}\textcolor{ForestGreen}{int}\ i\textcolor{BrickRed}{=}\textcolor{Purple}{0}\textcolor{BrickRed}{;}\ i\textcolor{BrickRed}{\textless{}}n\textcolor{BrickRed}{;}\ \textcolor{BrickRed}{++}i\textcolor{BrickRed}{)}\textcolor{Red}{\{} \\
\mbox{}\ \ \ \ \ \ \ \ T\textcolor{BrickRed}{[}i\textcolor{BrickRed}{]}\ \textcolor{BrickRed}{=}\ \textbf{\textcolor{Black}{random}}\textcolor{BrickRed}{();} \\
\mbox{}\ \ \ \ \textcolor{Red}{\}} \\
\mbox{}\ \ \ \  \\
\mbox{}\ \ \ \ \textbf{\textcolor{Black}{clock$\_$gettime}}\textcolor{BrickRed}{(}CLOCK$\_$REALTIME\textcolor{BrickRed}{,\&}t$\_$antes\textcolor{BrickRed}{);} \\
\mbox{}\ \ \ \ \textbf{\textcolor{Black}{seleccion}}\ \textcolor{BrickRed}{(}T\textcolor{BrickRed}{,}n\textcolor{BrickRed}{);} \\
\mbox{}\ \ \ \ \textbf{\textcolor{Black}{clock$\_$gettime}}\textcolor{BrickRed}{(}CLOCK$\_$REALTIME\textcolor{BrickRed}{,\&}t$\_$despues\textcolor{BrickRed}{);} \\
\mbox{}\ \ \ \  \\
\mbox{}\ \ \ \ cout\textcolor{BrickRed}{.}\textbf{\textcolor{Black}{precision}}\textcolor{BrickRed}{(}\textcolor{Purple}{3}\textcolor{BrickRed}{);} \\
\mbox{}\ \ \ \ cout\ \textcolor{BrickRed}{\textless{}\textless{}}\ \textcolor{BrickRed}{(}\textcolor{ForestGreen}{double}\textcolor{BrickRed}{)}\ \textcolor{BrickRed}{(}t$\_$despues\textcolor{BrickRed}{.}tv$\_$sec\textcolor{BrickRed}{-}t$\_$antes\textcolor{BrickRed}{.}tv$\_$sec\textcolor{BrickRed}{)+} \\
\mbox{}\ \ \ \ \ \ \ \ \textcolor{BrickRed}{(}\textcolor{ForestGreen}{double}\textcolor{BrickRed}{)}\ \textcolor{BrickRed}{((}t$\_$despues\textcolor{BrickRed}{.}tv$\_$nsec\textcolor{BrickRed}{-}t$\_$antes\textcolor{BrickRed}{.}tv$\_$nsec\textcolor{BrickRed}{)/(}\textcolor{Purple}{1}\textcolor{BrickRed}{.}e\textcolor{BrickRed}{+}\textcolor{Purple}{9}\textcolor{BrickRed}{))}\ \textcolor{BrickRed}{\textless{}\textless{}}\ endl\textcolor{BrickRed}{;} \\
\mbox{} \\
\mbox{}\ \ \ \  \\
\mbox{}\ \ \ \ \textbf{\textcolor{Blue}{delete}}\ \textcolor{BrickRed}{[]}\ T\textcolor{BrickRed}{;} \\
\mbox{}\ \ \ \  \\
\mbox{}\ \ \ \ \textbf{\textcolor{Blue}{return}}\ \textcolor{Purple}{0}\textcolor{BrickRed}{;} \\
\mbox{}\textcolor{Red}{\}}


\subsubsection{Ordenación heapsort}
% Generator: GNU source-highlight, by Lorenzo Bettini, http://www.gnu.org/software/src-highlite
\noindent
\mbox{}\textit{\textcolor{Brown}{/**}} \\
\mbox{}\textit{\textcolor{Brown}{\ *\ }}\textcolor{ForestGreen}{@file}\textit{\textcolor{Brown}{\ Ordenación\ por\ montones}} \\
\mbox{}\textit{\textcolor{Brown}{\ */}} \\
\mbox{} \\
\mbox{}\textbf{\textcolor{RoyalBlue}{\#include}}\ \texttt{\textcolor{Red}{\textless{}iostream\textgreater{}}} \\
\mbox{}\textbf{\textcolor{RoyalBlue}{\#include}}\ \texttt{\textcolor{Red}{\textless{}ctime\textgreater{}}} \\
\mbox{}\textbf{\textcolor{RoyalBlue}{\#include}}\ \texttt{\textcolor{Red}{\textless{}cstdlib\textgreater{}}} \\
\mbox{}\textbf{\textcolor{RoyalBlue}{\#include}}\ \texttt{\textcolor{Red}{\textless{}ctime\textgreater{}}} \\
\mbox{}\textbf{\textcolor{Blue}{using}}\ \textbf{\textcolor{Blue}{namespace}}\ std\textcolor{BrickRed}{;} \\
\mbox{} \\
\mbox{}\textbf{\textcolor{RoyalBlue}{\#define}}\ NUM$\_$VECES\ \textcolor{Purple}{10000} \\
\mbox{} \\
\mbox{}\textit{\textcolor{Brown}{/**}} \\
\mbox{}\textit{\textcolor{Brown}{\ *\ }}\textcolor{ForestGreen}{@brief}\textit{\textcolor{Brown}{\ Ordena\ un\ vector\ por\ el\ método\ de\ montones.}} \\
\mbox{}\textit{\textcolor{Brown}{\ *\ }}\textcolor{ForestGreen}{@param}\textit{\textcolor{Brown}{\ T:\ vector\ de\ elementos.\ Debe\ tener\ num$\_$elem\ elementos.}} \\
\mbox{}\textit{\textcolor{Brown}{\ *\ Es\ modificado.}} \\
\mbox{}\textit{\textcolor{Brown}{\ *\ }}\textcolor{ForestGreen}{@param}\textit{\textcolor{Brown}{\ num$\_$elem:\ número\ de\ elementos.\ num$\_$elem\ \textgreater{}\ 0.}} \\
\mbox{}\textit{\textcolor{Brown}{\ *\ }} \\
\mbox{}\textit{\textcolor{Brown}{\ *\ Cambia\ el\ orden\ de\ los\ elementos\ de\ T\ de\ forma\ que\ los\ dispone}} \\
\mbox{}\textit{\textcolor{Brown}{\ *\ en\ sentido\ creciente\ de\ menor\ a\ mayor.}} \\
\mbox{}\textit{\textcolor{Brown}{\ *\ Aplica\ el\ algoritmo\ de\ ordenación\ por\ montones.}} \\
\mbox{}\textit{\textcolor{Brown}{\ *\ }} \\
\mbox{}\textit{\textcolor{Brown}{\ */}} \\
\mbox{} \\
\mbox{}\textbf{\textcolor{Blue}{inline}}\ \textbf{\textcolor{Blue}{static}}\ \textcolor{ForestGreen}{void}\ \textbf{\textcolor{Black}{heapsort}}\textcolor{BrickRed}{(}\textcolor{ForestGreen}{int}\ T\textcolor{BrickRed}{[],}\ \textcolor{ForestGreen}{int}\ num$\_$elem\textcolor{BrickRed}{);} \\
\mbox{} \\
\mbox{}\textit{\textcolor{Brown}{/**}} \\
\mbox{}\textit{\textcolor{Brown}{\ *\ }}\textcolor{ForestGreen}{@brief}\textit{\textcolor{Brown}{\ Reajusta\ parte\ de\ un\ vector\ para\ que\ sea\ un\ montón.}} \\
\mbox{}\textit{\textcolor{Brown}{\ *\ }}\textcolor{ForestGreen}{@param}\textit{\textcolor{Brown}{\ T:\ vector\ de\ elementos.\ Debe\ tener\ num$\_$elem\ elementos.}} \\
\mbox{}\textit{\textcolor{Brown}{\ *\ Es\ modificado.}} \\
\mbox{}\textit{\textcolor{Brown}{\ *\ }}\textcolor{ForestGreen}{@param}\textit{\textcolor{Brown}{\ num$\_$elem:\ número\ de\ elementos.\ num$\_$elem\ \textgreater{}\ 0.}} \\
\mbox{}\textit{\textcolor{Brown}{\ *\ }}\textcolor{ForestGreen}{@param}\textit{\textcolor{Brown}{\ k:\ índice\ del\ elemento\ que\ se\ toma\ com\ raíz}} \\
\mbox{}\textit{\textcolor{Brown}{\ *\ \ \ }} \\
\mbox{}\textit{\textcolor{Brown}{\ *\ Reajusta\ los\ elementos\ entre\ las\ posiciones\ k\ y\ num$\_$elem\ -\ 1\ }} \\
\mbox{}\textit{\textcolor{Brown}{\ *\ de\ T\ para\ que\ cumpla\ la\ propiedad\ de\ un\ montón\ (APO),\ }} \\
\mbox{}\textit{\textcolor{Brown}{\ *\ considerando\ al\ elemento\ en\ la\ posición\ k\ como\ la\ raíz.}} \\
\mbox{}\textit{\textcolor{Brown}{\ */}} \\
\mbox{} \\
\mbox{}\textbf{\textcolor{Blue}{static}}\ \textcolor{ForestGreen}{void}\ \textbf{\textcolor{Black}{reajustar}}\textcolor{BrickRed}{(}\textcolor{ForestGreen}{int}\ T\textcolor{BrickRed}{[],}\ \textcolor{ForestGreen}{int}\ num$\_$elem\textcolor{BrickRed}{,}\ \textcolor{ForestGreen}{int}\ k\textcolor{BrickRed}{);} \\
\mbox{} \\
\mbox{}\textit{\textcolor{Brown}{//\ Implementación\ de\ las\ funciones}} \\
\mbox{} \\
\mbox{}\textbf{\textcolor{Blue}{static}}\ \textcolor{ForestGreen}{void}\ \textbf{\textcolor{Black}{heapsort}}\textcolor{BrickRed}{(}\textcolor{ForestGreen}{int}\ T\textcolor{BrickRed}{[],}\ \textcolor{ForestGreen}{int}\ num$\_$elem\textcolor{BrickRed}{)}\textcolor{Red}{\{} \\
\mbox{}\ \ \ \ \textbf{\textcolor{Blue}{for}}\ \textcolor{BrickRed}{(}\textcolor{ForestGreen}{int}\ i\ \textcolor{BrickRed}{=}\ num$\_$elem\textcolor{BrickRed}{/}\textcolor{Purple}{2}\textcolor{BrickRed}{;}\ i\ \textcolor{BrickRed}{\textgreater{}=}\ \textcolor{Purple}{0}\textcolor{BrickRed}{;}\ i\textcolor{BrickRed}{-\/-)} \\
\mbox{}\ \ \ \ \ \ \ \ \textbf{\textcolor{Black}{reajustar}}\textcolor{BrickRed}{(}T\textcolor{BrickRed}{,}\ num$\_$elem\textcolor{BrickRed}{,}\ i\textcolor{BrickRed}{);} \\
\mbox{} \\
\mbox{}\ \ \ \ \textbf{\textcolor{Blue}{for}}\ \textcolor{BrickRed}{(}\textcolor{ForestGreen}{int}\ i\ \textcolor{BrickRed}{=}\ num$\_$elem\ \textcolor{BrickRed}{-}\ \textcolor{Purple}{1}\textcolor{BrickRed}{;}\ i\ \textcolor{BrickRed}{\textgreater{}=}\ \textcolor{Purple}{1}\textcolor{BrickRed}{;}\ i\textcolor{BrickRed}{-\/-)}\textcolor{Red}{\{} \\
\mbox{}\ \ \ \ \ \ \ \ \textcolor{ForestGreen}{int}\ aux\ \textcolor{BrickRed}{=}\ T\textcolor{BrickRed}{[}\textcolor{Purple}{0}\textcolor{BrickRed}{];} \\
\mbox{}\ \ \ \ \ \ \ \ T\textcolor{BrickRed}{[}\textcolor{Purple}{0}\textcolor{BrickRed}{]}\ \textcolor{BrickRed}{=}\ T\textcolor{BrickRed}{[}i\textcolor{BrickRed}{];} \\
\mbox{}\ \ \ \ \ \ \ \ T\textcolor{BrickRed}{[}i\textcolor{BrickRed}{]}\ \textcolor{BrickRed}{=}\ aux\textcolor{BrickRed}{;} \\
\mbox{}\ \ \ \ \ \ \ \ \textbf{\textcolor{Black}{reajustar}}\textcolor{BrickRed}{(}T\textcolor{BrickRed}{,}\ i\textcolor{BrickRed}{,}\ \textcolor{Purple}{0}\textcolor{BrickRed}{);} \\
\mbox{}\ \ \ \ \textcolor{Red}{\}} \\
\mbox{}\textcolor{Red}{\}} \\
\mbox{} \\
\mbox{} \\
\mbox{}\textbf{\textcolor{Blue}{static}}\ \textcolor{ForestGreen}{void}\ \textbf{\textcolor{Black}{reajustar}}\textcolor{BrickRed}{(}\textcolor{ForestGreen}{int}\ T\textcolor{BrickRed}{[],}\ \textcolor{ForestGreen}{int}\ num$\_$elem\textcolor{BrickRed}{,}\ \textcolor{ForestGreen}{int}\ k\textcolor{BrickRed}{)}\textcolor{Red}{\{} \\
\mbox{}\ \ \ \ \textcolor{ForestGreen}{int}\ j\textcolor{BrickRed}{;} \\
\mbox{}\ \ \ \ \textcolor{ForestGreen}{int}\ v\textcolor{BrickRed}{;} \\
\mbox{}\ \ \ \ v\ \textcolor{BrickRed}{=}\ T\textcolor{BrickRed}{[}k\textcolor{BrickRed}{];} \\
\mbox{}\ \ \ \ \textcolor{ForestGreen}{bool}\ esAPO\ \textcolor{BrickRed}{=}\ \textbf{\textcolor{Blue}{false}}\textcolor{BrickRed}{;} \\
\mbox{}\ \ \ \ \textbf{\textcolor{Blue}{while}}\ \textcolor{BrickRed}{((}k\ \textcolor{BrickRed}{\textless{}}\ num$\_$elem\textcolor{BrickRed}{/}\textcolor{Purple}{2}\textcolor{BrickRed}{)}\ \textcolor{BrickRed}{\&\&}\ \textcolor{BrickRed}{!}esAPO\textcolor{BrickRed}{)}\textcolor{Red}{\{} \\
\mbox{}\ \ \ \ \ \ \ \ j\ \textcolor{BrickRed}{=}\ k\ \textcolor{BrickRed}{+}\ k\ \textcolor{BrickRed}{+}\ \textcolor{Purple}{1}\textcolor{BrickRed}{;} \\
\mbox{}\ \ \ \ \ \ \ \ \textbf{\textcolor{Blue}{if}}\ \textcolor{BrickRed}{((}j\ \textcolor{BrickRed}{\textless{}}\ \textcolor{BrickRed}{(}num$\_$elem\ \textcolor{BrickRed}{-}\ \textcolor{Purple}{1}\textcolor{BrickRed}{))}\ \textcolor{BrickRed}{\&\&}\ \textcolor{BrickRed}{(}T\textcolor{BrickRed}{[}j\textcolor{BrickRed}{]}\ \textcolor{BrickRed}{\textless{}}\ T\textcolor{BrickRed}{[}j\textcolor{BrickRed}{+}\textcolor{Purple}{1}\textcolor{BrickRed}{]))} \\
\mbox{}\ \ \ \ \ \ \ \ \ \ \ \ j\textcolor{BrickRed}{++;} \\
\mbox{}\ \ \ \ \ \ \ \ \textbf{\textcolor{Blue}{if}}\ \textcolor{BrickRed}{(}v\ \textcolor{BrickRed}{\textgreater{}=}\ T\textcolor{BrickRed}{[}j\textcolor{BrickRed}{])} \\
\mbox{}\ \ \ \ \ \ \ \ \ \ \ \ esAPO\ \textcolor{BrickRed}{=}\ \textbf{\textcolor{Blue}{true}}\textcolor{BrickRed}{;} \\
\mbox{}\ \ \ \ \ \ \ \ T\textcolor{BrickRed}{[}k\textcolor{BrickRed}{]}\ \textcolor{BrickRed}{=}\ T\textcolor{BrickRed}{[}j\textcolor{BrickRed}{];} \\
\mbox{}\ \ \ \ \ \ \ \ k\ \textcolor{BrickRed}{=}\ j\textcolor{BrickRed}{;} \\
\mbox{}\ \ \ \ \textcolor{Red}{\}} \\
\mbox{}\ \ \ \ T\textcolor{BrickRed}{[}k\textcolor{BrickRed}{]}\ \textcolor{BrickRed}{=}\ v\textcolor{BrickRed}{;} \\
\mbox{}\textcolor{Red}{\}} \\
\mbox{} \\
\mbox{}\textcolor{ForestGreen}{int}\ \textbf{\textcolor{Black}{main}}\textcolor{BrickRed}{(}\textcolor{ForestGreen}{int}\ argc\textcolor{BrickRed}{,}\ \textcolor{ForestGreen}{char}\textcolor{BrickRed}{*}\ argv\textcolor{BrickRed}{[])}\textcolor{Red}{\{} \\
\mbox{}\ \ \ \ \textbf{\textcolor{Blue}{if}}\ \textcolor{BrickRed}{(}argc\ \textcolor{BrickRed}{!=}\textcolor{Purple}{2}\textcolor{BrickRed}{)}\textcolor{Red}{\{} \\
\mbox{}\ \ \ \ \ \ \ \ cerr\ \textcolor{BrickRed}{\textless{}\textless{}}\ \texttt{\textcolor{Red}{"{}Uso\ del\ programa:\ "{}}}\ \textcolor{BrickRed}{+}\ \textcolor{BrickRed}{(}string\textcolor{BrickRed}{)(}argv\textcolor{BrickRed}{[}\textcolor{Purple}{0}\textcolor{BrickRed}{])}\ \textcolor{BrickRed}{+}\ \texttt{\textcolor{Red}{"{}\ \textless{}número\ positivo\textgreater{}"{}}}\ \textcolor{BrickRed}{\textless{}\textless{}}\ endl\textcolor{BrickRed}{;}\ \  \\
\mbox{}\ \ \ \ \ \ \ \ \textbf{\textcolor{Blue}{return}}\ \textcolor{BrickRed}{-}\textcolor{Purple}{1}\textcolor{BrickRed}{;} \\
\mbox{}\ \ \ \ \textcolor{Red}{\}} \\
\mbox{}\ \ \ \ \textcolor{ForestGreen}{int}\ n\ \textcolor{BrickRed}{=}\ \textbf{\textcolor{Black}{atoi}}\textcolor{BrickRed}{(}argv\textcolor{BrickRed}{[}\textcolor{Purple}{1}\textcolor{BrickRed}{]);}\ \ \ \  \\
\mbox{}\ \ \ \ \textbf{\textcolor{Blue}{if}}\ \textcolor{BrickRed}{(}n\textcolor{BrickRed}{\textless{}}\textcolor{Purple}{0}\textcolor{BrickRed}{)}\ \textbf{\textcolor{Blue}{return}}\ \textcolor{BrickRed}{-}\textcolor{Purple}{1}\textcolor{BrickRed}{;} \\
\mbox{}\ \ \ \  \\
\mbox{}\ \ \ \ \textcolor{ForestGreen}{int}\ \textcolor{BrickRed}{*}\ T\ \textcolor{BrickRed}{=}\ \textbf{\textcolor{Blue}{new}}\ \textcolor{ForestGreen}{int}\textcolor{BrickRed}{[}n\textcolor{BrickRed}{];} \\
\mbox{}\ \ \ \ \textbf{\textcolor{Blue}{struct}}\ \textcolor{TealBlue}{timespec}\ t$\_$antes\textcolor{BrickRed}{,}\ t$\_$despues\textcolor{BrickRed}{;} \\
\mbox{}\ \ \ \  \\
\mbox{}\ \ \ \ \textbf{\textcolor{Black}{srandom}}\textcolor{BrickRed}{(}\textbf{\textcolor{Black}{time}}\textcolor{BrickRed}{(}\textcolor{Purple}{0}\textcolor{BrickRed}{));} \\
\mbox{}\ \ \ \  \\
\mbox{}\ \ \ \ \textbf{\textcolor{Blue}{for}}\ \textcolor{BrickRed}{(}\textcolor{ForestGreen}{int}\ i\textcolor{BrickRed}{=}\textcolor{Purple}{0}\textcolor{BrickRed}{;}\ i\textcolor{BrickRed}{\textless{}}n\textcolor{BrickRed}{;}\ \textcolor{BrickRed}{++}i\textcolor{BrickRed}{)}\textcolor{Red}{\{} \\
\mbox{}\ \ \ \ \ \ \ \ T\textcolor{BrickRed}{[}i\textcolor{BrickRed}{]}\ \textcolor{BrickRed}{=}\ \textbf{\textcolor{Black}{random}}\textcolor{BrickRed}{();} \\
\mbox{}\ \ \ \ \textcolor{Red}{\}} \\
\mbox{}\ \ \ \  \\
\mbox{}\ \ \ \ \textbf{\textcolor{Black}{clock$\_$gettime}}\textcolor{BrickRed}{(}CLOCK$\_$REALTIME\textcolor{BrickRed}{,\&}t$\_$antes\textcolor{BrickRed}{);} \\
\mbox{}\ \ \ \ \textbf{\textcolor{Black}{heapsort}}\ \textcolor{BrickRed}{(}T\textcolor{BrickRed}{,}n\textcolor{BrickRed}{);} \\
\mbox{}\ \ \ \ \textbf{\textcolor{Black}{clock$\_$gettime}}\textcolor{BrickRed}{(}CLOCK$\_$REALTIME\textcolor{BrickRed}{,\&}t$\_$despues\textcolor{BrickRed}{);} \\
\mbox{}\ \ \ \  \\
\mbox{}\ \ \ \ cout\textcolor{BrickRed}{.}\textbf{\textcolor{Black}{precision}}\textcolor{BrickRed}{(}\textcolor{Purple}{3}\textcolor{BrickRed}{);} \\
\mbox{}\ \ \ \ cout\ \textcolor{BrickRed}{\textless{}\textless{}}\ \textcolor{BrickRed}{(}\textcolor{ForestGreen}{double}\textcolor{BrickRed}{)}\ \textcolor{BrickRed}{(}t$\_$despues\textcolor{BrickRed}{.}tv$\_$sec\textcolor{BrickRed}{-}t$\_$antes\textcolor{BrickRed}{.}tv$\_$sec\textcolor{BrickRed}{)+} \\
\mbox{}\ \ \ \ \ \ \ \ \textcolor{BrickRed}{(}\textcolor{ForestGreen}{double}\textcolor{BrickRed}{)}\ \textcolor{BrickRed}{((}t$\_$despues\textcolor{BrickRed}{.}tv$\_$nsec\textcolor{BrickRed}{-}t$\_$antes\textcolor{BrickRed}{.}tv$\_$nsec\textcolor{BrickRed}{)/(}\textcolor{Purple}{1}\textcolor{BrickRed}{.}e\textcolor{BrickRed}{+}\textcolor{Purple}{9}\textcolor{BrickRed}{))}\ \textcolor{BrickRed}{\textless{}\textless{}}\ endl\textcolor{BrickRed}{;} \\
\mbox{} \\
\mbox{}\ \ \ \  \\
\mbox{}\ \ \ \ \textbf{\textcolor{Blue}{delete}}\ \textcolor{BrickRed}{[]}\ T\textcolor{BrickRed}{;} \\
\mbox{}\ \ \ \  \\
\mbox{}\ \ \ \ \textbf{\textcolor{Blue}{return}}\ \textcolor{Purple}{0}\textcolor{BrickRed}{;} \\
\mbox{}\textcolor{Red}{\}} \\
\mbox{}


\subsubsection{Ordenación mergesort}
% Generator: GNU source-highlight, by Lorenzo Bettini, http://www.gnu.org/software/src-highlite
\noindent
\mbox{}\textit{\textcolor{Brown}{/**}} \\
\mbox{}\textit{\textcolor{Brown}{\ *\ }}\textcolor{ForestGreen}{@file}\textit{\textcolor{Brown}{\ mergesort.cpp}} \\
\mbox{}\textit{\textcolor{Brown}{\ *\ Ordenación\ por\ mezcla}} \\
\mbox{}\textit{\textcolor{Brown}{\ */}} \\
\mbox{} \\
\mbox{}\textbf{\textcolor{RoyalBlue}{\#include}}\ \texttt{\textcolor{Red}{\textless{}iostream\textgreater{}}} \\
\mbox{}\textbf{\textcolor{RoyalBlue}{\#include}}\ \texttt{\textcolor{Red}{\textless{}ctime\textgreater{}}} \\
\mbox{}\textbf{\textcolor{RoyalBlue}{\#include}}\ \texttt{\textcolor{Red}{\textless{}cstdlib\textgreater{}}} \\
\mbox{}\textbf{\textcolor{RoyalBlue}{\#include}}\ \texttt{\textcolor{Red}{\textless{}cassert\textgreater{}}} \\
\mbox{}\textbf{\textcolor{RoyalBlue}{\#include}}\ \texttt{\textcolor{Red}{\textless{}climits\textgreater{}}} \\
\mbox{}\textbf{\textcolor{Blue}{using}}\ \textbf{\textcolor{Blue}{namespace}}\ std\textcolor{BrickRed}{;} \\
\mbox{} \\
\mbox{}\textbf{\textcolor{RoyalBlue}{\#define}}\ NUM$\_$VECES\ \textcolor{Purple}{50} \\
\mbox{} \\
\mbox{}\textit{\textcolor{Brown}{/**}} \\
\mbox{}\textit{\textcolor{Brown}{\ *\ }}\textcolor{ForestGreen}{@brief}\textit{\textcolor{Brown}{\ Ordena\ un\ vector\ por\ el\ método\ de\ mezcla.}} \\
\mbox{}\textit{\textcolor{Brown}{\ *\ }}\textcolor{ForestGreen}{@param}\textit{\textcolor{Brown}{\ T\ Vector\ de\ elementos.\ Debe\ tener\ num$\_$elem\ elementos.}} \\
\mbox{}\textit{\textcolor{Brown}{\ *\ }}\textcolor{ForestGreen}{@param}\textit{\textcolor{Brown}{\ num$\_$elem:\ número\ de\ elementos.\ num$\_$elem\ \textgreater{}\ 0.}} \\
\mbox{}\textit{\textcolor{Brown}{\ *\ }}\textcolor{ForestGreen}{@pos}\textit{\textcolor{Brown}{\ El\ vector\ contiene\ los\ elementos\ ordenados.}} \\
\mbox{}\textit{\textcolor{Brown}{\ *\ }} \\
\mbox{}\textit{\textcolor{Brown}{\ *\ Cambia\ el\ orden\ de\ los\ elementos\ de\ T\ de\ forma\ que\ los\ dispone}} \\
\mbox{}\textit{\textcolor{Brown}{\ *\ en\ sentido\ creciente\ de\ menor\ a\ mayor.}} \\
\mbox{}\textit{\textcolor{Brown}{\ *\ Aplica\ el\ algoritmo\ de\ mezcla.}} \\
\mbox{}\textit{\textcolor{Brown}{\ */}} \\
\mbox{} \\
\mbox{}\textbf{\textcolor{Blue}{inline}}\ \textbf{\textcolor{Blue}{static}}\ \textcolor{ForestGreen}{void}\ \textbf{\textcolor{Black}{mergesort}}\textcolor{BrickRed}{(}\textcolor{ForestGreen}{int}\ T\textcolor{BrickRed}{[],}\ \textcolor{ForestGreen}{int}\ num$\_$elem\textcolor{BrickRed}{);} \\
\mbox{} \\
\mbox{}\textit{\textcolor{Brown}{/**}} \\
\mbox{}\textit{\textcolor{Brown}{\ *\ }}\textcolor{ForestGreen}{@brief}\textit{\textcolor{Brown}{\ Ordena\ parte\ de\ un\ vector\ por\ el\ método\ de\ mezcla.}} \\
\mbox{}\textit{\textcolor{Brown}{\ *\ }}\textcolor{ForestGreen}{@param}\textit{\textcolor{Brown}{\ T:\ vector\ de\ elementos.\ Tiene\ un\ número\ de\ elementos\ }} \\
\mbox{}\textit{\textcolor{Brown}{\ *\ mayor\ o\ igual\ a\ final.\ Es\ MODIFICADO.}} \\
\mbox{}\textit{\textcolor{Brown}{\ *\ }}\textcolor{ForestGreen}{@param}\textit{\textcolor{Brown}{\ inicial:\ Posición\ que\ marca\ el\ incio\ de\ la\ parte\ del}} \\
\mbox{}\textit{\textcolor{Brown}{\ *\ vector\ a\ ordenar.}} \\
\mbox{}\textit{\textcolor{Brown}{\ *\ }}\textcolor{ForestGreen}{@param}\textit{\textcolor{Brown}{\ final:\ Posición\ detrás\ de\ la\ última\ de\ la\ parte\ del}} \\
\mbox{}\textit{\textcolor{Brown}{\ *\ vector\ a\ ordenar.\ }} \\
\mbox{}\textit{\textcolor{Brown}{\ *\ }}\textcolor{ForestGreen}{@pre}\textit{\textcolor{Brown}{\ inicial\ \textless{}\ final.}} \\
\mbox{}\textit{\textcolor{Brown}{\ *\ }} \\
\mbox{}\textit{\textcolor{Brown}{\ *\ Cambia\ el\ orden\ de\ los\ elementos\ de\ T\ entre\ las\ posiciones}} \\
\mbox{}\textit{\textcolor{Brown}{\ *\ inicial\ y\ final\ -\ 1\ de\ forma\ que\ los\ dispone\ en\ sentido\ creciente}} \\
\mbox{}\textit{\textcolor{Brown}{\ *\ de\ menor\ a\ mayor.}} \\
\mbox{}\textit{\textcolor{Brown}{\ *\ Aplica\ el\ algoritmo\ de\ la\ mezcla.}} \\
\mbox{}\textit{\textcolor{Brown}{\ */}} \\
\mbox{} \\
\mbox{}\textbf{\textcolor{Blue}{static}}\ \textcolor{ForestGreen}{void}\ \textbf{\textcolor{Black}{mergesort$\_$lims}}\textcolor{BrickRed}{(}\textcolor{ForestGreen}{int}\ T\textcolor{BrickRed}{[],}\ \textcolor{ForestGreen}{int}\ inicial\textcolor{BrickRed}{,}\ \textcolor{ForestGreen}{int}\ final\textcolor{BrickRed}{);} \\
\mbox{} \\
\mbox{}\textit{\textcolor{Brown}{/**}} \\
\mbox{}\textit{\textcolor{Brown}{\ *\ }}\textcolor{ForestGreen}{@brief}\textit{\textcolor{Brown}{\ Ordena\ un\ vector\ por\ el\ método\ de\ inserción.}} \\
\mbox{}\textit{\textcolor{Brown}{\ *\ }}\textcolor{ForestGreen}{@param}\textit{\textcolor{Brown}{\ T:\ vector\ de\ elementos.\ Debe\ tener\ num$\_$elem\ elementos.}} \\
\mbox{}\textit{\textcolor{Brown}{\ *\ Es\ modificado.}} \\
\mbox{}\textit{\textcolor{Brown}{\ *\ }}\textcolor{ForestGreen}{@param}\textit{\textcolor{Brown}{\ num$\_$elem:\ número\ de\ elementos.\ num$\_$elem\ \textgreater{}\ 0.}} \\
\mbox{}\textit{\textcolor{Brown}{\ *\ }} \\
\mbox{}\textit{\textcolor{Brown}{\ *\ Cambia\ el\ orden\ de\ los\ elementos\ de\ T\ de\ forma\ que\ los\ dispone}} \\
\mbox{}\textit{\textcolor{Brown}{\ *\ en\ sentido\ creciente\ de\ menor\ a\ mayor.}} \\
\mbox{}\textit{\textcolor{Brown}{\ *\ Aplica\ el\ algoritmo\ de\ inserción.}} \\
\mbox{}\textit{\textcolor{Brown}{\ */}} \\
\mbox{} \\
\mbox{}\textbf{\textcolor{Blue}{inline}}\ \textbf{\textcolor{Blue}{static}}\ \textcolor{ForestGreen}{void}\ \textbf{\textcolor{Black}{insercion}}\textcolor{BrickRed}{(}\textcolor{ForestGreen}{int}\ T\textcolor{BrickRed}{[],}\ \textcolor{ForestGreen}{int}\ num$\_$elem\textcolor{BrickRed}{);} \\
\mbox{} \\
\mbox{}\textit{\textcolor{Brown}{/**}} \\
\mbox{}\textit{\textcolor{Brown}{\ *\ }}\textcolor{ForestGreen}{@brief}\textit{\textcolor{Brown}{\ Ordena\ parte\ de\ un\ vector\ por\ el\ método\ de\ inserción.}} \\
\mbox{}\textit{\textcolor{Brown}{\ *\ }}\textcolor{ForestGreen}{@param}\textit{\textcolor{Brown}{\ T:\ vector\ de\ elementos.\ Tiene\ un\ número\ de\ elementos\ }} \\
\mbox{}\textit{\textcolor{Brown}{\ *\ mayor\ o\ igual\ a\ final.\ Es\ MODIFICADO.}} \\
\mbox{}\textit{\textcolor{Brown}{\ *\ }}\textcolor{ForestGreen}{@param}\textit{\textcolor{Brown}{\ inicial:\ Posición\ que\ marca\ el\ incio\ de\ la\ parte\ del}} \\
\mbox{}\textit{\textcolor{Brown}{\ *\ vector\ a\ ordenar.}} \\
\mbox{}\textit{\textcolor{Brown}{\ *\ }}\textcolor{ForestGreen}{@param}\textit{\textcolor{Brown}{\ final:\ Posición\ detrás\ de\ la\ última\ de\ la\ parte\ del}} \\
\mbox{}\textit{\textcolor{Brown}{\ *\ vector\ a\ ordenar.\ }} \\
\mbox{}\textit{\textcolor{Brown}{\ *\ }}\textcolor{ForestGreen}{@pre}\textit{\textcolor{Brown}{\ inicial\ \textless{}\ final.}} \\
\mbox{}\textit{\textcolor{Brown}{\ *\ }} \\
\mbox{}\textit{\textcolor{Brown}{\ *\ Cambia\ el\ orden\ de\ los\ elementos\ de\ T\ entre\ las\ posiciones}} \\
\mbox{}\textit{\textcolor{Brown}{\ *\ inicial\ y\ final\ -\ 1\ de\ forma\ que\ los\ dispone\ en\ sentido\ creciente}} \\
\mbox{}\textit{\textcolor{Brown}{\ *\ de\ menor\ a\ mayor.}} \\
\mbox{}\textit{\textcolor{Brown}{\ *\ Aplica\ el\ algoritmo\ de\ la\ inserción.}} \\
\mbox{}\textit{\textcolor{Brown}{\ */}} \\
\mbox{} \\
\mbox{}\textbf{\textcolor{Blue}{static}}\ \textcolor{ForestGreen}{void}\ \textbf{\textcolor{Black}{insercion$\_$lims}}\textcolor{BrickRed}{(}\textcolor{ForestGreen}{int}\ T\textcolor{BrickRed}{[],}\ \textcolor{ForestGreen}{int}\ inicial\textcolor{BrickRed}{,}\ \textcolor{ForestGreen}{int}\ final\textcolor{BrickRed}{);} \\
\mbox{} \\
\mbox{}\textit{\textcolor{Brown}{/**}} \\
\mbox{}\textit{\textcolor{Brown}{\ *\ }}\textcolor{ForestGreen}{@brief}\textit{\textcolor{Brown}{\ Mezcla\ dos\ vectores\ ordenados\ sobre\ otro.}} \\
\mbox{}\textit{\textcolor{Brown}{\ *\ }}\textcolor{ForestGreen}{@param}\textit{\textcolor{Brown}{\ T:\ vector\ de\ elementos.\ Tiene\ un\ número\ de\ elementos\ }} \\
\mbox{}\textit{\textcolor{Brown}{\ *\ mayor\ o\ igual\ a\ final.\ Es\ MODIFICADO.}} \\
\mbox{}\textit{\textcolor{Brown}{\ *\ }}\textcolor{ForestGreen}{@param}\textit{\textcolor{Brown}{\ inicial:\ Posición\ que\ marca\ el\ incio\ de\ la\ parte\ del}} \\
\mbox{}\textit{\textcolor{Brown}{\ *\ vector\ a\ escribir.}} \\
\mbox{}\textit{\textcolor{Brown}{\ *\ }}\textcolor{ForestGreen}{@param}\textit{\textcolor{Brown}{\ final:\ Posición\ detrás\ de\ la\ última\ de\ la\ parte\ del}} \\
\mbox{}\textit{\textcolor{Brown}{\ *\ vector\ a\ escribir}} \\
\mbox{}\textit{\textcolor{Brown}{\ *\ inicial\ \textless{}\ final.}} \\
\mbox{}\textit{\textcolor{Brown}{\ *\ }}\textcolor{ForestGreen}{@param}\textit{\textcolor{Brown}{\ U:\ Vector\ con\ los\ elementos\ ordenados.}} \\
\mbox{}\textit{\textcolor{Brown}{\ *\ }}\textcolor{ForestGreen}{@param}\textit{\textcolor{Brown}{\ V:\ Vector\ con\ los\ elementos\ ordenados.}} \\
\mbox{}\textit{\textcolor{Brown}{\ *\ }}\textcolor{ForestGreen}{@pre}\textit{\textcolor{Brown}{\ El\ número\ de\ elementos\ de\ U\ y\ V\ sumados\ debe\ coincidir}} \\
\mbox{}\textit{\textcolor{Brown}{\ *\ con\ final\ -\ inicial.}} \\
\mbox{}\textit{\textcolor{Brown}{\ *\ }} \\
\mbox{}\textit{\textcolor{Brown}{\ *\ En\ los\ elementos\ de\ T\ entre\ las\ posiciones\ inicial\ y\ final\ -\ 1}} \\
\mbox{}\textit{\textcolor{Brown}{\ *\ pone\ ordenados\ en\ sentido\ creciente,\ de\ menor\ a\ mayor,\ los}} \\
\mbox{}\textit{\textcolor{Brown}{\ *\ elementos\ de\ los\ vectores\ U\ y\ V.}} \\
\mbox{}\textit{\textcolor{Brown}{\ */}} \\
\mbox{} \\
\mbox{}\textbf{\textcolor{Blue}{static}}\ \textcolor{ForestGreen}{void}\ \textbf{\textcolor{Black}{fusion}}\textcolor{BrickRed}{(}\textcolor{ForestGreen}{int}\ T\textcolor{BrickRed}{[],}\ \textcolor{ForestGreen}{int}\ inicial\textcolor{BrickRed}{,}\ \textcolor{ForestGreen}{int}\ final\textcolor{BrickRed}{,}\ \textcolor{ForestGreen}{int}\ U\textcolor{BrickRed}{[],}\ \textcolor{ForestGreen}{int}\ V\textcolor{BrickRed}{[]);} \\
\mbox{} \\
\mbox{}\textit{\textcolor{Brown}{//\ Implementación\ de\ las\ funciones}} \\
\mbox{} \\
\mbox{}\textbf{\textcolor{Blue}{inline}}\ \textbf{\textcolor{Blue}{static}}\ \textcolor{ForestGreen}{void}\ \textbf{\textcolor{Black}{insercion}}\textcolor{BrickRed}{(}\textcolor{ForestGreen}{int}\ T\textcolor{BrickRed}{[],}\ \textcolor{ForestGreen}{int}\ num$\_$elem\textcolor{BrickRed}{)}\textcolor{Red}{\{} \\
\mbox{}\ \ \ \ \textbf{\textcolor{Black}{insercion$\_$lims}}\textcolor{BrickRed}{(}T\textcolor{BrickRed}{,}\ \textcolor{Purple}{0}\textcolor{BrickRed}{,}\ num$\_$elem\textcolor{BrickRed}{);} \\
\mbox{}\textcolor{Red}{\}} \\
\mbox{} \\
\mbox{}\textbf{\textcolor{Blue}{static}}\ \textcolor{ForestGreen}{void}\ \textbf{\textcolor{Black}{insercion$\_$lims}}\textcolor{BrickRed}{(}\textcolor{ForestGreen}{int}\ T\textcolor{BrickRed}{[],}\ \textcolor{ForestGreen}{int}\ inicial\textcolor{BrickRed}{,}\ \textcolor{ForestGreen}{int}\ final\textcolor{BrickRed}{)}\textcolor{Red}{\{} \\
\mbox{}\ \ \ \ \textcolor{ForestGreen}{int}\ i\textcolor{BrickRed}{,}\ j\textcolor{BrickRed}{;} \\
\mbox{}\ \ \ \ \textcolor{ForestGreen}{int}\ aux\textcolor{BrickRed}{;} \\
\mbox{} \\
\mbox{}\ \ \ \ \textbf{\textcolor{Blue}{for}}\ \textcolor{BrickRed}{(}i\ \textcolor{BrickRed}{=}\ inicial\ \textcolor{BrickRed}{+}\ \textcolor{Purple}{1}\textcolor{BrickRed}{;}\ i\textcolor{BrickRed}{\textless{}}final\textcolor{BrickRed}{;}\ i\textcolor{BrickRed}{++)}\textcolor{Red}{\{} \\
\mbox{}\ \ \ \ \ \ \ \ j\ \textcolor{BrickRed}{=}\ i\textcolor{BrickRed}{;} \\
\mbox{}\ \ \ \ \ \ \ \ \textbf{\textcolor{Blue}{while}}\ \textcolor{BrickRed}{((}T\textcolor{BrickRed}{[}j\textcolor{BrickRed}{]}\ \textcolor{BrickRed}{\textless{}}\ T\textcolor{BrickRed}{[}j\textcolor{BrickRed}{-}\textcolor{Purple}{1}\textcolor{BrickRed}{])}\ \textcolor{BrickRed}{\&\&}\ \textcolor{BrickRed}{(}j\ \textcolor{BrickRed}{\textgreater{}}\ \textcolor{Purple}{0}\textcolor{BrickRed}{))}\textcolor{Red}{\{} \\
\mbox{}\ \ \ \ \ \ \ \ \ \ \ \ aux\ \textcolor{BrickRed}{=}\ T\textcolor{BrickRed}{[}j\textcolor{BrickRed}{];} \\
\mbox{}\ \ \ \ \ \ \ \ \ \ \ \ T\textcolor{BrickRed}{[}j\textcolor{BrickRed}{]}\ \textcolor{BrickRed}{=}\ T\textcolor{BrickRed}{[}j\textcolor{BrickRed}{-}\textcolor{Purple}{1}\textcolor{BrickRed}{];} \\
\mbox{}\ \ \ \ \ \ \ \ \ \ \ \ T\textcolor{BrickRed}{[}j\textcolor{BrickRed}{-}\textcolor{Purple}{1}\textcolor{BrickRed}{]}\ \textcolor{BrickRed}{=}\ aux\textcolor{BrickRed}{;} \\
\mbox{}\ \ \ \ \ \ \ \ \ \ \ \ j\textcolor{BrickRed}{-\/-;} \\
\mbox{}\ \ \ \ \ \ \ \ \textcolor{Red}{\}} \\
\mbox{}\ \ \ \ \textcolor{Red}{\}} \\
\mbox{}\textcolor{Red}{\}} \\
\mbox{} \\
\mbox{}\textbf{\textcolor{Blue}{const}}\ \textcolor{ForestGreen}{int}\ UMBRAL$\_$MS\ \textcolor{BrickRed}{=}\ \textcolor{Purple}{100}\textcolor{BrickRed}{;} \\
\mbox{} \\
\mbox{}\textcolor{ForestGreen}{void}\ \textbf{\textcolor{Black}{mergesort}}\textcolor{BrickRed}{(}\textcolor{ForestGreen}{int}\ T\textcolor{BrickRed}{[],}\ \textcolor{ForestGreen}{int}\ num$\_$elem\textcolor{BrickRed}{)}\textcolor{Red}{\{} \\
\mbox{}\ \ \ \ \textbf{\textcolor{Black}{mergesort$\_$lims}}\textcolor{BrickRed}{(}T\textcolor{BrickRed}{,}\ \textcolor{Purple}{0}\textcolor{BrickRed}{,}\ num$\_$elem\textcolor{BrickRed}{);} \\
\mbox{}\textcolor{Red}{\}} \\
\mbox{} \\
\mbox{}\textbf{\textcolor{Blue}{static}}\ \textcolor{ForestGreen}{void}\ \textbf{\textcolor{Black}{mergesort$\_$lims}}\textcolor{BrickRed}{(}\textcolor{ForestGreen}{int}\ T\textcolor{BrickRed}{[],}\ \textcolor{ForestGreen}{int}\ inicial\textcolor{BrickRed}{,}\ \textcolor{ForestGreen}{int}\ final\textcolor{BrickRed}{)}\textcolor{Red}{\{} \\
\mbox{}\ \ \ \ \textbf{\textcolor{Blue}{if}}\ \textcolor{BrickRed}{((}final\ \textcolor{BrickRed}{-}\ inicial\textcolor{BrickRed}{)}\ \textcolor{BrickRed}{\textless{}}\ UMBRAL$\_$MS\textcolor{BrickRed}{)}\ \textcolor{Red}{\{} \\
\mbox{}\ \ \ \ \ \ \ \ \textbf{\textcolor{Black}{insercion$\_$lims}}\textcolor{BrickRed}{(}T\textcolor{BrickRed}{,}\ inicial\textcolor{BrickRed}{,}\ final\textcolor{BrickRed}{);} \\
\mbox{}\ \ \ \ \textcolor{Red}{\}} \\
\mbox{}\ \ \ \ \textbf{\textcolor{Blue}{else}}\ \textcolor{Red}{\{} \\
\mbox{}\ \ \ \ \ \ \ \ \textcolor{ForestGreen}{int}\ k\ \textcolor{BrickRed}{=}\ \textcolor{BrickRed}{(}final\ \textcolor{BrickRed}{-}\ inicial\textcolor{BrickRed}{)/}\textcolor{Purple}{2}\textcolor{BrickRed}{;} \\
\mbox{}\ \ \ \ \ \ \ \  \\
\mbox{}\ \ \ \ \ \ \ \ \textcolor{ForestGreen}{int}\textcolor{BrickRed}{*}\ U\ \textcolor{BrickRed}{=}\ \textbf{\textcolor{Blue}{new}}\ \textcolor{ForestGreen}{int}\ \textcolor{BrickRed}{[}k\ \textcolor{BrickRed}{-}\ inicial\ \textcolor{BrickRed}{+}\ \textcolor{Purple}{1}\textcolor{BrickRed}{];} \\
\mbox{}\ \ \ \ \ \ \ \ \textbf{\textcolor{Black}{assert}}\textcolor{BrickRed}{(}U\textcolor{BrickRed}{);} \\
\mbox{}\ \ \ \ \ \ \ \ \textcolor{ForestGreen}{int}\ l\textcolor{BrickRed}{,}\ l2\textcolor{BrickRed}{;} \\
\mbox{}\ \ \ \ \ \ \ \ \textbf{\textcolor{Blue}{for}}\ \textcolor{BrickRed}{(}l\ \textcolor{BrickRed}{=}\ \textcolor{Purple}{0}\textcolor{BrickRed}{,}\ l2\ \textcolor{BrickRed}{=}\ inicial\textcolor{BrickRed}{;}\ l\ \textcolor{BrickRed}{\textless{}}\ k\textcolor{BrickRed}{;}\ l\textcolor{BrickRed}{++,}\ l2\textcolor{BrickRed}{++)} \\
\mbox{}\ \ \ \ \ \ \ \ \ \ \ \ U\textcolor{BrickRed}{[}l\textcolor{BrickRed}{]}\ \textcolor{BrickRed}{=}\ T\textcolor{BrickRed}{[}l2\textcolor{BrickRed}{];} \\
\mbox{}\ \ \ \ \ \ \ \ U\textcolor{BrickRed}{[}l\textcolor{BrickRed}{]}\ \textcolor{BrickRed}{=}\ INT$\_$MAX\textcolor{BrickRed}{;} \\
\mbox{}\ \ \ \ \ \ \ \  \\
\mbox{}\ \ \ \ \ \ \ \ \textcolor{ForestGreen}{int}\textcolor{BrickRed}{*}\ V\ \textcolor{BrickRed}{=}\ \textbf{\textcolor{Blue}{new}}\ \textcolor{ForestGreen}{int}\ \textcolor{BrickRed}{[}final\ \textcolor{BrickRed}{-}\ k\ \textcolor{BrickRed}{+}\ \textcolor{Purple}{1}\textcolor{BrickRed}{];} \\
\mbox{}\ \ \ \ \ \ \ \ \textbf{\textcolor{Black}{assert}}\textcolor{BrickRed}{(}V\textcolor{BrickRed}{);} \\
\mbox{}\ \ \ \ \ \ \ \ \textbf{\textcolor{Blue}{for}}\ \textcolor{BrickRed}{(}l\ \textcolor{BrickRed}{=}\ \textcolor{Purple}{0}\textcolor{BrickRed}{,}\ l2\ \textcolor{BrickRed}{=}\ k\textcolor{BrickRed}{;}\ l\ \textcolor{BrickRed}{\textless{}}\ final\ \textcolor{BrickRed}{-}\ k\textcolor{BrickRed}{;}\ l\textcolor{BrickRed}{++,}\ l2\textcolor{BrickRed}{++)} \\
\mbox{}\ \ \ \ \ \ \ \ \ \ \ \ V\textcolor{BrickRed}{[}l\textcolor{BrickRed}{]}\ \textcolor{BrickRed}{=}\ T\textcolor{BrickRed}{[}l2\textcolor{BrickRed}{];} \\
\mbox{}\ \ \ \ \ \ \ \ V\textcolor{BrickRed}{[}l\textcolor{BrickRed}{]}\ \textcolor{BrickRed}{=}\ INT$\_$MAX\textcolor{BrickRed}{;} \\
\mbox{}\ \ \ \ \ \ \ \  \\
\mbox{}\ \ \ \ \ \ \ \ \textbf{\textcolor{Black}{mergesort$\_$lims}}\textcolor{BrickRed}{(}U\textcolor{BrickRed}{,}\ \textcolor{Purple}{0}\textcolor{BrickRed}{,}\ k\textcolor{BrickRed}{);} \\
\mbox{}\ \ \ \ \ \ \ \ \textbf{\textcolor{Black}{mergesort$\_$lims}}\textcolor{BrickRed}{(}V\textcolor{BrickRed}{,}\ \textcolor{Purple}{0}\textcolor{BrickRed}{,}\ final\ \textcolor{BrickRed}{-}\ k\textcolor{BrickRed}{);} \\
\mbox{}\ \ \ \ \ \ \ \ \textbf{\textcolor{Black}{fusion}}\textcolor{BrickRed}{(}T\textcolor{BrickRed}{,}\ inicial\textcolor{BrickRed}{,}\ final\textcolor{BrickRed}{,}\ U\textcolor{BrickRed}{,}\ V\textcolor{BrickRed}{);} \\
\mbox{}\ \ \ \ \ \ \ \ \textbf{\textcolor{Blue}{delete}}\ \textcolor{BrickRed}{[]}\ U\textcolor{BrickRed}{;} \\
\mbox{}\ \ \ \ \ \ \ \ \textbf{\textcolor{Blue}{delete}}\ \textcolor{BrickRed}{[]}\ V\textcolor{BrickRed}{;} \\
\mbox{}\ \ \ \ \textcolor{Red}{\}} \\
\mbox{}\textcolor{Red}{\}} \\
\mbox{} \\
\mbox{} \\
\mbox{}\textbf{\textcolor{Blue}{static}}\ \textcolor{ForestGreen}{void}\ \textbf{\textcolor{Black}{fusion}}\textcolor{BrickRed}{(}\textcolor{ForestGreen}{int}\ T\textcolor{BrickRed}{[],}\ \textcolor{ForestGreen}{int}\ inicial\textcolor{BrickRed}{,}\ \textcolor{ForestGreen}{int}\ final\textcolor{BrickRed}{,}\ \textcolor{ForestGreen}{int}\ U\textcolor{BrickRed}{[],}\ \textcolor{ForestGreen}{int}\ V\textcolor{BrickRed}{[])}\ \textcolor{Red}{\{} \\
\mbox{}\ \ \ \ \textit{\textcolor{Brown}{//\ Toma\ el\ mínimo\ entre\ los\ restantes\ de\ los\ vectores\ U\ y\ V,}} \\
\mbox{}\ \ \ \ \textit{\textcolor{Brown}{//\ colocará\ ese\ mínimo\ en\ T\ y\ moverá\ el\ índice\ de\ lectura.}} \\
\mbox{}\ \ \ \ \textcolor{ForestGreen}{int}\ j\ \textcolor{BrickRed}{=}\ \textcolor{Purple}{0}\textcolor{BrickRed}{;} \\
\mbox{}\ \ \ \ \textcolor{ForestGreen}{int}\ k\ \textcolor{BrickRed}{=}\ \textcolor{Purple}{0}\textcolor{BrickRed}{;} \\
\mbox{} \\
\mbox{}\ \ \ \ \textbf{\textcolor{Blue}{for}}\ \textcolor{BrickRed}{(}\textcolor{ForestGreen}{int}\ i\ \textcolor{BrickRed}{=}\ inicial\textcolor{BrickRed}{;}\ i\ \textcolor{BrickRed}{\textless{}}\ final\textcolor{BrickRed}{;}\ i\textcolor{BrickRed}{++)}\ \textcolor{Red}{\{} \\
\mbox{}\ \ \ \ \ \ \ \ \textbf{\textcolor{Blue}{if}}\ \textcolor{BrickRed}{(}U\textcolor{BrickRed}{[}j\textcolor{BrickRed}{]}\ \textcolor{BrickRed}{\textless{}}\ V\textcolor{BrickRed}{[}k\textcolor{BrickRed}{])}\ \textcolor{Red}{\{} \\
\mbox{}\ \ \ \ \ \ \ \ \ \ \ \ T\textcolor{BrickRed}{[}i\textcolor{BrickRed}{]}\ \textcolor{BrickRed}{=}\ U\textcolor{BrickRed}{[}j\textcolor{BrickRed}{];} \\
\mbox{}\ \ \ \ \ \ \ \ \ \ \ \ j\textcolor{BrickRed}{++;} \\
\mbox{}\ \ \ \ \ \ \ \ \textcolor{Red}{\}} \\
\mbox{}\ \ \ \ \ \ \ \ \textbf{\textcolor{Blue}{else}}\ \textcolor{Red}{\{} \\
\mbox{}\ \ \ \ \ \ \ \ \ \ \ \ T\textcolor{BrickRed}{[}i\textcolor{BrickRed}{]}\ \textcolor{BrickRed}{=}\ V\textcolor{BrickRed}{[}k\textcolor{BrickRed}{];} \\
\mbox{}\ \ \ \ \ \ \ \ \ \ \ \ k\textcolor{BrickRed}{++;} \\
\mbox{}\ \ \ \ \ \ \ \ \textcolor{Red}{\}} \\
\mbox{}\ \ \ \ \textcolor{Red}{\}} \\
\mbox{}\textcolor{Red}{\}} \\
\mbox{} \\
\mbox{}\textcolor{ForestGreen}{int}\ \textbf{\textcolor{Black}{main}}\textcolor{BrickRed}{(}\textcolor{ForestGreen}{int}\ argc\textcolor{BrickRed}{,}\ \textcolor{ForestGreen}{char}\textcolor{BrickRed}{*}\ argv\textcolor{BrickRed}{[])}\ \textcolor{Red}{\{} \\
\mbox{}\ \ \ \ \textbf{\textcolor{Blue}{if}}\ \textcolor{BrickRed}{(}argc\ \textcolor{BrickRed}{!=}\ \textcolor{Purple}{2}\textcolor{BrickRed}{)}\textcolor{Red}{\{} \\
\mbox{}\ \ \ \ \ \ \ \ cerr\ \textcolor{BrickRed}{\textless{}\textless{}}\ \texttt{\textcolor{Red}{"{}Uso\ del\ programa:\ "{}}}\ \textcolor{BrickRed}{+}\ \textcolor{BrickRed}{(}string\textcolor{BrickRed}{)(}argv\textcolor{BrickRed}{[}\textcolor{Purple}{0}\textcolor{BrickRed}{])}\ \textcolor{BrickRed}{+}\ \texttt{\textcolor{Red}{"{}\ \textless{}número\ positivo\textgreater{}"{}}}\ \textcolor{BrickRed}{\textless{}\textless{}}\ endl\textcolor{BrickRed}{;}\ \  \\
\mbox{}\ \ \ \ \ \ \ \ \textbf{\textcolor{Blue}{return}}\ \textcolor{BrickRed}{-}\textcolor{Purple}{1}\textcolor{BrickRed}{;} \\
\mbox{}\ \ \ \ \textcolor{Red}{\}} \\
\mbox{} \\
\mbox{}\ \ \ \ \textcolor{ForestGreen}{int}\ n\ \textcolor{BrickRed}{=}\ \textbf{\textcolor{Black}{atoi}}\textcolor{BrickRed}{(}argv\textcolor{BrickRed}{[}\textcolor{Purple}{1}\textcolor{BrickRed}{]);}\ \ \ \  \\
\mbox{}\ \ \ \ \textbf{\textcolor{Blue}{if}}\ \textcolor{BrickRed}{(}n\textcolor{BrickRed}{\textless{}}\textcolor{Purple}{0}\textcolor{BrickRed}{)}\ \textbf{\textcolor{Blue}{return}}\ \textcolor{BrickRed}{-}\textcolor{Purple}{1}\textcolor{BrickRed}{;} \\
\mbox{}\ \ \ \  \\
\mbox{}\ \ \ \ \textcolor{ForestGreen}{int}\ \textcolor{BrickRed}{*}\ T\ \textcolor{BrickRed}{=}\ \textbf{\textcolor{Blue}{new}}\ \textcolor{ForestGreen}{int}\textcolor{BrickRed}{[}n\textcolor{BrickRed}{];} \\
\mbox{}\ \ \ \ \textbf{\textcolor{Blue}{struct}}\ \textcolor{TealBlue}{timespec}\ t$\_$antes\textcolor{BrickRed}{,}\ t$\_$despues\textcolor{BrickRed}{;} \\
\mbox{}\ \ \ \  \\
\mbox{}\ \ \ \ \textit{\textcolor{Brown}{//\ Vector\ aleatorio.}} \\
\mbox{}\ \ \ \ \textbf{\textcolor{Black}{srandom}}\textcolor{BrickRed}{(}\textbf{\textcolor{Black}{time}}\textcolor{BrickRed}{(}\textcolor{Purple}{0}\textcolor{BrickRed}{));} \\
\mbox{}\ \ \ \ \textbf{\textcolor{Blue}{for}}\ \textcolor{BrickRed}{(}\textcolor{ForestGreen}{int}\ i\textcolor{BrickRed}{=}\textcolor{Purple}{0}\textcolor{BrickRed}{;}\ i\textcolor{BrickRed}{\textless{}}n\textcolor{BrickRed}{;}\ \textcolor{BrickRed}{++}i\textcolor{BrickRed}{)}\ \textcolor{Red}{\{} \\
\mbox{}\ \ \ \ \ \ \ \ T\textcolor{BrickRed}{[}i\textcolor{BrickRed}{]}\ \textcolor{BrickRed}{=}\ \textbf{\textcolor{Black}{random}}\textcolor{BrickRed}{();} \\
\mbox{}\ \ \ \ \textcolor{Red}{\}} \\
\mbox{}\ \ \ \  \\
\mbox{}\ \ \ \ \textit{\textcolor{Brown}{//\ Medida\ del\ tiempo.\ Ejecución\ del\ algoritmo.}} \\
\mbox{}\ \ \ \ \textbf{\textcolor{Black}{clock$\_$gettime}}\textcolor{BrickRed}{(}CLOCK$\_$REALTIME\textcolor{BrickRed}{,\&}t$\_$antes\textcolor{BrickRed}{);} \\
\mbox{}\ \ \ \ \textbf{\textcolor{Black}{mergesort}}\textcolor{BrickRed}{(}T\textcolor{BrickRed}{,}n\textcolor{BrickRed}{);} \\
\mbox{}\ \ \ \ \textbf{\textcolor{Black}{clock$\_$gettime}}\textcolor{BrickRed}{(}CLOCK$\_$REALTIME\textcolor{BrickRed}{,\&}t$\_$despues\textcolor{BrickRed}{);} \\
\mbox{}\ \ \ \  \\
\mbox{}\ \ \ \  \\
\mbox{}\ \ \ \ cout\textcolor{BrickRed}{.}\textbf{\textcolor{Black}{precision}}\textcolor{BrickRed}{(}\textcolor{Purple}{3}\textcolor{BrickRed}{);} \\
\mbox{}\ \ \ \ cout\ \textcolor{BrickRed}{\textless{}\textless{}}\ \textcolor{BrickRed}{(}\textcolor{ForestGreen}{double}\textcolor{BrickRed}{)}\ \textcolor{BrickRed}{(}t$\_$despues\textcolor{BrickRed}{.}tv$\_$sec\textcolor{BrickRed}{-}t$\_$antes\textcolor{BrickRed}{.}tv$\_$sec\textcolor{BrickRed}{)+} \\
\mbox{}\ \ \ \ \ \ \ \ \textcolor{BrickRed}{(}\textcolor{ForestGreen}{double}\textcolor{BrickRed}{)}\ \textcolor{BrickRed}{((}t$\_$despues\textcolor{BrickRed}{.}tv$\_$nsec\textcolor{BrickRed}{-}t$\_$antes\textcolor{BrickRed}{.}tv$\_$nsec\textcolor{BrickRed}{)/(}\textcolor{Purple}{1}\textcolor{BrickRed}{.}e\textcolor{BrickRed}{+}\textcolor{Purple}{9}\textcolor{BrickRed}{))}\ \textcolor{BrickRed}{\textless{}\textless{}}\ endl\textcolor{BrickRed}{;} \\
\mbox{} \\
\mbox{}\ \ \ \ \textbf{\textcolor{Blue}{delete}}\ \textcolor{BrickRed}{[]}\ T\textcolor{BrickRed}{;} \\
\mbox{}\ \ \ \  \\
\mbox{}\ \ \ \ \textbf{\textcolor{Blue}{return}}\ \textcolor{Purple}{0}\textcolor{BrickRed}{;} \\
\mbox{}\textcolor{Red}{\}} \\
\mbox{}


\subsubsection{Algoritmo de Hoare}
% Generator: GNU source-highlight, by Lorenzo Bettini, http://www.gnu.org/software/src-highlite
\noindent
\mbox{}\textit{\textcolor{Brown}{/**}} \\
\mbox{}\textit{\textcolor{Brown}{\ *\ }}\textcolor{ForestGreen}{@file}\textit{\textcolor{Brown}{\ quicksort.cpp}} \\
\mbox{}\textit{\textcolor{Brown}{\ *\ Ordenación\ rápida\ (quicksort).}} \\
\mbox{}\textit{\textcolor{Brown}{\ */}} \\
\mbox{} \\
\mbox{}\textbf{\textcolor{RoyalBlue}{\#include}}\ \texttt{\textcolor{Red}{\textless{}iostream\textgreater{}}} \\
\mbox{}\textbf{\textcolor{RoyalBlue}{\#include}}\ \texttt{\textcolor{Red}{\textless{}ctime\textgreater{}}} \\
\mbox{}\textbf{\textcolor{RoyalBlue}{\#include}}\ \texttt{\textcolor{Red}{\textless{}cstdlib\textgreater{}}} \\
\mbox{}\textbf{\textcolor{Blue}{using}}\ \textbf{\textcolor{Blue}{namespace}}\ std\textcolor{BrickRed}{;} \\
\mbox{} \\
\mbox{}\textbf{\textcolor{RoyalBlue}{\#define}}\ NUM$\_$VECES\ \textcolor{Purple}{100} \\
\mbox{} \\
\mbox{}\textit{\textcolor{Brown}{/**}} \\
\mbox{}\textit{\textcolor{Brown}{\ *\ }}\textcolor{ForestGreen}{@brief}\textit{\textcolor{Brown}{\ Ordena\ un\ vector\ por\ el\ método\ quicksort.}} \\
\mbox{}\textit{\textcolor{Brown}{\ *\ }}\textcolor{ForestGreen}{@param}\textit{\textcolor{Brown}{\ T:\ vector\ de\ elementos.\ Debe\ tener\ num$\_$elem\ elementos.}} \\
\mbox{}\textit{\textcolor{Brown}{\ *\ Es\ modificado.}} \\
\mbox{}\textit{\textcolor{Brown}{\ *\ }}\textcolor{ForestGreen}{@param}\textit{\textcolor{Brown}{\ num$\_$elem:\ número\ de\ elementos.\ num$\_$elem\ \textgreater{}\ 0.}} \\
\mbox{}\textit{\textcolor{Brown}{\ *\ }} \\
\mbox{}\textit{\textcolor{Brown}{\ *\ Cambia\ el\ orden\ de\ los\ elementos\ de\ T\ de\ forma\ que\ los\ dispone}} \\
\mbox{}\textit{\textcolor{Brown}{\ *\ en\ sentido\ creciente\ de\ menor\ a\ mayor.}} \\
\mbox{}\textit{\textcolor{Brown}{\ *\ Aplica\ el\ algoritmo\ quicksort.}} \\
\mbox{}\textit{\textcolor{Brown}{\ */}} \\
\mbox{} \\
\mbox{}\textbf{\textcolor{Blue}{inline}}\ \textbf{\textcolor{Blue}{static}}\ \textcolor{ForestGreen}{void}\ \textbf{\textcolor{Black}{quicksort}}\textcolor{BrickRed}{(}\textcolor{ForestGreen}{int}\ T\textcolor{BrickRed}{[],}\ \textcolor{ForestGreen}{int}\ num$\_$elem\textcolor{BrickRed}{);} \\
\mbox{} \\
\mbox{}\textit{\textcolor{Brown}{/**}} \\
\mbox{}\textit{\textcolor{Brown}{\ *\ }}\textcolor{ForestGreen}{@brief}\textit{\textcolor{Brown}{\ Ordena\ parte\ de\ un\ vector\ por\ el\ método\ quicksort.}} \\
\mbox{}\textit{\textcolor{Brown}{\ *\ }}\textcolor{ForestGreen}{@param}\textit{\textcolor{Brown}{\ T:\ vector\ de\ elementos.\ Tiene\ un\ número\ de\ elementos\ }} \\
\mbox{}\textit{\textcolor{Brown}{\ *\ mayor\ o\ igual\ a\ final.\ Es\ MODIFICADO.}} \\
\mbox{}\textit{\textcolor{Brown}{\ *\ }}\textcolor{ForestGreen}{@param}\textit{\textcolor{Brown}{\ inicial:\ Posición\ que\ marca\ el\ incio\ de\ la\ parte\ del}} \\
\mbox{}\textit{\textcolor{Brown}{\ *\ vector\ a\ ordenar.}} \\
\mbox{}\textit{\textcolor{Brown}{\ *\ }}\textcolor{ForestGreen}{@param}\textit{\textcolor{Brown}{\ final:\ Posición\ detrás\ de\ la\ última\ de\ la\ parte\ del}} \\
\mbox{}\textit{\textcolor{Brown}{\ *\ vector\ a\ ordenar.\ }} \\
\mbox{}\textit{\textcolor{Brown}{\ *\ }}\textcolor{ForestGreen}{@pre}\textit{\textcolor{Brown}{\ inicial\ \textless{}\ final.}} \\
\mbox{}\textit{\textcolor{Brown}{\ *\ }} \\
\mbox{}\textit{\textcolor{Brown}{\ *\ Cambia\ el\ orden\ de\ los\ elementos\ de\ T\ entre\ las\ posiciones}} \\
\mbox{}\textit{\textcolor{Brown}{\ *\ inicial\ y\ final\ -\ 1\ de\ forma\ que\ los\ dispone\ en\ sentido\ creciente}} \\
\mbox{}\textit{\textcolor{Brown}{\ *\ de\ menor\ a\ mayor.}} \\
\mbox{}\textit{\textcolor{Brown}{\ *\ Aplica\ el\ algoritmo\ quicksort.}} \\
\mbox{}\textit{\textcolor{Brown}{\ */}} \\
\mbox{} \\
\mbox{}\textbf{\textcolor{Blue}{static}}\ \textcolor{ForestGreen}{void}\ \textbf{\textcolor{Black}{quicksort$\_$lims}}\textcolor{BrickRed}{(}\textcolor{ForestGreen}{int}\ T\textcolor{BrickRed}{[],}\ \textcolor{ForestGreen}{int}\ inicial\textcolor{BrickRed}{,}\ \textcolor{ForestGreen}{int}\ final\textcolor{BrickRed}{);} \\
\mbox{} \\
\mbox{}\textit{\textcolor{Brown}{/**}} \\
\mbox{}\textit{\textcolor{Brown}{\ *\ }}\textcolor{ForestGreen}{@brief}\textit{\textcolor{Brown}{\ Ordena\ un\ vector\ por\ el\ método\ de\ inserción.}} \\
\mbox{}\textit{\textcolor{Brown}{\ *\ }}\textcolor{ForestGreen}{@param}\textit{\textcolor{Brown}{\ T:\ vector\ de\ elementos.\ Debe\ tener\ num$\_$elem\ elementos.}} \\
\mbox{}\textit{\textcolor{Brown}{\ *\ Es\ modificado.}} \\
\mbox{}\textit{\textcolor{Brown}{\ *\ }}\textcolor{ForestGreen}{@param}\textit{\textcolor{Brown}{\ num$\_$elem:\ número\ de\ elementos.\ num$\_$elem\ \textgreater{}\ 0.}} \\
\mbox{}\textit{\textcolor{Brown}{\ *\ }} \\
\mbox{}\textit{\textcolor{Brown}{\ *\ Cambia\ el\ orden\ de\ los\ elementos\ de\ T\ de\ forma\ que\ los\ dispone}} \\
\mbox{}\textit{\textcolor{Brown}{\ *\ en\ sentido\ creciente\ de\ menor\ a\ mayor.}} \\
\mbox{}\textit{\textcolor{Brown}{\ *\ Aplica\ el\ algoritmo\ de\ inserción.}} \\
\mbox{}\textit{\textcolor{Brown}{\ */}} \\
\mbox{} \\
\mbox{}\textbf{\textcolor{Blue}{inline}}\ \textbf{\textcolor{Blue}{static}}\ \textcolor{ForestGreen}{void}\ \textbf{\textcolor{Black}{insercion}}\textcolor{BrickRed}{(}\textcolor{ForestGreen}{int}\ T\textcolor{BrickRed}{[],}\ \textcolor{ForestGreen}{int}\ num$\_$elem\textcolor{BrickRed}{);} \\
\mbox{} \\
\mbox{}\textit{\textcolor{Brown}{/**}} \\
\mbox{}\textit{\textcolor{Brown}{\ *\ }}\textcolor{ForestGreen}{@brief}\textit{\textcolor{Brown}{\ Ordena\ parte\ de\ un\ vector\ por\ el\ método\ de\ inserción.}} \\
\mbox{}\textit{\textcolor{Brown}{\ *\ }}\textcolor{ForestGreen}{@param}\textit{\textcolor{Brown}{\ T:\ vector\ de\ elementos.\ Tiene\ un\ número\ de\ elementos\ }} \\
\mbox{}\textit{\textcolor{Brown}{\ *\ mayor\ o\ igual\ a\ final.\ Es\ MODIFICADO.}} \\
\mbox{}\textit{\textcolor{Brown}{\ *\ }}\textcolor{ForestGreen}{@param}\textit{\textcolor{Brown}{\ inicial:\ Posición\ que\ marca\ el\ incio\ de\ la\ parte\ del}} \\
\mbox{}\textit{\textcolor{Brown}{\ *\ vector\ a\ ordenar.}} \\
\mbox{}\textit{\textcolor{Brown}{\ *\ }}\textcolor{ForestGreen}{@param}\textit{\textcolor{Brown}{\ final:\ Posición\ detrás\ de\ la\ última\ de\ la\ parte\ del}} \\
\mbox{}\textit{\textcolor{Brown}{\ *\ vector\ a\ ordenar.\ }} \\
\mbox{}\textit{\textcolor{Brown}{\ *\ }}\textcolor{ForestGreen}{@pre}\textit{\textcolor{Brown}{\ inicial\ \textless{}\ final.}} \\
\mbox{}\textit{\textcolor{Brown}{\ *\ }} \\
\mbox{}\textit{\textcolor{Brown}{\ *\ Cambia\ el\ orden\ de\ los\ elementos\ de\ T\ entre\ las\ posiciones}} \\
\mbox{}\textit{\textcolor{Brown}{\ *\ inicial\ y\ final\ -\ 1\ de\ forma\ que\ los\ dispone\ en\ sentido\ creciente}} \\
\mbox{}\textit{\textcolor{Brown}{\ *\ de\ menor\ a\ mayor.}} \\
\mbox{}\textit{\textcolor{Brown}{\ *\ Aplica\ el\ algoritmo\ de\ inserción.}} \\
\mbox{}\textit{\textcolor{Brown}{\ */}} \\
\mbox{} \\
\mbox{}\textbf{\textcolor{Blue}{static}}\ \textcolor{ForestGreen}{void}\ \textbf{\textcolor{Black}{insercion$\_$lims}}\textcolor{BrickRed}{(}\textcolor{ForestGreen}{int}\ T\textcolor{BrickRed}{[],}\ \textcolor{ForestGreen}{int}\ inicial\textcolor{BrickRed}{,}\ \textcolor{ForestGreen}{int}\ final\textcolor{BrickRed}{);} \\
\mbox{} \\
\mbox{}\textit{\textcolor{Brown}{/**}} \\
\mbox{}\textit{\textcolor{Brown}{\ *\ }}\textcolor{ForestGreen}{@brief}\textit{\textcolor{Brown}{\ Redistribuye\ los\ elementos\ de\ un\ vector\ según\ un\ pivote.}} \\
\mbox{}\textit{\textcolor{Brown}{\ *\ }}\textcolor{ForestGreen}{@param}\textit{\textcolor{Brown}{\ T:\ vector\ de\ elementos.\ Tiene\ un\ número\ de\ elementos\ }} \\
\mbox{}\textit{\textcolor{Brown}{\ *\ mayor\ o\ igual\ a\ final.\ Es\ MODIFICADO.}} \\
\mbox{}\textit{\textcolor{Brown}{\ *\ }}\textcolor{ForestGreen}{@param}\textit{\textcolor{Brown}{\ inicial:\ Posición\ que\ marca\ el\ incio\ de\ la\ parte\ del}} \\
\mbox{}\textit{\textcolor{Brown}{\ *\ vector\ a\ ordenar.}} \\
\mbox{}\textit{\textcolor{Brown}{\ *\ }}\textcolor{ForestGreen}{@param}\textit{\textcolor{Brown}{\ final:\ Posición\ detrás\ de\ la\ última\ de\ la\ parte\ del}} \\
\mbox{}\textit{\textcolor{Brown}{\ *\ vector\ a\ ordenar.\ }} \\
\mbox{}\textit{\textcolor{Brown}{\ *\ }}\textcolor{ForestGreen}{@pre}\textit{\textcolor{Brown}{\ inicial\ \textless{}\ final.\ \ }} \\
\mbox{}\textit{\textcolor{Brown}{\ *\ }}\textcolor{ForestGreen}{@param}\textit{\textcolor{Brown}{\ pp:\ Posición\ del\ pivote.\ Es\ MODIFICADO.}} \\
\mbox{}\textit{\textcolor{Brown}{\ *\ }} \\
\mbox{}\textit{\textcolor{Brown}{\ *\ Selecciona\ un\ pivote\ los\ elementos\ de\ T\ situados\ en\ las\ posiciones}} \\
\mbox{}\textit{\textcolor{Brown}{\ *\ entre\ inicial\ y\ final\ -\ 1.\ Redistribuye\ los\ elementos,\ situando\ los}} \\
\mbox{}\textit{\textcolor{Brown}{\ *\ menores\ que\ el\ pivote\ a\ su\ izquierda,\ después\ los\ iguales\ y\ a\ la}} \\
\mbox{}\textit{\textcolor{Brown}{\ *\ derecha\ los\ mayores.\ La\ posición\ del\ pivote\ se\ devuelve\ en\ pp.}} \\
\mbox{}\textit{\textcolor{Brown}{\ */}} \\
\mbox{} \\
\mbox{}\textbf{\textcolor{Blue}{static}}\ \textcolor{ForestGreen}{void}\ \textbf{\textcolor{Black}{dividir$\_$qs}}\textcolor{BrickRed}{(}\textcolor{ForestGreen}{int}\ T\textcolor{BrickRed}{[],}\ \textcolor{ForestGreen}{int}\ inicial\textcolor{BrickRed}{,}\ \textcolor{ForestGreen}{int}\ final\textcolor{BrickRed}{,}\ \textcolor{ForestGreen}{int}\ \textcolor{BrickRed}{\&}\ pp\textcolor{BrickRed}{);} \\
\mbox{} \\
\mbox{}\textit{\textcolor{Brown}{//\ Implementación\ de\ las\ funciones}} \\
\mbox{} \\
\mbox{}\textbf{\textcolor{Blue}{inline}}\ \textbf{\textcolor{Blue}{static}}\ \textcolor{ForestGreen}{void}\ \textbf{\textcolor{Black}{insercion}}\textcolor{BrickRed}{(}\textcolor{ForestGreen}{int}\ T\textcolor{BrickRed}{[],}\ \textcolor{ForestGreen}{int}\ num$\_$elem\textcolor{BrickRed}{)}\textcolor{Red}{\{} \\
\mbox{}\ \ \ \ \textbf{\textcolor{Black}{insercion$\_$lims}}\textcolor{BrickRed}{(}T\textcolor{BrickRed}{,}\ \textcolor{Purple}{0}\textcolor{BrickRed}{,}\ num$\_$elem\textcolor{BrickRed}{);} \\
\mbox{}\textcolor{Red}{\}} \\
\mbox{} \\
\mbox{}\textbf{\textcolor{Blue}{static}}\ \textcolor{ForestGreen}{void}\ \textbf{\textcolor{Black}{insercion$\_$lims}}\textcolor{BrickRed}{(}\textcolor{ForestGreen}{int}\ T\textcolor{BrickRed}{[],}\ \textcolor{ForestGreen}{int}\ inicial\textcolor{BrickRed}{,}\ \textcolor{ForestGreen}{int}\ final\textcolor{BrickRed}{)}\textcolor{Red}{\{} \\
\mbox{}\ \ \ \ \textcolor{ForestGreen}{int}\ i\textcolor{BrickRed}{,}\ j\textcolor{BrickRed}{;} \\
\mbox{}\ \ \ \ \textcolor{ForestGreen}{int}\ aux\textcolor{BrickRed}{;} \\
\mbox{}\ \ \ \ \textbf{\textcolor{Blue}{for}}\ \textcolor{BrickRed}{(}i\ \textcolor{BrickRed}{=}\ inicial\ \textcolor{BrickRed}{+}\ \textcolor{Purple}{1}\textcolor{BrickRed}{;}\ i\ \textcolor{BrickRed}{\textless{}}\ final\textcolor{BrickRed}{;}\ i\textcolor{BrickRed}{++)}\ \textcolor{Red}{\{} \\
\mbox{}\ \ \ \ \ \ \ \ j\ \textcolor{BrickRed}{=}\ i\textcolor{BrickRed}{;} \\
\mbox{}\ \ \ \ \ \ \ \ \textbf{\textcolor{Blue}{while}}\ \textcolor{BrickRed}{((}T\textcolor{BrickRed}{[}j\textcolor{BrickRed}{]}\ \textcolor{BrickRed}{\textless{}}\ T\textcolor{BrickRed}{[}j\textcolor{BrickRed}{-}\textcolor{Purple}{1}\textcolor{BrickRed}{])}\ \textbf{\textcolor{Black}{and}}\ \textcolor{BrickRed}{(}j\ \textcolor{BrickRed}{\textgreater{}}\ \textcolor{Purple}{0}\textcolor{BrickRed}{))}\ \textcolor{Red}{\{} \\
\mbox{}\ \ \ \ \ \ \ \ \ \ \ \ aux\ \textcolor{BrickRed}{=}\ T\textcolor{BrickRed}{[}j\textcolor{BrickRed}{];} \\
\mbox{}\ \ \ \ \ \ \ \ \ \ \ \ T\textcolor{BrickRed}{[}j\textcolor{BrickRed}{]}\ \textcolor{BrickRed}{=}\ T\textcolor{BrickRed}{[}j\textcolor{BrickRed}{-}\textcolor{Purple}{1}\textcolor{BrickRed}{];} \\
\mbox{}\ \ \ \ \ \ \ \ \ \ \ \ T\textcolor{BrickRed}{[}j\textcolor{BrickRed}{-}\textcolor{Purple}{1}\textcolor{BrickRed}{]}\ \textcolor{BrickRed}{=}\ aux\textcolor{BrickRed}{;} \\
\mbox{}\ \ \ \ \ \ \ \ \ \ \ \ j\textcolor{BrickRed}{-\/-;} \\
\mbox{}\ \ \ \ \ \ \ \ \textcolor{Red}{\}} \\
\mbox{}\ \ \ \ \textcolor{Red}{\}} \\
\mbox{}\textcolor{Red}{\}} \\
\mbox{} \\
\mbox{}\textbf{\textcolor{Blue}{const}}\ \textcolor{ForestGreen}{int}\ UMBRAL$\_$QS\ \textcolor{BrickRed}{=}\ \textcolor{Purple}{50}\textcolor{BrickRed}{;} \\
\mbox{} \\
\mbox{}\textbf{\textcolor{Blue}{inline}}\ \textcolor{ForestGreen}{void}\ \textbf{\textcolor{Black}{quicksort}}\textcolor{BrickRed}{(}\textcolor{ForestGreen}{int}\ T\textcolor{BrickRed}{[],}\ \textcolor{ForestGreen}{int}\ num$\_$elem\textcolor{BrickRed}{)}\textcolor{Red}{\{} \\
\mbox{}\ \ \ \ \textbf{\textcolor{Black}{quicksort$\_$lims}}\textcolor{BrickRed}{(}T\textcolor{BrickRed}{,}\ \textcolor{Purple}{0}\textcolor{BrickRed}{,}\ num$\_$elem\textcolor{BrickRed}{);} \\
\mbox{}\textcolor{Red}{\}} \\
\mbox{} \\
\mbox{}\textbf{\textcolor{Blue}{static}}\ \textcolor{ForestGreen}{void}\ \textbf{\textcolor{Black}{quicksort$\_$lims}}\textcolor{BrickRed}{(}\textcolor{ForestGreen}{int}\ T\textcolor{BrickRed}{[],}\ \textcolor{ForestGreen}{int}\ inicial\textcolor{BrickRed}{,}\ \textcolor{ForestGreen}{int}\ final\textcolor{BrickRed}{)}\textcolor{Red}{\{} \\
\mbox{}\ \ \ \ \textcolor{ForestGreen}{int}\ k\textcolor{BrickRed}{;} \\
\mbox{}\ \ \ \ \textbf{\textcolor{Blue}{if}}\ \textcolor{BrickRed}{(}final\ \textcolor{BrickRed}{-}\ inicial\ \textcolor{BrickRed}{\textless{}}\ UMBRAL$\_$QS\textcolor{BrickRed}{)}\ \textcolor{Red}{\{} \\
\mbox{}\ \ \ \ \ \ \ \ \textbf{\textcolor{Black}{insercion$\_$lims}}\textcolor{BrickRed}{(}T\textcolor{BrickRed}{,}\ inicial\textcolor{BrickRed}{,}\ final\textcolor{BrickRed}{);} \\
\mbox{}\ \ \ \ \textcolor{Red}{\}} \\
\mbox{}\ \ \ \ \textbf{\textcolor{Blue}{else}}\ \textcolor{Red}{\{} \\
\mbox{}\ \ \ \ \ \ \ \ \textbf{\textcolor{Black}{dividir$\_$qs}}\textcolor{BrickRed}{(}T\textcolor{BrickRed}{,}\ inicial\textcolor{BrickRed}{,}\ final\textcolor{BrickRed}{,}\ k\textcolor{BrickRed}{);} \\
\mbox{}\ \ \ \ \ \ \ \ \textbf{\textcolor{Black}{quicksort$\_$lims}}\textcolor{BrickRed}{(}T\textcolor{BrickRed}{,}\ inicial\textcolor{BrickRed}{,}\ k\textcolor{BrickRed}{);} \\
\mbox{}\ \ \ \ \ \ \ \ \textbf{\textcolor{Black}{quicksort$\_$lims}}\textcolor{BrickRed}{(}T\textcolor{BrickRed}{,}\ k\ \textcolor{BrickRed}{+}\ \textcolor{Purple}{1}\textcolor{BrickRed}{,}\ final\textcolor{BrickRed}{);} \\
\mbox{}\ \ \ \ \textcolor{Red}{\}} \\
\mbox{}\textcolor{Red}{\}} \\
\mbox{} \\
\mbox{}\textbf{\textcolor{Blue}{static}}\ \textcolor{ForestGreen}{void}\ \textbf{\textcolor{Black}{dividir$\_$qs}}\textcolor{BrickRed}{(}\textcolor{ForestGreen}{int}\ T\textcolor{BrickRed}{[],}\ \textcolor{ForestGreen}{int}\ inicial\textcolor{BrickRed}{,}\ \textcolor{ForestGreen}{int}\ final\textcolor{BrickRed}{,}\ \textcolor{ForestGreen}{int}\ \textcolor{BrickRed}{\&}\ pp\textcolor{BrickRed}{)}\textcolor{Red}{\{} \\
\mbox{}\ \ \ \ \textcolor{ForestGreen}{int}\ pivote\textcolor{BrickRed}{,}\ aux\textcolor{BrickRed}{;} \\
\mbox{}\ \ \ \ \textcolor{ForestGreen}{int}\ k\textcolor{BrickRed}{,}\ l\textcolor{BrickRed}{;} \\
\mbox{}\ \ \ \  \\
\mbox{}\ \ \ \ pivote\ \textcolor{BrickRed}{=}\ T\textcolor{BrickRed}{[}inicial\textcolor{BrickRed}{];} \\
\mbox{}\ \ \ \ k\ \textcolor{BrickRed}{=}\ inicial\textcolor{BrickRed}{;} \\
\mbox{}\ \ \ \ l\ \textcolor{BrickRed}{=}\ final\textcolor{BrickRed}{;} \\
\mbox{}\ \ \ \  \\
\mbox{}\ \ \ \ \textbf{\textcolor{Blue}{do}}\ k\textcolor{BrickRed}{++;}\ \textbf{\textcolor{Blue}{while}}\ \textcolor{BrickRed}{((}T\textcolor{BrickRed}{[}k\textcolor{BrickRed}{]}\ \textcolor{BrickRed}{\textless{}=}\ pivote\textcolor{BrickRed}{)}\ \textbf{\textcolor{Black}{and}}\ \textcolor{BrickRed}{(}k\ \textcolor{BrickRed}{\textless{}}\ final\textcolor{BrickRed}{-}\textcolor{Purple}{1}\textcolor{BrickRed}{));} \\
\mbox{}\ \ \ \ \textbf{\textcolor{Blue}{do}}\ l\textcolor{BrickRed}{-\/-;}\ \textbf{\textcolor{Blue}{while}}\ \textcolor{BrickRed}{(}T\textcolor{BrickRed}{[}l\textcolor{BrickRed}{]}\ \textcolor{BrickRed}{\textgreater{}}\ pivote\textcolor{BrickRed}{);} \\
\mbox{}\ \ \ \  \\
\mbox{}\ \ \ \ \textbf{\textcolor{Blue}{while}}\ \textcolor{BrickRed}{(}k\ \textcolor{BrickRed}{\textless{}}\ l\textcolor{BrickRed}{)}\ \textcolor{Red}{\{} \\
\mbox{}\ \ \ \ \ \ \ \ aux\ \textcolor{BrickRed}{=}\ T\textcolor{BrickRed}{[}k\textcolor{BrickRed}{];} \\
\mbox{}\ \ \ \ \ \ \ \ T\textcolor{BrickRed}{[}k\textcolor{BrickRed}{]}\ \textcolor{BrickRed}{=}\ T\textcolor{BrickRed}{[}l\textcolor{BrickRed}{];} \\
\mbox{}\ \ \ \ \ \ \ \ T\textcolor{BrickRed}{[}l\textcolor{BrickRed}{]}\ \textcolor{BrickRed}{=}\ aux\textcolor{BrickRed}{;} \\
\mbox{}\ \ \ \ \ \ \ \ \textbf{\textcolor{Blue}{do}}\ k\textcolor{BrickRed}{++;}\ \textbf{\textcolor{Blue}{while}}\ \textcolor{BrickRed}{(}T\textcolor{BrickRed}{[}k\textcolor{BrickRed}{]}\ \textcolor{BrickRed}{\textless{}=}\ pivote\textcolor{BrickRed}{);} \\
\mbox{}\ \ \ \ \ \ \ \ \textbf{\textcolor{Blue}{do}}\ l\textcolor{BrickRed}{-\/-;}\ \textbf{\textcolor{Blue}{while}}\ \textcolor{BrickRed}{(}T\textcolor{BrickRed}{[}l\textcolor{BrickRed}{]}\ \textcolor{BrickRed}{\textgreater{}}\ pivote\textcolor{BrickRed}{);} \\
\mbox{}\ \ \ \ \textcolor{Red}{\}} \\
\mbox{} \\
\mbox{}\ \ \ \ aux\ \textcolor{BrickRed}{=}\ T\textcolor{BrickRed}{[}inicial\textcolor{BrickRed}{];} \\
\mbox{}\ \ \ \ T\textcolor{BrickRed}{[}inicial\textcolor{BrickRed}{]}\ \textcolor{BrickRed}{=}\ T\textcolor{BrickRed}{[}l\textcolor{BrickRed}{];} \\
\mbox{}\ \ \ \ T\textcolor{BrickRed}{[}l\textcolor{BrickRed}{]}\ \textcolor{BrickRed}{=}\ aux\textcolor{BrickRed}{;} \\
\mbox{}\ \ \ \ pp\ \textcolor{BrickRed}{=}\ l\textcolor{BrickRed}{;} \\
\mbox{}\textcolor{Red}{\}} \\
\mbox{} \\
\mbox{}\textcolor{ForestGreen}{int}\ \textbf{\textcolor{Black}{main}}\textcolor{BrickRed}{(}\textcolor{ForestGreen}{int}\ argc\textcolor{BrickRed}{,}\ \textcolor{ForestGreen}{char}\textcolor{BrickRed}{*}\ argv\textcolor{BrickRed}{[])}\textcolor{Red}{\{} \\
\mbox{}\ \ \ \ \textbf{\textcolor{Blue}{if}}\ \textcolor{BrickRed}{(}argc\ \textcolor{BrickRed}{!=}\textcolor{Purple}{2}\textcolor{BrickRed}{)}\textcolor{Red}{\{} \\
\mbox{}\ \ \ \ \ \ \ \ cerr\ \textcolor{BrickRed}{\textless{}\textless{}}\ \texttt{\textcolor{Red}{"{}Uso\ del\ programa:\ "{}}}\ \textcolor{BrickRed}{+}\ \textcolor{BrickRed}{(}string\textcolor{BrickRed}{)(}argv\textcolor{BrickRed}{[}\textcolor{Purple}{0}\textcolor{BrickRed}{])}\ \textcolor{BrickRed}{+}\ \texttt{\textcolor{Red}{"{}\ \textless{}número\ positivo\textgreater{}"{}}}\ \textcolor{BrickRed}{\textless{}\textless{}}\ endl\textcolor{BrickRed}{;}\ \  \\
\mbox{}\ \ \ \ \ \ \ \ \textbf{\textcolor{Blue}{return}}\ \textcolor{BrickRed}{-}\textcolor{Purple}{1}\textcolor{BrickRed}{;} \\
\mbox{}\ \ \ \ \textcolor{Red}{\}} \\
\mbox{} \\
\mbox{}\ \ \ \ \textcolor{ForestGreen}{int}\ n\ \textcolor{BrickRed}{=}\ \textbf{\textcolor{Black}{atoi}}\textcolor{BrickRed}{(}argv\textcolor{BrickRed}{[}\textcolor{Purple}{1}\textcolor{BrickRed}{]);}\ \ \ \  \\
\mbox{}\ \ \ \ \textbf{\textcolor{Blue}{if}}\ \textcolor{BrickRed}{(}n\textcolor{BrickRed}{\textless{}}\textcolor{Purple}{0}\textcolor{BrickRed}{)}\ \textbf{\textcolor{Blue}{return}}\ \textcolor{BrickRed}{-}\textcolor{Purple}{1}\textcolor{BrickRed}{;} \\
\mbox{}\ \ \ \  \\
\mbox{}\ \ \ \ \textcolor{ForestGreen}{int}\ \textcolor{BrickRed}{*}\ T\ \textcolor{BrickRed}{=}\ \textbf{\textcolor{Blue}{new}}\ \textcolor{ForestGreen}{int}\textcolor{BrickRed}{[}n\textcolor{BrickRed}{];} \\
\mbox{}\ \ \ \ \textbf{\textcolor{Blue}{struct}}\ \textcolor{TealBlue}{timespec}\ t$\_$antes\textcolor{BrickRed}{,}\ t$\_$despues\textcolor{BrickRed}{;} \\
\mbox{}\ \ \ \  \\
\mbox{}\ \ \ \ \textbf{\textcolor{Black}{srandom}}\textcolor{BrickRed}{(}\textbf{\textcolor{Black}{time}}\textcolor{BrickRed}{(}\textcolor{Purple}{0}\textcolor{BrickRed}{));} \\
\mbox{}\ \ \ \  \\
\mbox{}\ \ \ \ \textbf{\textcolor{Blue}{for}}\ \textcolor{BrickRed}{(}\textcolor{ForestGreen}{int}\ i\textcolor{BrickRed}{=}\textcolor{Purple}{0}\textcolor{BrickRed}{;}\ i\textcolor{BrickRed}{\textless{}}n\textcolor{BrickRed}{;}\ \textcolor{BrickRed}{++}i\textcolor{BrickRed}{)}\textcolor{Red}{\{} \\
\mbox{}\ \ \ \ \ \ \ \ T\textcolor{BrickRed}{[}i\textcolor{BrickRed}{]}\ \textcolor{BrickRed}{=}\ \textbf{\textcolor{Black}{random}}\textcolor{BrickRed}{();} \\
\mbox{}\ \ \ \ \textcolor{Red}{\}} \\
\mbox{}\ \ \ \  \\
\mbox{}\ \ \ \ \textbf{\textcolor{Black}{clock$\_$gettime}}\textcolor{BrickRed}{(}CLOCK$\_$REALTIME\textcolor{BrickRed}{,\&}t$\_$antes\textcolor{BrickRed}{);} \\
\mbox{}\ \ \ \ \textbf{\textcolor{Black}{quicksort}}\ \textcolor{BrickRed}{(}T\textcolor{BrickRed}{,}n\textcolor{BrickRed}{);} \\
\mbox{}\ \ \ \ \textbf{\textcolor{Black}{clock$\_$gettime}}\textcolor{BrickRed}{(}CLOCK$\_$REALTIME\textcolor{BrickRed}{,\&}t$\_$despues\textcolor{BrickRed}{);} \\
\mbox{}\ \ \ \  \\
\mbox{}\ \ \ \ cout\textcolor{BrickRed}{.}\textbf{\textcolor{Black}{precision}}\textcolor{BrickRed}{(}\textcolor{Purple}{3}\textcolor{BrickRed}{);} \\
\mbox{}\ \ \ \ cout\ \textcolor{BrickRed}{\textless{}\textless{}}\ \textcolor{BrickRed}{(}\textcolor{ForestGreen}{double}\textcolor{BrickRed}{)}\ \textcolor{BrickRed}{(}t$\_$despues\textcolor{BrickRed}{.}tv$\_$sec\textcolor{BrickRed}{-}t$\_$antes\textcolor{BrickRed}{.}tv$\_$sec\textcolor{BrickRed}{)+} \\
\mbox{}\ \ \ \ \ \ \ \ \textcolor{BrickRed}{(}\textcolor{ForestGreen}{double}\textcolor{BrickRed}{)}\ \textcolor{BrickRed}{((}t$\_$despues\textcolor{BrickRed}{.}tv$\_$nsec\textcolor{BrickRed}{-}t$\_$antes\textcolor{BrickRed}{.}tv$\_$nsec\textcolor{BrickRed}{)/(}\textcolor{Purple}{1}\textcolor{BrickRed}{.}e\textcolor{BrickRed}{+}\textcolor{Purple}{9}\textcolor{BrickRed}{))}\ \textcolor{BrickRed}{\textless{}\textless{}}\ endl\textcolor{BrickRed}{;} \\
\mbox{} \\
\mbox{}\ \ \ \  \\
\mbox{}\ \ \ \ \textbf{\textcolor{Blue}{delete}}\ \textcolor{BrickRed}{[]}\ T\textcolor{BrickRed}{;} \\
\mbox{}\ \ \ \  \\
\mbox{}\ \ \ \ \textbf{\textcolor{Blue}{return}}\ \textcolor{Purple}{0}\textcolor{BrickRed}{;} \\
\mbox{}\textcolor{Red}{\}} \\
\mbox{}


\subsubsection{Sucesión de Fibonacci}
% Generator: GNU source-highlight, by Lorenzo Bettini, http://www.gnu.org/software/src-highlite
\noindent
\mbox{}\textit{\textcolor{Brown}{/**}} \\
\mbox{}\textit{\textcolor{Brown}{\ *\ }}\textcolor{ForestGreen}{@file}\textit{\textcolor{Brown}{\ Cálculo\ de\ la\ sucesión\ de\ Fibonacci}} \\
\mbox{}\textit{\textcolor{Brown}{\ */}} \\
\mbox{} \\
\mbox{}\textbf{\textcolor{RoyalBlue}{\#include}}\ \texttt{\textcolor{Red}{\textless{}iostream\textgreater{}}} \\
\mbox{}\textbf{\textcolor{RoyalBlue}{\#include}}\ \texttt{\textcolor{Red}{\textless{}ctime\textgreater{}}} \\
\mbox{}\textbf{\textcolor{RoyalBlue}{\#include}}\ \texttt{\textcolor{Red}{\textless{}cstdlib\textgreater{}}} \\
\mbox{}\textbf{\textcolor{Blue}{using}}\ \textbf{\textcolor{Blue}{namespace}}\ std\textcolor{BrickRed}{;} \\
\mbox{} \\
\mbox{}\textit{\textcolor{Brown}{/**}} \\
\mbox{}\textit{\textcolor{Brown}{\ *\ }}\textcolor{ForestGreen}{@brief}\textit{\textcolor{Brown}{\ Calcula\ el\ término\ n-ésimo\ de\ la\ sucesión\ de\ Fibonacci.}} \\
\mbox{}\textit{\textcolor{Brown}{\ *\ }}\textcolor{ForestGreen}{@param}\textit{\textcolor{Brown}{\ n:\ número\ de\ orden\ del\ término\ buscado.\ n\ \textgreater{}=\ 1.}} \\
\mbox{}\textit{\textcolor{Brown}{\ *\ }}\textcolor{ForestGreen}{@return}\textit{\textcolor{Brown}{:\ término\ n-ésimo\ de\ la\ sucesión\ de\ Fibonacci.}} \\
\mbox{}\textit{\textcolor{Brown}{\ */}} \\
\mbox{} \\
\mbox{}\textcolor{ForestGreen}{int}\ \textbf{\textcolor{Black}{fibo}}\textcolor{BrickRed}{(}\textcolor{ForestGreen}{int}\ n\textcolor{BrickRed}{)}\textcolor{Red}{\{} \\
\mbox{}\ \ \ \ \textbf{\textcolor{Blue}{if}}\ \textcolor{BrickRed}{(}n\ \textcolor{BrickRed}{\textless{}}\ \textcolor{Purple}{2}\textcolor{BrickRed}{)} \\
\mbox{}\ \ \ \ \ \ \ \ \textbf{\textcolor{Blue}{return}}\ \textcolor{Purple}{1}\textcolor{BrickRed}{;} \\
\mbox{}\ \ \ \ \textbf{\textcolor{Blue}{else}} \\
\mbox{}\ \ \ \ \ \ \ \ \textbf{\textcolor{Blue}{return}}\ \textbf{\textcolor{Black}{fibo}}\textcolor{BrickRed}{(}n\textcolor{BrickRed}{-}\textcolor{Purple}{1}\textcolor{BrickRed}{)}\ \textcolor{BrickRed}{+}\ \textbf{\textcolor{Black}{fibo}}\textcolor{BrickRed}{(}n\textcolor{BrickRed}{-}\textcolor{Purple}{2}\textcolor{BrickRed}{);} \\
\mbox{}\textcolor{Red}{\}} \\
\mbox{} \\
\mbox{}\textcolor{ForestGreen}{int}\ \textbf{\textcolor{Black}{main}}\textcolor{BrickRed}{(}\textcolor{ForestGreen}{int}\ argc\textcolor{BrickRed}{,}\ \textcolor{ForestGreen}{char}\textcolor{BrickRed}{*}\ argv\textcolor{BrickRed}{[])}\textcolor{Red}{\{} \\
\mbox{}\ \ \ \ \textbf{\textcolor{Blue}{if}}\ \textcolor{BrickRed}{(}argc\ \textcolor{BrickRed}{!=}\textcolor{Purple}{2}\textcolor{BrickRed}{)}\textcolor{Red}{\{} \\
\mbox{}\ \ \ \ \ \ \ \ cerr\ \textcolor{BrickRed}{\textless{}\textless{}}\ \texttt{\textcolor{Red}{"{}Uso\ del\ programa:\ "{}}}\ \textcolor{BrickRed}{+}\ \textcolor{BrickRed}{(}string\textcolor{BrickRed}{)(}argv\textcolor{BrickRed}{[}\textcolor{Purple}{0}\textcolor{BrickRed}{])}\ \textcolor{BrickRed}{+}\ \texttt{\textcolor{Red}{"{}\ \textless{}número\ positivo\textgreater{}"{}}}\ \textcolor{BrickRed}{\textless{}\textless{}}\ endl\textcolor{BrickRed}{;}\ \  \\
\mbox{}\ \ \ \ \ \ \ \ \textbf{\textcolor{Blue}{return}}\ \textcolor{BrickRed}{-}\textcolor{Purple}{1}\textcolor{BrickRed}{;} \\
\mbox{}\ \ \ \ \textcolor{Red}{\}} \\
\mbox{}\ \ \ \ \textcolor{ForestGreen}{int}\ n\ \textcolor{BrickRed}{=}\ \textbf{\textcolor{Black}{atoi}}\textcolor{BrickRed}{(}argv\textcolor{BrickRed}{[}\textcolor{Purple}{1}\textcolor{BrickRed}{]);}\ \ \ \  \\
\mbox{}\ \ \ \ \textbf{\textcolor{Blue}{if}}\ \textcolor{BrickRed}{(}n\textcolor{BrickRed}{\textless{}}\textcolor{Purple}{0}\textcolor{BrickRed}{)}\ \textbf{\textcolor{Blue}{return}}\ \textcolor{BrickRed}{-}\textcolor{Purple}{1}\textcolor{BrickRed}{;} \\
\mbox{} \\
\mbox{}\ \ \ \ \textbf{\textcolor{Blue}{struct}}\ \textcolor{TealBlue}{timespec}\ t$\_$antes\textcolor{BrickRed}{,}\ t$\_$despues\textcolor{BrickRed}{;} \\
\mbox{}\ \ \ \  \\
\mbox{}\ \ \ \ \textbf{\textcolor{Black}{clock$\_$gettime}}\textcolor{BrickRed}{(}CLOCK$\_$REALTIME\textcolor{BrickRed}{,\&}t$\_$antes\textcolor{BrickRed}{);} \\
\mbox{}\ \ \ \ \textbf{\textcolor{Black}{fibo}}\ \textcolor{BrickRed}{(}n\textcolor{BrickRed}{);} \\
\mbox{}\ \ \ \ \textbf{\textcolor{Black}{clock$\_$gettime}}\textcolor{BrickRed}{(}CLOCK$\_$REALTIME\textcolor{BrickRed}{,\&}t$\_$despues\textcolor{BrickRed}{);} \\
\mbox{}\ \ \ \  \\
\mbox{}\ \ \ \ cout\textcolor{BrickRed}{.}\textbf{\textcolor{Black}{precision}}\textcolor{BrickRed}{(}\textcolor{Purple}{3}\textcolor{BrickRed}{);} \\
\mbox{}\ \ \ \ cout\ \textcolor{BrickRed}{\textless{}\textless{}}\ \textcolor{BrickRed}{(}\textcolor{ForestGreen}{double}\textcolor{BrickRed}{)}\ \textcolor{BrickRed}{(}t$\_$despues\textcolor{BrickRed}{.}tv$\_$sec\textcolor{BrickRed}{-}t$\_$antes\textcolor{BrickRed}{.}tv$\_$sec\textcolor{BrickRed}{)+} \\
\mbox{}\ \ \ \ \ \ \ \ \textcolor{BrickRed}{(}\textcolor{ForestGreen}{double}\textcolor{BrickRed}{)}\ \textcolor{BrickRed}{((}t$\_$despues\textcolor{BrickRed}{.}tv$\_$nsec\textcolor{BrickRed}{-}t$\_$antes\textcolor{BrickRed}{.}tv$\_$nsec\textcolor{BrickRed}{)/(}\textcolor{Purple}{1}\textcolor{BrickRed}{.}e\textcolor{BrickRed}{+}\textcolor{Purple}{9}\textcolor{BrickRed}{))}\ \textcolor{BrickRed}{\textless{}\textless{}}\ endl\textcolor{BrickRed}{;} \\
\mbox{}\ \ \ \  \\
\mbox{}\ \ \ \ \textbf{\textcolor{Blue}{return}}\ \textcolor{Purple}{0}\textcolor{BrickRed}{;} \\
\mbox{}\textcolor{Red}{\}}


\subsubsection{Algoritmo de Floyd}
% Generator: GNU source-highlight, by Lorenzo Bettini, http://www.gnu.org/software/src-highlite
\noindent
\mbox{}\textit{\textcolor{Brown}{/**}} \\
\mbox{}\textit{\textcolor{Brown}{\ *\ }}\textcolor{ForestGreen}{@file}\textit{\textcolor{Brown}{\ Cálculo\ del\ coste\ de\ los\ caminos\ mínimos.\ Algoritmo\ de\ Floyd.}} \\
\mbox{}\textit{\textcolor{Brown}{\ */}} \\
\mbox{} \\
\mbox{} \\
\mbox{}\textbf{\textcolor{RoyalBlue}{\#include}}\ \texttt{\textcolor{Red}{\textless{}iostream\textgreater{}}} \\
\mbox{}\textbf{\textcolor{RoyalBlue}{\#include}}\ \texttt{\textcolor{Red}{\textless{}ctime\textgreater{}}} \\
\mbox{}\textbf{\textcolor{RoyalBlue}{\#include}}\ \texttt{\textcolor{Red}{\textless{}cstdlib\textgreater{}}} \\
\mbox{}\textbf{\textcolor{RoyalBlue}{\#include}}\ \texttt{\textcolor{Red}{\textless{}climits\textgreater{}}} \\
\mbox{}\textbf{\textcolor{RoyalBlue}{\#include}}\ \texttt{\textcolor{Red}{\textless{}cassert\textgreater{}}} \\
\mbox{}\textbf{\textcolor{RoyalBlue}{\#include}}\ \texttt{\textcolor{Red}{\textless{}cmath\textgreater{}}} \\
\mbox{}\textbf{\textcolor{Blue}{using}}\ \textbf{\textcolor{Blue}{namespace}}\ std\textcolor{BrickRed}{;} \\
\mbox{} \\
\mbox{}\textbf{\textcolor{Blue}{static}}\ \textcolor{ForestGreen}{int}\ \textbf{\textcolor{Blue}{const}}\ MAX$\_$LONG\ \ \textcolor{BrickRed}{=}\ \textcolor{Purple}{10}\textcolor{BrickRed}{;} \\
\mbox{} \\
\mbox{}\textit{\textcolor{Brown}{/**}} \\
\mbox{}\textit{\textcolor{Brown}{\ *\ }}\textcolor{ForestGreen}{@brief}\textit{\textcolor{Brown}{\ Reserva\ espacio\ en\ memoria\ dinámica\ para\ una\ matriz\ cuadrada.}} \\
\mbox{}\textit{\textcolor{Brown}{\ *\ }}\textcolor{ForestGreen}{@param}\textit{\textcolor{Brown}{\ dim:\ dimensión\ de\ la\ matriz.\ dim\ \textgreater{}\ 0.}} \\
\mbox{}\textit{\textcolor{Brown}{\ *\ }}\textcolor{ForestGreen}{@returns}\textit{\textcolor{Brown}{\ puntero\ a\ la\ zona\ de\ memoria\ reservada.}} \\
\mbox{}\textit{\textcolor{Brown}{\ */}} \\
\mbox{} \\
\mbox{}\textcolor{ForestGreen}{int}\ \textcolor{BrickRed}{**}\ \textbf{\textcolor{Black}{ReservaMatriz}}\textcolor{BrickRed}{(}\textcolor{ForestGreen}{int}\ dim\textcolor{BrickRed}{);} \\
\mbox{} \\
\mbox{}\textit{\textcolor{Brown}{/**}} \\
\mbox{}\textit{\textcolor{Brown}{\ *\ }}\textcolor{ForestGreen}{@brief}\textit{\textcolor{Brown}{\ Libera\ el\ espacio\ asignado\ a\ una\ matriz\ cuadrada.}} \\
\mbox{}\textit{\textcolor{Brown}{\ *\ }}\textcolor{ForestGreen}{@param}\textit{\textcolor{Brown}{\ M:\ puntero\ a\ la\ zona\ de\ memoria\ reservada.\ Es\ MODIFICADO.}} \\
\mbox{}\textit{\textcolor{Brown}{\ *\ }}\textcolor{ForestGreen}{@param}\textit{\textcolor{Brown}{\ dim:\ dimensión\ de\ la\ matriz.\ dim\ \textgreater{}\ 0.}} \\
\mbox{}\textit{\textcolor{Brown}{\ *\ }} \\
\mbox{}\textit{\textcolor{Brown}{\ *\ Liberar\ la\ zona\ memoria\ asignada\ a\ M\ y\ lo\ pone\ a\ NULL.}} \\
\mbox{}\textit{\textcolor{Brown}{\ */}} \\
\mbox{}\textcolor{ForestGreen}{void}\ \textbf{\textcolor{Black}{LiberaMatriz}}\textcolor{BrickRed}{(}\textcolor{ForestGreen}{int}\ \textcolor{BrickRed}{**}\ \textcolor{BrickRed}{\&}\ M\textcolor{BrickRed}{,}\ \textcolor{ForestGreen}{int}\ dim\textcolor{BrickRed}{);} \\
\mbox{} \\
\mbox{}\textit{\textcolor{Brown}{/**}} \\
\mbox{}\textit{\textcolor{Brown}{\ *\ }}\textcolor{ForestGreen}{@brief}\textit{\textcolor{Brown}{\ Rellena\ una\ matriz\ cuadrada\ con\ valores\ aleaotorias.}} \\
\mbox{}\textit{\textcolor{Brown}{\ *\ }}\textcolor{ForestGreen}{@param}\textit{\textcolor{Brown}{\ M:\ puntero\ a\ la\ zona\ de\ memoria\ reservada.\ Es\ MODIFICADO.}} \\
\mbox{}\textit{\textcolor{Brown}{\ *\ }}\textcolor{ForestGreen}{@param}\textit{\textcolor{Brown}{\ dim:\ dimensión\ de\ la\ matriz.\ dim\ \textgreater{}\ 0.}} \\
\mbox{}\textit{\textcolor{Brown}{\ *\ }} \\
\mbox{}\textit{\textcolor{Brown}{\ *\ Asigna\ un\ valor\ aleatorio\ entero\ de\ [0,\ MAX$\_$LONG\ -\ 1]\ a\ cada}} \\
\mbox{}\textit{\textcolor{Brown}{\ *\ elemento\ de\ la\ matriz\ M,\ salvo\ los\ de\ la\ diagonal\ principal}} \\
\mbox{}\textit{\textcolor{Brown}{\ *\ que\ quedan\ a\ 0..}} \\
\mbox{}\textit{\textcolor{Brown}{\ */}} \\
\mbox{} \\
\mbox{}\textcolor{ForestGreen}{void}\ \textbf{\textcolor{Black}{RellenaMatriz}}\textcolor{BrickRed}{(}\textcolor{ForestGreen}{int}\ \textcolor{BrickRed}{**}M\textcolor{BrickRed}{,}\ \textcolor{ForestGreen}{int}\ dim\textcolor{BrickRed}{);} \\
\mbox{}\  \\
\mbox{}\textit{\textcolor{Brown}{/**}} \\
\mbox{}\textit{\textcolor{Brown}{\ *\ }}\textcolor{ForestGreen}{@brief}\textit{\textcolor{Brown}{\ Cálculo\ de\ caminos\ mínimos.}} \\
\mbox{}\textit{\textcolor{Brown}{\ *\ }}\textcolor{ForestGreen}{@param}\textit{\textcolor{Brown}{\ M:\ Matriz\ de\ longitudes\ de\ los\ caminos.\ Es\ MODIFICADO.}} \\
\mbox{}\textit{\textcolor{Brown}{\ *\ }}\textcolor{ForestGreen}{@param}\textit{\textcolor{Brown}{\ dim:\ dimensión\ de\ la\ matriz.\ dim\ \textgreater{}\ 0.}} \\
\mbox{}\textit{\textcolor{Brown}{\ *\ }} \\
\mbox{}\textit{\textcolor{Brown}{\ *\ Calcula\ la\ longitud\ del\ camino\ mínimo\ entre\ cada\ par\ de\ nodos\ (i,j),}} \\
\mbox{}\textit{\textcolor{Brown}{\ *\ que\ se\ almacena\ en\ M[i][j].}} \\
\mbox{}\textit{\textcolor{Brown}{\ */}} \\
\mbox{} \\
\mbox{}\textcolor{ForestGreen}{void}\ \textbf{\textcolor{Black}{Floyd}}\textcolor{BrickRed}{(}\textcolor{ForestGreen}{int}\ \textcolor{BrickRed}{**}M\textcolor{BrickRed}{,}\ \textcolor{ForestGreen}{int}\ dim\textcolor{BrickRed}{);} \\
\mbox{} \\
\mbox{}\textit{\textcolor{Brown}{//\ Implementación\ de\ las\ funciones}} \\
\mbox{} \\
\mbox{}\textcolor{ForestGreen}{int}\ \textcolor{BrickRed}{**}\ \textbf{\textcolor{Black}{ReservaMatriz}}\textcolor{BrickRed}{(}\textcolor{ForestGreen}{int}\ dim\textcolor{BrickRed}{)}\textcolor{Red}{\{} \\
\mbox{}\ \ \ \ \textcolor{ForestGreen}{int}\ \textcolor{BrickRed}{**}M\textcolor{BrickRed}{;} \\
\mbox{}\ \ \ \ \textbf{\textcolor{Blue}{if}}\ \textcolor{BrickRed}{(}dim\ \ \textcolor{BrickRed}{\textless{}=}\ \textcolor{Purple}{0}\textcolor{BrickRed}{)} \\
\mbox{}\ \ \ \ \textcolor{Red}{\{} \\
\mbox{}\ \ \ \ \ \ \ \ cerr\textcolor{BrickRed}{\textless{}\textless{}}\ \texttt{\textcolor{Red}{"{}}}\texttt{\textcolor{CarnationPink}{\textbackslash{}n}}\texttt{\textcolor{Red}{\ ERROR:\ Dimension\ de\ la\ matriz\ debe\ ser\ mayor\ que\ 0"{}}}\ \textcolor{BrickRed}{\textless{}\textless{}}\ endl\textcolor{BrickRed}{;} \\
\mbox{}\ \ \ \ \ \ \ \ \textbf{\textcolor{Black}{exit}}\textcolor{BrickRed}{(}\textcolor{Purple}{1}\textcolor{BrickRed}{);} \\
\mbox{}\ \ \ \ \textcolor{Red}{\}} \\
\mbox{}\ \ \ \ M\ \textcolor{BrickRed}{=}\ \textbf{\textcolor{Blue}{new}}\ \textcolor{ForestGreen}{int}\textcolor{BrickRed}{*}\ \textcolor{BrickRed}{[}dim\textcolor{BrickRed}{];} \\
\mbox{}\ \ \ \ \textbf{\textcolor{Blue}{if}}\ \textcolor{BrickRed}{(}M\ \textcolor{BrickRed}{==}\ NULL\textcolor{BrickRed}{)} \\
\mbox{}\ \ \ \ \textcolor{Red}{\{} \\
\mbox{}\ \ \ \ \ \ \ \ cerr\ \textcolor{BrickRed}{\textless{}\textless{}}\ \texttt{\textcolor{Red}{"{}}}\texttt{\textcolor{CarnationPink}{\textbackslash{}n}}\texttt{\textcolor{Red}{\ ERROR:\ No\ puedo\ reservar\ memoria\ para\ un\ matriz\ de\ "{}}} \\
\mbox{}\ \ \ \ \ \ \ \ \textcolor{BrickRed}{\textless{}\textless{}}\ dim\ \textcolor{BrickRed}{\textless{}\textless{}}\ \texttt{\textcolor{Red}{"{}\ x\ "{}}}\ \textcolor{BrickRed}{\textless{}\textless{}}\ dim\ \textcolor{BrickRed}{\textless{}\textless{}}\ \texttt{\textcolor{Red}{"{}elementos"{}}}\ \textcolor{BrickRed}{\textless{}\textless{}}\ endl\textcolor{BrickRed}{;} \\
\mbox{}\ \ \ \ \ \ \ \ \textbf{\textcolor{Black}{exit}}\textcolor{BrickRed}{(}\textcolor{Purple}{1}\textcolor{BrickRed}{);} \\
\mbox{}\ \ \ \ \textcolor{Red}{\}} \\
\mbox{}\ \ \ \ \textbf{\textcolor{Blue}{for}}\ \textcolor{BrickRed}{(}\textcolor{ForestGreen}{int}\ i\ \textcolor{BrickRed}{=}\ \textcolor{Purple}{0}\textcolor{BrickRed}{;}\ i\ \textcolor{BrickRed}{\textless{}}\ dim\textcolor{BrickRed}{;}\ i\textcolor{BrickRed}{++)} \\
\mbox{}\ \ \ \ \textcolor{Red}{\{} \\
\mbox{}\ \ \ \ \ \ \ \ M\textcolor{BrickRed}{[}i\textcolor{BrickRed}{]=}\ \textbf{\textcolor{Blue}{new}}\ \textcolor{ForestGreen}{int}\ \textcolor{BrickRed}{[}dim\textcolor{BrickRed}{];} \\
\mbox{}\ \ \ \ \ \ \ \ \textbf{\textcolor{Blue}{if}}\ \textcolor{BrickRed}{(}M\textcolor{BrickRed}{[}i\textcolor{BrickRed}{]}\ \textcolor{BrickRed}{==}\ NULL\textcolor{BrickRed}{)} \\
\mbox{}\ \ \ \ \ \ \ \ \textcolor{Red}{\{} \\
\mbox{}\ \ \ \ \ \ \ \ \ \ \ \ cerr\ \textcolor{BrickRed}{\textless{}\textless{}}\ \texttt{\textcolor{Red}{"{}ERROR:\ No\ puedo\ reservar\ memoria\ para\ un\ matriz\ de\ "{}}} \\
\mbox{}\ \ \ \ \ \ \ \ \ \ \ \ \textcolor{BrickRed}{\textless{}\textless{}}\ dim\ \textcolor{BrickRed}{\textless{}\textless{}}\ \texttt{\textcolor{Red}{"{}\ x\ "{}}}\ \textcolor{BrickRed}{\textless{}\textless{}}\ dim\ \textcolor{BrickRed}{\textless{}\textless{}}\ endl\textcolor{BrickRed}{;} \\
\mbox{}\ \ \ \ \ \ \ \ \ \ \ \ \textbf{\textcolor{Blue}{for}}\ \textcolor{BrickRed}{(}\textcolor{ForestGreen}{int}\ j\ \textcolor{BrickRed}{=}\ \textcolor{Purple}{0}\textcolor{BrickRed}{;}\ j\ \textcolor{BrickRed}{\textless{}}\ i\textcolor{BrickRed}{;}\ j\textcolor{BrickRed}{++)} \\
\mbox{}\ \ \ \ \ \ \ \ \ \ \ \ \ \ \ \ \textbf{\textcolor{Blue}{delete}}\ \textcolor{BrickRed}{[]}\ M\textcolor{BrickRed}{[}i\textcolor{BrickRed}{];} \\
\mbox{}\ \ \ \ \ \ \ \ \ \ \ \ \textbf{\textcolor{Blue}{delete}}\ \textcolor{BrickRed}{[]}\ M\textcolor{BrickRed}{;} \\
\mbox{}\ \ \ \ \ \ \ \ \ \ \ \ \textbf{\textcolor{Black}{exit}}\textcolor{BrickRed}{(}\textcolor{Purple}{1}\textcolor{BrickRed}{);} \\
\mbox{}\ \ \ \ \ \ \ \ \textcolor{Red}{\}}\  \\
\mbox{}\ \ \ \ \textcolor{Red}{\}} \\
\mbox{}\ \ \ \ \textbf{\textcolor{Blue}{return}}\ M\textcolor{BrickRed}{;} \\
\mbox{}\textcolor{Red}{\}}\ \ \  \\
\mbox{} \\
\mbox{}\textcolor{ForestGreen}{void}\ \textbf{\textcolor{Black}{LiberaMatriz}}\textcolor{BrickRed}{(}\textcolor{ForestGreen}{int}\ \textcolor{BrickRed}{**}\ \textcolor{BrickRed}{\&}\ M\textcolor{BrickRed}{,}\ \textcolor{ForestGreen}{int}\ dim\textcolor{BrickRed}{)}\textcolor{Red}{\{} \\
\mbox{}\ \ \ \ \textbf{\textcolor{Blue}{for}}\ \textcolor{BrickRed}{(}\textcolor{ForestGreen}{int}\ i\ \textcolor{BrickRed}{=}\ \textcolor{Purple}{0}\textcolor{BrickRed}{;}\ i\ \textcolor{BrickRed}{\textless{}}\ dim\textcolor{BrickRed}{;}\ i\textcolor{BrickRed}{++)} \\
\mbox{}\ \ \ \ \ \ \ \ \textbf{\textcolor{Blue}{delete}}\ \textcolor{BrickRed}{[]}\ M\textcolor{BrickRed}{[}i\textcolor{BrickRed}{];} \\
\mbox{}\ \ \ \ \textbf{\textcolor{Blue}{delete}}\ \textcolor{BrickRed}{[]}\ M\textcolor{BrickRed}{;} \\
\mbox{}\ \ \ \ M\ \textcolor{BrickRed}{=}\ NULL\textcolor{BrickRed}{;} \\
\mbox{}\textcolor{Red}{\}}\ \ \ \ \ \ \  \\
\mbox{} \\
\mbox{}\textcolor{ForestGreen}{void}\ \textbf{\textcolor{Black}{RellenaMatriz}}\textcolor{BrickRed}{(}\textcolor{ForestGreen}{int}\ \textcolor{BrickRed}{**}M\textcolor{BrickRed}{,}\ \textcolor{ForestGreen}{int}\ dim\textcolor{BrickRed}{)}\textcolor{Red}{\{} \\
\mbox{}\ \ \ \ \textbf{\textcolor{Blue}{for}}\ \textcolor{BrickRed}{(}\textcolor{ForestGreen}{int}\ i\ \textcolor{BrickRed}{=}\ \textcolor{Purple}{0}\textcolor{BrickRed}{;}\ i\ \textcolor{BrickRed}{\textless{}}\ dim\textcolor{BrickRed}{;}\ i\textcolor{BrickRed}{++)} \\
\mbox{}\ \ \ \ \ \ \ \ \textbf{\textcolor{Blue}{for}}\ \textcolor{BrickRed}{(}\textcolor{ForestGreen}{int}\ j\ \textcolor{BrickRed}{=}\ \textcolor{Purple}{0}\textcolor{BrickRed}{;}\ j\ \textcolor{BrickRed}{\textless{}}\ dim\textcolor{BrickRed}{;}\ j\textcolor{BrickRed}{++)} \\
\mbox{}\ \ \ \ \ \ \ \ \ \ \ \ \textbf{\textcolor{Blue}{if}}\ \textcolor{BrickRed}{(}i\ \textcolor{BrickRed}{!=}\ j\textcolor{BrickRed}{)} \\
\mbox{}\ \ \ \ \ \ \ \ \ \ \ \ \ \ \ \ M\textcolor{BrickRed}{[}i\textcolor{BrickRed}{][}j\textcolor{BrickRed}{]=}\ \textcolor{BrickRed}{(}\textbf{\textcolor{Black}{rand}}\textcolor{BrickRed}{()}\ \textcolor{BrickRed}{\%}\ MAX$\_$LONG\textcolor{BrickRed}{);} \\
\mbox{}\textcolor{Red}{\}}\ \ \ \ \ \ \ \ \ \ \  \\
\mbox{}\ \ \ \  \\
\mbox{}\textcolor{ForestGreen}{void}\ \textbf{\textcolor{Black}{Floyd}}\textcolor{BrickRed}{(}\textcolor{ForestGreen}{int}\ \textcolor{BrickRed}{**}M\textcolor{BrickRed}{,}\ \textcolor{ForestGreen}{int}\ dim\textcolor{BrickRed}{)}\textcolor{Red}{\{} \\
\mbox{}\ \ \ \ \textbf{\textcolor{Blue}{for}}\ \textcolor{BrickRed}{(}\textcolor{ForestGreen}{int}\ k\ \textcolor{BrickRed}{=}\ \textcolor{Purple}{0}\textcolor{BrickRed}{;}\ k\ \textcolor{BrickRed}{\textless{}}\ dim\textcolor{BrickRed}{;}\ k\textcolor{BrickRed}{++)} \\
\mbox{}\ \ \ \ \ \ \ \ \textbf{\textcolor{Blue}{for}}\ \textcolor{BrickRed}{(}\textcolor{ForestGreen}{int}\ i\ \textcolor{BrickRed}{=}\ \textcolor{Purple}{0}\textcolor{BrickRed}{;}\ i\ \textcolor{BrickRed}{\textless{}}\ dim\textcolor{BrickRed}{;}i\textcolor{BrickRed}{++)} \\
\mbox{}\ \ \ \ \ \ \ \ \ \ \ \ \textbf{\textcolor{Blue}{for}}\ \textcolor{BrickRed}{(}\textcolor{ForestGreen}{int}\ j\ \textcolor{BrickRed}{=}\ \textcolor{Purple}{0}\textcolor{BrickRed}{;}\ j\ \textcolor{BrickRed}{\textless{}}\ dim\textcolor{BrickRed}{;}j\textcolor{BrickRed}{++)} \\
\mbox{}\ \ \ \ \ \ \ \ \ \ \ \ \textcolor{Red}{\{} \\
\mbox{}\ \ \ \ \ \ \ \ \ \ \ \ \ \ \ \ \textcolor{ForestGreen}{int}\ sum\ \textcolor{BrickRed}{=}\ M\textcolor{BrickRed}{[}i\textcolor{BrickRed}{][}k\textcolor{BrickRed}{]}\ \textcolor{BrickRed}{+}\ M\textcolor{BrickRed}{[}k\textcolor{BrickRed}{][}j\textcolor{BrickRed}{];}\ \ \ \ \ \ \ \  \\
\mbox{}\ \ \ \ \ \ \ \ \ \ \ \ \ \ \ \ M\textcolor{BrickRed}{[}i\textcolor{BrickRed}{][}j\textcolor{BrickRed}{]}\ \textcolor{BrickRed}{=}\ \textcolor{BrickRed}{(}M\textcolor{BrickRed}{[}i\textcolor{BrickRed}{][}j\textcolor{BrickRed}{]}\ \textcolor{BrickRed}{\textgreater{}}\ sum\textcolor{BrickRed}{)}\ \textcolor{BrickRed}{?}\ sum\ \textcolor{BrickRed}{:}\ M\textcolor{BrickRed}{[}i\textcolor{BrickRed}{][}j\textcolor{BrickRed}{];} \\
\mbox{}\ \ \ \ \ \ \ \ \ \ \ \ \textcolor{Red}{\}} \\
\mbox{}\textcolor{Red}{\}}\ \ \ \ \ \ \ \ \ \ \  \\
\mbox{} \\
\mbox{}\textcolor{ForestGreen}{int}\ \textbf{\textcolor{Black}{main}}\ \textcolor{BrickRed}{(}\textcolor{ForestGreen}{int}\ argc\textcolor{BrickRed}{,}\ \textcolor{ForestGreen}{char}\ \textcolor{BrickRed}{*}argv\textcolor{BrickRed}{[])}\textcolor{Red}{\{} \\
\mbox{}\ \ \ \ \textbf{\textcolor{Blue}{struct}}\ \textcolor{TealBlue}{timespec}\ t$\_$antes\textcolor{BrickRed}{,}\ t$\_$despues\textcolor{BrickRed}{;} \\
\mbox{}\ \ \ \ \textcolor{ForestGreen}{int}\ dim\textcolor{BrickRed}{;}\ \ \ \ \ \ \ \ \ \ \ \ \textit{\textcolor{Brown}{//\ Dimensión\ de\ la\ matriz}} \\
\mbox{}\ \ \ \  \\
\mbox{}\ \ \ \ \textit{\textcolor{Brown}{//Lectura\ de\ los\ parametros\ de\ entrada}} \\
\mbox{}\ \ \ \ \textbf{\textcolor{Blue}{if}}\ \textcolor{BrickRed}{(}argc\ \textcolor{BrickRed}{!=}\textcolor{Purple}{2}\textcolor{BrickRed}{)}\textcolor{Red}{\{} \\
\mbox{}\ \ \ \ \ \ \ \ cerr\ \textcolor{BrickRed}{\textless{}\textless{}}\ \texttt{\textcolor{Red}{"{}Uso\ del\ programa:\ "{}}}\ \textcolor{BrickRed}{+}\ \textcolor{BrickRed}{(}string\textcolor{BrickRed}{)(}argv\textcolor{BrickRed}{[}\textcolor{Purple}{0}\textcolor{BrickRed}{])}\ \textcolor{BrickRed}{+}\ \texttt{\textcolor{Red}{"{}\ \textless{}número\ positivo\textgreater{}"{}}}\ \textcolor{BrickRed}{\textless{}\textless{}}\ endl\textcolor{BrickRed}{;}\ \  \\
\mbox{}\ \ \ \ \ \ \ \ \textbf{\textcolor{Blue}{return}}\ \textcolor{BrickRed}{-}\textcolor{Purple}{1}\textcolor{BrickRed}{;} \\
\mbox{}\ \ \ \ \textcolor{Red}{\}} \\
\mbox{}\ \ \ \  \\
\mbox{}\ \ \ \ dim\ \textcolor{BrickRed}{=}\ \textbf{\textcolor{Black}{atoi}}\textcolor{BrickRed}{(}argv\textcolor{BrickRed}{[}\textcolor{Purple}{1}\textcolor{BrickRed}{]);} \\
\mbox{}\ \ \ \ \textbf{\textcolor{Blue}{if}}\ \textcolor{BrickRed}{(}dim\textcolor{BrickRed}{\textless{}}\textcolor{Purple}{0}\textcolor{BrickRed}{)}\ \textbf{\textcolor{Blue}{return}}\ \textcolor{BrickRed}{-}\textcolor{Purple}{1}\textcolor{BrickRed}{;} \\
\mbox{}\ \ \ \ \textcolor{ForestGreen}{int}\ \textcolor{BrickRed}{**}\ M\ \textcolor{BrickRed}{=}\ \textbf{\textcolor{Black}{ReservaMatriz}}\textcolor{BrickRed}{(}dim\textcolor{BrickRed}{);} \\
\mbox{}\ \ \ \  \\
\mbox{}\ \ \ \ \textbf{\textcolor{Black}{RellenaMatriz}}\textcolor{BrickRed}{(}M\textcolor{BrickRed}{,}dim\textcolor{BrickRed}{);} \\
\mbox{}\ \ \ \  \\
\mbox{}\ \ \ \  \\
\mbox{}\ \ \ \ \textit{\textcolor{Brown}{//\ Empieza\ el\ algoritmo\ de\ floyd}} \\
\mbox{}\ \ \ \  \\
\mbox{}\ \ \ \ \textbf{\textcolor{Black}{clock$\_$gettime}}\textcolor{BrickRed}{(}CLOCK$\_$REALTIME\textcolor{BrickRed}{,\&}t$\_$antes\textcolor{BrickRed}{);} \\
\mbox{}\ \ \ \ \textbf{\textcolor{Black}{Floyd}}\ \textcolor{BrickRed}{(}M\textcolor{BrickRed}{,}dim\textcolor{BrickRed}{);} \\
\mbox{}\ \ \ \ \textbf{\textcolor{Black}{clock$\_$gettime}}\textcolor{BrickRed}{(}CLOCK$\_$REALTIME\textcolor{BrickRed}{,\&}t$\_$despues\textcolor{BrickRed}{);} \\
\mbox{}\ \ \ \  \\
\mbox{}\ \ \ \ cout\textcolor{BrickRed}{.}\textbf{\textcolor{Black}{precision}}\textcolor{BrickRed}{(}\textcolor{Purple}{3}\textcolor{BrickRed}{);} \\
\mbox{}\ \ \ \ cout\ \textcolor{BrickRed}{\textless{}\textless{}}\ \textcolor{BrickRed}{(}\textcolor{ForestGreen}{double}\textcolor{BrickRed}{)}\ \textcolor{BrickRed}{(}t$\_$despues\textcolor{BrickRed}{.}tv$\_$sec\textcolor{BrickRed}{-}t$\_$antes\textcolor{BrickRed}{.}tv$\_$sec\textcolor{BrickRed}{)+} \\
\mbox{}\ \ \ \ \ \ \ \ \textcolor{BrickRed}{(}\textcolor{ForestGreen}{double}\textcolor{BrickRed}{)}\ \textcolor{BrickRed}{((}t$\_$despues\textcolor{BrickRed}{.}tv$\_$nsec\textcolor{BrickRed}{-}t$\_$antes\textcolor{BrickRed}{.}tv$\_$nsec\textcolor{BrickRed}{)/(}\textcolor{Purple}{1}\textcolor{BrickRed}{.}e\textcolor{BrickRed}{+}\textcolor{Purple}{9}\textcolor{BrickRed}{))}\ \textcolor{BrickRed}{\textless{}\textless{}}\ endl\textcolor{BrickRed}{;} \\
\mbox{} \\
\mbox{}\ \ \ \ \textbf{\textcolor{Black}{LiberaMatriz}}\textcolor{BrickRed}{(}M\textcolor{BrickRed}{,}dim\textcolor{BrickRed}{);} \\
\mbox{}\ \ \ \  \\
\mbox{}\ \ \ \ \textbf{\textcolor{Blue}{return}}\ \textcolor{Purple}{0}\textcolor{BrickRed}{;} \\
\mbox{}\textcolor{Red}{\}}\ \ \  \\
\mbox{} \\
\mbox{}


\subsubsection{Torres de Hanoi}
% Generator: GNU source-highlight, by Lorenzo Bettini, http://www.gnu.org/software/src-highlite
\noindent
\mbox{}\textit{\textcolor{Brown}{/**}} \\
\mbox{}\textit{\textcolor{Brown}{\ *\ }}\textcolor{ForestGreen}{@file}\textit{\textcolor{Brown}{\ Resolución\ del\ problema\ de\ las\ Torres\ de\ Hanoi}} \\
\mbox{}\textit{\textcolor{Brown}{\ */}} \\
\mbox{} \\
\mbox{}\textbf{\textcolor{RoyalBlue}{\#include}}\ \texttt{\textcolor{Red}{\textless{}iostream\textgreater{}}} \\
\mbox{}\textbf{\textcolor{RoyalBlue}{\#include}}\ \texttt{\textcolor{Red}{\textless{}cstdlib\textgreater{}}} \\
\mbox{}\textbf{\textcolor{RoyalBlue}{\#include}}\ \texttt{\textcolor{Red}{\textless{}ctime\textgreater{}}} \\
\mbox{}\textbf{\textcolor{Blue}{using}}\ \textbf{\textcolor{Blue}{namespace}}\ std\textcolor{BrickRed}{;} \\
\mbox{} \\
\mbox{}\textit{\textcolor{Brown}{/**}} \\
\mbox{}\textit{\textcolor{Brown}{\ *\ }}\textcolor{ForestGreen}{@brief}\textit{\textcolor{Brown}{\ Resuelve\ el\ problema\ de\ las\ Torres\ de\ Hanoi}} \\
\mbox{}\textit{\textcolor{Brown}{\ *\ }}\textcolor{ForestGreen}{@param}\textit{\textcolor{Brown}{\ M:\ número\ de\ discos.\ M\ \textgreater{}\ 1.}} \\
\mbox{}\textit{\textcolor{Brown}{\ *\ }}\textcolor{ForestGreen}{@param}\textit{\textcolor{Brown}{\ i:\ número\ de\ columna\ en\ que\ están\ los\ discos.}} \\
\mbox{}\textit{\textcolor{Brown}{\ *\ i\ es\ un\ valor\ de\ \{1,\ 2,\ 3\}.\ i\ !=\ j.}} \\
\mbox{}\textit{\textcolor{Brown}{\ *\ }}\textcolor{ForestGreen}{@param}\textit{\textcolor{Brown}{\ j:\ número\ de\ columna\ a\ que\ se\ llevan\ los\ discos.}} \\
\mbox{}\textit{\textcolor{Brown}{\ *\ j\ es\ un\ valor\ de\ \{1,\ 2,\ 3\}.\ j\ !=\ i.}} \\
\mbox{}\textit{\textcolor{Brown}{\ *\ }} \\
\mbox{}\textit{\textcolor{Brown}{\ *\ Esta\ función\ imprime\ en\ la\ salida\ estándar\ la\ secuencia\ de}} \\
\mbox{}\textit{\textcolor{Brown}{\ *\ movimientos\ necesarios\ para\ desplazar\ los\ M\ discos\ de\ la}} \\
\mbox{}\textit{\textcolor{Brown}{\ *\ columna\ i\ a\ la\ j,\ observando\ la\ restricción\ de\ que\ ningún}} \\
\mbox{}\textit{\textcolor{Brown}{\ *\ disco\ se\ puede\ situar\ sobre\ otro\ de\ tamaño\ menor.\ Utiliza}} \\
\mbox{}\textit{\textcolor{Brown}{\ *\ una\ única\ columna\ auxiliar.}} \\
\mbox{}\textit{\textcolor{Brown}{\ */}} \\
\mbox{} \\
\mbox{}\textcolor{ForestGreen}{void}\ \textbf{\textcolor{Black}{hanoi}}\ \textcolor{BrickRed}{(}\textcolor{ForestGreen}{int}\ M\textcolor{BrickRed}{,}\ \textcolor{ForestGreen}{int}\ i\textcolor{BrickRed}{,}\ \textcolor{ForestGreen}{int}\ j\textcolor{BrickRed}{);} \\
\mbox{} \\
\mbox{}\textcolor{ForestGreen}{void}\ \textbf{\textcolor{Black}{hanoi}}\ \textcolor{BrickRed}{(}\textcolor{ForestGreen}{int}\ M\textcolor{BrickRed}{,}\ \textcolor{ForestGreen}{int}\ i\textcolor{BrickRed}{,}\ \textcolor{ForestGreen}{int}\ j\textcolor{BrickRed}{)}\textcolor{Red}{\{} \\
\mbox{}\ \ \ \ \textbf{\textcolor{Blue}{if}}\ \textcolor{BrickRed}{(}M\ \textcolor{BrickRed}{\textgreater{}}\ \textcolor{Purple}{0}\textcolor{BrickRed}{)}\textcolor{Red}{\{} \\
\mbox{}\ \ \ \ \ \ \ \ \textbf{\textcolor{Black}{hanoi}}\textcolor{BrickRed}{(}M\textcolor{BrickRed}{-}\textcolor{Purple}{1}\textcolor{BrickRed}{,}\ i\textcolor{BrickRed}{,}\ \textcolor{Purple}{6}\textcolor{BrickRed}{-}i\textcolor{BrickRed}{-}j\textcolor{BrickRed}{);} \\
\mbox{}\ \ \ \ \ \ \ \ \textit{\textcolor{Brown}{//cout\ \textless{}\textless{}\ i\ \textless{}\textless{}\ "{}\ -\textgreater{}\ "{}\ \textless{}\textless{}\ j\ \textless{}\textless{}\ endl;}} \\
\mbox{}\ \ \ \ \ \ \ \ \textbf{\textcolor{Black}{hanoi}}\ \textcolor{BrickRed}{(}M\textcolor{BrickRed}{-}\textcolor{Purple}{1}\textcolor{BrickRed}{,}\ \textcolor{Purple}{6}\textcolor{BrickRed}{-}i\textcolor{BrickRed}{-}j\textcolor{BrickRed}{,}\ j\textcolor{BrickRed}{);} \\
\mbox{}\ \ \ \ \textcolor{Red}{\}} \\
\mbox{}\textcolor{Red}{\}} \\
\mbox{} \\
\mbox{}\textcolor{ForestGreen}{int}\ \textbf{\textcolor{Black}{main}}\textcolor{BrickRed}{(}\textcolor{ForestGreen}{int}\ argc\textcolor{BrickRed}{,}\ \textcolor{ForestGreen}{char}\textcolor{BrickRed}{*}\ argv\textcolor{BrickRed}{[])}\textcolor{Red}{\{} \\
\mbox{}\ \ \ \ \textbf{\textcolor{Blue}{if}}\ \textcolor{BrickRed}{(}argc\ \textcolor{BrickRed}{!=}\textcolor{Purple}{2}\textcolor{BrickRed}{)}\textcolor{Red}{\{} \\
\mbox{}\ \ \ \ \ \ \ \ cerr\ \textcolor{BrickRed}{\textless{}\textless{}}\ \texttt{\textcolor{Red}{"{}Uso\ del\ programa:\ "{}}}\ \textcolor{BrickRed}{+}\ \textcolor{BrickRed}{(}string\textcolor{BrickRed}{)(}argv\textcolor{BrickRed}{[}\textcolor{Purple}{0}\textcolor{BrickRed}{])}\ \textcolor{BrickRed}{+}\ \texttt{\textcolor{Red}{"{}\ \textless{}número\ positivo\textgreater{}"{}}}\ \textcolor{BrickRed}{\textless{}\textless{}}\ endl\textcolor{BrickRed}{;}\ \  \\
\mbox{}\ \ \ \ \ \ \ \ \textbf{\textcolor{Blue}{return}}\ \textcolor{BrickRed}{-}\textcolor{Purple}{1}\textcolor{BrickRed}{;} \\
\mbox{}\ \ \ \ \textcolor{Red}{\}} \\
\mbox{}\ \ \ \ \textcolor{ForestGreen}{int}\ n\ \textcolor{BrickRed}{=}\ \textbf{\textcolor{Black}{atoi}}\textcolor{BrickRed}{(}argv\textcolor{BrickRed}{[}\textcolor{Purple}{1}\textcolor{BrickRed}{]);}\ \ \ \  \\
\mbox{}\ \ \ \ \textbf{\textcolor{Blue}{if}}\ \textcolor{BrickRed}{(}n\textcolor{BrickRed}{\textless{}}\textcolor{Purple}{0}\textcolor{BrickRed}{)}\ \textbf{\textcolor{Blue}{return}}\ \textcolor{BrickRed}{-}\textcolor{Purple}{1}\textcolor{BrickRed}{;} \\
\mbox{} \\
\mbox{}\ \ \ \ \textbf{\textcolor{Blue}{struct}}\ \textcolor{TealBlue}{timespec}\ t$\_$antes\textcolor{BrickRed}{,}\ t$\_$despues\textcolor{BrickRed}{;} \\
\mbox{}\ \ \ \  \\
\mbox{}\ \ \ \ \textbf{\textcolor{Black}{clock$\_$gettime}}\textcolor{BrickRed}{(}CLOCK$\_$REALTIME\textcolor{BrickRed}{,\&}t$\_$antes\textcolor{BrickRed}{);} \\
\mbox{}\ \ \ \ \textbf{\textcolor{Black}{hanoi}}\ \textcolor{BrickRed}{(}n\textcolor{BrickRed}{,}\textcolor{Purple}{1}\textcolor{BrickRed}{,}\textcolor{Purple}{2}\textcolor{BrickRed}{);} \\
\mbox{}\ \ \ \ \textbf{\textcolor{Black}{clock$\_$gettime}}\textcolor{BrickRed}{(}CLOCK$\_$REALTIME\textcolor{BrickRed}{,\&}t$\_$despues\textcolor{BrickRed}{);} \\
\mbox{}\ \ \ \  \\
\mbox{}\ \ \ \ cout\textcolor{BrickRed}{.}\textbf{\textcolor{Black}{precision}}\textcolor{BrickRed}{(}\textcolor{Purple}{3}\textcolor{BrickRed}{);} \\
\mbox{}\ \ \ \ cout\ \textcolor{BrickRed}{\textless{}\textless{}}\ \textcolor{BrickRed}{(}\textcolor{ForestGreen}{double}\textcolor{BrickRed}{)}\ \textcolor{BrickRed}{(}t$\_$despues\textcolor{BrickRed}{.}tv$\_$sec\textcolor{BrickRed}{-}t$\_$antes\textcolor{BrickRed}{.}tv$\_$sec\textcolor{BrickRed}{)+} \\
\mbox{}\ \ \ \ \ \ \ \ \textcolor{BrickRed}{(}\textcolor{ForestGreen}{double}\textcolor{BrickRed}{)}\ \textcolor{BrickRed}{((}t$\_$despues\textcolor{BrickRed}{.}tv$\_$nsec\textcolor{BrickRed}{-}t$\_$antes\textcolor{BrickRed}{.}tv$\_$nsec\textcolor{BrickRed}{)/(}\textcolor{Purple}{1}\textcolor{BrickRed}{.}e\textcolor{BrickRed}{+}\textcolor{Purple}{9}\textcolor{BrickRed}{))}\ \textcolor{BrickRed}{\textless{}\textless{}}\ endl\textcolor{BrickRed}{;} \\
\mbox{}\ \ \ \  \\
\mbox{}\ \ \ \ \textbf{\textcolor{Blue}{return}}\ \textcolor{Purple}{0}\textcolor{BrickRed}{;} \\
\mbox{}\textcolor{Red}{\}}


\end{allintypewriter}
}


\normalsize

\subsection{Guion generador de datos y gráficos}
\label{gengraf}
Para automatizar algunas de las labores que se nos pedían en la práctica
y facilitar la obtención de datos en distintos equipos, se ha confeccionado
un script, cuyo código se incluye a continuación, y que se ha llamado desde
el \texttt{makefile} para cada algoritmo.

\scriptsize\texttt{% Generator: GNU source-highlight, by Lorenzo Bettini, http://www.gnu.org/software/src-highlite
\noindent
\mbox{}\textit{\textcolor{Brown}{\#!/bin/bash}} \\
\mbox{} \\
\mbox{}\textit{\textcolor{Brown}{\#\ Variables\ de\ ejecución}} \\
\mbox{}\textcolor{ForestGreen}{SCRIPT}\textcolor{BrickRed}{=}plot \\
\mbox{}\textit{\textcolor{Brown}{\#\ Límite\ e\ incremento\ de\ tamaño.}} \\
\mbox{}\textcolor{ForestGreen}{MAP}\textcolor{BrickRed}{=}\texttt{\textcolor{Red}{"{}burbuja\ 10000\ 200}} \\
\mbox{}\texttt{\textcolor{Red}{fibonacci\ 50\ 1}} \\
\mbox{}\texttt{\textcolor{Red}{floyd\ 1000\ 5}} \\
\mbox{}\texttt{\textcolor{Red}{hanoi\ 35\ 1}} \\
\mbox{}\texttt{\textcolor{Red}{heapsort\ 10000\ 100}} \\
\mbox{}\texttt{\textcolor{Red}{insercion\ 10000\ 200}} \\
\mbox{}\texttt{\textcolor{Red}{mergesort\ 10000\ 100}} \\
\mbox{}\texttt{\textcolor{Red}{quicksort\ 10000\ 100}} \\
\mbox{}\texttt{\textcolor{Red}{seleccion\ 10000\ 200"{}}} \\
\mbox{}\textit{\textcolor{Brown}{\#\ Constante\ del\ número\ áureo}} \\
\mbox{}\textcolor{ForestGreen}{aur}\textcolor{BrickRed}{=}`echo\ \texttt{\textcolor{Red}{"{}(1+sqrt(5))/2"{}}}\ \textcolor{BrickRed}{$|$}\ bc\ -l` \\
\mbox{}\textit{\textcolor{Brown}{\#\ Funciones\ de\ ajuste\ posibles}} \\
<<<<<<< HEAD
\mbox{}\textcolor{ForestGreen}{FUNCS}\textcolor{BrickRed}{=(}\ \texttt{\textcolor{Red}{"{}a0\ \ \ \ \ \ \ \ \ \ \ \ \ \ \ \ \ \ \ \ \ \ \ \ \ \ \ \ \ \ \ \ \ \ \ \ \ \ \ \ \ \ \ \ \ \ \ \ \ \ \ \ \ a0"{}}} \\
=======
\mbox{}\textcolor{ForestGreen}{FUNCS}\textcolor{BrickRed}{=(}\  \\
\mbox{}\ \ \ \ \ \ \ \ \texttt{\textcolor{Red}{"{}a0\ \ \ \ \ \ \ \ \ \ \ \ \ \ \ \ \ \ \ \ \ \ \ \ \ \ \ \ \ \ \ \ \ \ \ \ \ \ \ \ \ \ \ \ \ \ \ \ \ \ \ \ \ a0"{}}} \\
>>>>>>> 0fc97cf1a4e91ecc50f541e4896dd02722c4e6c8
\mbox{}\ \ \ \ \ \ \ \ \texttt{\textcolor{Red}{"{}a0*x+a1\ \ \ \ \ \ \ \ \ \ \ \ \ \ \ \ \ \ \ \ \ \ \ \ \ \ \ \ \ \ \ \ \ \ \ \ \ \ \ \ \ \ \ \ \ \ \ \ a0,a1"{}}} \\
\mbox{}\ \ \ \ \ \ \ \ \texttt{\textcolor{Red}{"{}a0*(x**2)+a1*x+a0\ \ \ \ \ \ \ \ \ \ \ \ \ \ \ \ \ \ \ \ \ \ \ \ \ \ \ \ \ \ \ \ \ \ \ \ \ \ a0,a1,a2"{}}}\ \ \ \ \ \  \\
\mbox{}\ \ \ \ \ \ \ \ \texttt{\textcolor{Red}{"{}a0*(x**3)+a1*(x**2)+a2*x+a3\ \ \ \ \ \ \ \ \ \ \ \ \ \ \ \ \ \ \ \ \ \ \ \ \ \ \ \ a0,a1,a2,a3"{}}} \\
\mbox{}\ \ \ \ \ \ \ \ \texttt{\textcolor{Red}{"{}a0*x*log(x)+a1*x+a2*log(x)+a3\ \ \ \ \ \ \ \ \ \ \ \ \ \ \ \ \ \ \ \ \ \ \ \ \ \ a0,a1,a2,a3"{}}} \\
\mbox{}\ \ \ \ \ \ \ \ \texttt{\textcolor{Red}{"{}a0*(2**x)+a1\ \ \ \ \ \ \ \ \ \ \ \ \ \ \ \ \ \ \ \ \ \ \ \ \ \ \ \ \ \ \ \ \ \ \ \ \ \ \ \ \ \ \ a0,a1"{}}} \\
\mbox{}\ \ \ \ \ \ \ \ \texttt{\textcolor{Red}{"{}a0*((\$aur)**x)+a1*((1/\$aur)**x)+a2*(x**2)+a3*x+a4\ \ \ \ \ \ a0,a1,a2,a3,a4"{}}} \\
\mbox{}\ \ \ \ \ \ \ \ \texttt{\textcolor{Red}{"{}a0*exp(a1*x)+a2\ \ \ \ \ \ \ \ \ \ \ \ \ \ \ \ \ \ \ \ \ \ \ \ \ \ \ \ \ \ \ \ \ \ \ \ \ \ \ \ a0,a1,a2"{}}} \\
\mbox{}\ \ \ \ \ \ \ \ \texttt{\textcolor{Red}{"{}a0*exp(a2*x)+a2*x+a3\ \ \ \ \ \ \ \ \ \ \ \ \ \ \ \ \ \ \ \ \ \ \ \ \ \ \ \ \ \ \ \ \ \ \ a0,a1,a2,a3"{}}} \\
\mbox{}\ \ \ \ \ \ \ \ \texttt{\textcolor{Red}{"{}a0*exp(a1*x)+a2*(x**2)+a3*x+a4\ \ \ \ \ \ \ \ \ \ \ \ \ \ \ \ \ \ \ \ \ \ \ \ \ a0,a1,a3,a4"{}}}\  \\
\mbox{}\ \ \ \ \ \ \ \textcolor{BrickRed}{)} \\
\mbox{}\textit{\textcolor{Brown}{\#\ Número\ de\ ejecuciones\ con\ las\ que\ se\ obtendrá\ el\ promedio}} \\
\mbox{}\textcolor{ForestGreen}{N$\_$ITER}\textcolor{BrickRed}{=}\textcolor{Purple}{5} \\
\mbox{} \\
\mbox{} \\
\mbox{} \\
\mbox{}\textit{\textcolor{Brown}{\#\ param\ \$1\ "{}programa\ a\ ejecutar"{}}} \\
\mbox{}\textit{\textcolor{Brown}{\#}} \\
\mbox{}\textit{\textcolor{Brown}{\#\ Genera\ un\ archivo\ de\ datos\ donde\ la\ primera\ columna\ correponde\ a\ tallas\ del\ problema,}} \\
\mbox{}\textit{\textcolor{Brown}{\#\ y\ la\ segunda\ a\ tiempo\ que\ necesita\ el\ algoritmo\ para\ resolver\ un\ problema\ de\ esa}} \\
\mbox{}\textit{\textcolor{Brown}{\#\ talla}} \\
\mbox{}\textit{\textcolor{Brown}{\#}} \\
\mbox{}\textbf{\textcolor{Black}{function\ gendata()}}\ \{ \\
\mbox{}\ \ \ \ \textit{\textcolor{Brown}{\#\ Se\ extraen\ los\ valores\ de\ talla\ máxima\ del\ problema\ e\ incremento\ del\ número}} \\
\mbox{}\ \ \ \ \textit{\textcolor{Brown}{\#\ de\ datos\ que\ se\ introducirán\ al\ algoritmo,\ así\ como\ la\ talla\ inicial}} \\
\mbox{}\ \ \ \ \textcolor{ForestGreen}{lim}\textcolor{BrickRed}{=}`echo\ \textcolor{ForestGreen}{\$MAP}\ \textcolor{BrickRed}{$|$}\ grep\ -o\ \texttt{\textcolor{Red}{"{}\$\{1\#\#*/\}\ [[:digit:]]*\ [[:digit:]]*"{}}}\textcolor{BrickRed}{$|$}\ cut\ -f\textcolor{Purple}{2}\ -d\texttt{\textcolor{Red}{"{}\ "{}}}` \\
\mbox{}\ \ \ \ \textcolor{ForestGreen}{inc}\textcolor{BrickRed}{=}`echo\ \textcolor{ForestGreen}{\$MAP}\ \textcolor{BrickRed}{$|$}\ grep\ -o\ \texttt{\textcolor{Red}{"{}\$\{1\#\#*/\}\ [[:digit:]]*\ [[:digit:]]*"{}}}\ \textcolor{BrickRed}{$|$}\ cut\ -f\textcolor{Purple}{3}\ -d\texttt{\textcolor{Red}{"{}\ "{}}}` \\
\mbox{}\ \ \ \ \textcolor{ForestGreen}{ini}\textcolor{BrickRed}{=}`\textcolor{BrickRed}{[[}\ \textcolor{ForestGreen}{\$inc}\ -eq\ \textcolor{Purple}{1}\ \textcolor{BrickRed}{]]}\ \textcolor{BrickRed}{\&\&}\ echo\ \textcolor{Purple}{1}\ \textcolor{BrickRed}{$|$$|$}\ echo\ \textcolor{Purple}{10}` \\
\mbox{}\ \ \ \  \\
\mbox{}\ \ \ \ echo\ -n\ \texttt{\textcolor{Red}{"{}"{}}} \\
\mbox{}\ \ \ \ \textbf{\textcolor{Blue}{for}}\ i\ \textbf{\textcolor{Blue}{in}}\ `seq\ \textcolor{ForestGreen}{\$ini}\ \textcolor{ForestGreen}{\$inc}\ \textcolor{ForestGreen}{\$lim}`\textcolor{BrickRed}{;}\ \textbf{\textcolor{Blue}{do}}\ \textit{\textcolor{Brown}{\#((\ i\ =\ \$ini;\ i\ \textless{}\ \$lim;\ i\ +=\ \$inc\ ));\ do}} \\
\mbox{}\ \ \ \ \ \ \ \ echo\ -n\ \texttt{\textcolor{Red}{"{}\$i\ "{}}} \\
\mbox{}\ \ \ \ \ \ \ \ \textcolor{ForestGreen}{sum}\textcolor{BrickRed}{=}\textcolor{Purple}{0} \\
\mbox{}\ \ \ \ \ \ \ \  \\
\mbox{}\ \ \ \ \ \ \ \ \textit{\textcolor{Brown}{\#\ Se\ obtiene\ un\ promedio\ en\ N$\_$ITER\ ejecuciones\ para\ la\ talla\ dada}} \\
\mbox{}\ \ \ \ \ \ \ \ \textbf{\textcolor{Blue}{for}}\ k\ \textbf{\textcolor{Blue}{in}}\ `seq\ \textcolor{Purple}{0}\ \textcolor{ForestGreen}{\$N$\_$ITER}` \\
\mbox{}\ \ \ \ \ \ \ \ \textbf{\textcolor{Blue}{do}} \\
\mbox{}\ \ \ \ \ \ \ \ \ \ \ \ \textcolor{ForestGreen}{exc}\textcolor{BrickRed}{=}`\textcolor{ForestGreen}{\$1}\ \textcolor{ForestGreen}{\$i}` \\
\mbox{}\ \ \ \ \ \ \ \ \ \ \ \ \textcolor{ForestGreen}{sum}\textcolor{BrickRed}{=}`echo\ \texttt{\textcolor{Red}{"{}\$sum+\$\{exc/e/*10\textasciicircum{}\}"{}}}\ \textcolor{BrickRed}{$|$}\ bc\ -l` \\
\mbox{}\ \ \ \ \ \ \ \ \textbf{\textcolor{Blue}{done}} \\
\mbox{}\ \ \ \ \ \ \ \  \\
\mbox{}\ \ \ \ \ \ \ \ echo\ `echo\ \textcolor{ForestGreen}{\$sum}\textcolor{BrickRed}{/}\textcolor{ForestGreen}{\$N$\_$ITER}\ \textcolor{BrickRed}{$|$}\ bc\ -l` \\
\mbox{}\ \ \ \ \textbf{\textcolor{Blue}{done}} \\
\mbox{}\} \\
\mbox{} \\
\mbox{} \\
\mbox{}\textit{\textcolor{Brown}{\#\ param\ \$1\ "{}nombre\ del\ algoritmo\ (mergesort,\ heapsort,...)"{}}} \\
\mbox{}\textit{\textcolor{Brown}{\#}} \\
\mbox{}\textit{\textcolor{Brown}{\#\ Función\ que\ permite\ hacer\ un\ plot\ de\ un\ archivo\ \$1.dat\ ya\ generado}} \\
\mbox{}\textit{\textcolor{Brown}{\#}} \\
\mbox{}\textbf{\textcolor{Black}{function\ genplot()}}\ \{ \\
\mbox{}\ \ \ \ echo\ \texttt{\textcolor{Red}{'set\ xlabel\ "{}Talla\ del\ problema(n)"{}}} \\
\mbox{}\texttt{\textcolor{Red}{\ \ \ \ \ \ \ \ set\ ylabel\ "{}Tiempo(s)"{}}} \\
\mbox{}\texttt{\textcolor{Red}{\ \ \ \ \ \ \ \ set\ terminal\ jpeg\ size\ 800,480}} \\
\mbox{}\texttt{\textcolor{Red}{\ \ \ \ \ \ \ \ set\ output\ "{}plots/"{}.basename."{}.jpg"{}}} \\
\mbox{}\texttt{\textcolor{Red}{\ \ \ \ \ \ \ \ plot\ "{}data/"{}.basename."{}.dat"{}\ title\ "{}Eficiencia\ "{}\ .basename\ with\ linespoints'}}\ \textcolor{BrickRed}{\textgreater{}}\ \textcolor{ForestGreen}{\$SCRIPT} \\
\mbox{}\ \ \ \ gnuplot\ -e\ \texttt{\textcolor{Red}{"{}basename='\$1'"{}}}\ \textcolor{ForestGreen}{\$SCRIPT} \\
\mbox{}\ \ \ \ rm\ \textcolor{ForestGreen}{\$SCRIPT} \\
\mbox{}\} \\
\mbox{} \\
\mbox{} \\
\mbox{}\textit{\textcolor{Brown}{\#\ param\ \$1\ "{}nombre\ del\ algoritmo\ (mergesort,\ heapsort,...)"{}}} \\
\mbox{}\textit{\textcolor{Brown}{\#\ param\ \$2\ "{}función\ de\ ajuste"{}}} \\
\mbox{}\textit{\textcolor{Brown}{\#\ param\ \$3\ "{}coeficientes\ de\ la\ función\ de\ ajuste\ correspondiente"{}}} \\
\mbox{}\textit{\textcolor{Brown}{\#\ return\ result\ "{}Residuos\ del\ ajuste"{}}} \\
\mbox{}\textit{\textcolor{Brown}{\#}} \\
\mbox{}\textbf{\textcolor{Black}{function\ bondadajuste()}}\ \{ \\
\mbox{}\ \ \ \ echo\ \texttt{\textcolor{Red}{"{}f(x)=\$2;\ fit\ f(x)\ '\$1'\ via\ \$3"{}}}\ \textcolor{BrickRed}{$|$}\ gnuplot\ \textcolor{Purple}{2}\textcolor{BrickRed}{\textgreater{}}\ tmp \\
\mbox{}\ \ \ \  \\
\mbox{}\ \ \ \ \textcolor{ForestGreen}{result}\textcolor{BrickRed}{=}`cat\ tmp\ \textcolor{BrickRed}{$|$}\ grep\ \texttt{\textcolor{Red}{"{}rms"{}}}\ \textcolor{BrickRed}{$|$}\ grep\ -o\ \texttt{\textcolor{Red}{"{}[[:digit:]]}}\texttt{\textcolor{CarnationPink}{\textbackslash{}+}}\texttt{\textcolor{Red}{.*\$"{}}}` \\
\mbox{}\ \ \ \ \textcolor{ForestGreen}{result}\textcolor{BrickRed}{=}`echo\ \textcolor{ForestGreen}{\$\{result/e/*10\textasciicircum{}\}}\ \textcolor{BrickRed}{$|$}\ tr\ -d\ \texttt{\textcolor{Red}{"{}+"{}}}`\  \\
\mbox{}\ \ \ \ \textcolor{ForestGreen}{result}\textcolor{BrickRed}{=}\textcolor{ForestGreen}{\$\{result:-\/-1\}} \\
\mbox{}\ \ \ \  \\
\mbox{}\ \ \ \ \textcolor{BrickRed}{[[}\ result\ \textcolor{BrickRed}{!=}\ -\textcolor{Purple}{1}\ \textcolor{BrickRed}{]]}\ \textcolor{BrickRed}{\&\&}\ echo\ -e\ \texttt{\textcolor{Red}{"{}Ajuste:\ f(x)=\$2}}\texttt{\textcolor{CarnationPink}{\textbackslash{}n}}\texttt{\textcolor{Red}{"{}}}\ \textcolor{BrickRed}{\&\&}\ cat\ tmp \\
\mbox{}\ \ \ \  \\
\mbox{}\ \ \ \ rm\ -f\ tmp \\
\mbox{}\ \ \ \  \\
\mbox{}\ \ \ \ echo\ -e\ \texttt{\textcolor{Red}{"{}}}\texttt{\textcolor{CarnationPink}{\textbackslash{}n\textbackslash{}n}}\texttt{\textcolor{Red}{\#\#\#\#\#\#\#\#\#\#\#\#\#\#\#\#\#\#\#\#\#\#\#\#\#\#\#\#\#\#\#\#\#\#\#\#\#\#\#\#\#\#\#\#\#\#\#\#\#\#\#\#\#\#\#\#\#\#\#\#\#\#\#\#\#\#\#\#\#\#\#\#\#\#}}\texttt{\textcolor{CarnationPink}{\textbackslash{}n\textbackslash{}n}}\texttt{\textcolor{Red}{"{}}} \\
\mbox{}\} \\
\mbox{} \\
\mbox{} \\
\mbox{}\textit{\textcolor{Brown}{\#\ param\ \$1\ "{}nombre\ del\ algoritmo\ (mergesort,\ heapsort,...)"{}}} \\
\mbox{}\textit{\textcolor{Brown}{\#\ param\ \$2\ "{}función\ de\ ajuste"{}}} \\
\mbox{}\textit{\textcolor{Brown}{\#\ param\ \$3\ "{}coeficientes\ de\ la\ función\ de\ ajuste\ correspondiente"{}}} \\
\mbox{}\textit{\textcolor{Brown}{\#}} \\
\mbox{}\textit{\textcolor{Brown}{\#\ Hace\ un\ plot\ del\ .dat\ junto\ a\ la\ función\ que\ más\ se\ le\ ajusta\ de}} \\
\mbox{}\textit{\textcolor{Brown}{\#\ entre\ las\ disponibles\ en\ FUNCS}} \\
\mbox{}\textit{\textcolor{Brown}{\#}} \\
\mbox{}\textbf{\textcolor{Black}{function\ plotajuste()}}\ \{ \\
\mbox{}\ \ \ \ echo\ \texttt{\textcolor{Red}{"{}set\ xlabel\ 'Talla\ del\ problema(n)'}} \\
\mbox{}\texttt{\textcolor{Red}{\ \ \ \ \ \ \ \ set\ ylabel\ 'Tiempo(s)'}} \\
\mbox{}\texttt{\textcolor{Red}{\ \ \ \ \ \ \ \ set\ terminal\ jpeg\ size\ 800,480}} \\
\mbox{}\texttt{\textcolor{Red}{\ \ \ \ \ \ \ \ set\ output\ 'regressionPlots/`echo\ \$\{1\#\#*/\}\ $|$\ rev\ $|$\ cut\ -f2\ -d.\ $|$\ rev`$\_$fit.jpg'}} \\
\mbox{}\texttt{\textcolor{Red}{\ \ \ \ \ \ \ \ f(x)=\$2}} \\
\mbox{}\texttt{\textcolor{Red}{\ \ \ \ \ \ \ \ fit\ f(x)\ '\$1'\ via\ \$3}} \\
\mbox{}\texttt{\textcolor{Red}{\ \ \ \ \ \ \ \ plot\ '\$1',f(x)\ title\ 'Curva\ ajustada'\ with\ linespoints"{}}}\ \textcolor{BrickRed}{\textgreater{}}\ \textcolor{ForestGreen}{\$SCRIPT} \\
\mbox{}\ \ \ \ echo\ \texttt{\textcolor{Red}{"{}****\ \ \ Función\ de\ mejor\ ajuste:\ \$2\ \ \ *****"{}}} \\
\mbox{}\ \ \ \ gnuplot\ \textcolor{ForestGreen}{\$SCRIPT}\ \textcolor{Purple}{2}\textcolor{BrickRed}{\textgreater{}}\ /dev/null \\
\mbox{}\ \ \ \ rm\ -f\ \textcolor{ForestGreen}{\$SCRIPT} \\
\mbox{}\} \\
\mbox{} \\
\mbox{} \\
\mbox{}\textit{\textcolor{Brown}{\#\ param\ \$1\ "{}índice(válido)\ de\ FUNCS"{}}} \\
\mbox{}\textit{\textcolor{Brown}{\#\ return\ func\ "{}función\ de\ ajuste\ i-ésima\ de\ FUNCS"{}}} \\
\mbox{}\textit{\textcolor{Brown}{\#\ return\ coefs\ "{}coeficientes\ de\ ajuste\ de\ la\ función\ i-ésima"{}}} \\
\mbox{}\textit{\textcolor{Brown}{\#}} \\
\mbox{}\textbf{\textcolor{Black}{function\ extrae$\_$f()}}\{ \\
\mbox{}\ \ \ \ \textcolor{ForestGreen}{func}\textcolor{BrickRed}{=}`echo\ \textcolor{ForestGreen}{\$\{FUNCS[\$1]\}}\ \textcolor{BrickRed}{$|$}\ cut\ -f\textcolor{Purple}{1}\ -d\ \texttt{\textcolor{Red}{"{}\ "{}}}` \\
\mbox{}\ \ \ \ \textcolor{ForestGreen}{coefs}\textcolor{BrickRed}{=}`echo\ \textcolor{ForestGreen}{\$\{FUNCS[\$1]\}}\textcolor{BrickRed}{$|$}\ cut\ -f\textcolor{Purple}{2}\ -d\ \texttt{\textcolor{Red}{"{}\ "{}}}` \\
\mbox{}\}\ \ \ \  \\
\mbox{} \\
\mbox{} \\
\mbox{}\textit{\textcolor{Brown}{\#\ param\ \$1\ "{}archivo\ de\ datos\ (mergesort.dat,\ heapsort.dat,...)"{}}} \\
\mbox{}\textit{\textcolor{Brown}{\#\ }} \\
\mbox{}\textit{\textcolor{Brown}{\#\ Genera\ el\ plot\ del\ mejor\ ajuste\ posible\ de\ entre\ los\ de\ FUNCS}} \\
\mbox{}\textit{\textcolor{Brown}{\#\ para\ \$1.dat}} \\
\mbox{}\textit{\textcolor{Brown}{\#}} \\
\mbox{}\textbf{\textcolor{Black}{function\ genajuste()}}\ \{ \\
\mbox{}\ \ \ \ \textit{\textcolor{Brown}{\#echo\ -n\ "{}"{}\ \textgreater{}\ "{}\$1.fit"{}}} \\
\mbox{}\ \ \ \ \textit{\textcolor{Brown}{\#\ Suponemos\ FUNCS\ no\ vacío}} \\
\mbox{}\ \ \ \ extrae$\_$f\ \textcolor{Purple}{0} \\
\mbox{}\ \ \ \ bondadajuste\ \textcolor{ForestGreen}{\$1}\ \textcolor{ForestGreen}{\$\{func\}}\ \textcolor{ForestGreen}{\$\{coefs\}} \\
\mbox{}\ \ \ \ \textcolor{ForestGreen}{mejor}\textcolor{BrickRed}{=}\textcolor{ForestGreen}{\$result} \\
\mbox{}\ \ \ \ \textcolor{ForestGreen}{chosen}\textcolor{BrickRed}{=}\textcolor{Purple}{0} \\
\mbox{}\ \ \ \  \\
\mbox{}\ \ \ \ \textit{\textcolor{Brown}{\#\ Para\ cada\ ajuste\ en\ FUNCS}} \\
\mbox{}\ \ \ \ \textit{\textcolor{Brown}{\#\ Calcula\ sus\ residuos}} \\
\mbox{}\ \ \ \ \textit{\textcolor{Brown}{\#\ Si\ es\ mejor\ ajuste\ que\ el\ mejor\ hasta\ el\ momento}} \\
\mbox{}\ \ \ \ \textit{\textcolor{Brown}{\#\ \ \ \ \ \ \ Actualizar\ mejor\ ajuste}} \\
\mbox{}\ \ \ \ \textbf{\textcolor{Blue}{for}}\ i\ \textbf{\textcolor{Blue}{in}}\ `seq\ \textcolor{Purple}{1}\ \textcolor{ForestGreen}{\$((\$\{\#FUNCS[*]\}-1))}` \\
\mbox{}\ \ \ \ \textbf{\textcolor{Blue}{do}} \\
\mbox{}\ \ \ \ \ \ \ \ extrae$\_$f\ \textcolor{ForestGreen}{\$i} \\
\mbox{}\ \ \ \ \ \ \ \ bondadajuste\ \textcolor{ForestGreen}{\$1}\ \textcolor{ForestGreen}{\$\{func\}}\ \textcolor{ForestGreen}{\$\{coefs\}} \\
\mbox{}\ \ \ \ \ \ \ \  \\
\mbox{}\ \ \ \ \ \ \ \ \textbf{\textcolor{Blue}{if}}\ \textcolor{BrickRed}{[[}\ `echo\ \texttt{\textcolor{Red}{"{}(\$result\ \textless{}\ \$mejor)"{}}}\ \textcolor{BrickRed}{$|$}\ bc\ -l`\ -eq\ \textcolor{Purple}{1}\ \textcolor{BrickRed}{]]}\ \textcolor{BrickRed}{\&\&}\ \textcolor{BrickRed}{[[}\ \textcolor{ForestGreen}{\$result}\ \textcolor{BrickRed}{!=}\ -\textcolor{Purple}{1}\ \textcolor{BrickRed}{]]} \\
\mbox{}\ \ \ \ \ \ \ \ \textbf{\textcolor{Blue}{then}} \\
\mbox{}\ \ \ \ \ \ \ \ \ \ \ \ \textcolor{ForestGreen}{mejor}\textcolor{BrickRed}{=}\textcolor{ForestGreen}{\$result} \\
\mbox{}\ \ \ \ \ \ \ \ \ \ \ \ \textcolor{ForestGreen}{chosen}\textcolor{BrickRed}{=}\textcolor{ForestGreen}{\$i} \\
\mbox{}\ \ \ \ \ \ \ \ \textbf{\textcolor{Blue}{fi}} \\
\mbox{}\ \ \ \ \textbf{\textcolor{Blue}{done}} \\
\mbox{} \\
\mbox{}\ \ \ \ plotajuste\ \textcolor{ForestGreen}{\$1}\ \textcolor{ForestGreen}{\$(}echo\ \textcolor{ForestGreen}{\$\{FUNCS[\$chosen]\}}\ \textcolor{BrickRed}{$|$}\ cut\ -f\textcolor{Purple}{1}\ -d\ \texttt{\textcolor{Red}{"{}\ "{}}}\textcolor{BrickRed}{)}\ \textcolor{ForestGreen}{\$(}echo\ \textcolor{ForestGreen}{\$\{FUNCS[\$chosen]\}}\ \textcolor{BrickRed}{$|$}\ cut\ -f\textcolor{Purple}{2}\ -d\ \texttt{\textcolor{Red}{"{}\ "{}}}\textcolor{BrickRed}{)} \\
\mbox{} \\
\mbox{}\} \\
\mbox{} \\
\mbox{}\textit{\textcolor{Brown}{\#\ param\ \$1\ "{}nombre\ del/los\ algoritmo(s)\ (mergesort,\ heapsort,...)"{}}} \\
\mbox{}\textit{\textcolor{Brown}{\#\ }} \\
\mbox{}\textit{\textcolor{Brown}{\#\ Genera\ una\ tabla\ de\ datos\ a\ partir\ del\ .dat\ de\ los\ algoritmos\ dados\ }} \\
\mbox{}\textit{\textcolor{Brown}{\#}} \\
\mbox{}\textbf{\textcolor{Black}{function\ gentable()}}\ \{ \\
\mbox{}\ \ \ \ \textcolor{ForestGreen}{TEXFILE}\textcolor{BrickRed}{=}\texttt{\textcolor{Red}{"{}table\$(echo\ "{}}}\textcolor{ForestGreen}{\$1}\texttt{\textcolor{Red}{"{}\ $|$\ egrep\ -o\ "{}}}\textcolor{BrickRed}{\textbackslash{}}b\textcolor{BrickRed}{[[:}alnum\textcolor{BrickRed}{:]]}\{\textcolor{Purple}{3}\}\texttt{\textcolor{Red}{"{}\ $|$\ tr\ -d\ "{}}}\textcolor{BrickRed}{\textbackslash{}}n\texttt{\textcolor{Red}{"{}).tex"{}}} \\
\mbox{}\ \ \ \ touch\ \textcolor{ForestGreen}{\$TEXFILE}\ \textcolor{BrickRed}{$|$$|$}\ \textbf{\textcolor{Blue}{exit}}\ -\textcolor{Purple}{1} \\
\mbox{} \\
\mbox{}\ \ \ \ echo\ -n\ \texttt{\textcolor{Red}{"{}}}\texttt{\textcolor{CarnationPink}{\textbackslash{}\textbackslash{}}}\texttt{\textcolor{Red}{begin\{center\}}} \\
\mbox{}\texttt{\textcolor{Red}{\ \ \ \ }}\texttt{\textcolor{CarnationPink}{\textbackslash{}\textbackslash{}}}\texttt{\textcolor{Red}{begin\{longtabu\}\ to\ }}\texttt{\textcolor{CarnationPink}{\textbackslash{}l}}\texttt{\textcolor{Red}{inewidth\{\ l\ $|$\ *\{`echo\ \$1\ $|$\ wc\ -w`\}\{d\{10\}\}\}\ \ \%\ máx\ 10\ decimales}} \\
\mbox{}\texttt{\textcolor{CarnationPink}{\textbackslash{}r}}\texttt{\textcolor{Red}{owfont}}\texttt{\textcolor{CarnationPink}{\textbackslash{}b}}\texttt{\textcolor{Red}{fseries\ Tamaño\ "{}}}\ \textcolor{BrickRed}{\textgreater{}}\ \textcolor{ForestGreen}{\$TEXFILE} \\
\mbox{} \\
\mbox{}\ \ \ \ \textcolor{ForestGreen}{first}\textcolor{BrickRed}{=}\textcolor{Purple}{0} \\
\mbox{} \\
\mbox{}\ \ \ \ \textbf{\textcolor{Blue}{for}}\ file\ \textbf{\textcolor{Blue}{in}}\ \textcolor{ForestGreen}{\$1}\textcolor{BrickRed}{;}\ \textbf{\textcolor{Blue}{do}} \\
\mbox{}\ \ \ \ \ \ \ \ \textcolor{BrickRed}{[[}\ \textcolor{ForestGreen}{\$first}\ -eq\ \textcolor{Purple}{0}\ \textcolor{BrickRed}{]]}\ \textcolor{BrickRed}{\&\&}\ \textcolor{ForestGreen}{NLINES}\textcolor{BrickRed}{=}`cat\ \textcolor{ForestGreen}{\$file}\textcolor{BrickRed}{.}dat\ \textcolor{BrickRed}{$|$}\ wc\ -l`\ \textcolor{BrickRed}{\&\&}\ \textcolor{ForestGreen}{first}\textcolor{BrickRed}{=}\textcolor{Purple}{1} \\
\mbox{}\ \ \ \ \ \ \ \ echo\ -n\ \texttt{\textcolor{Red}{"{}\&\ }}\texttt{\textcolor{CarnationPink}{\textbackslash{}m}}\texttt{\textcolor{Red}{ulticolumn\{1\}\{l\}\{\$file\}\ "{}}}\ \textcolor{BrickRed}{\textgreater{}\textgreater{}}\ \textcolor{ForestGreen}{\$TEXFILE} \\
\mbox{}\ \ \ \ \textbf{\textcolor{Blue}{done}} \\
\mbox{} \\
\mbox{}\ \ \ \ echo\ \texttt{\textcolor{Red}{"{}}}\texttt{\textcolor{CarnationPink}{\textbackslash{}\textbackslash{}\textbackslash{}\textbackslash{}}}\texttt{\textcolor{Red}{\ }}\texttt{\textcolor{CarnationPink}{\textbackslash{}\textbackslash{}}}\texttt{\textcolor{Red}{hline}} \\
\mbox{}\texttt{\textcolor{Red}{\ \ \ \ }}\texttt{\textcolor{CarnationPink}{\textbackslash{}e}}\texttt{\textcolor{Red}{ndhead}} \\
\mbox{}\texttt{\textcolor{Red}{\ \ \ \ }}\texttt{\textcolor{CarnationPink}{\textbackslash{}e}}\texttt{\textcolor{Red}{ndfoot}} \\
\mbox{}\texttt{\textcolor{Red}{\ \ \ \ }}\texttt{\textcolor{CarnationPink}{\textbackslash{}\textbackslash{}\textbackslash{}\textbackslash{}}}\texttt{\textcolor{Red}{\ }}\texttt{\textcolor{CarnationPink}{\textbackslash{}\textbackslash{}}}\texttt{\textcolor{Red}{hline}} \\
\mbox{}\texttt{\textcolor{Red}{\ \ \ \ }}\texttt{\textcolor{CarnationPink}{\textbackslash{}e}}\texttt{\textcolor{Red}{ndlastfoot"{}}}\ \textcolor{BrickRed}{\textgreater{}\textgreater{}}\ \textcolor{ForestGreen}{\$TEXFILE} \\
\mbox{} \\
\mbox{}\ \ \ \ \textbf{\textcolor{Blue}{for}}\ \textcolor{BrickRed}{((}\ i\ \textcolor{BrickRed}{=}\ \textcolor{Purple}{1}\textcolor{BrickRed}{;}\ i\ \textcolor{BrickRed}{\textless{}=}\ \textcolor{ForestGreen}{\$NLINES}\textcolor{BrickRed}{;}\ i\textcolor{BrickRed}{++}\ \textcolor{BrickRed}{));}\ \textbf{\textcolor{Blue}{do}} \\
\mbox{}\ \ \ \ \ \ \ \ \textcolor{ForestGreen}{first}\textcolor{BrickRed}{=}\textcolor{Purple}{0}\ \textit{\textcolor{Brown}{\#\ true}} \\
\mbox{}\ \ \ \ \ \ \ \ \textbf{\textcolor{Blue}{for}}\ file\ \textbf{\textcolor{Blue}{in}}\ \textcolor{ForestGreen}{\$1}\textcolor{BrickRed}{;}\ \textbf{\textcolor{Blue}{do}} \\
\mbox{}\ \ \ \ \ \ \ \ \ \ \ \ \textcolor{ForestGreen}{line}\textcolor{BrickRed}{=}`sed\ \texttt{\textcolor{Red}{"{}\$i\ q;d"{}}}\ \textcolor{ForestGreen}{\$file}\textcolor{BrickRed}{.}dat` \\
\mbox{}\ \ \ \ \ \ \ \ \ \ \ \ \textcolor{ForestGreen}{time}\textcolor{BrickRed}{=}`echo\ \textcolor{ForestGreen}{\$line}\ \textcolor{BrickRed}{$|$}\ cut\ -d\texttt{\textcolor{Red}{"{}\ "{}}}\ -f\textcolor{Purple}{2}\ \textcolor{BrickRed}{$|$}\ sed\ \texttt{\textcolor{Red}{"{}s/0}}\texttt{\textcolor{CarnationPink}{\textbackslash{}\{}}\texttt{\textcolor{Red}{1,}}\texttt{\textcolor{CarnationPink}{\textbackslash{}\}}}\texttt{\textcolor{Red}{\$//;s/\textasciicircum{}}}\texttt{\textcolor{CarnationPink}{\textbackslash{}.}}\texttt{\textcolor{Red}{/0}}\texttt{\textcolor{CarnationPink}{\textbackslash{}.}}\texttt{\textcolor{Red}{/"{}}}` \\
\mbox{}\ \ \ \ \ \ \ \ \ \ \ \  \\
\mbox{}\ \ \ \ \ \ \ \ \ \ \ \ \textcolor{BrickRed}{[[}\ \textcolor{ForestGreen}{\$first}\ -eq\ \textcolor{Purple}{0}\ \textcolor{BrickRed}{]]}\ \textcolor{BrickRed}{\&\&}\ \{ \\
\mbox{}\ \ \ \ \ \ \ \ \ \ \ \ \ \ \ \ \textcolor{ForestGreen}{tam}\textcolor{BrickRed}{=}`echo\ \textcolor{ForestGreen}{\$line}\ \textcolor{BrickRed}{$|$}\ cut\ -d\texttt{\textcolor{Red}{"{}\ "{}}}\ -f\textcolor{Purple}{1}` \\
\mbox{}\ \ \ \ \ \ \ \ \ \ \ \ \ \ \ \ echo\ -n\ \texttt{\textcolor{Red}{"{}\$tam\ "{}}}\ \textcolor{BrickRed}{\textgreater{}\textgreater{}}\ \textcolor{ForestGreen}{\$TEXFILE} \\
\mbox{}\ \ \ \ \ \ \ \ \ \ \ \ \ \ \ \ \textcolor{ForestGreen}{first}\textcolor{BrickRed}{=}\textcolor{Purple}{1}\ \textit{\textcolor{Brown}{\#\ false}} \\
\mbox{}\ \ \ \ \ \ \ \ \ \ \ \ \} \\
\mbox{} \\
\mbox{}\ \ \ \ \ \ \ \ \ \ \ \ echo\ -n\ \texttt{\textcolor{Red}{"{}\&\ \$time\ "{}}}\ \textcolor{BrickRed}{\textgreater{}\textgreater{}}\ \textcolor{ForestGreen}{\$TEXFILE} \\
\mbox{}\ \ \ \ \ \ \ \ \textbf{\textcolor{Blue}{done}} \\
\mbox{}\ \ \ \ \ \ \ \ echo\ \texttt{\textcolor{Red}{"{}}}\texttt{\textcolor{CarnationPink}{\textbackslash{}\textbackslash{}\textbackslash{}\textbackslash{}}}\texttt{\textcolor{Red}{"{}}}\ \textcolor{BrickRed}{\textgreater{}\textgreater{}}\ \textcolor{ForestGreen}{\$TEXFILE} \\
\mbox{}\ \ \ \ \textbf{\textcolor{Blue}{done}} \\
\mbox{} \\
\mbox{}\ \ \ \ echo\ \texttt{\textcolor{Red}{"{}}} \\
\mbox{}\texttt{\textcolor{Red}{\ \ \ \ }}\texttt{\textcolor{CarnationPink}{\textbackslash{}\textbackslash{}}}\texttt{\textcolor{Red}{end\{longtabu\}}} \\
\mbox{}\texttt{\textcolor{CarnationPink}{\textbackslash{}\textbackslash{}}}\texttt{\textcolor{Red}{end\{center\}"{}}}\ \textcolor{BrickRed}{\textgreater{}\textgreater{}}\ \textcolor{ForestGreen}{\$TEXFILE} \\
\mbox{}\} \\
\mbox{} \\
\mbox{}\textit{\textcolor{Brown}{\#\ Si\ el\ argumento\ del\ script\ es\ 0,\ generamos\ el\ .dat\ correspondiente\ al\ archivo\ pasado}} \\
\mbox{}\textit{\textcolor{Brown}{\#\ Si\ es\ 1,\ hacemos\ un\ plot\ de\ datos}} \\
\mbox{}\textit{\textcolor{Brown}{\#\ Si\ es\ 2,\ generamos\ un\ plot\ del\ ajuste\ al\ .dat\ del\ archivo\ pasado}} \\
\mbox{}\textit{\textcolor{Brown}{\#\ Si\ es\ 3,\ generamos\ una\ tabla\ LaTeX\ a\ partir\ de\ los\ .dat\ de\ los\ programas\ dados}} \\
\mbox{} \\
\mbox{}\textcolor{BrickRed}{[[}\ \textcolor{ForestGreen}{\$2}\ -eq\ \textcolor{Purple}{0}\ \textcolor{BrickRed}{]]}\ \textcolor{BrickRed}{\&\&}\ gendata\ \textcolor{ForestGreen}{\$1}\ \textcolor{BrickRed}{\&\&}\ \textbf{\textcolor{Blue}{exit}}\ \textcolor{Purple}{0} \\
\mbox{} \\
\mbox{}\textcolor{BrickRed}{[[}\ \textcolor{ForestGreen}{\$2}\ -eq\ \textcolor{Purple}{1}\ \textcolor{BrickRed}{]]}\ \textcolor{BrickRed}{\&\&}\ genplot\ `echo\ \textcolor{ForestGreen}{\$\{1\#\#*/\}}\ \textcolor{BrickRed}{$|$}\ rev\ \textcolor{BrickRed}{$|$}\ cut\ -f\textcolor{Purple}{2}\ -d\textcolor{BrickRed}{.}\ \textcolor{BrickRed}{$|$}\ rev`\ \textcolor{BrickRed}{\&\&}\ \textbf{\textcolor{Blue}{exit}}\ \textcolor{Purple}{0} \\
\mbox{} \\
\mbox{}\textcolor{BrickRed}{[[}\ \textcolor{ForestGreen}{\$2}\ -eq\ \textcolor{Purple}{2}\ \textcolor{BrickRed}{]]}\ \textcolor{BrickRed}{\&\&}\ genajuste\ \textcolor{ForestGreen}{\$1}\ \textcolor{BrickRed}{\&\&}\ \textbf{\textcolor{Blue}{exit}}\ \textcolor{Purple}{0} \\
\mbox{} \\
\mbox{}\textcolor{BrickRed}{[[}\ \textcolor{ForestGreen}{\$2}\ -eq\ \textcolor{Purple}{3}\ \textcolor{BrickRed}{]]}\ \textcolor{BrickRed}{\&\&}\ gentable\ \texttt{\textcolor{Red}{"{}\$1"{}}}\ \textcolor{BrickRed}{\&\&}\ \textbf{\textcolor{Blue}{exit}}\ \textcolor{Purple}{0} \\
\mbox{} \\
\mbox{}\textbf{\textcolor{Blue}{exit}}\ \textcolor{Purple}{1} \\
\mbox{}
}
\normalsize

\subsection{Makefile}
Para llamar al algoritmo para cada archivo de código fuente y encapsular tareas
como la generación de ejecutables, de gráficas y de gráficas y funciones de ajuste,
se ha redactado un makefile.
\begin{itemize}
\item \texttt{make data} genera los archivos de datos del apéndice posterior.
\item \texttt{make plot} genera para los algoritmos cuyo código fuente se incluye 
en \texttt{./src} y a partir de los \texttt{.dat} correspondientes, las gráficas
\item \texttt{make fit} genera para los \texttt{.dat} correspondientes a los algoritmos
un archivo de la forma \texttt{<nombre-algoritmo>.fit}, que contiene información
sobre la bondad de una serie de ajustes (recta de regresión, ajuste cuadrático, cúbico, 
exponencial,...) a las parejas de datos en los \texttt{.dat}
\end{itemize}

\scriptsize\texttt{% Generator: GNU source-highlight, by Lorenzo Bettini, http://www.gnu.org/software/src-highlite
\noindent
\mbox{}\textit{\textcolor{Brown}{\#\ Eficiencia\ de\ algoritmos.}} \\
\mbox{}\textit{\textcolor{Brown}{\#\ makefile.}} \\
\mbox{}\textit{\textcolor{Brown}{\#\ Basado\ en:\ http://stackoverflow.com/questions/9787160/makefile-that-compiles-all-cpp-files-in-a-directory-into-separate-executable}} \\
\mbox{} \\
\mbox{}\textcolor{ForestGreen}{BIN=}\textcolor{BrickRed}{.}/bin \\
\mbox{}\textcolor{ForestGreen}{SRC=}\textcolor{BrickRed}{.}/src \\
\mbox{}\textcolor{ForestGreen}{DATA=}\textcolor{BrickRed}{.}/data \\
\mbox{}\textcolor{ForestGreen}{PLOT=}\textcolor{BrickRed}{.}/plots \\
\mbox{}\textcolor{ForestGreen}{FIT=}\textcolor{BrickRed}{.}/regressionPlots \\
\mbox{}\textcolor{ForestGreen}{TEX=}\textcolor{BrickRed}{.}/tex \\
\mbox{}\textcolor{ForestGreen}{FLAGS=}-std\textcolor{BrickRed}{=}c\textcolor{BrickRed}{++}0x\ -Wall \\
\mbox{} \\
\mbox{}\textit{\textcolor{Brown}{\#\ make\ all:\ Compilar\ todos\ los\ programas\ }} \\
\mbox{}\textcolor{BrickRed}{all:}\ \textcolor{ForestGreen}{\$(}patsubst\ \textcolor{ForestGreen}{\$(SRC)}\textcolor{BrickRed}{/\%}.cpp\textcolor{BrickRed}{,}\ \textcolor{ForestGreen}{\$(BIN)}\textcolor{BrickRed}{/\%,}\ \textcolor{ForestGreen}{\$(}wildcard\ \textcolor{ForestGreen}{\$(SRC)/*.cpp))} \\
\mbox{} \\
\mbox{}\textit{\textcolor{Brown}{\#\ make\ data:\ Recalcular\ el\ archivo\ .dat\ de\ todos\ los\ programas}} \\
\mbox{}\textcolor{BrickRed}{data:}\ all\ gengraf.sh\ \textcolor{ForestGreen}{\$(}patsubst\ \textcolor{ForestGreen}{\$(SRC)}\textcolor{BrickRed}{/\%}.cpp\textcolor{BrickRed}{,}\ \textcolor{ForestGreen}{\$(DATA)}\textcolor{BrickRed}{/\%}.dat\textcolor{BrickRed}{,}\ \textcolor{ForestGreen}{\$(}wildcard\ \textcolor{ForestGreen}{\$(SRC)/*.cpp))} \\
\mbox{} \\
\mbox{}\textit{\textcolor{Brown}{\#\ make\ plot:\ Generar\ todas\ las\ imágenes\ a\ partir\ de\ los\ archivos\ .dat}} \\
\mbox{}\textcolor{BrickRed}{plot:}\ gengraf.sh\ \textcolor{ForestGreen}{\$(}patsubst\ \textcolor{ForestGreen}{\$(SRC)}\textcolor{BrickRed}{/\%}.cpp\textcolor{BrickRed}{,}\ \textcolor{ForestGreen}{\$(PLOT)}\textcolor{BrickRed}{/\%}.jpg\textcolor{BrickRed}{,}\ \textcolor{ForestGreen}{\$(}wildcard\ \textcolor{ForestGreen}{\$(SRC)/*.cpp))} \\
\mbox{} \\
\mbox{}\textit{\textcolor{Brown}{\#\ make\ fit:\ Crear\ las\ imágenes\ con\ las\ funciones\ de\ ajuste}} \\
\mbox{}\textcolor{BrickRed}{fit:}\ gengraf.sh\ \textcolor{ForestGreen}{\$(}patsubst\ \textcolor{ForestGreen}{\$(SRC)}\textcolor{BrickRed}{/\%}.cpp\textcolor{BrickRed}{,}\ \textcolor{ForestGreen}{\$(FIT)}\textcolor{BrickRed}{/\%}.fit\textcolor{BrickRed}{,}\ \textcolor{ForestGreen}{\$(}wildcard\ \textcolor{ForestGreen}{\$(SRC)/*.cpp))} \\
\mbox{} \\
\mbox{}\textit{\textcolor{Brown}{\#\ make\ codetex:\ Crear\ los\ archivos\ LaTeX\ de\ código\ resaltado}} \\
\mbox{}\textcolor{BrickRed}{codetex:}\ \textcolor{ForestGreen}{\$(}patsubst\ \textcolor{ForestGreen}{\$(SRC)}\textcolor{BrickRed}{/\%}.cpp\textcolor{BrickRed}{,}\ \textcolor{ForestGreen}{\$(TEX)}\textcolor{BrickRed}{/\%}.tex\textcolor{BrickRed}{,}\ \textcolor{ForestGreen}{\$(}wildcard\ \textcolor{ForestGreen}{\$(SRC)/*.cpp))} \\
\mbox{} \\
\mbox{}\textit{\textcolor{Brown}{\#\ Opciones\ individuales\ }} \\
\mbox{}\textcolor{ForestGreen}{\$(BIN)}\textcolor{BrickRed}{/\%:}\ \textcolor{ForestGreen}{\$(SRC)}\textcolor{BrickRed}{/\%}.cpp \\
\mbox{}\ \ \ \ \ \ \ \ g\textcolor{BrickRed}{++}\ \textcolor{ForestGreen}{\$\textless{}}\ -o\ \textcolor{ForestGreen}{\$@}\ \textcolor{ForestGreen}{\$(FLAGS)} \\
\mbox{} \\
\mbox{}\textcolor{ForestGreen}{\$(DATA)}\textcolor{BrickRed}{/\%}.dat\textcolor{BrickRed}{:}\ \textcolor{ForestGreen}{\$(BIN)}\textcolor{BrickRed}{/\%} \\
\mbox{}\ \ \ \ \ \ \ \ \textcolor{BrickRed}{.}/gengraf.sh\ \textcolor{ForestGreen}{\$\textless{}}\ \textcolor{Purple}{0}\ \textcolor{BrickRed}{\textgreater{}}\ \textcolor{ForestGreen}{\$@} \\
\mbox{} \\
\mbox{}\textcolor{ForestGreen}{\$(PLOT)}\textcolor{BrickRed}{/\%}.jpg\textcolor{BrickRed}{:}\ \textcolor{ForestGreen}{\$(DATA)}\textcolor{BrickRed}{/\%}.dat \\
\mbox{}\ \ \ \ \ \ \ \ \textcolor{BrickRed}{.}/gengraf.sh\ \textcolor{ForestGreen}{\$\textless{}}\ \textcolor{Purple}{1} \\
\mbox{} \\
\mbox{}\textcolor{ForestGreen}{\$(FIT)}\textcolor{BrickRed}{/\%}.fit\textcolor{BrickRed}{:}\ \textcolor{ForestGreen}{\$(DATA)}\textcolor{BrickRed}{/\%}.dat \\
\mbox{}\ \ \ \ \ \ \ \ \textcolor{BrickRed}{.}/gengraf.sh\ \textcolor{ForestGreen}{\$\textless{}}\ \textcolor{Purple}{2}\ \textcolor{BrickRed}{\textgreater{}}\ \textcolor{ForestGreen}{\$@} \\
\mbox{} \\
\mbox{}\textit{\textcolor{Brown}{\#\$(FIT)/\%.fit.jpg:\ \$(DATA)/\%.dat}} \\
\mbox{}\textit{\textcolor{Brown}{\#\ \ \ \ \ \ \ \ ./gengraf.sh\ \$\textless{}\ 2}} \\
\mbox{} \\
\mbox{}\textcolor{ForestGreen}{\$(TEX)}\textcolor{BrickRed}{/\%}.tex\textcolor{BrickRed}{:}\ \textcolor{ForestGreen}{\$(SRC)}\textcolor{BrickRed}{/\%}.cpp \\
\mbox{}\ \ \ \ \ \ \ \ source-highlight\ -f\ latexcolor\ -i\ \textcolor{ForestGreen}{\$\textless{}}\ -o\ \textcolor{ForestGreen}{\$@} \\
\mbox{}\textcolor{ForestGreen}{\$(TEX)}/makefile.tex\textcolor{BrickRed}{:}\ makefile \\
\mbox{}\ \ \ \ \ \ \ \ source-highlight\ -f\ latexcolor\ -i\ \textcolor{ForestGreen}{\$\textless{}}\ -o\ \textcolor{ForestGreen}{\$@} \\
\mbox{} \\
\mbox{}\textit{\textcolor{Brown}{\#\ Limpieza\ de\ los\ ejecutables}} \\
\mbox{}\textcolor{BrickRed}{clean:} \\
\mbox{}\ \ \ \ \ \ \ \ rm\ \textcolor{ForestGreen}{\$(BIN)}\textcolor{BrickRed}{/*} \\
\mbox{} \\
\mbox{}\textcolor{BrickRed}{cleanall:}\ clean \\
\mbox{}\ \ \ \ \ \ \ \ rm\ \textcolor{BrickRed}{*}.jpg\ \textcolor{BrickRed}{*}.dat\ tex\textcolor{BrickRed}{/*} \\
\mbox{}
}\normalsize

\subsection{Tablas de datos}
Se adjuntan las tablas de datos (talla del problema-tiempo de ejecución) para
cada algoritmo.
\begin{center}
    \begin{longtabu} to \linewidth{ l | *{3}{d{10}}}  % máx 10 decimales
\rowfont\bfseries Tamaño & \multicolumn{1}{l}{burbuja} & \multicolumn{1}{l}{insercion} & \multicolumn{1}{l}{seleccion} \\ \hline
    \endhead
    \endfoot
    \\ \hline
    \endlastfoot
10 & 0.0000008892 & 0.0000006192 & 0.0000008138 \\
110 & 0.00004446 & 0.00001924 & 0.00002938 \\
210 & 0.0001442 & 0.00006996 & 0.0000944 \\
310 & 0.0003004 & 0.0001594 & 0.000198 \\
410 & 0.0005086 & 0.0002868 & 0.0003384 \\
510 & 0.0007816 & 0.0004362 & 0.0005204 \\
610 & 0.0010908 & 0.0005976 & 0.0007372 \\
710 & 0.001462 & 0.000803 & 0.0009894 \\
810 & 0.001878 & 0.0010258 & 0.001272 \\
910 & 0.00236 & 0.001332 & 0.001594 \\
1010 & 0.002894 & 0.001654 & 0.001966 \\
1110 & 0.00348 & 0.001984 & 0.002322 \\
1210 & 0.004136 & 0.002382 & 0.002766 \\
1310 & 0.004932 & 0.002788 & 0.00327 \\
1410 & 0.005878 & 0.003218 & 0.003754 \\
1510 & 0.0066 & 0.00369 & 0.004288 \\
1610 & 0.00757 & 0.004096 & 0.004846 \\
1710 & 0.008366 & 0.004626 & 0.005474 \\
1810 & 0.009484 & 0.005244 & 0.00609 \\
1910 & 0.010456 & 0.005938 & 0.006768 \\
2010 & 0.011572 & 0.006502 & 0.007432 \\
2110 & 0.0125 & 0.007208 & 0.00831 \\
2210 & 0.01372 & 0.007818 & 0.009128 \\
2310 & 0.01652 & 0.00849 & 0.009954 \\
2410 & 0.01686 & 0.00906 & 0.010776 \\
2510 & 0.01812 & 0.009828 & 0.011506 \\
2610 & 0.01934 & 0.010628 & 0.01232 \\
2710 & 0.02104 & 0.011356 & 0.0131 \\
2810 & 0.02588 & 0.01226 & 0.01418 \\
2910 & 0.02424 & 0.01298 & 0.01514 \\
3010 & 0.02626 & 0.01382 & 0.01602 \\
3110 & 0.02826 & 0.01476 & 0.01708 \\
3210 & 0.03036 & 0.01588 & 0.01822 \\
3310 & 0.0333 & 0.0166 & 0.01946 \\
3410 & 0.03454 & 0.01776 & 0.02046 \\
3510 & 0.03676 & 0.01902 & 0.02172 \\
3610 & 0.03946 & 0.02008 & 0.02314 \\
3710 & 0.041 & 0.02126 & 0.02416 \\
3810 & 0.04564 & 0.02226 & 0.02542 \\
3910 & 0.04702 & 0.02354 & 0.02672 \\
4010 & 0.04964 & 0.02444 & 0.02812 \\
4110 & 0.05094 & 0.02578 & 0.0292 \\
4210 & 0.05558 & 0.02704 & 0.0308 \\
4310 & 0.05518 & 0.02844 & 0.03232 \\
4410 & 0.05848 & 0.02982 & 0.03346 \\
4510 & 0.06142 & 0.03106 & 0.03502 \\
4610 & 0.06258 & 0.03228 & 0.03664 \\
4710 & 0.06566 & 0.0338 & 0.0381 \\
4810 & 0.0696 & 0.03512 & 0.04002 \\
4910 & 0.07356 & 0.03604 & 0.04136 \\
5010 & 0.0783 & 0.03816 & 0.0432 \\
5110 & 0.07918 & 0.03882 & 0.04458 \\
5210 & 0.08968 & 0.0407 & 0.04716 \\
5310 & 0.0892 & 0.04244 & 0.04968 \\
5410 & 0.09464 & 0.04472 & 0.0555 \\
5510 & 0.10028 & 0.0457 & 0.05168 \\
5610 & 0.10722 & 0.04762 & 0.05368 \\
5710 & 0.10374 & 0.04912 & 0.05538 \\
5810 & 0.10634 & 0.05132 & 0.05758 \\
5910 & 0.10982 & 0.0524 & 0.0596 \\
6010 & 0.11438 & 0.0541 & 0.06114 \\
6110 & 0.11898 & 0.05498 & 0.06358 \\
6210 & 0.1232 & 0.05726 & 0.06666 \\
6310 & 0.1306 & 0.05864 & 0.06742 \\
6410 & 0.1394 & 0.06146 & 0.06994 \\
6510 & 0.1402 & 0.06388 & 0.07186 \\
6610 & 0.1442 & 0.06518 & 0.07386 \\
6710 & 0.1458 & 0.06674 & 0.07638 \\
6810 & 0.1492 & 0.07028 & 0.07828 \\
6910 & 0.1588 & 0.0703 & 0.08092 \\
7010 & 0.165 & 0.07852 & 0.12216 \\
7110 & 0.17 & 0.07768 & 0.0904 \\
7210 & 0.1762 & 0.07674 & 0.09202 \\
7310 & 0.196 & 0.08146 & 0.08936 \\
7410 & 0.1878 & 0.0821 & 0.10202 \\
7510 & 0.2044 & 0.08376 & 0.10954 \\
7610 & 0.2028 & 0.08484 & 0.0987 \\
7710 & 0.203 & 0.08808 & 0.10064 \\
7810 & 0.2138 & 0.09026 & 0.1028 \\
7910 & 0.2206 & 0.09288 & 0.10632 \\
8010 & 0.2324 & 0.09484 & 0.1079 \\
8110 & 0.2328 & 0.09744 & 0.11234 \\
8210 & 0.2398 & 0.09942 & 0.11326 \\
8310 & 0.239 & 0.1021 & 0.11802 \\
8410 & 0.2474 & 0.10586 & 0.1205 \\
8510 & 0.2652 & 0.10826 & 0.1216 \\
8610 & 0.266 & 0.10982 & 0.1252 \\
8710 & 0.2624 & 0.1124 & 0.1278 \\
8810 & 0.2746 & 0.11512 & 0.1342 \\
8910 & 0.2846 & 0.11892 & 0.1458 \\
9010 & 0.29 & 0.12212 & 0.1392 \\
9110 & 0.3062 & 0.126 & 0.1392 \\
9210 & 0.3132 & 0.127 & 0.1442 \\
9310 & 0.3134 & 0.1304 & 0.1478 \\
9410 & 0.3156 & 0.1304 & 0.1494 \\
9510 & 0.333 & 0.1334 & 0.1514 \\
9610 & 0.3404 & 0.137 & 0.1556 \\
9710 & 0.3378 & 0.1418 & 0.1592 \\
9810 & 0.3466 & 0.1462 & 0.1622 \\
9910 & 0.347 & 0.1484 & 0.1648 \\

    \end{longtabu}
\end{center}

\begin{center}
    \begin{longtabu} to \linewidth{ l | *{1}{d{10}}}  % máx 10 decimales
\rowfont\bfseries Tamaño & \multicolumn{1}{l}{floyd} \\ \hline
    \endhead
    \endfoot
    \\ \hline
    \endlastfoot
10 & 0.000013586 \\
15 & 0.0000337 \\
20 & 0.00009132 \\
25 & 0.0001544 \\
30 & 0.0002582 \\
35 & 0.0004036 \\
40 & 0.0005962 \\
45 & 0.0009006 \\
50 & 0.0011426 \\
55 & 0.001504 \\
60 & 0.001934 \\
65 & 0.002458 \\
70 & 0.003336 \\
75 & 0.004736 \\
80 & 0.00512 \\
85 & 0.005722 \\
90 & 0.006434 \\
95 & 0.00824 \\
100 & 0.009808 \\
105 & 0.011378 \\
110 & 0.011724 \\
115 & 0.01372 \\
120 & 0.01648 \\
125 & 0.01794 \\
130 & 0.01944 \\
135 & 0.02204 \\
140 & 0.02414 \\
145 & 0.03024 \\
150 & 0.03006 \\
155 & 0.03534 \\
160 & 0.03576 \\
165 & 0.03974 \\
170 & 0.04692 \\
175 & 0.04604 \\
180 & 0.0497 \\
185 & 0.05576 \\
190 & 0.06016 \\
195 & 0.0645 \\
200 & 0.07116 \\
205 & 0.07624 \\
210 & 0.08262 \\
215 & 0.08622 \\
220 & 0.09184 \\
225 & 0.102 \\
230 & 0.10298 \\
235 & 0.11146 \\
240 & 0.12502 \\
245 & 0.1262 \\
250 & 0.1342 \\
255 & 0.145 \\
260 & 0.1504 \\
265 & 0.1616 \\
270 & 0.1672 \\
275 & 0.1778 \\
280 & 0.1904 \\
285 & 0.1894 \\
290 & 0.2038 \\
295 & 0.219 \\
300 & 0.228 \\
305 & 0.2384 \\
310 & 0.2652 \\
315 & 0.2598 \\
320 & 0.271 \\
325 & 0.2922 \\
330 & 0.2968 \\
335 & 0.309 \\
340 & 0.3346 \\
345 & 0.344 \\
350 & 0.3638 \\
355 & 0.3738 \\
360 & 0.3798 \\
365 & 0.4068 \\
370 & 0.4194 \\
375 & 0.4412 \\
380 & 0.451 \\
385 & 0.4936 \\
390 & 0.4994 \\
395 & 0.5022 \\
400 & 0.54 \\
405 & 0.5524 \\
410 & 0.5864 \\
415 & 0.6408 \\
420 & 0.623 \\
425 & 0.659 \\
430 & 0.6798 \\
435 & 0.7106 \\
440 & 0.7148 \\
445 & 0.7378 \\
450 & 0.749 \\
455 & 0.7714 \\
460 & 0.8122 \\
465 & 0.916 \\
470 & 0.8836 \\
475 & 0.9008 \\
480 & 0.9232 \\
485 & 0.9262 \\
490 & 0.9456 \\
495 & 0.9642 \\
500 & 1.0046 \\
505 & 1.0406 \\
510 & 1.0782 \\
515 & 1.1282 \\
520 & 1.1226 \\
525 & 1.1528 \\
530 & 1.2152 \\
535 & 1.258 \\
540 & 1.336 \\
545 & 1.342 \\
550 & 1.372 \\
555 & 1.376 \\
560 & 1.468 \\
565 & 1.502 \\
570 & 1.56 \\
575 & 1.584 \\
580 & 1.636 \\
585 & 1.642 \\
590 & 1.7 \\
595 & 1.7 \\
600 & 1.748 \\
605 & 1.826 \\
610 & 1.888 \\
615 & 1.94 \\
620 & 1.964 \\
625 & 1.936 \\
630 & 1.982 \\
635 & 2.02 \\
640 & 2.336 \\
645 & 2.224 \\
650 & 2.25 \\
655 & 2.224 \\
660 & 2.296 \\
665 & 2.326 \\
670 & 2.44 \\
675 & 2.426 \\
680 & 2.492 \\
685 & 2.562 \\
690 & 2.69 \\
695 & 2.736 \\
700 & 2.752 \\
705 & 2.87 \\
710 & 2.798 \\
715 & 2.81 \\
720 & 2.894 \\
725 & 2.934 \\
730 & 3. \\
735 & 3.124 \\
740 & 3.154 \\
745 & 3.2 \\
750 & 3.35 \\
755 & 3.336 \\
760 & 3.406 \\
765 & 3.432 \\
770 & 3.53 \\
775 & 3.574 \\
780 & 3.656 \\
785 & 3.772 \\
790 & 3.82 \\
795 & 3.902 \\
800 & 3.932 \\
805 & 4.034 \\
810 & 4.166 \\
815 & 4.43 \\
820 & 4.356 \\
825 & 4.368 \\
830 & 4.46 \\
835 & 4.492 \\
840 & 4.606 \\
845 & 4.674 \\
850 & 4.828 \\
855 & 4.856 \\
860 & 4.884 \\
865 & 5.026 \\
870 & 5.214 \\
875 & 5.19 \\
880 & 5.276 \\
885 & 5.334 \\
890 & 5.464 \\
895 & 5.578 \\
900 & 5.624 \\
905 & 5.762 \\
910 & 5.868 \\
915 & 5.896 \\
920 & 6.01 \\
925 & 6.116 \\
930 & 6.232 \\
935 & 6.354 \\
940 & 6.446 \\
945 & 6.57 \\
950 & 6.714 \\
955 & 6.726 \\
960 & 6.854 \\
965 & 7.008 \\
970 & 7.038 \\
975 & 7.166 \\
980 & 7.29 \\
985 & 7.39 \\
990 & 7.488 \\
995 & 7.76 \\

    \end{longtabu}
\end{center}

\begin{center}
    \begin{longtabu} to \linewidth{ l | *{1}{d{10}}}  % máx 10 decimales
\rowfont\bfseries Tamaño & \multicolumn{1}{l}{hanoi} \\ \hline
    \endhead
    \endfoot
    \\ \hline
    \endlastfoot
1 & 0.0000001792 \\
2 & 0.0000001948 \\
3 & 0.0000002414 \\
4 & 0.0000004294 \\
5 & 0.0000005908 \\
6 & 0.0000010442 \\
7 & 0.000001544 \\
8 & 0.000002784 \\
9 & 0.000005286 \\
10 & 0.000010166 \\
11 & 0.00002014 \\
12 & 0.00003998 \\
13 & 0.00008108 \\
14 & 0.0001586 \\
15 & 0.0003176 \\
16 & 0.0006344 \\
17 & 0.00125 \\
18 & 0.002484 \\
19 & 0.00489 \\
20 & 0.010266 \\
21 & 0.01886 \\
22 & 0.03656 \\
23 & 0.07274 \\
24 & 0.1452 \\
25 & 0.2938 \\
26 & 0.6256 \\
27 & 1.1712 \\
28 & 2.344 \\
29 & 4.684 \\
30 & 9.402 \\
31 & 18.8 \\
32 & 37.64 \\
33 & 75.6 \\
34 & 156.8 \\

    \end{longtabu}
\end{center}

\begin{center}
    \begin{longtabu} to \linewidth{ l | *{3}{d{10}}}  % máx 10 decimales
\rowfont\bfseries Tamaño & \multicolumn{1}{l}{mergesort} & \multicolumn{1}{l}{quicksort} & \multicolumn{1}{l}{heapsort} \\ \hline
    \endhead
    \endfoot
    \\ \hline
    \endlastfoot
10 & 0.0000006618 & 0.0000006912 & 0.0000010894 \\
110 & 0.00001646 & 0.00001059 & 0.000012012 \\
210 & 0.00002944 & 0.00002244 & 0.00002546 \\
310 & 0.00005618 & 0.00003332 & 0.00003888 \\
410 & 0.00006322 & 0.00004572 & 0.0000547 \\
510 & 0.00008668 & 0.00005474 & 0.00007006 \\
610 & 0.00011682 & 0.0000675 & 0.0000861 \\
710 & 0.0001488 & 0.00008238 & 0.00010288 \\
810 & 0.0001312 & 0.00009086 & 0.00011534 \\
910 & 0.0001614 & 0.00011104 & 0.0001324 \\
1010 & 0.0001858 & 0.00012022 & 0.0001518 \\
1110 & 0.0002164 & 0.0001306 & 0.0001682 \\
1210 & 0.0002476 & 0.0001422 & 0.0001882 \\
1310 & 0.0002782 & 0.0001584 & 0.0002034 \\
1410 & 0.0003156 & 0.0001742 & 0.0002196 \\
1510 & 0.0003848 & 0.0001886 & 0.0002374 \\
1610 & 0.0002852 & 0.0002048 & 0.000255 \\
1710 & 0.000306 & 0.0002128 & 0.000275 \\
1810 & 0.0003416 & 0.0002316 & 0.0002914 \\
1910 & 0.0003666 & 0.000245 & 0.000309 \\
2010 & 0.0003988 & 0.0002604 & 0.0003258 \\
2110 & 0.000425 & 0.000269 & 0.0003468 \\
2210 & 0.0004594 & 0.0002818 & 0.0003614 \\
2310 & 0.0004882 & 0.0003018 & 0.0003804 \\
2410 & 0.0005208 & 0.0003166 & 0.0003994 \\
2510 & 0.000548 & 0.0003292 & 0.00042 \\
2610 & 0.0005954 & 0.0003478 & 0.0004366 \\
2710 & 0.0006232 & 0.0003518 & 0.000458 \\
2810 & 0.0006586 & 0.0003782 & 0.000478 \\
2910 & 0.0007026 & 0.000388 & 0.0004918 \\
3010 & 0.0007398 & 0.0004044 & 0.00051 \\
3110 & 0.0007712 & 0.0004112 & 0.0005274 \\
3210 & 0.0006026 & 0.0004358 & 0.0005508 \\
3310 & 0.0006398 & 0.0004418 & 0.0005698 \\
3410 & 0.0006742 & 0.0004602 & 0.000589 \\
3510 & 0.000686 & 0.000475 & 0.000613 \\
3610 & 0.0007292 & 0.0004834 & 0.0006298 \\
3710 & 0.0007472 & 0.0004986 & 0.0006476 \\
3810 & 0.0007828 & 0.0005252 & 0.0006664 \\
3910 & 0.0008158 & 0.0005432 & 0.0007014 \\
4010 & 0.0008448 & 0.0005416 & 0.000708 \\
4110 & 0.000869 & 0.0005646 & 0.0007234 \\
4210 & 0.0009106 & 0.0005746 & 0.0007476 \\
4310 & 0.0009404 & 0.0005972 & 0.0007706 \\
4410 & 0.0009708 & 0.000634 & 0.0007844 \\
4510 & 0.0010004 & 0.0006348 & 0.0008044 \\
4610 & 0.0010402 & 0.000658 & 0.0008242 \\
4710 & 0.001063 & 0.0006734 & 0.0008434 \\
4810 & 0.001105 & 0.000692 & 0.0008682 \\
4910 & 0.001138 & 0.000696 & 0.000884 \\
5010 & 0.0011494 & 0.0007138 & 0.0008988 \\
5110 & 0.001218 & 0.0007438 & 0.0009244 \\
5210 & 0.001242 & 0.0007508 & 0.0009416 \\
5310 & 0.001248 & 0.0007692 & 0.000963 \\
5410 & 0.001298 & 0.0007818 & 0.0009886 \\
5510 & 0.001812 & 0.0007966 & 0.001003 \\
5610 & 0.001384 & 0.0008218 & 0.0010266 \\
5710 & 0.00143 & 0.00082 & 0.0010322 \\
5810 & 0.00148 & 0.0008394 & 0.0010628 \\
5910 & 0.001504 & 0.0008664 & 0.0010874 \\
6010 & 0.00156 & 0.0008726 & 0.001105 \\
6110 & 0.002034 & 0.000895 & 0.0011246 \\
6210 & 0.001654 & 0.000925 & 0.0011308 \\
6310 & 0.00166 & 0.0009102 & 0.001162 \\
6410 & 0.001314 & 0.0009348 & 0.0012062 \\
6510 & 0.001334 & 0.0009656 & 0.001212 \\
6610 & 0.001362 & 0.0009802 & 0.001238 \\
6710 & 0.001394 & 0.0009954 & 0.001252 \\
6810 & 0.001416 & 0.0009918 & 0.00127 \\
6910 & 0.001452 & 0.0010228 & 0.001294 \\
7010 & 0.00148 & 0.0010612 & 0.001314 \\
7110 & 0.001516 & 0.001054 & 0.001332 \\
7210 & 0.00154 & 0.001065 & 0.001352 \\
7310 & 0.001578 & 0.0010828 & 0.001338 \\
7410 & 0.001598 & 0.001103 & 0.001386 \\
7510 & 0.00162 & 0.001115 & 0.00142 \\
7610 & 0.001778 & 0.0011308 & 0.001438 \\
7710 & 0.00169 & 0.00117 & 0.00146 \\
7810 & 0.00172 & 0.0011822 & 0.001466 \\
7910 & 0.00176 & 0.0011676 & 0.00149 \\
8010 & 0.001806 & 0.001214 & 0.001508 \\
8110 & 0.001812 & 0.001212 & 0.001538 \\
8210 & 0.001848 & 0.001244 & 0.001562 \\
8310 & 0.001894 & 0.001264 & 0.001564 \\
8410 & 0.00191 & 0.001266 & 0.001602 \\
8510 & 0.001962 & 0.001284 & 0.001634 \\
8610 & 0.001972 & 0.001298 & 0.001656 \\
8710 & 0.001992 & 0.001292 & 0.00167 \\
8810 & 0.002106 & 0.001328 & 0.00167 \\
8910 & 0.00207 & 0.001356 & 0.001694 \\
9010 & 0.002108 & 0.001366 & 0.001752 \\
9110 & 0.002132 & 0.001368 & 0.00175 \\
9210 & 0.002184 & 0.0014 & 0.001774 \\
9310 & 0.002216 & 0.001406 & 0.001796 \\
9410 & 0.002262 & 0.001426 & 0.001816 \\
9510 & 0.002308 & 0.001414 & 0.001842 \\
9610 & 0.00233 & 0.001432 & 0.001858 \\
9710 & 0.002356 & 0.001478 & 0.001878 \\
9810 & 0.002408 & 0.00148 & 0.001908 \\
9910 & 0.002448 & 0.001478 & 0.001924 \\

    \end{longtabu}
\end{center}



%----------------------------------------------------------------------------------------
%	ABSTRACT AND KEYWORDS
%----------------------------------------------------------------------------------------

%\renewcommand{\abstractname}{Summary} % Uncomment to change the name of the abstract to something else

%\begin{abstract}
%Morbi tempor congue porta. Proin semper, leo vitae faucibus dictum, metus mauris lacinia lorem, ac congue leo felis eu turpis. Sed nec nunc pellentesque, gravida eros at, porttitor ipsum. Praesent consequat urna a lacus lobortis ultrices eget ac metus. In tempus hendrerit rhoncus. Mauris dignissim turpis id sollicitudin lacinia. Praesent libero tellus, fringilla nec ullamcorper at, ultrices id nulla. Phasellus placerat a tellus a malesuada.
%\end{abstract}

%\hspace*{3,6mm}\textit{Keywords:} lorem , ipsum , dolor , sit amet , lectus % Keywords

%\vspace{30pt} % Some vertical space between the abstract and first section

%----------------------------------------------------------------------------------------
%	ESSAY BODY
%----------------------------------------------------------------------------------------

% \section*{Introduction}
% 
% This statement requires citation \cite{Smith:2012qr}; this one does too \cite{Smith:2013jd}. Lorem ipsum dolor sit amet, consectetur adipiscing elit. Aenean dictum lacus sem, ut varius ante dignissim ac. Sed a mi quis lectus feugiat aliquam. Nunc sed vulputate velit. Sed commodo metus vel felis semper, quis rutrum odio vulputate. Donec a elit porttitor, facilisis nisl sit amet, dignissim arcu. Vivamus accumsan pellentesque nulla at euismod. Duis porta rutrum sem, eu facilisis mi varius sed. Suspendisse potenti. Mauris rhoncus neque nisi, ut laoreet augue pretium luctus. Vestibulum sit amet luctus sem, luctus ultrices leo. Aenean vitae sem leo.
% 
% Nullam semper quam at ante convallis posuere. Ut faucibus tellus ac massa luctus consectetur. Nulla pellentesque tortor et aliquam vehicula. Maecenas imperdiet euismod enim ut pharetra. Suspendisse pulvinar sapien vitae placerat pellentesque. Nulla facilisi. Aenean vitae nunc venenatis, vehicula neque in, congue ligula.
% 
% Pellentesque quis neque fringilla, varius ligula quis, malesuada dolor. Aenean malesuada urna porta, condimentum nisl sed, scelerisque nisi. Suspendisse ac orci quis massa porta dignissim. Morbi sollicitudin, felis eget tristique laoreet, ante lacus pretium lacus, nec ornare sem lorem a velit. Pellentesque eu erat congue, ullamcorper ante ut, tristique turpis. Nam sodales mi sed nisl tincidunt vestibulum. Interdum et malesuada fames ac ante ipsum primis in faucibus.
% 
% %------------------------------------------------
% 
% \section*{Section Name}
% 
% Cras gravida, est vel interdum euismod, tortor mi lobortis mi, quis adipiscing elit lacus ut orci. Phasellus nec fringilla nisi, ut vestibulum neque. Aenean non risus eu nunc accumsan condimentum at sed ipsum.
% \begin{wrapfigure}{l}{0.4\textwidth} % Inline image example
% \begin{center}
% \includegraphics[width=0.38\textwidth]{fish.png}
% \end{center}
% \caption{Fish}
% \end{wrapfigure}
% Aliquam fringilla non diam sed varius. Suspendisse tellus felis, hendrerit non bibendum ut, adipiscing vitae diam. Lorem ipsum dolor sit amet, consectetur adipiscing elit. Nulla lobortis purus eget nisl scelerisque, commodo rhoncus lacus porta. Vestibulum vitae turpis tincidunt, varius dolor in, dictum lectus. Aenean ac ornare augue, ac facilisis purus. Sed leo lorem, molestie sit amet fermentum id, suscipit ut sem. Vestibulum orci arcu, vehicula sed tortor id, ornare dapibus lorem. Praesent aliquet iaculis lacus nec fermentum. Morbi eleifend blandit dolor, pharetra hendrerit neque ornare vel. Nulla ornare, nisl eget imperdiet ornare, libero enim interdum mi, ut lobortis quam velit bibendum nibh.
% 
% Morbi tempor congue porta. Proin semper, leo vitae faucibus dictum, metus mauris lacinia lorem, ac congue leo felis eu turpis. Sed nec nunc pellentesque, gravida eros at, porttitor ipsum. Praesent consequat urna a lacus lobortis ultrices eget ac metus. In tempus hendrerit rhoncus. Mauris dignissim turpis id sollicitudin lacinia. Praesent libero tellus, fringilla nec ullamcorper at, ultrices id nulla. Phasellus placerat a tellus a malesuada.
% 
% %------------------------------------------------
% 
% \section*{Conclusion}
% 
% Fusce in nibh augue. Cum sociis natoque penatibus et magnis dis parturient montes, nascetur ridiculus mus. In dictum accumsan sapien, ut hendrerit nisi. Phasellus ut nulla mauris. Phasellus sagittis nec odio sed posuere. Vestibulum porttitor dolor quis suscipit bibendum. Mauris risus lectus, cursus vitae hendrerit posuere, congue ac est. Suspendisse commodo eu eros non cursus. Mauris ultrices venenatis dolor, sed aliquet odio tempor pellentesque. Duis ultricies, mauris id lobortis vulputate, tellus turpis eleifend elit, in gravida leo tortor ultricies est. Maecenas vitae ipsum at dui sodales condimentum a quis dui. Nam mi sapien, lobortis ac blandit eget, dignissim quis nunc.
% 
% \begin{enumerate}
% \item First numbered list item
% \item Second numbered list item
% \end{enumerate}
% 
% Donec luctus tincidunt mauris, non ultrices ligula aliquam id. Sed varius, magna a faucibus congue, arcu tellus pellentesque nisl, vel laoreet magna eros et magna. Vivamus lobortis elit eu dignissim ultrices. Fusce erat nulla, ornare at dolor quis, rhoncus venenatis velit. Donec sed elit mi. Sed semper tellus a convallis viverra. Maecenas mi lorem, placerat sit amet sem quis, adipiscing tincidunt turpis. Cras a urna et tellus dictum eleifend. Fusce dignissim lectus risus, in bibendum tortor lacinia interdum.

%----------------------------------------------------------------------------------------
%	BIBLIOGRAPHY
%----------------------------------------------------------------------------------------

% \bibliographystyle{unsrt}
% 
% \bibliography{sample}

%----------------------------------------------------------------------------------------



\end{document}