%%%
% Modificación de una plantilla de Latex para adaptarla al castellano.
%%%

%%%%%%%%%%%%%%%%%%%%%%%%%%%%%%%%%%%%%%%%%
% Thin Sectioned Essay
% LaTeX Template
% Version 1.0 (3/8/13)
%
% This template has been downloaded from:
% http://www.LaTeXTemplates.com
%
% Original Author:
% Nicolas Diaz (nsdiaz@uc.cl) with extensive modifications by:
% Vel (vel@latextemplates.com)
%
% License:
% CC BY-NC-SA 3.0 (http://creativecommons.org/licenses/by-nc-sa/3.0/)
%
%%%%%%%%%%%%%%%%%%%%%%%%%%%%%%%%%%%%%%%%%

%----------------------------------------------------------------------------------------
%	PACKAGES AND OTHER DOCUMENT CONFIGURATIONS
%----------------------------------------------------------------------------------------

\documentclass[a4paper, 11pt]{article} % Font size (can be 10pt, 11pt or 12pt) and paper size (remove a4paper for US letter paper)

\usepackage[protrusion=true,expansion=true]{microtype} % Better typography
\usepackage{graphicx} % Required for including pictures
\usepackage[usenames,dvipsnames]{color} % Coloring code
\usepackage{wrapfig} % Allows in-line images
% Sorry, no me compila el LaTeX con estas opciones, descomentadlas para compilarlo vosotros si queréis -David
%\usepackage[spanish]{babel} % English language/hyphenation
%\selectlanguage{spanish}
\usepackage[utf8]{inputenc}
\usepackage{dcolumn} % Para el alineamiento del punto decimal
\newcolumntype{d}[1]{D{.}{.}{#1} } % Idem
\usepackage{longtable}
\usepackage{tabu}

\usepackage[section]{placeins} % Para gráficas en su sección.
\usepackage{mathpazo} % Use the Palatino font
\usepackage[T1]{fontenc} % Required for accented characters
\newenvironment{allintypewriter}{\ttfamily}{\par}
\linespread{1.05} % Change line spacing here, Palatino benefits from a slight increase by default

\makeatletter
\renewcommand\@biblabel[1]{\textbf{#1.}} % Change the square brackets for each bibliography item from '[1]' to '1.'
\renewcommand{\@listI}{\itemsep=0pt} % Reduce the space between items in the itemize and enumerate environments and the bibliography
\newcommand{\imagen}[2]{\begin{figure*}[ht!] \centering \includegraphics[width=90mm]{#1} \\ #2 \end{figure*}}


\renewcommand{\maketitle}{ % Customize the title - do not edit title and author name here, see the TITLE block below
\begin{flushright} % Right align
{\LARGE\@title} % Increase the font size of the title

\vspace{50pt} % Some vertical space between the title and author name

{\large\@author} % Author name
\\\@date % Date

\vspace{40pt} % Some vertical space between the author block and abstract
\end{flushright}
}

%----------------------------------------------------------------------------------------
%	TITLE
%----------------------------------------------------------------------------------------

\title{\textbf{Práctica 1}\\ % Title
Análisis de eficiencia} % Subtitle

\author{\textsc{Óscar Bermúdez,\\Francisco David Charte,\\Ignacio Cordón,\\José Carlos Entrena,\\Mario Román} % Author
\\{\textit{Universidad de Granada}}} % Institution

\date{\today} % Date

%----------------------------------------------------------------------------------------

\begin{document}

\maketitle % Print the title section

\renewcommand{\abstractname}{Resumen} % Uncomment to change the name of the abstract to something else
\begin{abstract}
La eficiencia de los algoritmos se puede medir de forma teórica 
y empírica. En esta memoria se recogen estudios acerca de la 
eficiencia de algunos algoritmos muy conocidos, y se exponen 
implementaciones de los mismos y datos de su ejecución en 
distintas máquinas y con diferentes niveles de optimización.
Se representan estos resultados en una serie de gráficas, y por 
último, se resume la información obtenida en unas conclusiones 
finales.
\end{abstract}
\tableofcontents

\pagebreak

\section {Preparación del código}
%Hay un problema con el espacio entre estos párrafos, que no deja separación enre ambos. Vale asi?
Con motivo de facilitarnos un poco el trabajo a la hora de realizar esta práctica,
hemos añadido algunos archivos y realizado ciertas modificaciones en los códigos, que
detallamos a continuación: 

\medskip
Inicialmente, hemos hecho algunos cambios léxicos en el código de los programas, con el
objetivo de que sean más legibles para nosotros y poder comprenderlos mejor. También hemos
cambiado la función que se utilizaba para medir el tiempo por una con más precisión, lo 
que nos permite ejecutar los algoritmos con un volumen de entrada menor. 

Para facilitarnos la compilación y ejecución de los programas, hemos creado un archivo 
makefile que compila los códigos y genera los datos y gráficas necesarios para la práctica.

Además, para la generación de dichas gráficas, hemos escrito un programa en bash que usa gnuplot 
para obtenerlas, al igual que los ajustes de la eficiencia. 

\section {Análisis teórico}
\subsection {Ordenación de la burbuja}
El algoritmo lo forman dos bucles anidados ejecutando un conjunto de sentencias elementales, $\mathcal{O}(1)$.
El bucle externo se repite $n$ veces y el interno depende del índice del bucle externo. Ejecutándose en total:
\begin{equation}
 T(n) = \sum_{i=0}^n i = \frac{n(n+1)}{2} = \frac{n^2}{2} + \frac{n}{2}
\end{equation}
por lo que el algoritmo es $\mathcal{O}(n^2)$.

\subsection{Ordenación por inserción}
De nuevo, el algoritmo está compuesto por dos bucles anidados, de los cuales el primero ejecuta $n-1$ iteraciones y el segundo realiza como máximo $i$ (el índice del primer bucle). La función que nos da el tiempo de ejecución en el peor caso es, por tanto:
\begin{equation}
 T(n) = \sum_{i=0}^{n-1} i = \frac{(n-1)n}{2} = \frac{n^2}{2} - \frac{n}{2}
\end{equation}
y el algoritmo es $\mathcal{O}(n^2)$ en el caso peor.
\subsection{Ordenación por selección}
Al igual que los anteriores, consta de dos bucles anidados. El exterior realiza $n-1$ iteraciones, y el interior tantas como indica el índice del primero, $i$, lo que nos da un tiempo de ejecución:
\begin{equation}
 T(n) = \sum_{i=0}^{n-1} i = \frac{(n-1)n}{2} = \frac{n^2}{2} - \frac{n}{2}
\end{equation}
por lo que este algoritmo también es $\mathcal{O}(n^2)$ en el caso peor.

\subsection{Ordenación \textit{heapsort}}
En este algoritmo insertaremos los datos en un árbol tipo heap y los extraeremos de nuevo ya ordenados.
La inserción en una estructura heap es del orden $\mathcal{O}(log n)$ y la extracción del mínimo, que involucra
reajustar el árbol, es también $\mathcal{O}(log n)$. En conjunto, y sabiendo que insertaremos y extraeremos $n$ elementos,
el orden del algoritmo es:
\begin{equation}
 T(n) = 2 n log(n)
\end{equation}
por lo que el algoritmo tiene complejidad $\mathcal{O}(nlog(n))$.


\subsection{Ordenación por mezcla (\textit{mergesort})}
El algoritmo mergesort es un algoritmo divide y vencerás para ordenación, basado en la ordenación de las dos mitades del
array independientemente. En nuestro caso usaremos \textit{insertion sort} para los problemas que quedan por debajo de una talla umbral.
Sabiendo que la ordenación en un caso suficientemente pequeño se acota por una constante y que el algoritmo recurre a la ordenación
de dos casos de tamaño mitad y una mezcla de ambos lineal:
\begin{equation}
 T(n) = 2T\left(\frac{n}{2}\right) + n
\end{equation}
Que tiene soluciones del tipo $an + bnlog(n)$, el algoritmo es de complejidad $\mathcal{O}(nlog(n))$.


\subsection{Algoritmo de Hoare (\textit{quicksort})}
El algoritmo quicksort toma un pivote para partir los elementos a ordenar en aquellos mayores y aquellos menores que el pivote.
En su caso medio, promediaremos las elecciones del pivote. Siempre se tendrá una fase de complejidad lineal previa en la que los
elementos se particionan por el pivote, y, posteriormente, habrá una fase en la que se analizarán las otras dos mitades. Tenemos:
\begin{equation}
 T(n) = n - 1 + \frac{1}{n}\left(\sum_{i=0}^{n-1} T(i) + T(n-1-i) \right)
\end{equation}
Que tiene una solución de complejidad $\mathcal{O}(nlog(n))$.


\subsection {Sucesión de Fibonacci}
El algoritmo está escrito recursivamente y sin memoización. La llamada a la función con tamaño $n$ produce 
dos llamadas recursivas con tamaños $n-1$ y $n-2$. Tendremos la ecuación en diferencias:
\begin{equation}
 T(n) = T(n-1) + T(n-2)
\end{equation}
con la solución conocida de complejidad $\mathcal{O}(\phi^n)$.

\subsection{Algoritmo de Floyd}
El algoritmo de Floyd guarda en una matriz los actuales caminos mínimos entre dos vértices de un grafo de $n$ vértices
y realiza un intercambio entre el camino de dos vértices si este puede mejorarse pasando por el vértice $n$. En resumen,
tendremos tres bucles anidados realizando una operación constante. Cada uno de ellos realizando $n$ iteraciones.

La complejidad del algoritmo es $\mathcal{O}(n^3)$.


\subsection{Torres de Hanoi}
El algoritmo de las torres de Hanoi de talla $n$ se llama a sí mismo recursivamente dos veces y cada una de ellas con
talla $n-1$. Resuelve la torre para los $n-1$ discos superiores, mueve la base, y la resuelve de nuevo para colocarlos sobre ella.
Tendremos entonces como ecuación principal:
\begin{equation}
 T(n) = 2T(n-1) + 1
\end{equation}
Con soluciones de complejidad $\mathcal{O}(2^n)$.

\section {Análisis práctico}
\subsection{Códigos modificados}
A continuación se incluyen los códigos modificados que se han usado en la práctica.
\scriptsize {
\begin{allintypewriter}

\subsubsection{Ordenación de la burbuja}
% Generator: GNU source-highlight, by Lorenzo Bettini, http://www.gnu.org/software/src-highlite
\noindent
\mbox{}\textit{\textcolor{Brown}{/**}} \\
\mbox{}\textit{\textcolor{Brown}{\ *\ }}\textcolor{ForestGreen}{@file}\textit{\textcolor{Brown}{\ Ordenación\ por\ burbuja}} \\
\mbox{}\textit{\textcolor{Brown}{\ */}} \\
\mbox{} \\
\mbox{}\textbf{\textcolor{RoyalBlue}{\#include}}\ \texttt{\textcolor{Red}{\textless{}iostream\textgreater{}}} \\
\mbox{}\textbf{\textcolor{RoyalBlue}{\#include}}\ \texttt{\textcolor{Red}{\textless{}ctime\textgreater{}}} \\
\mbox{}\textbf{\textcolor{RoyalBlue}{\#include}}\ \texttt{\textcolor{Red}{\textless{}cstdlib\textgreater{}}} \\
\mbox{}\textbf{\textcolor{RoyalBlue}{\#include}}\ \texttt{\textcolor{Red}{\textless{}ctime\textgreater{}}} \\
\mbox{}\textbf{\textcolor{Blue}{using}}\ \textbf{\textcolor{Blue}{namespace}}\ std\textcolor{BrickRed}{;} \\
\mbox{} \\
\mbox{}\textit{\textcolor{Brown}{/**}} \\
\mbox{}\textit{\textcolor{Brown}{\ *\ }}\textcolor{ForestGreen}{@brief}\textit{\textcolor{Brown}{\ Ordena\ un\ vector\ por\ el\ método\ de\ la\ burbuja.}} \\
\mbox{}\textit{\textcolor{Brown}{\ *\ }}\textcolor{ForestGreen}{@param}\textit{\textcolor{Brown}{\ T:\ vector\ de\ elementos.\ Debe\ tener\ num$\_$elem\ elementos.}} \\
\mbox{}\textit{\textcolor{Brown}{\ *\ Es\ modificado.}} \\
\mbox{}\textit{\textcolor{Brown}{\ *\ }}\textcolor{ForestGreen}{@param}\textit{\textcolor{Brown}{\ num$\_$elem:\ número\ de\ elementos.\ num$\_$elem\ \textgreater{}\ 0.}} \\
\mbox{}\textit{\textcolor{Brown}{\ *\ }} \\
\mbox{}\textit{\textcolor{Brown}{\ *\ Cambia\ el\ orden\ de\ los\ elementos\ de\ T\ de\ forma\ que\ los\ dispone}} \\
\mbox{}\textit{\textcolor{Brown}{\ *\ en\ sentido\ creciente\ de\ menor\ a\ mayor.}} \\
\mbox{}\textit{\textcolor{Brown}{\ *\ Aplica\ el\ algoritmo\ de\ la\ burbuja.}} \\
\mbox{}\textit{\textcolor{Brown}{\ */}} \\
\mbox{} \\
\mbox{}\textbf{\textcolor{Blue}{inline}}\ \textbf{\textcolor{Blue}{static}}\ \textcolor{ForestGreen}{void}\ \textbf{\textcolor{Black}{burbuja}}\textcolor{BrickRed}{(}\textcolor{ForestGreen}{int}\ T\textcolor{BrickRed}{[],}\ \textcolor{ForestGreen}{int}\ num$\_$elem\textcolor{BrickRed}{);} \\
\mbox{} \\
\mbox{}\textit{\textcolor{Brown}{/**}} \\
\mbox{}\textit{\textcolor{Brown}{\ *\ }}\textcolor{ForestGreen}{@brief}\textit{\textcolor{Brown}{\ Ordena\ parte\ de\ un\ vector\ por\ el\ método\ de\ la\ burbuja.}} \\
\mbox{}\textit{\textcolor{Brown}{\ *\ }}\textcolor{ForestGreen}{@param}\textit{\textcolor{Brown}{\ T:\ vector\ de\ elementos.\ Tiene\ un\ número\ de\ elementos}} \\
\mbox{}\textit{\textcolor{Brown}{\ *\ mayor\ o\ igual\ a\ final.Es\ MODIFICADO.}} \\
\mbox{}\textit{\textcolor{Brown}{\ *\ }}\textcolor{ForestGreen}{@param}\textit{\textcolor{Brown}{\ inicial:\ Posición\ que\ marca\ el\ incio\ de\ la\ parte\ del}} \\
\mbox{}\textit{\textcolor{Brown}{\ *\ vector\ a\ ordenar.}} \\
\mbox{}\textit{\textcolor{Brown}{\ *\ }}\textcolor{ForestGreen}{@param}\textit{\textcolor{Brown}{\ final:\ Posición\ detrás\ de\ la\ última\ de\ la\ parte\ del}} \\
\mbox{}\textit{\textcolor{Brown}{\ *\ vector\ a\ ordenar.}} \\
\mbox{}\textit{\textcolor{Brown}{\ *\ inicial\ \textless{}\ final.}} \\
\mbox{}\textit{\textcolor{Brown}{\ *\ }} \\
\mbox{}\textit{\textcolor{Brown}{\ *\ Cambia\ el\ orden\ de\ los\ elementos\ de\ T\ entre\ las\ posiciones}} \\
\mbox{}\textit{\textcolor{Brown}{\ *\ inicial\ y\ final\ -\ 1de\ forma\ que\ los\ dispone\ en\ sentido\ creciente}} \\
\mbox{}\textit{\textcolor{Brown}{\ *\ de\ menor\ a\ mayor.}} \\
\mbox{}\textit{\textcolor{Brown}{\ *\ Aplica\ el\ algoritmo\ de\ la\ burbuja.}} \\
\mbox{}\textit{\textcolor{Brown}{\ */}} \\
\mbox{} \\
\mbox{}\textbf{\textcolor{Blue}{static}}\ \textcolor{ForestGreen}{void}\ \textbf{\textcolor{Black}{burbuja$\_$lims}}\textcolor{BrickRed}{(}\textcolor{ForestGreen}{int}\ T\textcolor{BrickRed}{[],}\ \textcolor{ForestGreen}{int}\ inicial\textcolor{BrickRed}{,}\ \textcolor{ForestGreen}{int}\ final\textcolor{BrickRed}{);} \\
\mbox{} \\
\mbox{}\textit{\textcolor{Brown}{//\ Implementación}} \\
\mbox{} \\
\mbox{}\textbf{\textcolor{Blue}{inline}}\ \textcolor{ForestGreen}{void}\ \textbf{\textcolor{Black}{burbuja}}\textcolor{BrickRed}{(}\textcolor{ForestGreen}{int}\ T\textcolor{BrickRed}{[],}\ \textcolor{ForestGreen}{int}\ num$\_$elem\textcolor{BrickRed}{)}\textcolor{Red}{\{} \\
\mbox{}\ \ \ \ \textbf{\textcolor{Black}{burbuja$\_$lims}}\textcolor{BrickRed}{(}T\textcolor{BrickRed}{,}\ \textcolor{Purple}{0}\textcolor{BrickRed}{,}\ num$\_$elem\textcolor{BrickRed}{);} \\
\mbox{}\textcolor{Red}{\}} \\
\mbox{} \\
\mbox{}\textbf{\textcolor{Blue}{static}}\ \textcolor{ForestGreen}{void}\ \textbf{\textcolor{Black}{burbuja$\_$lims}}\textcolor{BrickRed}{(}\textcolor{ForestGreen}{int}\ T\textcolor{BrickRed}{[],}\ \textcolor{ForestGreen}{int}\ inicial\textcolor{BrickRed}{,}\ \textcolor{ForestGreen}{int}\ final\textcolor{BrickRed}{)}\textcolor{Red}{\{} \\
\mbox{}\ \ \ \ \textcolor{ForestGreen}{int}\ i\textcolor{BrickRed}{,}\ j\textcolor{BrickRed}{;} \\
\mbox{}\ \ \ \ \textcolor{ForestGreen}{int}\ aux\textcolor{BrickRed}{;} \\
\mbox{}\ \ \ \ \textbf{\textcolor{Blue}{for}}\ \textcolor{BrickRed}{(}i\ \textcolor{BrickRed}{=}\ inicial\textcolor{BrickRed}{;}\ i\ \textcolor{BrickRed}{\textless{}}\ final\ \textcolor{BrickRed}{-}\ \textcolor{Purple}{1}\textcolor{BrickRed}{;}\ i\textcolor{BrickRed}{++)} \\
\mbox{}\ \ \ \ \ \ \ \ \textbf{\textcolor{Blue}{for}}\ \textcolor{BrickRed}{(}j\ \textcolor{BrickRed}{=}\ final\ \textcolor{BrickRed}{-}\ \textcolor{Purple}{1}\textcolor{BrickRed}{;}\ j\ \textcolor{BrickRed}{\textgreater{}}\ i\textcolor{BrickRed}{;}\ j\textcolor{BrickRed}{-\/-)} \\
\mbox{}\ \ \ \ \ \ \ \ \ \ \ \ \textbf{\textcolor{Blue}{if}}\ \textcolor{BrickRed}{(}T\textcolor{BrickRed}{[}j\textcolor{BrickRed}{]}\ \textcolor{BrickRed}{\textless{}}\ T\textcolor{BrickRed}{[}j\textcolor{BrickRed}{-}\textcolor{Purple}{1}\textcolor{BrickRed}{])} \\
\mbox{}\ \ \ \ \ \ \ \ \ \ \ \ \textcolor{Red}{\{} \\
\mbox{}\ \ \ \ \ \ \ \ \ \ \ \ \ \ \ \ aux\ \textcolor{BrickRed}{=}\ T\textcolor{BrickRed}{[}j\textcolor{BrickRed}{];} \\
\mbox{}\ \ \ \ \ \ \ \ \ \ \ \ \ \ \ \ T\textcolor{BrickRed}{[}j\textcolor{BrickRed}{]}\ \textcolor{BrickRed}{=}\ T\textcolor{BrickRed}{[}j\textcolor{BrickRed}{-}\textcolor{Purple}{1}\textcolor{BrickRed}{];} \\
\mbox{}\ \ \ \ \ \ \ \ \ \ \ \ \ \ \ \ T\textcolor{BrickRed}{[}j\textcolor{BrickRed}{-}\textcolor{Purple}{1}\textcolor{BrickRed}{]}\ \textcolor{BrickRed}{=}\ aux\textcolor{BrickRed}{;} \\
\mbox{}\ \ \ \ \ \ \ \ \ \ \ \ \textcolor{Red}{\}} \\
\mbox{}\textcolor{Red}{\}} \\
\mbox{} \\
\mbox{}\textcolor{ForestGreen}{int}\ \textbf{\textcolor{Black}{main}}\textcolor{BrickRed}{(}\textcolor{ForestGreen}{int}\ argc\textcolor{BrickRed}{,}\ \textcolor{ForestGreen}{char}\textcolor{BrickRed}{*}\ argv\textcolor{BrickRed}{[])}\textcolor{Red}{\{} \\
\mbox{}\ \ \ \ \textbf{\textcolor{Blue}{if}}\ \textcolor{BrickRed}{(}argc\ \textcolor{BrickRed}{!=}\textcolor{Purple}{2}\textcolor{BrickRed}{)}\textcolor{Red}{\{} \\
\mbox{}\ \ \ \ \ \ \ \ cerr\ \textcolor{BrickRed}{\textless{}\textless{}}\ \texttt{\textcolor{Red}{"{}Uso\ del\ programa:\ "{}}}\ \textcolor{BrickRed}{+}\ \textcolor{BrickRed}{(}string\textcolor{BrickRed}{)(}argv\textcolor{BrickRed}{[}\textcolor{Purple}{0}\textcolor{BrickRed}{])}\ \textcolor{BrickRed}{+}\ \texttt{\textcolor{Red}{"{}\ \textless{}número\ positivo\textgreater{}"{}}}\ \textcolor{BrickRed}{\textless{}\textless{}}\ endl\textcolor{BrickRed}{;}\ \  \\
\mbox{}\ \ \ \ \ \ \ \ \textbf{\textcolor{Blue}{return}}\ \textcolor{BrickRed}{-}\textcolor{Purple}{1}\textcolor{BrickRed}{;} \\
\mbox{}\ \ \ \ \textcolor{Red}{\}} \\
\mbox{}\ \ \ \ \textcolor{ForestGreen}{int}\ n\ \textcolor{BrickRed}{=}\ \textbf{\textcolor{Black}{atoi}}\textcolor{BrickRed}{(}argv\textcolor{BrickRed}{[}\textcolor{Purple}{1}\textcolor{BrickRed}{]);}\ \ \ \  \\
\mbox{}\ \ \ \ \textbf{\textcolor{Blue}{if}}\ \textcolor{BrickRed}{(}n\textcolor{BrickRed}{\textless{}}\textcolor{Purple}{0}\textcolor{BrickRed}{)}\ \textbf{\textcolor{Blue}{return}}\ \textcolor{BrickRed}{-}\textcolor{Purple}{1}\textcolor{BrickRed}{;} \\
\mbox{}\ \ \ \  \\
\mbox{}\ \ \ \ \textcolor{ForestGreen}{int}\ \textcolor{BrickRed}{*}\ T\ \textcolor{BrickRed}{=}\ \textbf{\textcolor{Blue}{new}}\ \textcolor{ForestGreen}{int}\textcolor{BrickRed}{[}n\textcolor{BrickRed}{];} \\
\mbox{}\ \ \ \ \textbf{\textcolor{Blue}{struct}}\ \textcolor{TealBlue}{timespec}\ t$\_$antes\textcolor{BrickRed}{,}\ t$\_$despues\textcolor{BrickRed}{;} \\
\mbox{}\ \ \ \  \\
\mbox{}\ \ \ \ \textbf{\textcolor{Black}{srandom}}\textcolor{BrickRed}{(}\textbf{\textcolor{Black}{time}}\textcolor{BrickRed}{(}\textcolor{Purple}{0}\textcolor{BrickRed}{));} \\
\mbox{}\ \ \ \  \\
\mbox{}\ \ \ \ \textbf{\textcolor{Blue}{for}}\ \textcolor{BrickRed}{(}\textcolor{ForestGreen}{int}\ i\textcolor{BrickRed}{=}\textcolor{Purple}{0}\textcolor{BrickRed}{;}\ i\textcolor{BrickRed}{\textless{}}n\textcolor{BrickRed}{;}\ \textcolor{BrickRed}{++}i\textcolor{BrickRed}{)}\textcolor{Red}{\{} \\
\mbox{}\ \ \ \ \ \ \ \ T\textcolor{BrickRed}{[}i\textcolor{BrickRed}{]}\ \textcolor{BrickRed}{=}\ \textbf{\textcolor{Black}{random}}\textcolor{BrickRed}{();} \\
\mbox{}\ \ \ \ \textcolor{Red}{\}} \\
\mbox{}\ \ \ \  \\
\mbox{}\ \ \ \ \textbf{\textcolor{Black}{clock$\_$gettime}}\textcolor{BrickRed}{(}CLOCK$\_$REALTIME\textcolor{BrickRed}{,\&}t$\_$antes\textcolor{BrickRed}{);} \\
\mbox{}\ \ \ \ \textbf{\textcolor{Black}{burbuja}}\ \textcolor{BrickRed}{(}T\textcolor{BrickRed}{,}n\textcolor{BrickRed}{);} \\
\mbox{}\ \ \ \ \textbf{\textcolor{Black}{clock$\_$gettime}}\textcolor{BrickRed}{(}CLOCK$\_$REALTIME\textcolor{BrickRed}{,\&}t$\_$despues\textcolor{BrickRed}{);} \\
\mbox{}\ \ \ \  \\
\mbox{}\ \ \ \ cout\textcolor{BrickRed}{.}\textbf{\textcolor{Black}{precision}}\textcolor{BrickRed}{(}\textcolor{Purple}{3}\textcolor{BrickRed}{);} \\
\mbox{}\ \ \ \ cout\ \textcolor{BrickRed}{\textless{}\textless{}}\ \textcolor{BrickRed}{(}\textcolor{ForestGreen}{double}\textcolor{BrickRed}{)}\ \textcolor{BrickRed}{(}t$\_$despues\textcolor{BrickRed}{.}tv$\_$sec\textcolor{BrickRed}{-}t$\_$antes\textcolor{BrickRed}{.}tv$\_$sec\textcolor{BrickRed}{)+} \\
\mbox{}\ \ \ \ \ \ \ \ \textcolor{BrickRed}{(}\textcolor{ForestGreen}{double}\textcolor{BrickRed}{)}\ \textcolor{BrickRed}{((}t$\_$despues\textcolor{BrickRed}{.}tv$\_$nsec\textcolor{BrickRed}{-}t$\_$antes\textcolor{BrickRed}{.}tv$\_$nsec\textcolor{BrickRed}{)/(}\textcolor{Purple}{1}\textcolor{BrickRed}{.}e\textcolor{BrickRed}{+}\textcolor{Purple}{9}\textcolor{BrickRed}{))}\ \textcolor{BrickRed}{\textless{}\textless{}}\ endl\textcolor{BrickRed}{;} \\
\mbox{} \\
\mbox{}\ \ \ \  \\
\mbox{}\ \ \ \ \textbf{\textcolor{Blue}{delete}}\ \textcolor{BrickRed}{[]}\ T\textcolor{BrickRed}{;} \\
\mbox{}\ \ \ \  \\
\mbox{}\ \ \ \ \textbf{\textcolor{Blue}{return}}\ \textcolor{Purple}{0}\textcolor{BrickRed}{;} \\
\mbox{}\textcolor{Red}{\}}


\subsubsection{Ordenación por inserción}
% Generator: GNU source-highlight, by Lorenzo Bettini, http://www.gnu.org/software/src-highlite
\noindent
\mbox{}\textit{\textcolor{Brown}{/**}} \\
\mbox{}\textit{\textcolor{Brown}{\ *\ }}\textcolor{ForestGreen}{@file}\textit{\textcolor{Brown}{\ Ordenación\ por\ inserción}} \\
\mbox{}\textit{\textcolor{Brown}{\ */}} \\
\mbox{} \\
\mbox{}\textbf{\textcolor{RoyalBlue}{\#include}}\ \texttt{\textcolor{Red}{\textless{}iostream\textgreater{}}} \\
\mbox{}\textbf{\textcolor{RoyalBlue}{\#include}}\ \texttt{\textcolor{Red}{\textless{}ctime\textgreater{}}} \\
\mbox{}\textbf{\textcolor{RoyalBlue}{\#include}}\ \texttt{\textcolor{Red}{\textless{}cstdlib\textgreater{}}} \\
\mbox{}\textbf{\textcolor{RoyalBlue}{\#include}}\ \texttt{\textcolor{Red}{\textless{}ctime\textgreater{}}} \\
\mbox{}\textbf{\textcolor{Blue}{using}}\ \textbf{\textcolor{Blue}{namespace}}\ std\textcolor{BrickRed}{;} \\
\mbox{} \\
\mbox{}\textbf{\textcolor{RoyalBlue}{\#define}}\ NUM$\_$VECES\ \textcolor{Purple}{500} \\
\mbox{}\textit{\textcolor{Brown}{/**}} \\
\mbox{}\textit{\textcolor{Brown}{\ *\ }}\textcolor{ForestGreen}{@brief}\textit{\textcolor{Brown}{\ Ordena\ un\ vector\ por\ el\ método\ de\ inserción.}} \\
\mbox{}\textit{\textcolor{Brown}{\ *\ }}\textcolor{ForestGreen}{@param}\textit{\textcolor{Brown}{\ T:\ vector\ de\ elementos.\ Debe\ tener\ num$\_$elem\ elementos.}} \\
\mbox{}\textit{\textcolor{Brown}{\ *\ Es\ modificado.}} \\
\mbox{}\textit{\textcolor{Brown}{\ *\ }}\textcolor{ForestGreen}{@param}\textit{\textcolor{Brown}{\ num$\_$elem:\ número\ de\ elementos.\ num$\_$elem\ \textgreater{}\ 0.}} \\
\mbox{}\textit{\textcolor{Brown}{\ *\ }} \\
\mbox{}\textit{\textcolor{Brown}{\ *\ Cambia\ el\ orden\ de\ los\ elementos\ de\ T\ de\ forma\ que\ los\ dispone}} \\
\mbox{}\textit{\textcolor{Brown}{\ *\ en\ sentido\ creciente\ de\ menor\ a\ mayor.}} \\
\mbox{}\textit{\textcolor{Brown}{\ *\ Aplica\ el\ algoritmo\ de\ inserción.}} \\
\mbox{}\textit{\textcolor{Brown}{\ */}} \\
\mbox{} \\
\mbox{}\textbf{\textcolor{Blue}{inline}}\ \textbf{\textcolor{Blue}{static}}\ \textcolor{ForestGreen}{void}\ \textbf{\textcolor{Black}{insercion}}\textcolor{BrickRed}{(}\textcolor{ForestGreen}{int}\ T\textcolor{BrickRed}{[],}\ \textcolor{ForestGreen}{int}\ num$\_$elem\textcolor{BrickRed}{);} \\
\mbox{} \\
\mbox{}\textit{\textcolor{Brown}{/**}} \\
\mbox{}\textit{\textcolor{Brown}{\ *\ }}\textcolor{ForestGreen}{@brief}\textit{\textcolor{Brown}{\ Ordena\ parte\ de\ un\ vector\ por\ el\ método\ de\ inserción.}} \\
\mbox{}\textit{\textcolor{Brown}{\ *\ }}\textcolor{ForestGreen}{@param}\textit{\textcolor{Brown}{\ T:\ vector\ de\ elementos.\ Tiene\ un\ número\ de\ elementos\ }} \\
\mbox{}\textit{\textcolor{Brown}{\ *\ mayor\ o\ igual\ a\ final.\ Es\ MODIFICADO.}} \\
\mbox{}\textit{\textcolor{Brown}{\ *\ }}\textcolor{ForestGreen}{@param}\textit{\textcolor{Brown}{\ inicial:\ Posición\ que\ marca\ el\ incio\ de\ la\ parte\ del}} \\
\mbox{}\textit{\textcolor{Brown}{\ *\ vector\ a\ ordenar.}} \\
\mbox{}\textit{\textcolor{Brown}{\ *\ }}\textcolor{ForestGreen}{@param}\textit{\textcolor{Brown}{\ final:\ Posición\ detrás\ de\ la\ última\ de\ la\ parte\ del}} \\
\mbox{}\textit{\textcolor{Brown}{\ *\ vector\ a\ ordenar.\ }} \\
\mbox{}\textit{\textcolor{Brown}{\ *\ }}\textcolor{ForestGreen}{@pre}\textit{\textcolor{Brown}{\ inicial\ \textless{}\ final.}} \\
\mbox{}\textit{\textcolor{Brown}{\ *\ }} \\
\mbox{}\textit{\textcolor{Brown}{\ *\ Cambia\ el\ orden\ de\ los\ elementos\ de\ T\ entre\ las\ posiciones}} \\
\mbox{}\textit{\textcolor{Brown}{\ *\ inicial\ y\ final\ -\ 1de\ forma\ que\ los\ dispone\ en\ sentido\ creciente}} \\
\mbox{}\textit{\textcolor{Brown}{\ *\ de\ menor\ a\ mayor.}} \\
\mbox{}\textit{\textcolor{Brown}{\ *\ Aplica\ el\ algoritmo\ de\ inserción.}} \\
\mbox{}\textit{\textcolor{Brown}{\ */}} \\
\mbox{} \\
\mbox{}\textbf{\textcolor{Blue}{static}}\ \textcolor{ForestGreen}{void}\ \textbf{\textcolor{Black}{insercion$\_$lims}}\textcolor{BrickRed}{(}\textcolor{ForestGreen}{int}\ T\textcolor{BrickRed}{[],}\ \textcolor{ForestGreen}{int}\ inicial\textcolor{BrickRed}{,}\ \textcolor{ForestGreen}{int}\ final\textcolor{BrickRed}{);} \\
\mbox{} \\
\mbox{}\textit{\textcolor{Brown}{//\ Implementación\ de\ las\ funciones}} \\
\mbox{} \\
\mbox{}\textbf{\textcolor{Blue}{inline}}\ \textbf{\textcolor{Blue}{static}}\ \textcolor{ForestGreen}{void}\ \textbf{\textcolor{Black}{insercion}}\textcolor{BrickRed}{(}\textcolor{ForestGreen}{int}\ T\textcolor{BrickRed}{[],}\ \textcolor{ForestGreen}{int}\ num$\_$elem\textcolor{BrickRed}{)}\textcolor{Red}{\{} \\
\mbox{}\ \ \ \ \textbf{\textcolor{Black}{insercion$\_$lims}}\textcolor{BrickRed}{(}T\textcolor{BrickRed}{,}\ \textcolor{Purple}{0}\textcolor{BrickRed}{,}\ num$\_$elem\textcolor{BrickRed}{);} \\
\mbox{}\textcolor{Red}{\}} \\
\mbox{} \\
\mbox{}\textbf{\textcolor{Blue}{static}}\ \textcolor{ForestGreen}{void}\ \textbf{\textcolor{Black}{insercion$\_$lims}}\textcolor{BrickRed}{(}\textcolor{ForestGreen}{int}\ T\textcolor{BrickRed}{[],}\ \textcolor{ForestGreen}{int}\ inicial\textcolor{BrickRed}{,}\ \textcolor{ForestGreen}{int}\ final\textcolor{BrickRed}{)}\textcolor{Red}{\{} \\
\mbox{}\ \ \ \ \textcolor{ForestGreen}{int}\ i\textcolor{BrickRed}{,}\ j\textcolor{BrickRed}{;} \\
\mbox{}\ \ \ \ \textcolor{ForestGreen}{int}\ aux\textcolor{BrickRed}{;} \\
\mbox{}\ \ \ \ \textbf{\textcolor{Blue}{for}}\ \textcolor{BrickRed}{(}i\ \textcolor{BrickRed}{=}\ inicial\ \textcolor{BrickRed}{+}\ \textcolor{Purple}{1}\textcolor{BrickRed}{;}\ i\ \textcolor{BrickRed}{\textless{}}\ final\textcolor{BrickRed}{;}\ i\textcolor{BrickRed}{++)}\textcolor{Red}{\{} \\
\mbox{}\ \ \ \ \ \ \ \ j\ \textcolor{BrickRed}{=}\ i\textcolor{BrickRed}{;} \\
\mbox{}\ \ \ \ \ \ \ \ \textbf{\textcolor{Blue}{while}}\ \textcolor{BrickRed}{((}T\textcolor{BrickRed}{[}j\textcolor{BrickRed}{]}\ \textcolor{BrickRed}{\textless{}}\ T\textcolor{BrickRed}{[}j\textcolor{BrickRed}{-}\textcolor{Purple}{1}\textcolor{BrickRed}{])}\ \textcolor{BrickRed}{\&\&}\ \textcolor{BrickRed}{(}j\ \textcolor{BrickRed}{\textgreater{}}\ \textcolor{Purple}{0}\textcolor{BrickRed}{))}\textcolor{Red}{\{} \\
\mbox{}\ \ \ \ \ \ \ \ \ \ \ \ aux\ \textcolor{BrickRed}{=}\ T\textcolor{BrickRed}{[}j\textcolor{BrickRed}{];} \\
\mbox{}\ \ \ \ \ \ \ \ \ \ \ \ T\textcolor{BrickRed}{[}j\textcolor{BrickRed}{]}\ \textcolor{BrickRed}{=}\ T\textcolor{BrickRed}{[}j\textcolor{BrickRed}{-}\textcolor{Purple}{1}\textcolor{BrickRed}{];} \\
\mbox{}\ \ \ \ \ \ \ \ \ \ \ \ T\textcolor{BrickRed}{[}j\textcolor{BrickRed}{-}\textcolor{Purple}{1}\textcolor{BrickRed}{]}\ \textcolor{BrickRed}{=}\ aux\textcolor{BrickRed}{;} \\
\mbox{}\ \ \ \ \ \ \ \ \ \ \ \ j\textcolor{BrickRed}{-\/-;} \\
\mbox{}\ \ \ \ \ \ \ \ \textcolor{Red}{\}} \\
\mbox{}\ \ \ \ \textcolor{Red}{\}} \\
\mbox{}\textcolor{Red}{\}} \\
\mbox{} \\
\mbox{}\textcolor{ForestGreen}{int}\ \textbf{\textcolor{Black}{main}}\textcolor{BrickRed}{(}\textcolor{ForestGreen}{int}\ argc\textcolor{BrickRed}{,}\ \textcolor{ForestGreen}{char}\textcolor{BrickRed}{*}\ argv\textcolor{BrickRed}{[])}\textcolor{Red}{\{} \\
\mbox{}\ \ \ \ \textbf{\textcolor{Blue}{if}}\ \textcolor{BrickRed}{(}argc\ \textcolor{BrickRed}{!=}\textcolor{Purple}{2}\textcolor{BrickRed}{)}\textcolor{Red}{\{} \\
\mbox{}\ \ \ \ \ \ \ \ cerr\ \textcolor{BrickRed}{\textless{}\textless{}}\ \texttt{\textcolor{Red}{"{}Uso\ del\ programa:\ "{}}}\ \textcolor{BrickRed}{+}\ \textcolor{BrickRed}{(}string\textcolor{BrickRed}{)(}argv\textcolor{BrickRed}{[}\textcolor{Purple}{0}\textcolor{BrickRed}{])}\ \textcolor{BrickRed}{+}\ \texttt{\textcolor{Red}{"{}\ \textless{}número\ positivo\textgreater{}"{}}}\ \textcolor{BrickRed}{\textless{}\textless{}}\ endl\textcolor{BrickRed}{;}\ \  \\
\mbox{}\ \ \ \ \ \ \ \ \textbf{\textcolor{Blue}{return}}\ \textcolor{BrickRed}{-}\textcolor{Purple}{1}\textcolor{BrickRed}{;} \\
\mbox{}\ \ \ \ \textcolor{Red}{\}} \\
\mbox{}\ \ \ \ \textcolor{ForestGreen}{int}\ n\ \textcolor{BrickRed}{=}\ \textbf{\textcolor{Black}{atoi}}\textcolor{BrickRed}{(}argv\textcolor{BrickRed}{[}\textcolor{Purple}{1}\textcolor{BrickRed}{]);}\ \ \ \  \\
\mbox{}\ \ \ \ \textbf{\textcolor{Blue}{if}}\ \textcolor{BrickRed}{(}n\textcolor{BrickRed}{\textless{}}\textcolor{Purple}{0}\textcolor{BrickRed}{)}\ \textbf{\textcolor{Blue}{return}}\ \textcolor{BrickRed}{-}\textcolor{Purple}{1}\textcolor{BrickRed}{;} \\
\mbox{}\ \ \ \  \\
\mbox{}\ \ \ \ \textcolor{ForestGreen}{int}\ \textcolor{BrickRed}{*}\ T\ \textcolor{BrickRed}{=}\ \textbf{\textcolor{Blue}{new}}\ \textcolor{ForestGreen}{int}\textcolor{BrickRed}{[}n\textcolor{BrickRed}{];} \\
\mbox{}\ \ \ \ \textbf{\textcolor{Blue}{struct}}\ \textcolor{TealBlue}{timespec}\ t$\_$antes\textcolor{BrickRed}{,}\ t$\_$despues\textcolor{BrickRed}{;} \\
\mbox{}\ \ \ \  \\
\mbox{}\ \ \ \ \textbf{\textcolor{Black}{srandom}}\textcolor{BrickRed}{(}\textbf{\textcolor{Black}{time}}\textcolor{BrickRed}{(}\textcolor{Purple}{0}\textcolor{BrickRed}{));} \\
\mbox{}\ \ \ \  \\
\mbox{}\ \ \ \ \textbf{\textcolor{Blue}{for}}\ \textcolor{BrickRed}{(}\textcolor{ForestGreen}{int}\ i\textcolor{BrickRed}{=}\textcolor{Purple}{0}\textcolor{BrickRed}{;}\ i\textcolor{BrickRed}{\textless{}}n\textcolor{BrickRed}{;}\ \textcolor{BrickRed}{++}i\textcolor{BrickRed}{)}\textcolor{Red}{\{} \\
\mbox{}\ \ \ \ \ \ \ \ T\textcolor{BrickRed}{[}i\textcolor{BrickRed}{]}\ \textcolor{BrickRed}{=}\ \textbf{\textcolor{Black}{random}}\textcolor{BrickRed}{();} \\
\mbox{}\ \ \ \ \textcolor{Red}{\}} \\
\mbox{}\ \ \ \  \\
\mbox{}\ \ \ \ \textbf{\textcolor{Black}{clock$\_$gettime}}\textcolor{BrickRed}{(}CLOCK$\_$REALTIME\textcolor{BrickRed}{,\&}t$\_$antes\textcolor{BrickRed}{);} \\
\mbox{}\ \ \ \ \textbf{\textcolor{Black}{insercion}}\ \textcolor{BrickRed}{(}T\textcolor{BrickRed}{,}n\textcolor{BrickRed}{);} \\
\mbox{}\ \ \ \ \textbf{\textcolor{Black}{clock$\_$gettime}}\textcolor{BrickRed}{(}CLOCK$\_$REALTIME\textcolor{BrickRed}{,\&}t$\_$despues\textcolor{BrickRed}{);} \\
\mbox{}\ \ \ \  \\
\mbox{}\ \ \ \ cout\textcolor{BrickRed}{.}\textbf{\textcolor{Black}{precision}}\textcolor{BrickRed}{(}\textcolor{Purple}{3}\textcolor{BrickRed}{);} \\
\mbox{}\ \ \ \ cout\ \textcolor{BrickRed}{\textless{}\textless{}}\ \textcolor{BrickRed}{(}\textcolor{ForestGreen}{double}\textcolor{BrickRed}{)}\ \textcolor{BrickRed}{(}t$\_$despues\textcolor{BrickRed}{.}tv$\_$sec\textcolor{BrickRed}{-}t$\_$antes\textcolor{BrickRed}{.}tv$\_$sec\textcolor{BrickRed}{)+} \\
\mbox{}\ \ \ \ \ \ \ \ \textcolor{BrickRed}{(}\textcolor{ForestGreen}{double}\textcolor{BrickRed}{)}\ \textcolor{BrickRed}{((}t$\_$despues\textcolor{BrickRed}{.}tv$\_$nsec\textcolor{BrickRed}{-}t$\_$antes\textcolor{BrickRed}{.}tv$\_$nsec\textcolor{BrickRed}{)/(}\textcolor{Purple}{1}\textcolor{BrickRed}{.}e\textcolor{BrickRed}{+}\textcolor{Purple}{9}\textcolor{BrickRed}{))}\ \textcolor{BrickRed}{\textless{}\textless{}}\ endl\textcolor{BrickRed}{;} \\
\mbox{} \\
\mbox{}\ \ \ \  \\
\mbox{}\ \ \ \ \textbf{\textcolor{Blue}{delete}}\ \textcolor{BrickRed}{[]}\ T\textcolor{BrickRed}{;} \\
\mbox{}\ \ \ \  \\
\mbox{}\ \ \ \ \textbf{\textcolor{Blue}{return}}\ \textcolor{Purple}{0}\textcolor{BrickRed}{;} \\
\mbox{}\textcolor{Red}{\}} \\
\mbox{}


\subsubsection{Ordenación por selección}
% Generator: GNU source-highlight, by Lorenzo Bettini, http://www.gnu.org/software/src-highlite
\noindent
\mbox{}\textit{\textcolor{Brown}{/**}} \\
\mbox{}\textit{\textcolor{Brown}{\ *\ }}\textcolor{ForestGreen}{@file}\textit{\textcolor{Brown}{\ Ordenación\ por\ selección}} \\
\mbox{}\textit{\textcolor{Brown}{\ */}} \\
\mbox{} \\
\mbox{}\textbf{\textcolor{RoyalBlue}{\#include}}\ \texttt{\textcolor{Red}{\textless{}iostream\textgreater{}}} \\
\mbox{}\textbf{\textcolor{RoyalBlue}{\#include}}\ \texttt{\textcolor{Red}{\textless{}ctime\textgreater{}}} \\
\mbox{}\textbf{\textcolor{RoyalBlue}{\#include}}\ \texttt{\textcolor{Red}{\textless{}cstdlib\textgreater{}}} \\
\mbox{}\textbf{\textcolor{RoyalBlue}{\#include}}\ \texttt{\textcolor{Red}{\textless{}ctime\textgreater{}}} \\
\mbox{}\textbf{\textcolor{RoyalBlue}{\#include}}\ \texttt{\textcolor{Red}{\textless{}cassert\textgreater{}}} \\
\mbox{}\textbf{\textcolor{RoyalBlue}{\#include}}\ \texttt{\textcolor{Red}{\textless{}climits\textgreater{}}} \\
\mbox{}\textbf{\textcolor{Blue}{using}}\ \textbf{\textcolor{Blue}{namespace}}\ std\textcolor{BrickRed}{;} \\
\mbox{} \\
\mbox{}\textit{\textcolor{Brown}{/**}} \\
\mbox{}\textit{\textcolor{Brown}{\ *\ }}\textcolor{ForestGreen}{@brief}\textit{\textcolor{Brown}{\ Ordena\ un\ vector\ por\ el\ método\ de\ selección.}} \\
\mbox{}\textit{\textcolor{Brown}{\ *\ }}\textcolor{ForestGreen}{@param}\textit{\textcolor{Brown}{\ T:\ vector\ de\ elementos.\ Debe\ tener\ num$\_$elem\ elementos.}} \\
\mbox{}\textit{\textcolor{Brown}{\ *\ Es\ modificado.}} \\
\mbox{}\textit{\textcolor{Brown}{\ *\ }}\textcolor{ForestGreen}{@param}\textit{\textcolor{Brown}{\ num$\_$elem:\ número\ de\ elementos.\ num$\_$elem\ \textgreater{}\ 0.}} \\
\mbox{}\textit{\textcolor{Brown}{\ *\ }} \\
\mbox{}\textit{\textcolor{Brown}{\ *\ Cambia\ el\ orden\ de\ los\ elementos\ de\ T\ de\ forma\ que\ los\ dispone}} \\
\mbox{}\textit{\textcolor{Brown}{\ *\ en\ sentido\ creciente\ de\ menor\ a\ mayor.}} \\
\mbox{}\textit{\textcolor{Brown}{\ *\ Aplica\ el\ algoritmo\ de\ selección.}} \\
\mbox{}\textit{\textcolor{Brown}{\ */}} \\
\mbox{} \\
\mbox{}\textbf{\textcolor{Blue}{inline}}\ \textbf{\textcolor{Blue}{static}}\ \textcolor{ForestGreen}{void}\ \textbf{\textcolor{Black}{seleccion}}\textcolor{BrickRed}{(}\textcolor{ForestGreen}{int}\ T\textcolor{BrickRed}{[],}\ \textcolor{ForestGreen}{int}\ num$\_$elem\textcolor{BrickRed}{);} \\
\mbox{} \\
\mbox{}\textit{\textcolor{Brown}{/**}} \\
\mbox{}\textit{\textcolor{Brown}{\ *\ }}\textcolor{ForestGreen}{@brief}\textit{\textcolor{Brown}{\ Ordena\ parte\ de\ un\ vector\ por\ el\ método\ de\ selección.}} \\
\mbox{}\textit{\textcolor{Brown}{\ *\ }}\textcolor{ForestGreen}{@param}\textit{\textcolor{Brown}{\ T:\ vector\ de\ elementos.\ Tiene\ un\ número\ de\ elementos\ }} \\
\mbox{}\textit{\textcolor{Brown}{\ *\ mayor\ o\ igual\ a\ final.\ Es\ MODIFICADO.}} \\
\mbox{}\textit{\textcolor{Brown}{\ *\ }}\textcolor{ForestGreen}{@param}\textit{\textcolor{Brown}{\ inicial:\ Posición\ que\ marca\ el\ incio\ de\ la\ parte\ del}} \\
\mbox{}\textit{\textcolor{Brown}{\ *\ vector\ a\ ordenar.}} \\
\mbox{}\textit{\textcolor{Brown}{\ *\ }}\textcolor{ForestGreen}{@param}\textit{\textcolor{Brown}{\ final:\ Posición\ detrás\ de\ la\ última\ de\ la\ parte\ del}} \\
\mbox{}\textit{\textcolor{Brown}{\ *\ vector\ a\ ordenar.\ }} \\
\mbox{}\textit{\textcolor{Brown}{\ *\ }}\textcolor{ForestGreen}{@pre}\textit{\textcolor{Brown}{\ inicial\ \textless{}\ final.}} \\
\mbox{}\textit{\textcolor{Brown}{\ *\ }} \\
\mbox{}\textit{\textcolor{Brown}{\ *\ Cambia\ el\ orden\ de\ los\ elementos\ de\ T\ entre\ las\ posiciones}} \\
\mbox{}\textit{\textcolor{Brown}{\ *\ inicial\ y\ final\ -\ 1de\ forma\ que\ los\ dispone\ en\ sentido\ creciente}} \\
\mbox{}\textit{\textcolor{Brown}{\ *\ de\ menor\ a\ mayor.}} \\
\mbox{}\textit{\textcolor{Brown}{\ *\ Aplica\ el\ algoritmo\ de\ selección.}} \\
\mbox{}\textit{\textcolor{Brown}{\ */}} \\
\mbox{} \\
\mbox{}\textbf{\textcolor{Blue}{static}}\ \textcolor{ForestGreen}{void}\ \textbf{\textcolor{Black}{seleccion$\_$lims}}\textcolor{BrickRed}{(}\textcolor{ForestGreen}{int}\ T\textcolor{BrickRed}{[],}\ \textcolor{ForestGreen}{int}\ inicial\textcolor{BrickRed}{,}\ \textcolor{ForestGreen}{int}\ final\textcolor{BrickRed}{);} \\
\mbox{} \\
\mbox{}\textit{\textcolor{Brown}{//\ Implementación\ de\ las\ funciones}} \\
\mbox{} \\
\mbox{}\textcolor{ForestGreen}{void}\ \textbf{\textcolor{Black}{seleccion}}\textcolor{BrickRed}{(}\textcolor{ForestGreen}{int}\ T\textcolor{BrickRed}{[],}\ \textcolor{ForestGreen}{int}\ num$\_$elem\textcolor{BrickRed}{)}\textcolor{Red}{\{} \\
\mbox{}\ \ \ \ \textbf{\textcolor{Black}{seleccion$\_$lims}}\textcolor{BrickRed}{(}T\textcolor{BrickRed}{,}\ \textcolor{Purple}{0}\textcolor{BrickRed}{,}\ num$\_$elem\textcolor{BrickRed}{);} \\
\mbox{}\textcolor{Red}{\}} \\
\mbox{} \\
\mbox{}\textbf{\textcolor{Blue}{static}}\ \textcolor{ForestGreen}{void}\ \textbf{\textcolor{Black}{seleccion$\_$lims}}\textcolor{BrickRed}{(}\textcolor{ForestGreen}{int}\ T\textcolor{BrickRed}{[],}\ \textcolor{ForestGreen}{int}\ inicial\textcolor{BrickRed}{,}\ \textcolor{ForestGreen}{int}\ final\textcolor{BrickRed}{)}\textcolor{Red}{\{} \\
\mbox{}\ \ \ \ \textcolor{ForestGreen}{int}\ i\textcolor{BrickRed}{,}\ j\textcolor{BrickRed}{,}\ indice$\_$menor\textcolor{BrickRed}{;} \\
\mbox{}\ \ \ \ \textcolor{ForestGreen}{int}\ menor\textcolor{BrickRed}{,}\ aux\textcolor{BrickRed}{;} \\
\mbox{}\ \ \ \ \textbf{\textcolor{Blue}{for}}\ \textcolor{BrickRed}{(}i\ \textcolor{BrickRed}{=}\ inicial\textcolor{BrickRed}{;}\ i\ \textcolor{BrickRed}{\textless{}}\ final\ \textcolor{BrickRed}{-}\ \textcolor{Purple}{1}\textcolor{BrickRed}{;}\ i\textcolor{BrickRed}{++)}\ \textcolor{Red}{\{} \\
\mbox{}\ \ \ \ \ \ \ \ indice$\_$menor\ \textcolor{BrickRed}{=}\ i\textcolor{BrickRed}{;} \\
\mbox{}\ \ \ \ \ \ \ \ menor\ \textcolor{BrickRed}{=}\ T\textcolor{BrickRed}{[}i\textcolor{BrickRed}{];} \\
\mbox{}\ \ \ \ \ \ \ \ \textbf{\textcolor{Blue}{for}}\ \textcolor{BrickRed}{(}j\ \textcolor{BrickRed}{=}\ i\textcolor{BrickRed}{;}\ j\ \textcolor{BrickRed}{\textless{}}\ final\textcolor{BrickRed}{;}\ j\textcolor{BrickRed}{++)} \\
\mbox{}\ \ \ \ \ \ \ \ \ \ \ \ \textbf{\textcolor{Blue}{if}}\ \textcolor{BrickRed}{(}T\textcolor{BrickRed}{[}j\textcolor{BrickRed}{]}\ \textcolor{BrickRed}{\textless{}}\ menor\textcolor{BrickRed}{)}\ \textcolor{Red}{\{} \\
\mbox{}\ \ \ \ \ \ \ \ \ \ \ \ \ \ \ \ indice$\_$menor\ \textcolor{BrickRed}{=}\ j\textcolor{BrickRed}{;} \\
\mbox{}\ \ \ \ \ \ \ \ \ \ \ \ \ \ \ \ menor\ \textcolor{BrickRed}{=}\ T\textcolor{BrickRed}{[}j\textcolor{BrickRed}{];} \\
\mbox{}\ \ \ \ \ \ \ \ \ \ \ \ \textcolor{Red}{\}} \\
\mbox{}\ \ \ \ \ \ \ \ \ \ \ \ aux\ \textcolor{BrickRed}{=}\ T\textcolor{BrickRed}{[}i\textcolor{BrickRed}{];} \\
\mbox{}\ \ \ \ \ \ \ \ T\textcolor{BrickRed}{[}i\textcolor{BrickRed}{]}\ \textcolor{BrickRed}{=}\ T\textcolor{BrickRed}{[}indice$\_$menor\textcolor{BrickRed}{];} \\
\mbox{}\ \ \ \ \ \ \ \ T\textcolor{BrickRed}{[}indice$\_$menor\textcolor{BrickRed}{]}\ \textcolor{BrickRed}{=}\ aux\textcolor{BrickRed}{;} \\
\mbox{}\ \ \ \ \textcolor{Red}{\}} \\
\mbox{}\textcolor{Red}{\}} \\
\mbox{}\  \\
\mbox{}\textit{\textcolor{Brown}{/**}} \\
\mbox{}\textit{\textcolor{Brown}{\ *\ }}\textcolor{ForestGreen}{@brief}\textit{\textcolor{Brown}{\ Permite\ duplicar\ un\ vector\ de\ enteros}} \\
\mbox{}\textit{\textcolor{Brown}{\ *\ }}\textcolor{ForestGreen}{@param}\textit{\textcolor{Brown}{\ T\ puntero\ a\ un\ vector\ de\ enteros}} \\
\mbox{}\textit{\textcolor{Brown}{\ *\ }}\textcolor{ForestGreen}{@param}\textit{\textcolor{Brown}{\ U\ puntero\ a\ otro\ vector\ de\ enteros}} \\
\mbox{}\textit{\textcolor{Brown}{\ *\ }}\textcolor{ForestGreen}{@param}\textit{\textcolor{Brown}{\ n\ tamanio\ de\ ambos\ vectores}} \\
\mbox{}\textit{\textcolor{Brown}{\ *\ }}\textcolor{ForestGreen}{@pre}\textit{\textcolor{Brown}{\ Han\ de\ tener\ el\ mismo\ tamanio}} \\
\mbox{}\textit{\textcolor{Brown}{\ */}} \\
\mbox{} \\
\mbox{}\textcolor{ForestGreen}{void}\ \textbf{\textcolor{Black}{duplicaVector}}\textcolor{BrickRed}{(}\textcolor{ForestGreen}{int}\textcolor{BrickRed}{*}\ T\textcolor{BrickRed}{,}\textcolor{ForestGreen}{int}\textcolor{BrickRed}{*}\ U\textcolor{BrickRed}{,}\textcolor{ForestGreen}{int}\ tam\textcolor{BrickRed}{)}\textcolor{Red}{\{} \\
\mbox{}\ \ \ \ \textbf{\textcolor{Blue}{for}}\ \textcolor{BrickRed}{(}\textcolor{ForestGreen}{int}\ i\textcolor{BrickRed}{=}\textcolor{Purple}{0}\textcolor{BrickRed}{;}\ i\textcolor{BrickRed}{\textless{}}tam\textcolor{BrickRed}{;}\ \textcolor{BrickRed}{++}i\textcolor{BrickRed}{)}\textcolor{Red}{\{} \\
\mbox{}\ \ \ \ \ \ \ \ U\textcolor{BrickRed}{[}i\textcolor{BrickRed}{]=}T\textcolor{BrickRed}{[}i\textcolor{BrickRed}{];} \\
\mbox{}\ \ \ \ \textcolor{Red}{\}} \\
\mbox{}\textcolor{Red}{\}} \\
\mbox{} \\
\mbox{}\textcolor{ForestGreen}{int}\ \textbf{\textcolor{Black}{main}}\textcolor{BrickRed}{(}\textcolor{ForestGreen}{int}\ argc\textcolor{BrickRed}{,}\ \textcolor{ForestGreen}{char}\textcolor{BrickRed}{*}\ argv\textcolor{BrickRed}{[])}\textcolor{Red}{\{} \\
\mbox{}\ \ \ \ \textbf{\textcolor{Blue}{if}}\ \textcolor{BrickRed}{(}argc\ \textcolor{BrickRed}{!=}\textcolor{Purple}{2}\textcolor{BrickRed}{)}\textcolor{Red}{\{} \\
\mbox{}\ \ \ \ \ \ \ \ cerr\ \textcolor{BrickRed}{\textless{}\textless{}}\ \texttt{\textcolor{Red}{"{}Uso\ del\ programa:\ "{}}}\ \textcolor{BrickRed}{+}\ \textcolor{BrickRed}{(}string\textcolor{BrickRed}{)(}argv\textcolor{BrickRed}{[}\textcolor{Purple}{0}\textcolor{BrickRed}{])}\ \textcolor{BrickRed}{+}\ \texttt{\textcolor{Red}{"{}\ \textless{}número\ positivo\textgreater{}"{}}}\ \textcolor{BrickRed}{\textless{}\textless{}}\ endl\textcolor{BrickRed}{;}\ \  \\
\mbox{}\ \ \ \ \ \ \ \ \textbf{\textcolor{Blue}{return}}\ \textcolor{BrickRed}{-}\textcolor{Purple}{1}\textcolor{BrickRed}{;} \\
\mbox{}\ \ \ \ \textcolor{Red}{\}} \\
\mbox{}\ \ \ \ \textcolor{ForestGreen}{int}\ n\ \textcolor{BrickRed}{=}\ \textbf{\textcolor{Black}{atoi}}\textcolor{BrickRed}{(}argv\textcolor{BrickRed}{[}\textcolor{Purple}{1}\textcolor{BrickRed}{]);}\ \ \ \  \\
\mbox{}\ \ \ \ \textbf{\textcolor{Blue}{if}}\ \textcolor{BrickRed}{(}n\textcolor{BrickRed}{\textless{}}\textcolor{Purple}{0}\textcolor{BrickRed}{)}\ \textbf{\textcolor{Blue}{return}}\ \textcolor{BrickRed}{-}\textcolor{Purple}{1}\textcolor{BrickRed}{;} \\
\mbox{}\ \ \ \  \\
\mbox{}\ \ \ \ \textcolor{ForestGreen}{int}\ \textcolor{BrickRed}{*}\ T\ \textcolor{BrickRed}{=}\ \textbf{\textcolor{Blue}{new}}\ \textcolor{ForestGreen}{int}\textcolor{BrickRed}{[}n\textcolor{BrickRed}{];} \\
\mbox{}\ \ \ \ \textbf{\textcolor{Blue}{struct}}\ \textcolor{TealBlue}{timespec}\ t$\_$antes\textcolor{BrickRed}{,}\ t$\_$despues\textcolor{BrickRed}{;} \\
\mbox{}\ \ \ \  \\
\mbox{}\ \ \ \ \textbf{\textcolor{Black}{srandom}}\textcolor{BrickRed}{(}\textbf{\textcolor{Black}{time}}\textcolor{BrickRed}{(}\textcolor{Purple}{0}\textcolor{BrickRed}{));} \\
\mbox{}\ \ \ \  \\
\mbox{}\ \ \ \ \textbf{\textcolor{Blue}{for}}\ \textcolor{BrickRed}{(}\textcolor{ForestGreen}{int}\ i\textcolor{BrickRed}{=}\textcolor{Purple}{0}\textcolor{BrickRed}{;}\ i\textcolor{BrickRed}{\textless{}}n\textcolor{BrickRed}{;}\ \textcolor{BrickRed}{++}i\textcolor{BrickRed}{)}\textcolor{Red}{\{} \\
\mbox{}\ \ \ \ \ \ \ \ T\textcolor{BrickRed}{[}i\textcolor{BrickRed}{]}\ \textcolor{BrickRed}{=}\ \textbf{\textcolor{Black}{random}}\textcolor{BrickRed}{();} \\
\mbox{}\ \ \ \ \textcolor{Red}{\}} \\
\mbox{}\ \ \ \  \\
\mbox{}\ \ \ \ \textbf{\textcolor{Black}{clock$\_$gettime}}\textcolor{BrickRed}{(}CLOCK$\_$REALTIME\textcolor{BrickRed}{,\&}t$\_$antes\textcolor{BrickRed}{);} \\
\mbox{}\ \ \ \ \textbf{\textcolor{Black}{seleccion}}\ \textcolor{BrickRed}{(}T\textcolor{BrickRed}{,}n\textcolor{BrickRed}{);} \\
\mbox{}\ \ \ \ \textbf{\textcolor{Black}{clock$\_$gettime}}\textcolor{BrickRed}{(}CLOCK$\_$REALTIME\textcolor{BrickRed}{,\&}t$\_$despues\textcolor{BrickRed}{);} \\
\mbox{}\ \ \ \  \\
\mbox{}\ \ \ \ cout\textcolor{BrickRed}{.}\textbf{\textcolor{Black}{precision}}\textcolor{BrickRed}{(}\textcolor{Purple}{3}\textcolor{BrickRed}{);} \\
\mbox{}\ \ \ \ cout\ \textcolor{BrickRed}{\textless{}\textless{}}\ \textcolor{BrickRed}{(}\textcolor{ForestGreen}{double}\textcolor{BrickRed}{)}\ \textcolor{BrickRed}{(}t$\_$despues\textcolor{BrickRed}{.}tv$\_$sec\textcolor{BrickRed}{-}t$\_$antes\textcolor{BrickRed}{.}tv$\_$sec\textcolor{BrickRed}{)+} \\
\mbox{}\ \ \ \ \ \ \ \ \textcolor{BrickRed}{(}\textcolor{ForestGreen}{double}\textcolor{BrickRed}{)}\ \textcolor{BrickRed}{((}t$\_$despues\textcolor{BrickRed}{.}tv$\_$nsec\textcolor{BrickRed}{-}t$\_$antes\textcolor{BrickRed}{.}tv$\_$nsec\textcolor{BrickRed}{)/(}\textcolor{Purple}{1}\textcolor{BrickRed}{.}e\textcolor{BrickRed}{+}\textcolor{Purple}{9}\textcolor{BrickRed}{))}\ \textcolor{BrickRed}{\textless{}\textless{}}\ endl\textcolor{BrickRed}{;} \\
\mbox{} \\
\mbox{}\ \ \ \  \\
\mbox{}\ \ \ \ \textbf{\textcolor{Blue}{delete}}\ \textcolor{BrickRed}{[]}\ T\textcolor{BrickRed}{;} \\
\mbox{}\ \ \ \  \\
\mbox{}\ \ \ \ \textbf{\textcolor{Blue}{return}}\ \textcolor{Purple}{0}\textcolor{BrickRed}{;} \\
\mbox{}\textcolor{Red}{\}}


\subsubsection{Ordenación heapsort}
% Generator: GNU source-highlight, by Lorenzo Bettini, http://www.gnu.org/software/src-highlite
\noindent
\mbox{}\textit{\textcolor{Brown}{/**}} \\
\mbox{}\textit{\textcolor{Brown}{\ *\ }}\textcolor{ForestGreen}{@file}\textit{\textcolor{Brown}{\ Ordenación\ por\ montones}} \\
\mbox{}\textit{\textcolor{Brown}{\ */}} \\
\mbox{} \\
\mbox{}\textbf{\textcolor{RoyalBlue}{\#include}}\ \texttt{\textcolor{Red}{\textless{}iostream\textgreater{}}} \\
\mbox{}\textbf{\textcolor{RoyalBlue}{\#include}}\ \texttt{\textcolor{Red}{\textless{}ctime\textgreater{}}} \\
\mbox{}\textbf{\textcolor{RoyalBlue}{\#include}}\ \texttt{\textcolor{Red}{\textless{}cstdlib\textgreater{}}} \\
\mbox{}\textbf{\textcolor{RoyalBlue}{\#include}}\ \texttt{\textcolor{Red}{\textless{}ctime\textgreater{}}} \\
\mbox{}\textbf{\textcolor{Blue}{using}}\ \textbf{\textcolor{Blue}{namespace}}\ std\textcolor{BrickRed}{;} \\
\mbox{} \\
\mbox{}\textbf{\textcolor{RoyalBlue}{\#define}}\ NUM$\_$VECES\ \textcolor{Purple}{10000} \\
\mbox{} \\
\mbox{}\textit{\textcolor{Brown}{/**}} \\
\mbox{}\textit{\textcolor{Brown}{\ *\ }}\textcolor{ForestGreen}{@brief}\textit{\textcolor{Brown}{\ Ordena\ un\ vector\ por\ el\ método\ de\ montones.}} \\
\mbox{}\textit{\textcolor{Brown}{\ *\ }}\textcolor{ForestGreen}{@param}\textit{\textcolor{Brown}{\ T:\ vector\ de\ elementos.\ Debe\ tener\ num$\_$elem\ elementos.}} \\
\mbox{}\textit{\textcolor{Brown}{\ *\ Es\ modificado.}} \\
\mbox{}\textit{\textcolor{Brown}{\ *\ }}\textcolor{ForestGreen}{@param}\textit{\textcolor{Brown}{\ num$\_$elem:\ número\ de\ elementos.\ num$\_$elem\ \textgreater{}\ 0.}} \\
\mbox{}\textit{\textcolor{Brown}{\ *\ }} \\
\mbox{}\textit{\textcolor{Brown}{\ *\ Cambia\ el\ orden\ de\ los\ elementos\ de\ T\ de\ forma\ que\ los\ dispone}} \\
\mbox{}\textit{\textcolor{Brown}{\ *\ en\ sentido\ creciente\ de\ menor\ a\ mayor.}} \\
\mbox{}\textit{\textcolor{Brown}{\ *\ Aplica\ el\ algoritmo\ de\ ordenación\ por\ montones.}} \\
\mbox{}\textit{\textcolor{Brown}{\ *\ }} \\
\mbox{}\textit{\textcolor{Brown}{\ */}} \\
\mbox{} \\
\mbox{}\textbf{\textcolor{Blue}{inline}}\ \textbf{\textcolor{Blue}{static}}\ \textcolor{ForestGreen}{void}\ \textbf{\textcolor{Black}{heapsort}}\textcolor{BrickRed}{(}\textcolor{ForestGreen}{int}\ T\textcolor{BrickRed}{[],}\ \textcolor{ForestGreen}{int}\ num$\_$elem\textcolor{BrickRed}{);} \\
\mbox{} \\
\mbox{}\textit{\textcolor{Brown}{/**}} \\
\mbox{}\textit{\textcolor{Brown}{\ *\ }}\textcolor{ForestGreen}{@brief}\textit{\textcolor{Brown}{\ Reajusta\ parte\ de\ un\ vector\ para\ que\ sea\ un\ montón.}} \\
\mbox{}\textit{\textcolor{Brown}{\ *\ }}\textcolor{ForestGreen}{@param}\textit{\textcolor{Brown}{\ T:\ vector\ de\ elementos.\ Debe\ tener\ num$\_$elem\ elementos.}} \\
\mbox{}\textit{\textcolor{Brown}{\ *\ Es\ modificado.}} \\
\mbox{}\textit{\textcolor{Brown}{\ *\ }}\textcolor{ForestGreen}{@param}\textit{\textcolor{Brown}{\ num$\_$elem:\ número\ de\ elementos.\ num$\_$elem\ \textgreater{}\ 0.}} \\
\mbox{}\textit{\textcolor{Brown}{\ *\ }}\textcolor{ForestGreen}{@param}\textit{\textcolor{Brown}{\ k:\ índice\ del\ elemento\ que\ se\ toma\ com\ raíz}} \\
\mbox{}\textit{\textcolor{Brown}{\ *\ \ \ }} \\
\mbox{}\textit{\textcolor{Brown}{\ *\ Reajusta\ los\ elementos\ entre\ las\ posiciones\ k\ y\ num$\_$elem\ -\ 1\ }} \\
\mbox{}\textit{\textcolor{Brown}{\ *\ de\ T\ para\ que\ cumpla\ la\ propiedad\ de\ un\ montón\ (APO),\ }} \\
\mbox{}\textit{\textcolor{Brown}{\ *\ considerando\ al\ elemento\ en\ la\ posición\ k\ como\ la\ raíz.}} \\
\mbox{}\textit{\textcolor{Brown}{\ */}} \\
\mbox{} \\
\mbox{}\textbf{\textcolor{Blue}{static}}\ \textcolor{ForestGreen}{void}\ \textbf{\textcolor{Black}{reajustar}}\textcolor{BrickRed}{(}\textcolor{ForestGreen}{int}\ T\textcolor{BrickRed}{[],}\ \textcolor{ForestGreen}{int}\ num$\_$elem\textcolor{BrickRed}{,}\ \textcolor{ForestGreen}{int}\ k\textcolor{BrickRed}{);} \\
\mbox{} \\
\mbox{}\textit{\textcolor{Brown}{//\ Implementación\ de\ las\ funciones}} \\
\mbox{} \\
\mbox{}\textbf{\textcolor{Blue}{static}}\ \textcolor{ForestGreen}{void}\ \textbf{\textcolor{Black}{heapsort}}\textcolor{BrickRed}{(}\textcolor{ForestGreen}{int}\ T\textcolor{BrickRed}{[],}\ \textcolor{ForestGreen}{int}\ num$\_$elem\textcolor{BrickRed}{)}\textcolor{Red}{\{} \\
\mbox{}\ \ \ \ \textbf{\textcolor{Blue}{for}}\ \textcolor{BrickRed}{(}\textcolor{ForestGreen}{int}\ i\ \textcolor{BrickRed}{=}\ num$\_$elem\textcolor{BrickRed}{/}\textcolor{Purple}{2}\textcolor{BrickRed}{;}\ i\ \textcolor{BrickRed}{\textgreater{}=}\ \textcolor{Purple}{0}\textcolor{BrickRed}{;}\ i\textcolor{BrickRed}{-\/-)} \\
\mbox{}\ \ \ \ \ \ \ \ \textbf{\textcolor{Black}{reajustar}}\textcolor{BrickRed}{(}T\textcolor{BrickRed}{,}\ num$\_$elem\textcolor{BrickRed}{,}\ i\textcolor{BrickRed}{);} \\
\mbox{} \\
\mbox{}\ \ \ \ \textbf{\textcolor{Blue}{for}}\ \textcolor{BrickRed}{(}\textcolor{ForestGreen}{int}\ i\ \textcolor{BrickRed}{=}\ num$\_$elem\ \textcolor{BrickRed}{-}\ \textcolor{Purple}{1}\textcolor{BrickRed}{;}\ i\ \textcolor{BrickRed}{\textgreater{}=}\ \textcolor{Purple}{1}\textcolor{BrickRed}{;}\ i\textcolor{BrickRed}{-\/-)}\textcolor{Red}{\{} \\
\mbox{}\ \ \ \ \ \ \ \ \textcolor{ForestGreen}{int}\ aux\ \textcolor{BrickRed}{=}\ T\textcolor{BrickRed}{[}\textcolor{Purple}{0}\textcolor{BrickRed}{];} \\
\mbox{}\ \ \ \ \ \ \ \ T\textcolor{BrickRed}{[}\textcolor{Purple}{0}\textcolor{BrickRed}{]}\ \textcolor{BrickRed}{=}\ T\textcolor{BrickRed}{[}i\textcolor{BrickRed}{];} \\
\mbox{}\ \ \ \ \ \ \ \ T\textcolor{BrickRed}{[}i\textcolor{BrickRed}{]}\ \textcolor{BrickRed}{=}\ aux\textcolor{BrickRed}{;} \\
\mbox{}\ \ \ \ \ \ \ \ \textbf{\textcolor{Black}{reajustar}}\textcolor{BrickRed}{(}T\textcolor{BrickRed}{,}\ i\textcolor{BrickRed}{,}\ \textcolor{Purple}{0}\textcolor{BrickRed}{);} \\
\mbox{}\ \ \ \ \textcolor{Red}{\}} \\
\mbox{}\textcolor{Red}{\}} \\
\mbox{} \\
\mbox{} \\
\mbox{}\textbf{\textcolor{Blue}{static}}\ \textcolor{ForestGreen}{void}\ \textbf{\textcolor{Black}{reajustar}}\textcolor{BrickRed}{(}\textcolor{ForestGreen}{int}\ T\textcolor{BrickRed}{[],}\ \textcolor{ForestGreen}{int}\ num$\_$elem\textcolor{BrickRed}{,}\ \textcolor{ForestGreen}{int}\ k\textcolor{BrickRed}{)}\textcolor{Red}{\{} \\
\mbox{}\ \ \ \ \textcolor{ForestGreen}{int}\ j\textcolor{BrickRed}{;} \\
\mbox{}\ \ \ \ \textcolor{ForestGreen}{int}\ v\textcolor{BrickRed}{;} \\
\mbox{}\ \ \ \ v\ \textcolor{BrickRed}{=}\ T\textcolor{BrickRed}{[}k\textcolor{BrickRed}{];} \\
\mbox{}\ \ \ \ \textcolor{ForestGreen}{bool}\ esAPO\ \textcolor{BrickRed}{=}\ \textbf{\textcolor{Blue}{false}}\textcolor{BrickRed}{;} \\
\mbox{}\ \ \ \ \textbf{\textcolor{Blue}{while}}\ \textcolor{BrickRed}{((}k\ \textcolor{BrickRed}{\textless{}}\ num$\_$elem\textcolor{BrickRed}{/}\textcolor{Purple}{2}\textcolor{BrickRed}{)}\ \textcolor{BrickRed}{\&\&}\ \textcolor{BrickRed}{!}esAPO\textcolor{BrickRed}{)}\textcolor{Red}{\{} \\
\mbox{}\ \ \ \ \ \ \ \ j\ \textcolor{BrickRed}{=}\ k\ \textcolor{BrickRed}{+}\ k\ \textcolor{BrickRed}{+}\ \textcolor{Purple}{1}\textcolor{BrickRed}{;} \\
\mbox{}\ \ \ \ \ \ \ \ \textbf{\textcolor{Blue}{if}}\ \textcolor{BrickRed}{((}j\ \textcolor{BrickRed}{\textless{}}\ \textcolor{BrickRed}{(}num$\_$elem\ \textcolor{BrickRed}{-}\ \textcolor{Purple}{1}\textcolor{BrickRed}{))}\ \textcolor{BrickRed}{\&\&}\ \textcolor{BrickRed}{(}T\textcolor{BrickRed}{[}j\textcolor{BrickRed}{]}\ \textcolor{BrickRed}{\textless{}}\ T\textcolor{BrickRed}{[}j\textcolor{BrickRed}{+}\textcolor{Purple}{1}\textcolor{BrickRed}{]))} \\
\mbox{}\ \ \ \ \ \ \ \ \ \ \ \ j\textcolor{BrickRed}{++;} \\
\mbox{}\ \ \ \ \ \ \ \ \textbf{\textcolor{Blue}{if}}\ \textcolor{BrickRed}{(}v\ \textcolor{BrickRed}{\textgreater{}=}\ T\textcolor{BrickRed}{[}j\textcolor{BrickRed}{])} \\
\mbox{}\ \ \ \ \ \ \ \ \ \ \ \ esAPO\ \textcolor{BrickRed}{=}\ \textbf{\textcolor{Blue}{true}}\textcolor{BrickRed}{;} \\
\mbox{}\ \ \ \ \ \ \ \ T\textcolor{BrickRed}{[}k\textcolor{BrickRed}{]}\ \textcolor{BrickRed}{=}\ T\textcolor{BrickRed}{[}j\textcolor{BrickRed}{];} \\
\mbox{}\ \ \ \ \ \ \ \ k\ \textcolor{BrickRed}{=}\ j\textcolor{BrickRed}{;} \\
\mbox{}\ \ \ \ \textcolor{Red}{\}} \\
\mbox{}\ \ \ \ T\textcolor{BrickRed}{[}k\textcolor{BrickRed}{]}\ \textcolor{BrickRed}{=}\ v\textcolor{BrickRed}{;} \\
\mbox{}\textcolor{Red}{\}} \\
\mbox{} \\
\mbox{}\textcolor{ForestGreen}{int}\ \textbf{\textcolor{Black}{main}}\textcolor{BrickRed}{(}\textcolor{ForestGreen}{int}\ argc\textcolor{BrickRed}{,}\ \textcolor{ForestGreen}{char}\textcolor{BrickRed}{*}\ argv\textcolor{BrickRed}{[])}\textcolor{Red}{\{} \\
\mbox{}\ \ \ \ \textbf{\textcolor{Blue}{if}}\ \textcolor{BrickRed}{(}argc\ \textcolor{BrickRed}{!=}\textcolor{Purple}{2}\textcolor{BrickRed}{)}\textcolor{Red}{\{} \\
\mbox{}\ \ \ \ \ \ \ \ cerr\ \textcolor{BrickRed}{\textless{}\textless{}}\ \texttt{\textcolor{Red}{"{}Uso\ del\ programa:\ "{}}}\ \textcolor{BrickRed}{+}\ \textcolor{BrickRed}{(}string\textcolor{BrickRed}{)(}argv\textcolor{BrickRed}{[}\textcolor{Purple}{0}\textcolor{BrickRed}{])}\ \textcolor{BrickRed}{+}\ \texttt{\textcolor{Red}{"{}\ \textless{}número\ positivo\textgreater{}"{}}}\ \textcolor{BrickRed}{\textless{}\textless{}}\ endl\textcolor{BrickRed}{;}\ \  \\
\mbox{}\ \ \ \ \ \ \ \ \textbf{\textcolor{Blue}{return}}\ \textcolor{BrickRed}{-}\textcolor{Purple}{1}\textcolor{BrickRed}{;} \\
\mbox{}\ \ \ \ \textcolor{Red}{\}} \\
\mbox{}\ \ \ \ \textcolor{ForestGreen}{int}\ n\ \textcolor{BrickRed}{=}\ \textbf{\textcolor{Black}{atoi}}\textcolor{BrickRed}{(}argv\textcolor{BrickRed}{[}\textcolor{Purple}{1}\textcolor{BrickRed}{]);}\ \ \ \  \\
\mbox{}\ \ \ \ \textbf{\textcolor{Blue}{if}}\ \textcolor{BrickRed}{(}n\textcolor{BrickRed}{\textless{}}\textcolor{Purple}{0}\textcolor{BrickRed}{)}\ \textbf{\textcolor{Blue}{return}}\ \textcolor{BrickRed}{-}\textcolor{Purple}{1}\textcolor{BrickRed}{;} \\
\mbox{}\ \ \ \  \\
\mbox{}\ \ \ \ \textcolor{ForestGreen}{int}\ \textcolor{BrickRed}{*}\ T\ \textcolor{BrickRed}{=}\ \textbf{\textcolor{Blue}{new}}\ \textcolor{ForestGreen}{int}\textcolor{BrickRed}{[}n\textcolor{BrickRed}{];} \\
\mbox{}\ \ \ \ \textbf{\textcolor{Blue}{struct}}\ \textcolor{TealBlue}{timespec}\ t$\_$antes\textcolor{BrickRed}{,}\ t$\_$despues\textcolor{BrickRed}{;} \\
\mbox{}\ \ \ \  \\
\mbox{}\ \ \ \ \textbf{\textcolor{Black}{srandom}}\textcolor{BrickRed}{(}\textbf{\textcolor{Black}{time}}\textcolor{BrickRed}{(}\textcolor{Purple}{0}\textcolor{BrickRed}{));} \\
\mbox{}\ \ \ \  \\
\mbox{}\ \ \ \ \textbf{\textcolor{Blue}{for}}\ \textcolor{BrickRed}{(}\textcolor{ForestGreen}{int}\ i\textcolor{BrickRed}{=}\textcolor{Purple}{0}\textcolor{BrickRed}{;}\ i\textcolor{BrickRed}{\textless{}}n\textcolor{BrickRed}{;}\ \textcolor{BrickRed}{++}i\textcolor{BrickRed}{)}\textcolor{Red}{\{} \\
\mbox{}\ \ \ \ \ \ \ \ T\textcolor{BrickRed}{[}i\textcolor{BrickRed}{]}\ \textcolor{BrickRed}{=}\ \textbf{\textcolor{Black}{random}}\textcolor{BrickRed}{();} \\
\mbox{}\ \ \ \ \textcolor{Red}{\}} \\
\mbox{}\ \ \ \  \\
\mbox{}\ \ \ \ \textbf{\textcolor{Black}{clock$\_$gettime}}\textcolor{BrickRed}{(}CLOCK$\_$REALTIME\textcolor{BrickRed}{,\&}t$\_$antes\textcolor{BrickRed}{);} \\
\mbox{}\ \ \ \ \textbf{\textcolor{Black}{heapsort}}\ \textcolor{BrickRed}{(}T\textcolor{BrickRed}{,}n\textcolor{BrickRed}{);} \\
\mbox{}\ \ \ \ \textbf{\textcolor{Black}{clock$\_$gettime}}\textcolor{BrickRed}{(}CLOCK$\_$REALTIME\textcolor{BrickRed}{,\&}t$\_$despues\textcolor{BrickRed}{);} \\
\mbox{}\ \ \ \  \\
\mbox{}\ \ \ \ cout\textcolor{BrickRed}{.}\textbf{\textcolor{Black}{precision}}\textcolor{BrickRed}{(}\textcolor{Purple}{3}\textcolor{BrickRed}{);} \\
\mbox{}\ \ \ \ cout\ \textcolor{BrickRed}{\textless{}\textless{}}\ \textcolor{BrickRed}{(}\textcolor{ForestGreen}{double}\textcolor{BrickRed}{)}\ \textcolor{BrickRed}{(}t$\_$despues\textcolor{BrickRed}{.}tv$\_$sec\textcolor{BrickRed}{-}t$\_$antes\textcolor{BrickRed}{.}tv$\_$sec\textcolor{BrickRed}{)+} \\
\mbox{}\ \ \ \ \ \ \ \ \textcolor{BrickRed}{(}\textcolor{ForestGreen}{double}\textcolor{BrickRed}{)}\ \textcolor{BrickRed}{((}t$\_$despues\textcolor{BrickRed}{.}tv$\_$nsec\textcolor{BrickRed}{-}t$\_$antes\textcolor{BrickRed}{.}tv$\_$nsec\textcolor{BrickRed}{)/(}\textcolor{Purple}{1}\textcolor{BrickRed}{.}e\textcolor{BrickRed}{+}\textcolor{Purple}{9}\textcolor{BrickRed}{))}\ \textcolor{BrickRed}{\textless{}\textless{}}\ endl\textcolor{BrickRed}{;} \\
\mbox{} \\
\mbox{}\ \ \ \  \\
\mbox{}\ \ \ \ \textbf{\textcolor{Blue}{delete}}\ \textcolor{BrickRed}{[]}\ T\textcolor{BrickRed}{;} \\
\mbox{}\ \ \ \  \\
\mbox{}\ \ \ \ \textbf{\textcolor{Blue}{return}}\ \textcolor{Purple}{0}\textcolor{BrickRed}{;} \\
\mbox{}\textcolor{Red}{\}} \\
\mbox{}


\subsubsection{Ordenación mergesort}
% Generator: GNU source-highlight, by Lorenzo Bettini, http://www.gnu.org/software/src-highlite
\noindent
\mbox{}\textit{\textcolor{Brown}{/**}} \\
\mbox{}\textit{\textcolor{Brown}{\ *\ }}\textcolor{ForestGreen}{@file}\textit{\textcolor{Brown}{\ mergesort.cpp}} \\
\mbox{}\textit{\textcolor{Brown}{\ *\ Ordenación\ por\ mezcla}} \\
\mbox{}\textit{\textcolor{Brown}{\ */}} \\
\mbox{} \\
\mbox{}\textbf{\textcolor{RoyalBlue}{\#include}}\ \texttt{\textcolor{Red}{\textless{}iostream\textgreater{}}} \\
\mbox{}\textbf{\textcolor{RoyalBlue}{\#include}}\ \texttt{\textcolor{Red}{\textless{}ctime\textgreater{}}} \\
\mbox{}\textbf{\textcolor{RoyalBlue}{\#include}}\ \texttt{\textcolor{Red}{\textless{}cstdlib\textgreater{}}} \\
\mbox{}\textbf{\textcolor{RoyalBlue}{\#include}}\ \texttt{\textcolor{Red}{\textless{}cassert\textgreater{}}} \\
\mbox{}\textbf{\textcolor{RoyalBlue}{\#include}}\ \texttt{\textcolor{Red}{\textless{}climits\textgreater{}}} \\
\mbox{}\textbf{\textcolor{Blue}{using}}\ \textbf{\textcolor{Blue}{namespace}}\ std\textcolor{BrickRed}{;} \\
\mbox{} \\
\mbox{}\textbf{\textcolor{RoyalBlue}{\#define}}\ NUM$\_$VECES\ \textcolor{Purple}{50} \\
\mbox{} \\
\mbox{}\textit{\textcolor{Brown}{/**}} \\
\mbox{}\textit{\textcolor{Brown}{\ *\ }}\textcolor{ForestGreen}{@brief}\textit{\textcolor{Brown}{\ Ordena\ un\ vector\ por\ el\ método\ de\ mezcla.}} \\
\mbox{}\textit{\textcolor{Brown}{\ *\ }}\textcolor{ForestGreen}{@param}\textit{\textcolor{Brown}{\ T\ Vector\ de\ elementos.\ Debe\ tener\ num$\_$elem\ elementos.}} \\
\mbox{}\textit{\textcolor{Brown}{\ *\ }}\textcolor{ForestGreen}{@param}\textit{\textcolor{Brown}{\ num$\_$elem:\ número\ de\ elementos.\ num$\_$elem\ \textgreater{}\ 0.}} \\
\mbox{}\textit{\textcolor{Brown}{\ *\ }}\textcolor{ForestGreen}{@pos}\textit{\textcolor{Brown}{\ El\ vector\ contiene\ los\ elementos\ ordenados.}} \\
\mbox{}\textit{\textcolor{Brown}{\ *\ }} \\
\mbox{}\textit{\textcolor{Brown}{\ *\ Cambia\ el\ orden\ de\ los\ elementos\ de\ T\ de\ forma\ que\ los\ dispone}} \\
\mbox{}\textit{\textcolor{Brown}{\ *\ en\ sentido\ creciente\ de\ menor\ a\ mayor.}} \\
\mbox{}\textit{\textcolor{Brown}{\ *\ Aplica\ el\ algoritmo\ de\ mezcla.}} \\
\mbox{}\textit{\textcolor{Brown}{\ */}} \\
\mbox{} \\
\mbox{}\textbf{\textcolor{Blue}{inline}}\ \textbf{\textcolor{Blue}{static}}\ \textcolor{ForestGreen}{void}\ \textbf{\textcolor{Black}{mergesort}}\textcolor{BrickRed}{(}\textcolor{ForestGreen}{int}\ T\textcolor{BrickRed}{[],}\ \textcolor{ForestGreen}{int}\ num$\_$elem\textcolor{BrickRed}{);} \\
\mbox{} \\
\mbox{}\textit{\textcolor{Brown}{/**}} \\
\mbox{}\textit{\textcolor{Brown}{\ *\ }}\textcolor{ForestGreen}{@brief}\textit{\textcolor{Brown}{\ Ordena\ parte\ de\ un\ vector\ por\ el\ método\ de\ mezcla.}} \\
\mbox{}\textit{\textcolor{Brown}{\ *\ }}\textcolor{ForestGreen}{@param}\textit{\textcolor{Brown}{\ T:\ vector\ de\ elementos.\ Tiene\ un\ número\ de\ elementos\ }} \\
\mbox{}\textit{\textcolor{Brown}{\ *\ mayor\ o\ igual\ a\ final.\ Es\ MODIFICADO.}} \\
\mbox{}\textit{\textcolor{Brown}{\ *\ }}\textcolor{ForestGreen}{@param}\textit{\textcolor{Brown}{\ inicial:\ Posición\ que\ marca\ el\ incio\ de\ la\ parte\ del}} \\
\mbox{}\textit{\textcolor{Brown}{\ *\ vector\ a\ ordenar.}} \\
\mbox{}\textit{\textcolor{Brown}{\ *\ }}\textcolor{ForestGreen}{@param}\textit{\textcolor{Brown}{\ final:\ Posición\ detrás\ de\ la\ última\ de\ la\ parte\ del}} \\
\mbox{}\textit{\textcolor{Brown}{\ *\ vector\ a\ ordenar.\ }} \\
\mbox{}\textit{\textcolor{Brown}{\ *\ }}\textcolor{ForestGreen}{@pre}\textit{\textcolor{Brown}{\ inicial\ \textless{}\ final.}} \\
\mbox{}\textit{\textcolor{Brown}{\ *\ }} \\
\mbox{}\textit{\textcolor{Brown}{\ *\ Cambia\ el\ orden\ de\ los\ elementos\ de\ T\ entre\ las\ posiciones}} \\
\mbox{}\textit{\textcolor{Brown}{\ *\ inicial\ y\ final\ -\ 1\ de\ forma\ que\ los\ dispone\ en\ sentido\ creciente}} \\
\mbox{}\textit{\textcolor{Brown}{\ *\ de\ menor\ a\ mayor.}} \\
\mbox{}\textit{\textcolor{Brown}{\ *\ Aplica\ el\ algoritmo\ de\ la\ mezcla.}} \\
\mbox{}\textit{\textcolor{Brown}{\ */}} \\
\mbox{} \\
\mbox{}\textbf{\textcolor{Blue}{static}}\ \textcolor{ForestGreen}{void}\ \textbf{\textcolor{Black}{mergesort$\_$lims}}\textcolor{BrickRed}{(}\textcolor{ForestGreen}{int}\ T\textcolor{BrickRed}{[],}\ \textcolor{ForestGreen}{int}\ inicial\textcolor{BrickRed}{,}\ \textcolor{ForestGreen}{int}\ final\textcolor{BrickRed}{);} \\
\mbox{} \\
\mbox{}\textit{\textcolor{Brown}{/**}} \\
\mbox{}\textit{\textcolor{Brown}{\ *\ }}\textcolor{ForestGreen}{@brief}\textit{\textcolor{Brown}{\ Ordena\ un\ vector\ por\ el\ método\ de\ inserción.}} \\
\mbox{}\textit{\textcolor{Brown}{\ *\ }}\textcolor{ForestGreen}{@param}\textit{\textcolor{Brown}{\ T:\ vector\ de\ elementos.\ Debe\ tener\ num$\_$elem\ elementos.}} \\
\mbox{}\textit{\textcolor{Brown}{\ *\ Es\ modificado.}} \\
\mbox{}\textit{\textcolor{Brown}{\ *\ }}\textcolor{ForestGreen}{@param}\textit{\textcolor{Brown}{\ num$\_$elem:\ número\ de\ elementos.\ num$\_$elem\ \textgreater{}\ 0.}} \\
\mbox{}\textit{\textcolor{Brown}{\ *\ }} \\
\mbox{}\textit{\textcolor{Brown}{\ *\ Cambia\ el\ orden\ de\ los\ elementos\ de\ T\ de\ forma\ que\ los\ dispone}} \\
\mbox{}\textit{\textcolor{Brown}{\ *\ en\ sentido\ creciente\ de\ menor\ a\ mayor.}} \\
\mbox{}\textit{\textcolor{Brown}{\ *\ Aplica\ el\ algoritmo\ de\ inserción.}} \\
\mbox{}\textit{\textcolor{Brown}{\ */}} \\
\mbox{} \\
\mbox{}\textbf{\textcolor{Blue}{inline}}\ \textbf{\textcolor{Blue}{static}}\ \textcolor{ForestGreen}{void}\ \textbf{\textcolor{Black}{insercion}}\textcolor{BrickRed}{(}\textcolor{ForestGreen}{int}\ T\textcolor{BrickRed}{[],}\ \textcolor{ForestGreen}{int}\ num$\_$elem\textcolor{BrickRed}{);} \\
\mbox{} \\
\mbox{}\textit{\textcolor{Brown}{/**}} \\
\mbox{}\textit{\textcolor{Brown}{\ *\ }}\textcolor{ForestGreen}{@brief}\textit{\textcolor{Brown}{\ Ordena\ parte\ de\ un\ vector\ por\ el\ método\ de\ inserción.}} \\
\mbox{}\textit{\textcolor{Brown}{\ *\ }}\textcolor{ForestGreen}{@param}\textit{\textcolor{Brown}{\ T:\ vector\ de\ elementos.\ Tiene\ un\ número\ de\ elementos\ }} \\
\mbox{}\textit{\textcolor{Brown}{\ *\ mayor\ o\ igual\ a\ final.\ Es\ MODIFICADO.}} \\
\mbox{}\textit{\textcolor{Brown}{\ *\ }}\textcolor{ForestGreen}{@param}\textit{\textcolor{Brown}{\ inicial:\ Posición\ que\ marca\ el\ incio\ de\ la\ parte\ del}} \\
\mbox{}\textit{\textcolor{Brown}{\ *\ vector\ a\ ordenar.}} \\
\mbox{}\textit{\textcolor{Brown}{\ *\ }}\textcolor{ForestGreen}{@param}\textit{\textcolor{Brown}{\ final:\ Posición\ detrás\ de\ la\ última\ de\ la\ parte\ del}} \\
\mbox{}\textit{\textcolor{Brown}{\ *\ vector\ a\ ordenar.\ }} \\
\mbox{}\textit{\textcolor{Brown}{\ *\ }}\textcolor{ForestGreen}{@pre}\textit{\textcolor{Brown}{\ inicial\ \textless{}\ final.}} \\
\mbox{}\textit{\textcolor{Brown}{\ *\ }} \\
\mbox{}\textit{\textcolor{Brown}{\ *\ Cambia\ el\ orden\ de\ los\ elementos\ de\ T\ entre\ las\ posiciones}} \\
\mbox{}\textit{\textcolor{Brown}{\ *\ inicial\ y\ final\ -\ 1\ de\ forma\ que\ los\ dispone\ en\ sentido\ creciente}} \\
\mbox{}\textit{\textcolor{Brown}{\ *\ de\ menor\ a\ mayor.}} \\
\mbox{}\textit{\textcolor{Brown}{\ *\ Aplica\ el\ algoritmo\ de\ la\ inserción.}} \\
\mbox{}\textit{\textcolor{Brown}{\ */}} \\
\mbox{} \\
\mbox{}\textbf{\textcolor{Blue}{static}}\ \textcolor{ForestGreen}{void}\ \textbf{\textcolor{Black}{insercion$\_$lims}}\textcolor{BrickRed}{(}\textcolor{ForestGreen}{int}\ T\textcolor{BrickRed}{[],}\ \textcolor{ForestGreen}{int}\ inicial\textcolor{BrickRed}{,}\ \textcolor{ForestGreen}{int}\ final\textcolor{BrickRed}{);} \\
\mbox{} \\
\mbox{}\textit{\textcolor{Brown}{/**}} \\
\mbox{}\textit{\textcolor{Brown}{\ *\ }}\textcolor{ForestGreen}{@brief}\textit{\textcolor{Brown}{\ Mezcla\ dos\ vectores\ ordenados\ sobre\ otro.}} \\
\mbox{}\textit{\textcolor{Brown}{\ *\ }}\textcolor{ForestGreen}{@param}\textit{\textcolor{Brown}{\ T:\ vector\ de\ elementos.\ Tiene\ un\ número\ de\ elementos\ }} \\
\mbox{}\textit{\textcolor{Brown}{\ *\ mayor\ o\ igual\ a\ final.\ Es\ MODIFICADO.}} \\
\mbox{}\textit{\textcolor{Brown}{\ *\ }}\textcolor{ForestGreen}{@param}\textit{\textcolor{Brown}{\ inicial:\ Posición\ que\ marca\ el\ incio\ de\ la\ parte\ del}} \\
\mbox{}\textit{\textcolor{Brown}{\ *\ vector\ a\ escribir.}} \\
\mbox{}\textit{\textcolor{Brown}{\ *\ }}\textcolor{ForestGreen}{@param}\textit{\textcolor{Brown}{\ final:\ Posición\ detrás\ de\ la\ última\ de\ la\ parte\ del}} \\
\mbox{}\textit{\textcolor{Brown}{\ *\ vector\ a\ escribir}} \\
\mbox{}\textit{\textcolor{Brown}{\ *\ inicial\ \textless{}\ final.}} \\
\mbox{}\textit{\textcolor{Brown}{\ *\ }}\textcolor{ForestGreen}{@param}\textit{\textcolor{Brown}{\ U:\ Vector\ con\ los\ elementos\ ordenados.}} \\
\mbox{}\textit{\textcolor{Brown}{\ *\ }}\textcolor{ForestGreen}{@param}\textit{\textcolor{Brown}{\ V:\ Vector\ con\ los\ elementos\ ordenados.}} \\
\mbox{}\textit{\textcolor{Brown}{\ *\ }}\textcolor{ForestGreen}{@pre}\textit{\textcolor{Brown}{\ El\ número\ de\ elementos\ de\ U\ y\ V\ sumados\ debe\ coincidir}} \\
\mbox{}\textit{\textcolor{Brown}{\ *\ con\ final\ -\ inicial.}} \\
\mbox{}\textit{\textcolor{Brown}{\ *\ }} \\
\mbox{}\textit{\textcolor{Brown}{\ *\ En\ los\ elementos\ de\ T\ entre\ las\ posiciones\ inicial\ y\ final\ -\ 1}} \\
\mbox{}\textit{\textcolor{Brown}{\ *\ pone\ ordenados\ en\ sentido\ creciente,\ de\ menor\ a\ mayor,\ los}} \\
\mbox{}\textit{\textcolor{Brown}{\ *\ elementos\ de\ los\ vectores\ U\ y\ V.}} \\
\mbox{}\textit{\textcolor{Brown}{\ */}} \\
\mbox{} \\
\mbox{}\textbf{\textcolor{Blue}{static}}\ \textcolor{ForestGreen}{void}\ \textbf{\textcolor{Black}{fusion}}\textcolor{BrickRed}{(}\textcolor{ForestGreen}{int}\ T\textcolor{BrickRed}{[],}\ \textcolor{ForestGreen}{int}\ inicial\textcolor{BrickRed}{,}\ \textcolor{ForestGreen}{int}\ final\textcolor{BrickRed}{,}\ \textcolor{ForestGreen}{int}\ U\textcolor{BrickRed}{[],}\ \textcolor{ForestGreen}{int}\ V\textcolor{BrickRed}{[]);} \\
\mbox{} \\
\mbox{}\textit{\textcolor{Brown}{//\ Implementación\ de\ las\ funciones}} \\
\mbox{} \\
\mbox{}\textbf{\textcolor{Blue}{inline}}\ \textbf{\textcolor{Blue}{static}}\ \textcolor{ForestGreen}{void}\ \textbf{\textcolor{Black}{insercion}}\textcolor{BrickRed}{(}\textcolor{ForestGreen}{int}\ T\textcolor{BrickRed}{[],}\ \textcolor{ForestGreen}{int}\ num$\_$elem\textcolor{BrickRed}{)}\textcolor{Red}{\{} \\
\mbox{}\ \ \ \ \textbf{\textcolor{Black}{insercion$\_$lims}}\textcolor{BrickRed}{(}T\textcolor{BrickRed}{,}\ \textcolor{Purple}{0}\textcolor{BrickRed}{,}\ num$\_$elem\textcolor{BrickRed}{);} \\
\mbox{}\textcolor{Red}{\}} \\
\mbox{} \\
\mbox{}\textbf{\textcolor{Blue}{static}}\ \textcolor{ForestGreen}{void}\ \textbf{\textcolor{Black}{insercion$\_$lims}}\textcolor{BrickRed}{(}\textcolor{ForestGreen}{int}\ T\textcolor{BrickRed}{[],}\ \textcolor{ForestGreen}{int}\ inicial\textcolor{BrickRed}{,}\ \textcolor{ForestGreen}{int}\ final\textcolor{BrickRed}{)}\textcolor{Red}{\{} \\
\mbox{}\ \ \ \ \textcolor{ForestGreen}{int}\ i\textcolor{BrickRed}{,}\ j\textcolor{BrickRed}{;} \\
\mbox{}\ \ \ \ \textcolor{ForestGreen}{int}\ aux\textcolor{BrickRed}{;} \\
\mbox{} \\
\mbox{}\ \ \ \ \textbf{\textcolor{Blue}{for}}\ \textcolor{BrickRed}{(}i\ \textcolor{BrickRed}{=}\ inicial\ \textcolor{BrickRed}{+}\ \textcolor{Purple}{1}\textcolor{BrickRed}{;}\ i\textcolor{BrickRed}{\textless{}}final\textcolor{BrickRed}{;}\ i\textcolor{BrickRed}{++)}\textcolor{Red}{\{} \\
\mbox{}\ \ \ \ \ \ \ \ j\ \textcolor{BrickRed}{=}\ i\textcolor{BrickRed}{;} \\
\mbox{}\ \ \ \ \ \ \ \ \textbf{\textcolor{Blue}{while}}\ \textcolor{BrickRed}{((}T\textcolor{BrickRed}{[}j\textcolor{BrickRed}{]}\ \textcolor{BrickRed}{\textless{}}\ T\textcolor{BrickRed}{[}j\textcolor{BrickRed}{-}\textcolor{Purple}{1}\textcolor{BrickRed}{])}\ \textcolor{BrickRed}{\&\&}\ \textcolor{BrickRed}{(}j\ \textcolor{BrickRed}{\textgreater{}}\ \textcolor{Purple}{0}\textcolor{BrickRed}{))}\textcolor{Red}{\{} \\
\mbox{}\ \ \ \ \ \ \ \ \ \ \ \ aux\ \textcolor{BrickRed}{=}\ T\textcolor{BrickRed}{[}j\textcolor{BrickRed}{];} \\
\mbox{}\ \ \ \ \ \ \ \ \ \ \ \ T\textcolor{BrickRed}{[}j\textcolor{BrickRed}{]}\ \textcolor{BrickRed}{=}\ T\textcolor{BrickRed}{[}j\textcolor{BrickRed}{-}\textcolor{Purple}{1}\textcolor{BrickRed}{];} \\
\mbox{}\ \ \ \ \ \ \ \ \ \ \ \ T\textcolor{BrickRed}{[}j\textcolor{BrickRed}{-}\textcolor{Purple}{1}\textcolor{BrickRed}{]}\ \textcolor{BrickRed}{=}\ aux\textcolor{BrickRed}{;} \\
\mbox{}\ \ \ \ \ \ \ \ \ \ \ \ j\textcolor{BrickRed}{-\/-;} \\
\mbox{}\ \ \ \ \ \ \ \ \textcolor{Red}{\}} \\
\mbox{}\ \ \ \ \textcolor{Red}{\}} \\
\mbox{}\textcolor{Red}{\}} \\
\mbox{} \\
\mbox{}\textbf{\textcolor{Blue}{const}}\ \textcolor{ForestGreen}{int}\ UMBRAL$\_$MS\ \textcolor{BrickRed}{=}\ \textcolor{Purple}{100}\textcolor{BrickRed}{;} \\
\mbox{} \\
\mbox{}\textcolor{ForestGreen}{void}\ \textbf{\textcolor{Black}{mergesort}}\textcolor{BrickRed}{(}\textcolor{ForestGreen}{int}\ T\textcolor{BrickRed}{[],}\ \textcolor{ForestGreen}{int}\ num$\_$elem\textcolor{BrickRed}{)}\textcolor{Red}{\{} \\
\mbox{}\ \ \ \ \textbf{\textcolor{Black}{mergesort$\_$lims}}\textcolor{BrickRed}{(}T\textcolor{BrickRed}{,}\ \textcolor{Purple}{0}\textcolor{BrickRed}{,}\ num$\_$elem\textcolor{BrickRed}{);} \\
\mbox{}\textcolor{Red}{\}} \\
\mbox{} \\
\mbox{}\textbf{\textcolor{Blue}{static}}\ \textcolor{ForestGreen}{void}\ \textbf{\textcolor{Black}{mergesort$\_$lims}}\textcolor{BrickRed}{(}\textcolor{ForestGreen}{int}\ T\textcolor{BrickRed}{[],}\ \textcolor{ForestGreen}{int}\ inicial\textcolor{BrickRed}{,}\ \textcolor{ForestGreen}{int}\ final\textcolor{BrickRed}{)}\textcolor{Red}{\{} \\
\mbox{}\ \ \ \ \textbf{\textcolor{Blue}{if}}\ \textcolor{BrickRed}{((}final\ \textcolor{BrickRed}{-}\ inicial\textcolor{BrickRed}{)}\ \textcolor{BrickRed}{\textless{}}\ UMBRAL$\_$MS\textcolor{BrickRed}{)}\ \textcolor{Red}{\{} \\
\mbox{}\ \ \ \ \ \ \ \ \textbf{\textcolor{Black}{insercion$\_$lims}}\textcolor{BrickRed}{(}T\textcolor{BrickRed}{,}\ inicial\textcolor{BrickRed}{,}\ final\textcolor{BrickRed}{);} \\
\mbox{}\ \ \ \ \textcolor{Red}{\}} \\
\mbox{}\ \ \ \ \textbf{\textcolor{Blue}{else}}\ \textcolor{Red}{\{} \\
\mbox{}\ \ \ \ \ \ \ \ \textcolor{ForestGreen}{int}\ k\ \textcolor{BrickRed}{=}\ \textcolor{BrickRed}{(}final\ \textcolor{BrickRed}{-}\ inicial\textcolor{BrickRed}{)/}\textcolor{Purple}{2}\textcolor{BrickRed}{;} \\
\mbox{}\ \ \ \ \ \ \ \  \\
\mbox{}\ \ \ \ \ \ \ \ \textcolor{ForestGreen}{int}\textcolor{BrickRed}{*}\ U\ \textcolor{BrickRed}{=}\ \textbf{\textcolor{Blue}{new}}\ \textcolor{ForestGreen}{int}\ \textcolor{BrickRed}{[}k\ \textcolor{BrickRed}{-}\ inicial\ \textcolor{BrickRed}{+}\ \textcolor{Purple}{1}\textcolor{BrickRed}{];} \\
\mbox{}\ \ \ \ \ \ \ \ \textbf{\textcolor{Black}{assert}}\textcolor{BrickRed}{(}U\textcolor{BrickRed}{);} \\
\mbox{}\ \ \ \ \ \ \ \ \textcolor{ForestGreen}{int}\ l\textcolor{BrickRed}{,}\ l2\textcolor{BrickRed}{;} \\
\mbox{}\ \ \ \ \ \ \ \ \textbf{\textcolor{Blue}{for}}\ \textcolor{BrickRed}{(}l\ \textcolor{BrickRed}{=}\ \textcolor{Purple}{0}\textcolor{BrickRed}{,}\ l2\ \textcolor{BrickRed}{=}\ inicial\textcolor{BrickRed}{;}\ l\ \textcolor{BrickRed}{\textless{}}\ k\textcolor{BrickRed}{;}\ l\textcolor{BrickRed}{++,}\ l2\textcolor{BrickRed}{++)} \\
\mbox{}\ \ \ \ \ \ \ \ \ \ \ \ U\textcolor{BrickRed}{[}l\textcolor{BrickRed}{]}\ \textcolor{BrickRed}{=}\ T\textcolor{BrickRed}{[}l2\textcolor{BrickRed}{];} \\
\mbox{}\ \ \ \ \ \ \ \ U\textcolor{BrickRed}{[}l\textcolor{BrickRed}{]}\ \textcolor{BrickRed}{=}\ INT$\_$MAX\textcolor{BrickRed}{;} \\
\mbox{}\ \ \ \ \ \ \ \  \\
\mbox{}\ \ \ \ \ \ \ \ \textcolor{ForestGreen}{int}\textcolor{BrickRed}{*}\ V\ \textcolor{BrickRed}{=}\ \textbf{\textcolor{Blue}{new}}\ \textcolor{ForestGreen}{int}\ \textcolor{BrickRed}{[}final\ \textcolor{BrickRed}{-}\ k\ \textcolor{BrickRed}{+}\ \textcolor{Purple}{1}\textcolor{BrickRed}{];} \\
\mbox{}\ \ \ \ \ \ \ \ \textbf{\textcolor{Black}{assert}}\textcolor{BrickRed}{(}V\textcolor{BrickRed}{);} \\
\mbox{}\ \ \ \ \ \ \ \ \textbf{\textcolor{Blue}{for}}\ \textcolor{BrickRed}{(}l\ \textcolor{BrickRed}{=}\ \textcolor{Purple}{0}\textcolor{BrickRed}{,}\ l2\ \textcolor{BrickRed}{=}\ k\textcolor{BrickRed}{;}\ l\ \textcolor{BrickRed}{\textless{}}\ final\ \textcolor{BrickRed}{-}\ k\textcolor{BrickRed}{;}\ l\textcolor{BrickRed}{++,}\ l2\textcolor{BrickRed}{++)} \\
\mbox{}\ \ \ \ \ \ \ \ \ \ \ \ V\textcolor{BrickRed}{[}l\textcolor{BrickRed}{]}\ \textcolor{BrickRed}{=}\ T\textcolor{BrickRed}{[}l2\textcolor{BrickRed}{];} \\
\mbox{}\ \ \ \ \ \ \ \ V\textcolor{BrickRed}{[}l\textcolor{BrickRed}{]}\ \textcolor{BrickRed}{=}\ INT$\_$MAX\textcolor{BrickRed}{;} \\
\mbox{}\ \ \ \ \ \ \ \  \\
\mbox{}\ \ \ \ \ \ \ \ \textbf{\textcolor{Black}{mergesort$\_$lims}}\textcolor{BrickRed}{(}U\textcolor{BrickRed}{,}\ \textcolor{Purple}{0}\textcolor{BrickRed}{,}\ k\textcolor{BrickRed}{);} \\
\mbox{}\ \ \ \ \ \ \ \ \textbf{\textcolor{Black}{mergesort$\_$lims}}\textcolor{BrickRed}{(}V\textcolor{BrickRed}{,}\ \textcolor{Purple}{0}\textcolor{BrickRed}{,}\ final\ \textcolor{BrickRed}{-}\ k\textcolor{BrickRed}{);} \\
\mbox{}\ \ \ \ \ \ \ \ \textbf{\textcolor{Black}{fusion}}\textcolor{BrickRed}{(}T\textcolor{BrickRed}{,}\ inicial\textcolor{BrickRed}{,}\ final\textcolor{BrickRed}{,}\ U\textcolor{BrickRed}{,}\ V\textcolor{BrickRed}{);} \\
\mbox{}\ \ \ \ \ \ \ \ \textbf{\textcolor{Blue}{delete}}\ \textcolor{BrickRed}{[]}\ U\textcolor{BrickRed}{;} \\
\mbox{}\ \ \ \ \ \ \ \ \textbf{\textcolor{Blue}{delete}}\ \textcolor{BrickRed}{[]}\ V\textcolor{BrickRed}{;} \\
\mbox{}\ \ \ \ \textcolor{Red}{\}} \\
\mbox{}\textcolor{Red}{\}} \\
\mbox{} \\
\mbox{} \\
\mbox{}\textbf{\textcolor{Blue}{static}}\ \textcolor{ForestGreen}{void}\ \textbf{\textcolor{Black}{fusion}}\textcolor{BrickRed}{(}\textcolor{ForestGreen}{int}\ T\textcolor{BrickRed}{[],}\ \textcolor{ForestGreen}{int}\ inicial\textcolor{BrickRed}{,}\ \textcolor{ForestGreen}{int}\ final\textcolor{BrickRed}{,}\ \textcolor{ForestGreen}{int}\ U\textcolor{BrickRed}{[],}\ \textcolor{ForestGreen}{int}\ V\textcolor{BrickRed}{[])}\ \textcolor{Red}{\{} \\
\mbox{}\ \ \ \ \textit{\textcolor{Brown}{//\ Toma\ el\ mínimo\ entre\ los\ restantes\ de\ los\ vectores\ U\ y\ V,}} \\
\mbox{}\ \ \ \ \textit{\textcolor{Brown}{//\ colocará\ ese\ mínimo\ en\ T\ y\ moverá\ el\ índice\ de\ lectura.}} \\
\mbox{}\ \ \ \ \textcolor{ForestGreen}{int}\ j\ \textcolor{BrickRed}{=}\ \textcolor{Purple}{0}\textcolor{BrickRed}{;} \\
\mbox{}\ \ \ \ \textcolor{ForestGreen}{int}\ k\ \textcolor{BrickRed}{=}\ \textcolor{Purple}{0}\textcolor{BrickRed}{;} \\
\mbox{} \\
\mbox{}\ \ \ \ \textbf{\textcolor{Blue}{for}}\ \textcolor{BrickRed}{(}\textcolor{ForestGreen}{int}\ i\ \textcolor{BrickRed}{=}\ inicial\textcolor{BrickRed}{;}\ i\ \textcolor{BrickRed}{\textless{}}\ final\textcolor{BrickRed}{;}\ i\textcolor{BrickRed}{++)}\ \textcolor{Red}{\{} \\
\mbox{}\ \ \ \ \ \ \ \ \textbf{\textcolor{Blue}{if}}\ \textcolor{BrickRed}{(}U\textcolor{BrickRed}{[}j\textcolor{BrickRed}{]}\ \textcolor{BrickRed}{\textless{}}\ V\textcolor{BrickRed}{[}k\textcolor{BrickRed}{])}\ \textcolor{Red}{\{} \\
\mbox{}\ \ \ \ \ \ \ \ \ \ \ \ T\textcolor{BrickRed}{[}i\textcolor{BrickRed}{]}\ \textcolor{BrickRed}{=}\ U\textcolor{BrickRed}{[}j\textcolor{BrickRed}{];} \\
\mbox{}\ \ \ \ \ \ \ \ \ \ \ \ j\textcolor{BrickRed}{++;} \\
\mbox{}\ \ \ \ \ \ \ \ \textcolor{Red}{\}} \\
\mbox{}\ \ \ \ \ \ \ \ \textbf{\textcolor{Blue}{else}}\ \textcolor{Red}{\{} \\
\mbox{}\ \ \ \ \ \ \ \ \ \ \ \ T\textcolor{BrickRed}{[}i\textcolor{BrickRed}{]}\ \textcolor{BrickRed}{=}\ V\textcolor{BrickRed}{[}k\textcolor{BrickRed}{];} \\
\mbox{}\ \ \ \ \ \ \ \ \ \ \ \ k\textcolor{BrickRed}{++;} \\
\mbox{}\ \ \ \ \ \ \ \ \textcolor{Red}{\}} \\
\mbox{}\ \ \ \ \textcolor{Red}{\}} \\
\mbox{}\textcolor{Red}{\}} \\
\mbox{} \\
\mbox{}\textcolor{ForestGreen}{int}\ \textbf{\textcolor{Black}{main}}\textcolor{BrickRed}{(}\textcolor{ForestGreen}{int}\ argc\textcolor{BrickRed}{,}\ \textcolor{ForestGreen}{char}\textcolor{BrickRed}{*}\ argv\textcolor{BrickRed}{[])}\ \textcolor{Red}{\{} \\
\mbox{}\ \ \ \ \textbf{\textcolor{Blue}{if}}\ \textcolor{BrickRed}{(}argc\ \textcolor{BrickRed}{!=}\ \textcolor{Purple}{2}\textcolor{BrickRed}{)}\textcolor{Red}{\{} \\
\mbox{}\ \ \ \ \ \ \ \ cerr\ \textcolor{BrickRed}{\textless{}\textless{}}\ \texttt{\textcolor{Red}{"{}Uso\ del\ programa:\ "{}}}\ \textcolor{BrickRed}{+}\ \textcolor{BrickRed}{(}string\textcolor{BrickRed}{)(}argv\textcolor{BrickRed}{[}\textcolor{Purple}{0}\textcolor{BrickRed}{])}\ \textcolor{BrickRed}{+}\ \texttt{\textcolor{Red}{"{}\ \textless{}número\ positivo\textgreater{}"{}}}\ \textcolor{BrickRed}{\textless{}\textless{}}\ endl\textcolor{BrickRed}{;}\ \  \\
\mbox{}\ \ \ \ \ \ \ \ \textbf{\textcolor{Blue}{return}}\ \textcolor{BrickRed}{-}\textcolor{Purple}{1}\textcolor{BrickRed}{;} \\
\mbox{}\ \ \ \ \textcolor{Red}{\}} \\
\mbox{} \\
\mbox{}\ \ \ \ \textcolor{ForestGreen}{int}\ n\ \textcolor{BrickRed}{=}\ \textbf{\textcolor{Black}{atoi}}\textcolor{BrickRed}{(}argv\textcolor{BrickRed}{[}\textcolor{Purple}{1}\textcolor{BrickRed}{]);}\ \ \ \  \\
\mbox{}\ \ \ \ \textbf{\textcolor{Blue}{if}}\ \textcolor{BrickRed}{(}n\textcolor{BrickRed}{\textless{}}\textcolor{Purple}{0}\textcolor{BrickRed}{)}\ \textbf{\textcolor{Blue}{return}}\ \textcolor{BrickRed}{-}\textcolor{Purple}{1}\textcolor{BrickRed}{;} \\
\mbox{}\ \ \ \  \\
\mbox{}\ \ \ \ \textcolor{ForestGreen}{int}\ \textcolor{BrickRed}{*}\ T\ \textcolor{BrickRed}{=}\ \textbf{\textcolor{Blue}{new}}\ \textcolor{ForestGreen}{int}\textcolor{BrickRed}{[}n\textcolor{BrickRed}{];} \\
\mbox{}\ \ \ \ \textbf{\textcolor{Blue}{struct}}\ \textcolor{TealBlue}{timespec}\ t$\_$antes\textcolor{BrickRed}{,}\ t$\_$despues\textcolor{BrickRed}{;} \\
\mbox{}\ \ \ \  \\
\mbox{}\ \ \ \ \textit{\textcolor{Brown}{//\ Vector\ aleatorio.}} \\
\mbox{}\ \ \ \ \textbf{\textcolor{Black}{srandom}}\textcolor{BrickRed}{(}\textbf{\textcolor{Black}{time}}\textcolor{BrickRed}{(}\textcolor{Purple}{0}\textcolor{BrickRed}{));} \\
\mbox{}\ \ \ \ \textbf{\textcolor{Blue}{for}}\ \textcolor{BrickRed}{(}\textcolor{ForestGreen}{int}\ i\textcolor{BrickRed}{=}\textcolor{Purple}{0}\textcolor{BrickRed}{;}\ i\textcolor{BrickRed}{\textless{}}n\textcolor{BrickRed}{;}\ \textcolor{BrickRed}{++}i\textcolor{BrickRed}{)}\ \textcolor{Red}{\{} \\
\mbox{}\ \ \ \ \ \ \ \ T\textcolor{BrickRed}{[}i\textcolor{BrickRed}{]}\ \textcolor{BrickRed}{=}\ \textbf{\textcolor{Black}{random}}\textcolor{BrickRed}{();} \\
\mbox{}\ \ \ \ \textcolor{Red}{\}} \\
\mbox{}\ \ \ \  \\
\mbox{}\ \ \ \ \textit{\textcolor{Brown}{//\ Medida\ del\ tiempo.\ Ejecución\ del\ algoritmo.}} \\
\mbox{}\ \ \ \ \textbf{\textcolor{Black}{clock$\_$gettime}}\textcolor{BrickRed}{(}CLOCK$\_$REALTIME\textcolor{BrickRed}{,\&}t$\_$antes\textcolor{BrickRed}{);} \\
\mbox{}\ \ \ \ \textbf{\textcolor{Black}{mergesort}}\textcolor{BrickRed}{(}T\textcolor{BrickRed}{,}n\textcolor{BrickRed}{);} \\
\mbox{}\ \ \ \ \textbf{\textcolor{Black}{clock$\_$gettime}}\textcolor{BrickRed}{(}CLOCK$\_$REALTIME\textcolor{BrickRed}{,\&}t$\_$despues\textcolor{BrickRed}{);} \\
\mbox{}\ \ \ \  \\
\mbox{}\ \ \ \  \\
\mbox{}\ \ \ \ cout\textcolor{BrickRed}{.}\textbf{\textcolor{Black}{precision}}\textcolor{BrickRed}{(}\textcolor{Purple}{3}\textcolor{BrickRed}{);} \\
\mbox{}\ \ \ \ cout\ \textcolor{BrickRed}{\textless{}\textless{}}\ \textcolor{BrickRed}{(}\textcolor{ForestGreen}{double}\textcolor{BrickRed}{)}\ \textcolor{BrickRed}{(}t$\_$despues\textcolor{BrickRed}{.}tv$\_$sec\textcolor{BrickRed}{-}t$\_$antes\textcolor{BrickRed}{.}tv$\_$sec\textcolor{BrickRed}{)+} \\
\mbox{}\ \ \ \ \ \ \ \ \textcolor{BrickRed}{(}\textcolor{ForestGreen}{double}\textcolor{BrickRed}{)}\ \textcolor{BrickRed}{((}t$\_$despues\textcolor{BrickRed}{.}tv$\_$nsec\textcolor{BrickRed}{-}t$\_$antes\textcolor{BrickRed}{.}tv$\_$nsec\textcolor{BrickRed}{)/(}\textcolor{Purple}{1}\textcolor{BrickRed}{.}e\textcolor{BrickRed}{+}\textcolor{Purple}{9}\textcolor{BrickRed}{))}\ \textcolor{BrickRed}{\textless{}\textless{}}\ endl\textcolor{BrickRed}{;} \\
\mbox{} \\
\mbox{}\ \ \ \ \textbf{\textcolor{Blue}{delete}}\ \textcolor{BrickRed}{[]}\ T\textcolor{BrickRed}{;} \\
\mbox{}\ \ \ \  \\
\mbox{}\ \ \ \ \textbf{\textcolor{Blue}{return}}\ \textcolor{Purple}{0}\textcolor{BrickRed}{;} \\
\mbox{}\textcolor{Red}{\}} \\
\mbox{}


\subsubsection{Algoritmo de Hoare}
% Generator: GNU source-highlight, by Lorenzo Bettini, http://www.gnu.org/software/src-highlite
\noindent
\mbox{}\textit{\textcolor{Brown}{/**}} \\
\mbox{}\textit{\textcolor{Brown}{\ *\ }}\textcolor{ForestGreen}{@file}\textit{\textcolor{Brown}{\ quicksort.cpp}} \\
\mbox{}\textit{\textcolor{Brown}{\ *\ Ordenación\ rápida\ (quicksort).}} \\
\mbox{}\textit{\textcolor{Brown}{\ */}} \\
\mbox{} \\
\mbox{}\textbf{\textcolor{RoyalBlue}{\#include}}\ \texttt{\textcolor{Red}{\textless{}iostream\textgreater{}}} \\
\mbox{}\textbf{\textcolor{RoyalBlue}{\#include}}\ \texttt{\textcolor{Red}{\textless{}ctime\textgreater{}}} \\
\mbox{}\textbf{\textcolor{RoyalBlue}{\#include}}\ \texttt{\textcolor{Red}{\textless{}cstdlib\textgreater{}}} \\
\mbox{}\textbf{\textcolor{Blue}{using}}\ \textbf{\textcolor{Blue}{namespace}}\ std\textcolor{BrickRed}{;} \\
\mbox{} \\
\mbox{}\textbf{\textcolor{RoyalBlue}{\#define}}\ NUM$\_$VECES\ \textcolor{Purple}{100} \\
\mbox{} \\
\mbox{}\textit{\textcolor{Brown}{/**}} \\
\mbox{}\textit{\textcolor{Brown}{\ *\ }}\textcolor{ForestGreen}{@brief}\textit{\textcolor{Brown}{\ Ordena\ un\ vector\ por\ el\ método\ quicksort.}} \\
\mbox{}\textit{\textcolor{Brown}{\ *\ }}\textcolor{ForestGreen}{@param}\textit{\textcolor{Brown}{\ T:\ vector\ de\ elementos.\ Debe\ tener\ num$\_$elem\ elementos.}} \\
\mbox{}\textit{\textcolor{Brown}{\ *\ Es\ modificado.}} \\
\mbox{}\textit{\textcolor{Brown}{\ *\ }}\textcolor{ForestGreen}{@param}\textit{\textcolor{Brown}{\ num$\_$elem:\ número\ de\ elementos.\ num$\_$elem\ \textgreater{}\ 0.}} \\
\mbox{}\textit{\textcolor{Brown}{\ *\ }} \\
\mbox{}\textit{\textcolor{Brown}{\ *\ Cambia\ el\ orden\ de\ los\ elementos\ de\ T\ de\ forma\ que\ los\ dispone}} \\
\mbox{}\textit{\textcolor{Brown}{\ *\ en\ sentido\ creciente\ de\ menor\ a\ mayor.}} \\
\mbox{}\textit{\textcolor{Brown}{\ *\ Aplica\ el\ algoritmo\ quicksort.}} \\
\mbox{}\textit{\textcolor{Brown}{\ */}} \\
\mbox{} \\
\mbox{}\textbf{\textcolor{Blue}{inline}}\ \textbf{\textcolor{Blue}{static}}\ \textcolor{ForestGreen}{void}\ \textbf{\textcolor{Black}{quicksort}}\textcolor{BrickRed}{(}\textcolor{ForestGreen}{int}\ T\textcolor{BrickRed}{[],}\ \textcolor{ForestGreen}{int}\ num$\_$elem\textcolor{BrickRed}{);} \\
\mbox{} \\
\mbox{}\textit{\textcolor{Brown}{/**}} \\
\mbox{}\textit{\textcolor{Brown}{\ *\ }}\textcolor{ForestGreen}{@brief}\textit{\textcolor{Brown}{\ Ordena\ parte\ de\ un\ vector\ por\ el\ método\ quicksort.}} \\
\mbox{}\textit{\textcolor{Brown}{\ *\ }}\textcolor{ForestGreen}{@param}\textit{\textcolor{Brown}{\ T:\ vector\ de\ elementos.\ Tiene\ un\ número\ de\ elementos\ }} \\
\mbox{}\textit{\textcolor{Brown}{\ *\ mayor\ o\ igual\ a\ final.\ Es\ MODIFICADO.}} \\
\mbox{}\textit{\textcolor{Brown}{\ *\ }}\textcolor{ForestGreen}{@param}\textit{\textcolor{Brown}{\ inicial:\ Posición\ que\ marca\ el\ incio\ de\ la\ parte\ del}} \\
\mbox{}\textit{\textcolor{Brown}{\ *\ vector\ a\ ordenar.}} \\
\mbox{}\textit{\textcolor{Brown}{\ *\ }}\textcolor{ForestGreen}{@param}\textit{\textcolor{Brown}{\ final:\ Posición\ detrás\ de\ la\ última\ de\ la\ parte\ del}} \\
\mbox{}\textit{\textcolor{Brown}{\ *\ vector\ a\ ordenar.\ }} \\
\mbox{}\textit{\textcolor{Brown}{\ *\ }}\textcolor{ForestGreen}{@pre}\textit{\textcolor{Brown}{\ inicial\ \textless{}\ final.}} \\
\mbox{}\textit{\textcolor{Brown}{\ *\ }} \\
\mbox{}\textit{\textcolor{Brown}{\ *\ Cambia\ el\ orden\ de\ los\ elementos\ de\ T\ entre\ las\ posiciones}} \\
\mbox{}\textit{\textcolor{Brown}{\ *\ inicial\ y\ final\ -\ 1\ de\ forma\ que\ los\ dispone\ en\ sentido\ creciente}} \\
\mbox{}\textit{\textcolor{Brown}{\ *\ de\ menor\ a\ mayor.}} \\
\mbox{}\textit{\textcolor{Brown}{\ *\ Aplica\ el\ algoritmo\ quicksort.}} \\
\mbox{}\textit{\textcolor{Brown}{\ */}} \\
\mbox{} \\
\mbox{}\textbf{\textcolor{Blue}{static}}\ \textcolor{ForestGreen}{void}\ \textbf{\textcolor{Black}{quicksort$\_$lims}}\textcolor{BrickRed}{(}\textcolor{ForestGreen}{int}\ T\textcolor{BrickRed}{[],}\ \textcolor{ForestGreen}{int}\ inicial\textcolor{BrickRed}{,}\ \textcolor{ForestGreen}{int}\ final\textcolor{BrickRed}{);} \\
\mbox{} \\
\mbox{}\textit{\textcolor{Brown}{/**}} \\
\mbox{}\textit{\textcolor{Brown}{\ *\ }}\textcolor{ForestGreen}{@brief}\textit{\textcolor{Brown}{\ Ordena\ un\ vector\ por\ el\ método\ de\ inserción.}} \\
\mbox{}\textit{\textcolor{Brown}{\ *\ }}\textcolor{ForestGreen}{@param}\textit{\textcolor{Brown}{\ T:\ vector\ de\ elementos.\ Debe\ tener\ num$\_$elem\ elementos.}} \\
\mbox{}\textit{\textcolor{Brown}{\ *\ Es\ modificado.}} \\
\mbox{}\textit{\textcolor{Brown}{\ *\ }}\textcolor{ForestGreen}{@param}\textit{\textcolor{Brown}{\ num$\_$elem:\ número\ de\ elementos.\ num$\_$elem\ \textgreater{}\ 0.}} \\
\mbox{}\textit{\textcolor{Brown}{\ *\ }} \\
\mbox{}\textit{\textcolor{Brown}{\ *\ Cambia\ el\ orden\ de\ los\ elementos\ de\ T\ de\ forma\ que\ los\ dispone}} \\
\mbox{}\textit{\textcolor{Brown}{\ *\ en\ sentido\ creciente\ de\ menor\ a\ mayor.}} \\
\mbox{}\textit{\textcolor{Brown}{\ *\ Aplica\ el\ algoritmo\ de\ inserción.}} \\
\mbox{}\textit{\textcolor{Brown}{\ */}} \\
\mbox{} \\
\mbox{}\textbf{\textcolor{Blue}{inline}}\ \textbf{\textcolor{Blue}{static}}\ \textcolor{ForestGreen}{void}\ \textbf{\textcolor{Black}{insercion}}\textcolor{BrickRed}{(}\textcolor{ForestGreen}{int}\ T\textcolor{BrickRed}{[],}\ \textcolor{ForestGreen}{int}\ num$\_$elem\textcolor{BrickRed}{);} \\
\mbox{} \\
\mbox{}\textit{\textcolor{Brown}{/**}} \\
\mbox{}\textit{\textcolor{Brown}{\ *\ }}\textcolor{ForestGreen}{@brief}\textit{\textcolor{Brown}{\ Ordena\ parte\ de\ un\ vector\ por\ el\ método\ de\ inserción.}} \\
\mbox{}\textit{\textcolor{Brown}{\ *\ }}\textcolor{ForestGreen}{@param}\textit{\textcolor{Brown}{\ T:\ vector\ de\ elementos.\ Tiene\ un\ número\ de\ elementos\ }} \\
\mbox{}\textit{\textcolor{Brown}{\ *\ mayor\ o\ igual\ a\ final.\ Es\ MODIFICADO.}} \\
\mbox{}\textit{\textcolor{Brown}{\ *\ }}\textcolor{ForestGreen}{@param}\textit{\textcolor{Brown}{\ inicial:\ Posición\ que\ marca\ el\ incio\ de\ la\ parte\ del}} \\
\mbox{}\textit{\textcolor{Brown}{\ *\ vector\ a\ ordenar.}} \\
\mbox{}\textit{\textcolor{Brown}{\ *\ }}\textcolor{ForestGreen}{@param}\textit{\textcolor{Brown}{\ final:\ Posición\ detrás\ de\ la\ última\ de\ la\ parte\ del}} \\
\mbox{}\textit{\textcolor{Brown}{\ *\ vector\ a\ ordenar.\ }} \\
\mbox{}\textit{\textcolor{Brown}{\ *\ }}\textcolor{ForestGreen}{@pre}\textit{\textcolor{Brown}{\ inicial\ \textless{}\ final.}} \\
\mbox{}\textit{\textcolor{Brown}{\ *\ }} \\
\mbox{}\textit{\textcolor{Brown}{\ *\ Cambia\ el\ orden\ de\ los\ elementos\ de\ T\ entre\ las\ posiciones}} \\
\mbox{}\textit{\textcolor{Brown}{\ *\ inicial\ y\ final\ -\ 1\ de\ forma\ que\ los\ dispone\ en\ sentido\ creciente}} \\
\mbox{}\textit{\textcolor{Brown}{\ *\ de\ menor\ a\ mayor.}} \\
\mbox{}\textit{\textcolor{Brown}{\ *\ Aplica\ el\ algoritmo\ de\ inserción.}} \\
\mbox{}\textit{\textcolor{Brown}{\ */}} \\
\mbox{} \\
\mbox{}\textbf{\textcolor{Blue}{static}}\ \textcolor{ForestGreen}{void}\ \textbf{\textcolor{Black}{insercion$\_$lims}}\textcolor{BrickRed}{(}\textcolor{ForestGreen}{int}\ T\textcolor{BrickRed}{[],}\ \textcolor{ForestGreen}{int}\ inicial\textcolor{BrickRed}{,}\ \textcolor{ForestGreen}{int}\ final\textcolor{BrickRed}{);} \\
\mbox{} \\
\mbox{}\textit{\textcolor{Brown}{/**}} \\
\mbox{}\textit{\textcolor{Brown}{\ *\ }}\textcolor{ForestGreen}{@brief}\textit{\textcolor{Brown}{\ Redistribuye\ los\ elementos\ de\ un\ vector\ según\ un\ pivote.}} \\
\mbox{}\textit{\textcolor{Brown}{\ *\ }}\textcolor{ForestGreen}{@param}\textit{\textcolor{Brown}{\ T:\ vector\ de\ elementos.\ Tiene\ un\ número\ de\ elementos\ }} \\
\mbox{}\textit{\textcolor{Brown}{\ *\ mayor\ o\ igual\ a\ final.\ Es\ MODIFICADO.}} \\
\mbox{}\textit{\textcolor{Brown}{\ *\ }}\textcolor{ForestGreen}{@param}\textit{\textcolor{Brown}{\ inicial:\ Posición\ que\ marca\ el\ incio\ de\ la\ parte\ del}} \\
\mbox{}\textit{\textcolor{Brown}{\ *\ vector\ a\ ordenar.}} \\
\mbox{}\textit{\textcolor{Brown}{\ *\ }}\textcolor{ForestGreen}{@param}\textit{\textcolor{Brown}{\ final:\ Posición\ detrás\ de\ la\ última\ de\ la\ parte\ del}} \\
\mbox{}\textit{\textcolor{Brown}{\ *\ vector\ a\ ordenar.\ }} \\
\mbox{}\textit{\textcolor{Brown}{\ *\ }}\textcolor{ForestGreen}{@pre}\textit{\textcolor{Brown}{\ inicial\ \textless{}\ final.\ \ }} \\
\mbox{}\textit{\textcolor{Brown}{\ *\ }}\textcolor{ForestGreen}{@param}\textit{\textcolor{Brown}{\ pp:\ Posición\ del\ pivote.\ Es\ MODIFICADO.}} \\
\mbox{}\textit{\textcolor{Brown}{\ *\ }} \\
\mbox{}\textit{\textcolor{Brown}{\ *\ Selecciona\ un\ pivote\ los\ elementos\ de\ T\ situados\ en\ las\ posiciones}} \\
\mbox{}\textit{\textcolor{Brown}{\ *\ entre\ inicial\ y\ final\ -\ 1.\ Redistribuye\ los\ elementos,\ situando\ los}} \\
\mbox{}\textit{\textcolor{Brown}{\ *\ menores\ que\ el\ pivote\ a\ su\ izquierda,\ después\ los\ iguales\ y\ a\ la}} \\
\mbox{}\textit{\textcolor{Brown}{\ *\ derecha\ los\ mayores.\ La\ posición\ del\ pivote\ se\ devuelve\ en\ pp.}} \\
\mbox{}\textit{\textcolor{Brown}{\ */}} \\
\mbox{} \\
\mbox{}\textbf{\textcolor{Blue}{static}}\ \textcolor{ForestGreen}{void}\ \textbf{\textcolor{Black}{dividir$\_$qs}}\textcolor{BrickRed}{(}\textcolor{ForestGreen}{int}\ T\textcolor{BrickRed}{[],}\ \textcolor{ForestGreen}{int}\ inicial\textcolor{BrickRed}{,}\ \textcolor{ForestGreen}{int}\ final\textcolor{BrickRed}{,}\ \textcolor{ForestGreen}{int}\ \textcolor{BrickRed}{\&}\ pp\textcolor{BrickRed}{);} \\
\mbox{} \\
\mbox{}\textit{\textcolor{Brown}{//\ Implementación\ de\ las\ funciones}} \\
\mbox{} \\
\mbox{}\textbf{\textcolor{Blue}{inline}}\ \textbf{\textcolor{Blue}{static}}\ \textcolor{ForestGreen}{void}\ \textbf{\textcolor{Black}{insercion}}\textcolor{BrickRed}{(}\textcolor{ForestGreen}{int}\ T\textcolor{BrickRed}{[],}\ \textcolor{ForestGreen}{int}\ num$\_$elem\textcolor{BrickRed}{)}\textcolor{Red}{\{} \\
\mbox{}\ \ \ \ \textbf{\textcolor{Black}{insercion$\_$lims}}\textcolor{BrickRed}{(}T\textcolor{BrickRed}{,}\ \textcolor{Purple}{0}\textcolor{BrickRed}{,}\ num$\_$elem\textcolor{BrickRed}{);} \\
\mbox{}\textcolor{Red}{\}} \\
\mbox{} \\
\mbox{}\textbf{\textcolor{Blue}{static}}\ \textcolor{ForestGreen}{void}\ \textbf{\textcolor{Black}{insercion$\_$lims}}\textcolor{BrickRed}{(}\textcolor{ForestGreen}{int}\ T\textcolor{BrickRed}{[],}\ \textcolor{ForestGreen}{int}\ inicial\textcolor{BrickRed}{,}\ \textcolor{ForestGreen}{int}\ final\textcolor{BrickRed}{)}\textcolor{Red}{\{} \\
\mbox{}\ \ \ \ \textcolor{ForestGreen}{int}\ i\textcolor{BrickRed}{,}\ j\textcolor{BrickRed}{;} \\
\mbox{}\ \ \ \ \textcolor{ForestGreen}{int}\ aux\textcolor{BrickRed}{;} \\
\mbox{}\ \ \ \ \textbf{\textcolor{Blue}{for}}\ \textcolor{BrickRed}{(}i\ \textcolor{BrickRed}{=}\ inicial\ \textcolor{BrickRed}{+}\ \textcolor{Purple}{1}\textcolor{BrickRed}{;}\ i\ \textcolor{BrickRed}{\textless{}}\ final\textcolor{BrickRed}{;}\ i\textcolor{BrickRed}{++)}\ \textcolor{Red}{\{} \\
\mbox{}\ \ \ \ \ \ \ \ j\ \textcolor{BrickRed}{=}\ i\textcolor{BrickRed}{;} \\
\mbox{}\ \ \ \ \ \ \ \ \textbf{\textcolor{Blue}{while}}\ \textcolor{BrickRed}{((}T\textcolor{BrickRed}{[}j\textcolor{BrickRed}{]}\ \textcolor{BrickRed}{\textless{}}\ T\textcolor{BrickRed}{[}j\textcolor{BrickRed}{-}\textcolor{Purple}{1}\textcolor{BrickRed}{])}\ \textbf{\textcolor{Black}{and}}\ \textcolor{BrickRed}{(}j\ \textcolor{BrickRed}{\textgreater{}}\ \textcolor{Purple}{0}\textcolor{BrickRed}{))}\ \textcolor{Red}{\{} \\
\mbox{}\ \ \ \ \ \ \ \ \ \ \ \ aux\ \textcolor{BrickRed}{=}\ T\textcolor{BrickRed}{[}j\textcolor{BrickRed}{];} \\
\mbox{}\ \ \ \ \ \ \ \ \ \ \ \ T\textcolor{BrickRed}{[}j\textcolor{BrickRed}{]}\ \textcolor{BrickRed}{=}\ T\textcolor{BrickRed}{[}j\textcolor{BrickRed}{-}\textcolor{Purple}{1}\textcolor{BrickRed}{];} \\
\mbox{}\ \ \ \ \ \ \ \ \ \ \ \ T\textcolor{BrickRed}{[}j\textcolor{BrickRed}{-}\textcolor{Purple}{1}\textcolor{BrickRed}{]}\ \textcolor{BrickRed}{=}\ aux\textcolor{BrickRed}{;} \\
\mbox{}\ \ \ \ \ \ \ \ \ \ \ \ j\textcolor{BrickRed}{-\/-;} \\
\mbox{}\ \ \ \ \ \ \ \ \textcolor{Red}{\}} \\
\mbox{}\ \ \ \ \textcolor{Red}{\}} \\
\mbox{}\textcolor{Red}{\}} \\
\mbox{} \\
\mbox{}\textbf{\textcolor{Blue}{const}}\ \textcolor{ForestGreen}{int}\ UMBRAL$\_$QS\ \textcolor{BrickRed}{=}\ \textcolor{Purple}{50}\textcolor{BrickRed}{;} \\
\mbox{} \\
\mbox{}\textbf{\textcolor{Blue}{inline}}\ \textcolor{ForestGreen}{void}\ \textbf{\textcolor{Black}{quicksort}}\textcolor{BrickRed}{(}\textcolor{ForestGreen}{int}\ T\textcolor{BrickRed}{[],}\ \textcolor{ForestGreen}{int}\ num$\_$elem\textcolor{BrickRed}{)}\textcolor{Red}{\{} \\
\mbox{}\ \ \ \ \textbf{\textcolor{Black}{quicksort$\_$lims}}\textcolor{BrickRed}{(}T\textcolor{BrickRed}{,}\ \textcolor{Purple}{0}\textcolor{BrickRed}{,}\ num$\_$elem\textcolor{BrickRed}{);} \\
\mbox{}\textcolor{Red}{\}} \\
\mbox{} \\
\mbox{}\textbf{\textcolor{Blue}{static}}\ \textcolor{ForestGreen}{void}\ \textbf{\textcolor{Black}{quicksort$\_$lims}}\textcolor{BrickRed}{(}\textcolor{ForestGreen}{int}\ T\textcolor{BrickRed}{[],}\ \textcolor{ForestGreen}{int}\ inicial\textcolor{BrickRed}{,}\ \textcolor{ForestGreen}{int}\ final\textcolor{BrickRed}{)}\textcolor{Red}{\{} \\
\mbox{}\ \ \ \ \textcolor{ForestGreen}{int}\ k\textcolor{BrickRed}{;} \\
\mbox{}\ \ \ \ \textbf{\textcolor{Blue}{if}}\ \textcolor{BrickRed}{(}final\ \textcolor{BrickRed}{-}\ inicial\ \textcolor{BrickRed}{\textless{}}\ UMBRAL$\_$QS\textcolor{BrickRed}{)}\ \textcolor{Red}{\{} \\
\mbox{}\ \ \ \ \ \ \ \ \textbf{\textcolor{Black}{insercion$\_$lims}}\textcolor{BrickRed}{(}T\textcolor{BrickRed}{,}\ inicial\textcolor{BrickRed}{,}\ final\textcolor{BrickRed}{);} \\
\mbox{}\ \ \ \ \textcolor{Red}{\}} \\
\mbox{}\ \ \ \ \textbf{\textcolor{Blue}{else}}\ \textcolor{Red}{\{} \\
\mbox{}\ \ \ \ \ \ \ \ \textbf{\textcolor{Black}{dividir$\_$qs}}\textcolor{BrickRed}{(}T\textcolor{BrickRed}{,}\ inicial\textcolor{BrickRed}{,}\ final\textcolor{BrickRed}{,}\ k\textcolor{BrickRed}{);} \\
\mbox{}\ \ \ \ \ \ \ \ \textbf{\textcolor{Black}{quicksort$\_$lims}}\textcolor{BrickRed}{(}T\textcolor{BrickRed}{,}\ inicial\textcolor{BrickRed}{,}\ k\textcolor{BrickRed}{);} \\
\mbox{}\ \ \ \ \ \ \ \ \textbf{\textcolor{Black}{quicksort$\_$lims}}\textcolor{BrickRed}{(}T\textcolor{BrickRed}{,}\ k\ \textcolor{BrickRed}{+}\ \textcolor{Purple}{1}\textcolor{BrickRed}{,}\ final\textcolor{BrickRed}{);} \\
\mbox{}\ \ \ \ \textcolor{Red}{\}} \\
\mbox{}\textcolor{Red}{\}} \\
\mbox{} \\
\mbox{}\textbf{\textcolor{Blue}{static}}\ \textcolor{ForestGreen}{void}\ \textbf{\textcolor{Black}{dividir$\_$qs}}\textcolor{BrickRed}{(}\textcolor{ForestGreen}{int}\ T\textcolor{BrickRed}{[],}\ \textcolor{ForestGreen}{int}\ inicial\textcolor{BrickRed}{,}\ \textcolor{ForestGreen}{int}\ final\textcolor{BrickRed}{,}\ \textcolor{ForestGreen}{int}\ \textcolor{BrickRed}{\&}\ pp\textcolor{BrickRed}{)}\textcolor{Red}{\{} \\
\mbox{}\ \ \ \ \textcolor{ForestGreen}{int}\ pivote\textcolor{BrickRed}{,}\ aux\textcolor{BrickRed}{;} \\
\mbox{}\ \ \ \ \textcolor{ForestGreen}{int}\ k\textcolor{BrickRed}{,}\ l\textcolor{BrickRed}{;} \\
\mbox{}\ \ \ \  \\
\mbox{}\ \ \ \ pivote\ \textcolor{BrickRed}{=}\ T\textcolor{BrickRed}{[}inicial\textcolor{BrickRed}{];} \\
\mbox{}\ \ \ \ k\ \textcolor{BrickRed}{=}\ inicial\textcolor{BrickRed}{;} \\
\mbox{}\ \ \ \ l\ \textcolor{BrickRed}{=}\ final\textcolor{BrickRed}{;} \\
\mbox{}\ \ \ \  \\
\mbox{}\ \ \ \ \textbf{\textcolor{Blue}{do}}\ k\textcolor{BrickRed}{++;}\ \textbf{\textcolor{Blue}{while}}\ \textcolor{BrickRed}{((}T\textcolor{BrickRed}{[}k\textcolor{BrickRed}{]}\ \textcolor{BrickRed}{\textless{}=}\ pivote\textcolor{BrickRed}{)}\ \textbf{\textcolor{Black}{and}}\ \textcolor{BrickRed}{(}k\ \textcolor{BrickRed}{\textless{}}\ final\textcolor{BrickRed}{-}\textcolor{Purple}{1}\textcolor{BrickRed}{));} \\
\mbox{}\ \ \ \ \textbf{\textcolor{Blue}{do}}\ l\textcolor{BrickRed}{-\/-;}\ \textbf{\textcolor{Blue}{while}}\ \textcolor{BrickRed}{(}T\textcolor{BrickRed}{[}l\textcolor{BrickRed}{]}\ \textcolor{BrickRed}{\textgreater{}}\ pivote\textcolor{BrickRed}{);} \\
\mbox{}\ \ \ \  \\
\mbox{}\ \ \ \ \textbf{\textcolor{Blue}{while}}\ \textcolor{BrickRed}{(}k\ \textcolor{BrickRed}{\textless{}}\ l\textcolor{BrickRed}{)}\ \textcolor{Red}{\{} \\
\mbox{}\ \ \ \ \ \ \ \ aux\ \textcolor{BrickRed}{=}\ T\textcolor{BrickRed}{[}k\textcolor{BrickRed}{];} \\
\mbox{}\ \ \ \ \ \ \ \ T\textcolor{BrickRed}{[}k\textcolor{BrickRed}{]}\ \textcolor{BrickRed}{=}\ T\textcolor{BrickRed}{[}l\textcolor{BrickRed}{];} \\
\mbox{}\ \ \ \ \ \ \ \ T\textcolor{BrickRed}{[}l\textcolor{BrickRed}{]}\ \textcolor{BrickRed}{=}\ aux\textcolor{BrickRed}{;} \\
\mbox{}\ \ \ \ \ \ \ \ \textbf{\textcolor{Blue}{do}}\ k\textcolor{BrickRed}{++;}\ \textbf{\textcolor{Blue}{while}}\ \textcolor{BrickRed}{(}T\textcolor{BrickRed}{[}k\textcolor{BrickRed}{]}\ \textcolor{BrickRed}{\textless{}=}\ pivote\textcolor{BrickRed}{);} \\
\mbox{}\ \ \ \ \ \ \ \ \textbf{\textcolor{Blue}{do}}\ l\textcolor{BrickRed}{-\/-;}\ \textbf{\textcolor{Blue}{while}}\ \textcolor{BrickRed}{(}T\textcolor{BrickRed}{[}l\textcolor{BrickRed}{]}\ \textcolor{BrickRed}{\textgreater{}}\ pivote\textcolor{BrickRed}{);} \\
\mbox{}\ \ \ \ \textcolor{Red}{\}} \\
\mbox{} \\
\mbox{}\ \ \ \ aux\ \textcolor{BrickRed}{=}\ T\textcolor{BrickRed}{[}inicial\textcolor{BrickRed}{];} \\
\mbox{}\ \ \ \ T\textcolor{BrickRed}{[}inicial\textcolor{BrickRed}{]}\ \textcolor{BrickRed}{=}\ T\textcolor{BrickRed}{[}l\textcolor{BrickRed}{];} \\
\mbox{}\ \ \ \ T\textcolor{BrickRed}{[}l\textcolor{BrickRed}{]}\ \textcolor{BrickRed}{=}\ aux\textcolor{BrickRed}{;} \\
\mbox{}\ \ \ \ pp\ \textcolor{BrickRed}{=}\ l\textcolor{BrickRed}{;} \\
\mbox{}\textcolor{Red}{\}} \\
\mbox{} \\
\mbox{}\textcolor{ForestGreen}{int}\ \textbf{\textcolor{Black}{main}}\textcolor{BrickRed}{(}\textcolor{ForestGreen}{int}\ argc\textcolor{BrickRed}{,}\ \textcolor{ForestGreen}{char}\textcolor{BrickRed}{*}\ argv\textcolor{BrickRed}{[])}\textcolor{Red}{\{} \\
\mbox{}\ \ \ \ \textbf{\textcolor{Blue}{if}}\ \textcolor{BrickRed}{(}argc\ \textcolor{BrickRed}{!=}\textcolor{Purple}{2}\textcolor{BrickRed}{)}\textcolor{Red}{\{} \\
\mbox{}\ \ \ \ \ \ \ \ cerr\ \textcolor{BrickRed}{\textless{}\textless{}}\ \texttt{\textcolor{Red}{"{}Uso\ del\ programa:\ "{}}}\ \textcolor{BrickRed}{+}\ \textcolor{BrickRed}{(}string\textcolor{BrickRed}{)(}argv\textcolor{BrickRed}{[}\textcolor{Purple}{0}\textcolor{BrickRed}{])}\ \textcolor{BrickRed}{+}\ \texttt{\textcolor{Red}{"{}\ \textless{}número\ positivo\textgreater{}"{}}}\ \textcolor{BrickRed}{\textless{}\textless{}}\ endl\textcolor{BrickRed}{;}\ \  \\
\mbox{}\ \ \ \ \ \ \ \ \textbf{\textcolor{Blue}{return}}\ \textcolor{BrickRed}{-}\textcolor{Purple}{1}\textcolor{BrickRed}{;} \\
\mbox{}\ \ \ \ \textcolor{Red}{\}} \\
\mbox{} \\
\mbox{}\ \ \ \ \textcolor{ForestGreen}{int}\ n\ \textcolor{BrickRed}{=}\ \textbf{\textcolor{Black}{atoi}}\textcolor{BrickRed}{(}argv\textcolor{BrickRed}{[}\textcolor{Purple}{1}\textcolor{BrickRed}{]);}\ \ \ \  \\
\mbox{}\ \ \ \ \textbf{\textcolor{Blue}{if}}\ \textcolor{BrickRed}{(}n\textcolor{BrickRed}{\textless{}}\textcolor{Purple}{0}\textcolor{BrickRed}{)}\ \textbf{\textcolor{Blue}{return}}\ \textcolor{BrickRed}{-}\textcolor{Purple}{1}\textcolor{BrickRed}{;} \\
\mbox{}\ \ \ \  \\
\mbox{}\ \ \ \ \textcolor{ForestGreen}{int}\ \textcolor{BrickRed}{*}\ T\ \textcolor{BrickRed}{=}\ \textbf{\textcolor{Blue}{new}}\ \textcolor{ForestGreen}{int}\textcolor{BrickRed}{[}n\textcolor{BrickRed}{];} \\
\mbox{}\ \ \ \ \textbf{\textcolor{Blue}{struct}}\ \textcolor{TealBlue}{timespec}\ t$\_$antes\textcolor{BrickRed}{,}\ t$\_$despues\textcolor{BrickRed}{;} \\
\mbox{}\ \ \ \  \\
\mbox{}\ \ \ \ \textbf{\textcolor{Black}{srandom}}\textcolor{BrickRed}{(}\textbf{\textcolor{Black}{time}}\textcolor{BrickRed}{(}\textcolor{Purple}{0}\textcolor{BrickRed}{));} \\
\mbox{}\ \ \ \  \\
\mbox{}\ \ \ \ \textbf{\textcolor{Blue}{for}}\ \textcolor{BrickRed}{(}\textcolor{ForestGreen}{int}\ i\textcolor{BrickRed}{=}\textcolor{Purple}{0}\textcolor{BrickRed}{;}\ i\textcolor{BrickRed}{\textless{}}n\textcolor{BrickRed}{;}\ \textcolor{BrickRed}{++}i\textcolor{BrickRed}{)}\textcolor{Red}{\{} \\
\mbox{}\ \ \ \ \ \ \ \ T\textcolor{BrickRed}{[}i\textcolor{BrickRed}{]}\ \textcolor{BrickRed}{=}\ \textbf{\textcolor{Black}{random}}\textcolor{BrickRed}{();} \\
\mbox{}\ \ \ \ \textcolor{Red}{\}} \\
\mbox{}\ \ \ \  \\
\mbox{}\ \ \ \ \textbf{\textcolor{Black}{clock$\_$gettime}}\textcolor{BrickRed}{(}CLOCK$\_$REALTIME\textcolor{BrickRed}{,\&}t$\_$antes\textcolor{BrickRed}{);} \\
\mbox{}\ \ \ \ \textbf{\textcolor{Black}{quicksort}}\ \textcolor{BrickRed}{(}T\textcolor{BrickRed}{,}n\textcolor{BrickRed}{);} \\
\mbox{}\ \ \ \ \textbf{\textcolor{Black}{clock$\_$gettime}}\textcolor{BrickRed}{(}CLOCK$\_$REALTIME\textcolor{BrickRed}{,\&}t$\_$despues\textcolor{BrickRed}{);} \\
\mbox{}\ \ \ \  \\
\mbox{}\ \ \ \ cout\textcolor{BrickRed}{.}\textbf{\textcolor{Black}{precision}}\textcolor{BrickRed}{(}\textcolor{Purple}{3}\textcolor{BrickRed}{);} \\
\mbox{}\ \ \ \ cout\ \textcolor{BrickRed}{\textless{}\textless{}}\ \textcolor{BrickRed}{(}\textcolor{ForestGreen}{double}\textcolor{BrickRed}{)}\ \textcolor{BrickRed}{(}t$\_$despues\textcolor{BrickRed}{.}tv$\_$sec\textcolor{BrickRed}{-}t$\_$antes\textcolor{BrickRed}{.}tv$\_$sec\textcolor{BrickRed}{)+} \\
\mbox{}\ \ \ \ \ \ \ \ \textcolor{BrickRed}{(}\textcolor{ForestGreen}{double}\textcolor{BrickRed}{)}\ \textcolor{BrickRed}{((}t$\_$despues\textcolor{BrickRed}{.}tv$\_$nsec\textcolor{BrickRed}{-}t$\_$antes\textcolor{BrickRed}{.}tv$\_$nsec\textcolor{BrickRed}{)/(}\textcolor{Purple}{1}\textcolor{BrickRed}{.}e\textcolor{BrickRed}{+}\textcolor{Purple}{9}\textcolor{BrickRed}{))}\ \textcolor{BrickRed}{\textless{}\textless{}}\ endl\textcolor{BrickRed}{;} \\
\mbox{} \\
\mbox{}\ \ \ \  \\
\mbox{}\ \ \ \ \textbf{\textcolor{Blue}{delete}}\ \textcolor{BrickRed}{[]}\ T\textcolor{BrickRed}{;} \\
\mbox{}\ \ \ \  \\
\mbox{}\ \ \ \ \textbf{\textcolor{Blue}{return}}\ \textcolor{Purple}{0}\textcolor{BrickRed}{;} \\
\mbox{}\textcolor{Red}{\}} \\
\mbox{}


\subsubsection{Sucesión de Fibonacci}
% Generator: GNU source-highlight, by Lorenzo Bettini, http://www.gnu.org/software/src-highlite
\noindent
\mbox{}\textit{\textcolor{Brown}{/**}} \\
\mbox{}\textit{\textcolor{Brown}{\ *\ }}\textcolor{ForestGreen}{@file}\textit{\textcolor{Brown}{\ Cálculo\ de\ la\ sucesión\ de\ Fibonacci}} \\
\mbox{}\textit{\textcolor{Brown}{\ */}} \\
\mbox{} \\
\mbox{}\textbf{\textcolor{RoyalBlue}{\#include}}\ \texttt{\textcolor{Red}{\textless{}iostream\textgreater{}}} \\
\mbox{}\textbf{\textcolor{RoyalBlue}{\#include}}\ \texttt{\textcolor{Red}{\textless{}ctime\textgreater{}}} \\
\mbox{}\textbf{\textcolor{RoyalBlue}{\#include}}\ \texttt{\textcolor{Red}{\textless{}cstdlib\textgreater{}}} \\
\mbox{}\textbf{\textcolor{Blue}{using}}\ \textbf{\textcolor{Blue}{namespace}}\ std\textcolor{BrickRed}{;} \\
\mbox{} \\
\mbox{}\textit{\textcolor{Brown}{/**}} \\
\mbox{}\textit{\textcolor{Brown}{\ *\ }}\textcolor{ForestGreen}{@brief}\textit{\textcolor{Brown}{\ Calcula\ el\ término\ n-ésimo\ de\ la\ sucesión\ de\ Fibonacci.}} \\
\mbox{}\textit{\textcolor{Brown}{\ *\ }}\textcolor{ForestGreen}{@param}\textit{\textcolor{Brown}{\ n:\ número\ de\ orden\ del\ término\ buscado.\ n\ \textgreater{}=\ 1.}} \\
\mbox{}\textit{\textcolor{Brown}{\ *\ }}\textcolor{ForestGreen}{@return}\textit{\textcolor{Brown}{:\ término\ n-ésimo\ de\ la\ sucesión\ de\ Fibonacci.}} \\
\mbox{}\textit{\textcolor{Brown}{\ */}} \\
\mbox{} \\
\mbox{}\textcolor{ForestGreen}{int}\ \textbf{\textcolor{Black}{fibo}}\textcolor{BrickRed}{(}\textcolor{ForestGreen}{int}\ n\textcolor{BrickRed}{)}\textcolor{Red}{\{} \\
\mbox{}\ \ \ \ \textbf{\textcolor{Blue}{if}}\ \textcolor{BrickRed}{(}n\ \textcolor{BrickRed}{\textless{}}\ \textcolor{Purple}{2}\textcolor{BrickRed}{)} \\
\mbox{}\ \ \ \ \ \ \ \ \textbf{\textcolor{Blue}{return}}\ \textcolor{Purple}{1}\textcolor{BrickRed}{;} \\
\mbox{}\ \ \ \ \textbf{\textcolor{Blue}{else}} \\
\mbox{}\ \ \ \ \ \ \ \ \textbf{\textcolor{Blue}{return}}\ \textbf{\textcolor{Black}{fibo}}\textcolor{BrickRed}{(}n\textcolor{BrickRed}{-}\textcolor{Purple}{1}\textcolor{BrickRed}{)}\ \textcolor{BrickRed}{+}\ \textbf{\textcolor{Black}{fibo}}\textcolor{BrickRed}{(}n\textcolor{BrickRed}{-}\textcolor{Purple}{2}\textcolor{BrickRed}{);} \\
\mbox{}\textcolor{Red}{\}} \\
\mbox{} \\
\mbox{}\textcolor{ForestGreen}{int}\ \textbf{\textcolor{Black}{main}}\textcolor{BrickRed}{(}\textcolor{ForestGreen}{int}\ argc\textcolor{BrickRed}{,}\ \textcolor{ForestGreen}{char}\textcolor{BrickRed}{*}\ argv\textcolor{BrickRed}{[])}\textcolor{Red}{\{} \\
\mbox{}\ \ \ \ \textbf{\textcolor{Blue}{if}}\ \textcolor{BrickRed}{(}argc\ \textcolor{BrickRed}{!=}\textcolor{Purple}{2}\textcolor{BrickRed}{)}\textcolor{Red}{\{} \\
\mbox{}\ \ \ \ \ \ \ \ cerr\ \textcolor{BrickRed}{\textless{}\textless{}}\ \texttt{\textcolor{Red}{"{}Uso\ del\ programa:\ "{}}}\ \textcolor{BrickRed}{+}\ \textcolor{BrickRed}{(}string\textcolor{BrickRed}{)(}argv\textcolor{BrickRed}{[}\textcolor{Purple}{0}\textcolor{BrickRed}{])}\ \textcolor{BrickRed}{+}\ \texttt{\textcolor{Red}{"{}\ \textless{}número\ positivo\textgreater{}"{}}}\ \textcolor{BrickRed}{\textless{}\textless{}}\ endl\textcolor{BrickRed}{;}\ \  \\
\mbox{}\ \ \ \ \ \ \ \ \textbf{\textcolor{Blue}{return}}\ \textcolor{BrickRed}{-}\textcolor{Purple}{1}\textcolor{BrickRed}{;} \\
\mbox{}\ \ \ \ \textcolor{Red}{\}} \\
\mbox{}\ \ \ \ \textcolor{ForestGreen}{int}\ n\ \textcolor{BrickRed}{=}\ \textbf{\textcolor{Black}{atoi}}\textcolor{BrickRed}{(}argv\textcolor{BrickRed}{[}\textcolor{Purple}{1}\textcolor{BrickRed}{]);}\ \ \ \  \\
\mbox{}\ \ \ \ \textbf{\textcolor{Blue}{if}}\ \textcolor{BrickRed}{(}n\textcolor{BrickRed}{\textless{}}\textcolor{Purple}{0}\textcolor{BrickRed}{)}\ \textbf{\textcolor{Blue}{return}}\ \textcolor{BrickRed}{-}\textcolor{Purple}{1}\textcolor{BrickRed}{;} \\
\mbox{} \\
\mbox{}\ \ \ \ \textbf{\textcolor{Blue}{struct}}\ \textcolor{TealBlue}{timespec}\ t$\_$antes\textcolor{BrickRed}{,}\ t$\_$despues\textcolor{BrickRed}{;} \\
\mbox{}\ \ \ \  \\
\mbox{}\ \ \ \ \textbf{\textcolor{Black}{clock$\_$gettime}}\textcolor{BrickRed}{(}CLOCK$\_$REALTIME\textcolor{BrickRed}{,\&}t$\_$antes\textcolor{BrickRed}{);} \\
\mbox{}\ \ \ \ \textbf{\textcolor{Black}{fibo}}\ \textcolor{BrickRed}{(}n\textcolor{BrickRed}{);} \\
\mbox{}\ \ \ \ \textbf{\textcolor{Black}{clock$\_$gettime}}\textcolor{BrickRed}{(}CLOCK$\_$REALTIME\textcolor{BrickRed}{,\&}t$\_$despues\textcolor{BrickRed}{);} \\
\mbox{}\ \ \ \  \\
\mbox{}\ \ \ \ cout\textcolor{BrickRed}{.}\textbf{\textcolor{Black}{precision}}\textcolor{BrickRed}{(}\textcolor{Purple}{3}\textcolor{BrickRed}{);} \\
\mbox{}\ \ \ \ cout\ \textcolor{BrickRed}{\textless{}\textless{}}\ \textcolor{BrickRed}{(}\textcolor{ForestGreen}{double}\textcolor{BrickRed}{)}\ \textcolor{BrickRed}{(}t$\_$despues\textcolor{BrickRed}{.}tv$\_$sec\textcolor{BrickRed}{-}t$\_$antes\textcolor{BrickRed}{.}tv$\_$sec\textcolor{BrickRed}{)+} \\
\mbox{}\ \ \ \ \ \ \ \ \textcolor{BrickRed}{(}\textcolor{ForestGreen}{double}\textcolor{BrickRed}{)}\ \textcolor{BrickRed}{((}t$\_$despues\textcolor{BrickRed}{.}tv$\_$nsec\textcolor{BrickRed}{-}t$\_$antes\textcolor{BrickRed}{.}tv$\_$nsec\textcolor{BrickRed}{)/(}\textcolor{Purple}{1}\textcolor{BrickRed}{.}e\textcolor{BrickRed}{+}\textcolor{Purple}{9}\textcolor{BrickRed}{))}\ \textcolor{BrickRed}{\textless{}\textless{}}\ endl\textcolor{BrickRed}{;} \\
\mbox{}\ \ \ \  \\
\mbox{}\ \ \ \ \textbf{\textcolor{Blue}{return}}\ \textcolor{Purple}{0}\textcolor{BrickRed}{;} \\
\mbox{}\textcolor{Red}{\}}


\subsubsection{Algoritmo de Floyd}
% Generator: GNU source-highlight, by Lorenzo Bettini, http://www.gnu.org/software/src-highlite
\noindent
\mbox{}\textit{\textcolor{Brown}{/**}} \\
\mbox{}\textit{\textcolor{Brown}{\ *\ }}\textcolor{ForestGreen}{@file}\textit{\textcolor{Brown}{\ Cálculo\ del\ coste\ de\ los\ caminos\ mínimos.\ Algoritmo\ de\ Floyd.}} \\
\mbox{}\textit{\textcolor{Brown}{\ */}} \\
\mbox{} \\
\mbox{} \\
\mbox{}\textbf{\textcolor{RoyalBlue}{\#include}}\ \texttt{\textcolor{Red}{\textless{}iostream\textgreater{}}} \\
\mbox{}\textbf{\textcolor{RoyalBlue}{\#include}}\ \texttt{\textcolor{Red}{\textless{}ctime\textgreater{}}} \\
\mbox{}\textbf{\textcolor{RoyalBlue}{\#include}}\ \texttt{\textcolor{Red}{\textless{}cstdlib\textgreater{}}} \\
\mbox{}\textbf{\textcolor{RoyalBlue}{\#include}}\ \texttt{\textcolor{Red}{\textless{}climits\textgreater{}}} \\
\mbox{}\textbf{\textcolor{RoyalBlue}{\#include}}\ \texttt{\textcolor{Red}{\textless{}cassert\textgreater{}}} \\
\mbox{}\textbf{\textcolor{RoyalBlue}{\#include}}\ \texttt{\textcolor{Red}{\textless{}cmath\textgreater{}}} \\
\mbox{}\textbf{\textcolor{Blue}{using}}\ \textbf{\textcolor{Blue}{namespace}}\ std\textcolor{BrickRed}{;} \\
\mbox{} \\
\mbox{}\textbf{\textcolor{Blue}{static}}\ \textcolor{ForestGreen}{int}\ \textbf{\textcolor{Blue}{const}}\ MAX$\_$LONG\ \ \textcolor{BrickRed}{=}\ \textcolor{Purple}{10}\textcolor{BrickRed}{;} \\
\mbox{} \\
\mbox{}\textit{\textcolor{Brown}{/**}} \\
\mbox{}\textit{\textcolor{Brown}{\ *\ }}\textcolor{ForestGreen}{@brief}\textit{\textcolor{Brown}{\ Reserva\ espacio\ en\ memoria\ dinámica\ para\ una\ matriz\ cuadrada.}} \\
\mbox{}\textit{\textcolor{Brown}{\ *\ }}\textcolor{ForestGreen}{@param}\textit{\textcolor{Brown}{\ dim:\ dimensión\ de\ la\ matriz.\ dim\ \textgreater{}\ 0.}} \\
\mbox{}\textit{\textcolor{Brown}{\ *\ }}\textcolor{ForestGreen}{@returns}\textit{\textcolor{Brown}{\ puntero\ a\ la\ zona\ de\ memoria\ reservada.}} \\
\mbox{}\textit{\textcolor{Brown}{\ */}} \\
\mbox{} \\
\mbox{}\textcolor{ForestGreen}{int}\ \textcolor{BrickRed}{**}\ \textbf{\textcolor{Black}{ReservaMatriz}}\textcolor{BrickRed}{(}\textcolor{ForestGreen}{int}\ dim\textcolor{BrickRed}{);} \\
\mbox{} \\
\mbox{}\textit{\textcolor{Brown}{/**}} \\
\mbox{}\textit{\textcolor{Brown}{\ *\ }}\textcolor{ForestGreen}{@brief}\textit{\textcolor{Brown}{\ Libera\ el\ espacio\ asignado\ a\ una\ matriz\ cuadrada.}} \\
\mbox{}\textit{\textcolor{Brown}{\ *\ }}\textcolor{ForestGreen}{@param}\textit{\textcolor{Brown}{\ M:\ puntero\ a\ la\ zona\ de\ memoria\ reservada.\ Es\ MODIFICADO.}} \\
\mbox{}\textit{\textcolor{Brown}{\ *\ }}\textcolor{ForestGreen}{@param}\textit{\textcolor{Brown}{\ dim:\ dimensión\ de\ la\ matriz.\ dim\ \textgreater{}\ 0.}} \\
\mbox{}\textit{\textcolor{Brown}{\ *\ }} \\
\mbox{}\textit{\textcolor{Brown}{\ *\ Liberar\ la\ zona\ memoria\ asignada\ a\ M\ y\ lo\ pone\ a\ NULL.}} \\
\mbox{}\textit{\textcolor{Brown}{\ */}} \\
\mbox{}\textcolor{ForestGreen}{void}\ \textbf{\textcolor{Black}{LiberaMatriz}}\textcolor{BrickRed}{(}\textcolor{ForestGreen}{int}\ \textcolor{BrickRed}{**}\ \textcolor{BrickRed}{\&}\ M\textcolor{BrickRed}{,}\ \textcolor{ForestGreen}{int}\ dim\textcolor{BrickRed}{);} \\
\mbox{} \\
\mbox{}\textit{\textcolor{Brown}{/**}} \\
\mbox{}\textit{\textcolor{Brown}{\ *\ }}\textcolor{ForestGreen}{@brief}\textit{\textcolor{Brown}{\ Rellena\ una\ matriz\ cuadrada\ con\ valores\ aleaotorias.}} \\
\mbox{}\textit{\textcolor{Brown}{\ *\ }}\textcolor{ForestGreen}{@param}\textit{\textcolor{Brown}{\ M:\ puntero\ a\ la\ zona\ de\ memoria\ reservada.\ Es\ MODIFICADO.}} \\
\mbox{}\textit{\textcolor{Brown}{\ *\ }}\textcolor{ForestGreen}{@param}\textit{\textcolor{Brown}{\ dim:\ dimensión\ de\ la\ matriz.\ dim\ \textgreater{}\ 0.}} \\
\mbox{}\textit{\textcolor{Brown}{\ *\ }} \\
\mbox{}\textit{\textcolor{Brown}{\ *\ Asigna\ un\ valor\ aleatorio\ entero\ de\ [0,\ MAX$\_$LONG\ -\ 1]\ a\ cada}} \\
\mbox{}\textit{\textcolor{Brown}{\ *\ elemento\ de\ la\ matriz\ M,\ salvo\ los\ de\ la\ diagonal\ principal}} \\
\mbox{}\textit{\textcolor{Brown}{\ *\ que\ quedan\ a\ 0..}} \\
\mbox{}\textit{\textcolor{Brown}{\ */}} \\
\mbox{} \\
\mbox{}\textcolor{ForestGreen}{void}\ \textbf{\textcolor{Black}{RellenaMatriz}}\textcolor{BrickRed}{(}\textcolor{ForestGreen}{int}\ \textcolor{BrickRed}{**}M\textcolor{BrickRed}{,}\ \textcolor{ForestGreen}{int}\ dim\textcolor{BrickRed}{);} \\
\mbox{}\  \\
\mbox{}\textit{\textcolor{Brown}{/**}} \\
\mbox{}\textit{\textcolor{Brown}{\ *\ }}\textcolor{ForestGreen}{@brief}\textit{\textcolor{Brown}{\ Cálculo\ de\ caminos\ mínimos.}} \\
\mbox{}\textit{\textcolor{Brown}{\ *\ }}\textcolor{ForestGreen}{@param}\textit{\textcolor{Brown}{\ M:\ Matriz\ de\ longitudes\ de\ los\ caminos.\ Es\ MODIFICADO.}} \\
\mbox{}\textit{\textcolor{Brown}{\ *\ }}\textcolor{ForestGreen}{@param}\textit{\textcolor{Brown}{\ dim:\ dimensión\ de\ la\ matriz.\ dim\ \textgreater{}\ 0.}} \\
\mbox{}\textit{\textcolor{Brown}{\ *\ }} \\
\mbox{}\textit{\textcolor{Brown}{\ *\ Calcula\ la\ longitud\ del\ camino\ mínimo\ entre\ cada\ par\ de\ nodos\ (i,j),}} \\
\mbox{}\textit{\textcolor{Brown}{\ *\ que\ se\ almacena\ en\ M[i][j].}} \\
\mbox{}\textit{\textcolor{Brown}{\ */}} \\
\mbox{} \\
\mbox{}\textcolor{ForestGreen}{void}\ \textbf{\textcolor{Black}{Floyd}}\textcolor{BrickRed}{(}\textcolor{ForestGreen}{int}\ \textcolor{BrickRed}{**}M\textcolor{BrickRed}{,}\ \textcolor{ForestGreen}{int}\ dim\textcolor{BrickRed}{);} \\
\mbox{} \\
\mbox{}\textit{\textcolor{Brown}{//\ Implementación\ de\ las\ funciones}} \\
\mbox{} \\
\mbox{}\textcolor{ForestGreen}{int}\ \textcolor{BrickRed}{**}\ \textbf{\textcolor{Black}{ReservaMatriz}}\textcolor{BrickRed}{(}\textcolor{ForestGreen}{int}\ dim\textcolor{BrickRed}{)}\textcolor{Red}{\{} \\
\mbox{}\ \ \ \ \textcolor{ForestGreen}{int}\ \textcolor{BrickRed}{**}M\textcolor{BrickRed}{;} \\
\mbox{}\ \ \ \ \textbf{\textcolor{Blue}{if}}\ \textcolor{BrickRed}{(}dim\ \ \textcolor{BrickRed}{\textless{}=}\ \textcolor{Purple}{0}\textcolor{BrickRed}{)} \\
\mbox{}\ \ \ \ \textcolor{Red}{\{} \\
\mbox{}\ \ \ \ \ \ \ \ cerr\textcolor{BrickRed}{\textless{}\textless{}}\ \texttt{\textcolor{Red}{"{}}}\texttt{\textcolor{CarnationPink}{\textbackslash{}n}}\texttt{\textcolor{Red}{\ ERROR:\ Dimension\ de\ la\ matriz\ debe\ ser\ mayor\ que\ 0"{}}}\ \textcolor{BrickRed}{\textless{}\textless{}}\ endl\textcolor{BrickRed}{;} \\
\mbox{}\ \ \ \ \ \ \ \ \textbf{\textcolor{Black}{exit}}\textcolor{BrickRed}{(}\textcolor{Purple}{1}\textcolor{BrickRed}{);} \\
\mbox{}\ \ \ \ \textcolor{Red}{\}} \\
\mbox{}\ \ \ \ M\ \textcolor{BrickRed}{=}\ \textbf{\textcolor{Blue}{new}}\ \textcolor{ForestGreen}{int}\textcolor{BrickRed}{*}\ \textcolor{BrickRed}{[}dim\textcolor{BrickRed}{];} \\
\mbox{}\ \ \ \ \textbf{\textcolor{Blue}{if}}\ \textcolor{BrickRed}{(}M\ \textcolor{BrickRed}{==}\ NULL\textcolor{BrickRed}{)} \\
\mbox{}\ \ \ \ \textcolor{Red}{\{} \\
\mbox{}\ \ \ \ \ \ \ \ cerr\ \textcolor{BrickRed}{\textless{}\textless{}}\ \texttt{\textcolor{Red}{"{}}}\texttt{\textcolor{CarnationPink}{\textbackslash{}n}}\texttt{\textcolor{Red}{\ ERROR:\ No\ puedo\ reservar\ memoria\ para\ un\ matriz\ de\ "{}}} \\
\mbox{}\ \ \ \ \ \ \ \ \textcolor{BrickRed}{\textless{}\textless{}}\ dim\ \textcolor{BrickRed}{\textless{}\textless{}}\ \texttt{\textcolor{Red}{"{}\ x\ "{}}}\ \textcolor{BrickRed}{\textless{}\textless{}}\ dim\ \textcolor{BrickRed}{\textless{}\textless{}}\ \texttt{\textcolor{Red}{"{}elementos"{}}}\ \textcolor{BrickRed}{\textless{}\textless{}}\ endl\textcolor{BrickRed}{;} \\
\mbox{}\ \ \ \ \ \ \ \ \textbf{\textcolor{Black}{exit}}\textcolor{BrickRed}{(}\textcolor{Purple}{1}\textcolor{BrickRed}{);} \\
\mbox{}\ \ \ \ \textcolor{Red}{\}} \\
\mbox{}\ \ \ \ \textbf{\textcolor{Blue}{for}}\ \textcolor{BrickRed}{(}\textcolor{ForestGreen}{int}\ i\ \textcolor{BrickRed}{=}\ \textcolor{Purple}{0}\textcolor{BrickRed}{;}\ i\ \textcolor{BrickRed}{\textless{}}\ dim\textcolor{BrickRed}{;}\ i\textcolor{BrickRed}{++)} \\
\mbox{}\ \ \ \ \textcolor{Red}{\{} \\
\mbox{}\ \ \ \ \ \ \ \ M\textcolor{BrickRed}{[}i\textcolor{BrickRed}{]=}\ \textbf{\textcolor{Blue}{new}}\ \textcolor{ForestGreen}{int}\ \textcolor{BrickRed}{[}dim\textcolor{BrickRed}{];} \\
\mbox{}\ \ \ \ \ \ \ \ \textbf{\textcolor{Blue}{if}}\ \textcolor{BrickRed}{(}M\textcolor{BrickRed}{[}i\textcolor{BrickRed}{]}\ \textcolor{BrickRed}{==}\ NULL\textcolor{BrickRed}{)} \\
\mbox{}\ \ \ \ \ \ \ \ \textcolor{Red}{\{} \\
\mbox{}\ \ \ \ \ \ \ \ \ \ \ \ cerr\ \textcolor{BrickRed}{\textless{}\textless{}}\ \texttt{\textcolor{Red}{"{}ERROR:\ No\ puedo\ reservar\ memoria\ para\ un\ matriz\ de\ "{}}} \\
\mbox{}\ \ \ \ \ \ \ \ \ \ \ \ \textcolor{BrickRed}{\textless{}\textless{}}\ dim\ \textcolor{BrickRed}{\textless{}\textless{}}\ \texttt{\textcolor{Red}{"{}\ x\ "{}}}\ \textcolor{BrickRed}{\textless{}\textless{}}\ dim\ \textcolor{BrickRed}{\textless{}\textless{}}\ endl\textcolor{BrickRed}{;} \\
\mbox{}\ \ \ \ \ \ \ \ \ \ \ \ \textbf{\textcolor{Blue}{for}}\ \textcolor{BrickRed}{(}\textcolor{ForestGreen}{int}\ j\ \textcolor{BrickRed}{=}\ \textcolor{Purple}{0}\textcolor{BrickRed}{;}\ j\ \textcolor{BrickRed}{\textless{}}\ i\textcolor{BrickRed}{;}\ j\textcolor{BrickRed}{++)} \\
\mbox{}\ \ \ \ \ \ \ \ \ \ \ \ \ \ \ \ \textbf{\textcolor{Blue}{delete}}\ \textcolor{BrickRed}{[]}\ M\textcolor{BrickRed}{[}i\textcolor{BrickRed}{];} \\
\mbox{}\ \ \ \ \ \ \ \ \ \ \ \ \textbf{\textcolor{Blue}{delete}}\ \textcolor{BrickRed}{[]}\ M\textcolor{BrickRed}{;} \\
\mbox{}\ \ \ \ \ \ \ \ \ \ \ \ \textbf{\textcolor{Black}{exit}}\textcolor{BrickRed}{(}\textcolor{Purple}{1}\textcolor{BrickRed}{);} \\
\mbox{}\ \ \ \ \ \ \ \ \textcolor{Red}{\}}\  \\
\mbox{}\ \ \ \ \textcolor{Red}{\}} \\
\mbox{}\ \ \ \ \textbf{\textcolor{Blue}{return}}\ M\textcolor{BrickRed}{;} \\
\mbox{}\textcolor{Red}{\}}\ \ \  \\
\mbox{} \\
\mbox{}\textcolor{ForestGreen}{void}\ \textbf{\textcolor{Black}{LiberaMatriz}}\textcolor{BrickRed}{(}\textcolor{ForestGreen}{int}\ \textcolor{BrickRed}{**}\ \textcolor{BrickRed}{\&}\ M\textcolor{BrickRed}{,}\ \textcolor{ForestGreen}{int}\ dim\textcolor{BrickRed}{)}\textcolor{Red}{\{} \\
\mbox{}\ \ \ \ \textbf{\textcolor{Blue}{for}}\ \textcolor{BrickRed}{(}\textcolor{ForestGreen}{int}\ i\ \textcolor{BrickRed}{=}\ \textcolor{Purple}{0}\textcolor{BrickRed}{;}\ i\ \textcolor{BrickRed}{\textless{}}\ dim\textcolor{BrickRed}{;}\ i\textcolor{BrickRed}{++)} \\
\mbox{}\ \ \ \ \ \ \ \ \textbf{\textcolor{Blue}{delete}}\ \textcolor{BrickRed}{[]}\ M\textcolor{BrickRed}{[}i\textcolor{BrickRed}{];} \\
\mbox{}\ \ \ \ \textbf{\textcolor{Blue}{delete}}\ \textcolor{BrickRed}{[]}\ M\textcolor{BrickRed}{;} \\
\mbox{}\ \ \ \ M\ \textcolor{BrickRed}{=}\ NULL\textcolor{BrickRed}{;} \\
\mbox{}\textcolor{Red}{\}}\ \ \ \ \ \ \  \\
\mbox{} \\
\mbox{}\textcolor{ForestGreen}{void}\ \textbf{\textcolor{Black}{RellenaMatriz}}\textcolor{BrickRed}{(}\textcolor{ForestGreen}{int}\ \textcolor{BrickRed}{**}M\textcolor{BrickRed}{,}\ \textcolor{ForestGreen}{int}\ dim\textcolor{BrickRed}{)}\textcolor{Red}{\{} \\
\mbox{}\ \ \ \ \textbf{\textcolor{Blue}{for}}\ \textcolor{BrickRed}{(}\textcolor{ForestGreen}{int}\ i\ \textcolor{BrickRed}{=}\ \textcolor{Purple}{0}\textcolor{BrickRed}{;}\ i\ \textcolor{BrickRed}{\textless{}}\ dim\textcolor{BrickRed}{;}\ i\textcolor{BrickRed}{++)} \\
\mbox{}\ \ \ \ \ \ \ \ \textbf{\textcolor{Blue}{for}}\ \textcolor{BrickRed}{(}\textcolor{ForestGreen}{int}\ j\ \textcolor{BrickRed}{=}\ \textcolor{Purple}{0}\textcolor{BrickRed}{;}\ j\ \textcolor{BrickRed}{\textless{}}\ dim\textcolor{BrickRed}{;}\ j\textcolor{BrickRed}{++)} \\
\mbox{}\ \ \ \ \ \ \ \ \ \ \ \ \textbf{\textcolor{Blue}{if}}\ \textcolor{BrickRed}{(}i\ \textcolor{BrickRed}{!=}\ j\textcolor{BrickRed}{)} \\
\mbox{}\ \ \ \ \ \ \ \ \ \ \ \ \ \ \ \ M\textcolor{BrickRed}{[}i\textcolor{BrickRed}{][}j\textcolor{BrickRed}{]=}\ \textcolor{BrickRed}{(}\textbf{\textcolor{Black}{rand}}\textcolor{BrickRed}{()}\ \textcolor{BrickRed}{\%}\ MAX$\_$LONG\textcolor{BrickRed}{);} \\
\mbox{}\textcolor{Red}{\}}\ \ \ \ \ \ \ \ \ \ \  \\
\mbox{}\ \ \ \  \\
\mbox{}\textcolor{ForestGreen}{void}\ \textbf{\textcolor{Black}{Floyd}}\textcolor{BrickRed}{(}\textcolor{ForestGreen}{int}\ \textcolor{BrickRed}{**}M\textcolor{BrickRed}{,}\ \textcolor{ForestGreen}{int}\ dim\textcolor{BrickRed}{)}\textcolor{Red}{\{} \\
\mbox{}\ \ \ \ \textbf{\textcolor{Blue}{for}}\ \textcolor{BrickRed}{(}\textcolor{ForestGreen}{int}\ k\ \textcolor{BrickRed}{=}\ \textcolor{Purple}{0}\textcolor{BrickRed}{;}\ k\ \textcolor{BrickRed}{\textless{}}\ dim\textcolor{BrickRed}{;}\ k\textcolor{BrickRed}{++)} \\
\mbox{}\ \ \ \ \ \ \ \ \textbf{\textcolor{Blue}{for}}\ \textcolor{BrickRed}{(}\textcolor{ForestGreen}{int}\ i\ \textcolor{BrickRed}{=}\ \textcolor{Purple}{0}\textcolor{BrickRed}{;}\ i\ \textcolor{BrickRed}{\textless{}}\ dim\textcolor{BrickRed}{;}i\textcolor{BrickRed}{++)} \\
\mbox{}\ \ \ \ \ \ \ \ \ \ \ \ \textbf{\textcolor{Blue}{for}}\ \textcolor{BrickRed}{(}\textcolor{ForestGreen}{int}\ j\ \textcolor{BrickRed}{=}\ \textcolor{Purple}{0}\textcolor{BrickRed}{;}\ j\ \textcolor{BrickRed}{\textless{}}\ dim\textcolor{BrickRed}{;}j\textcolor{BrickRed}{++)} \\
\mbox{}\ \ \ \ \ \ \ \ \ \ \ \ \textcolor{Red}{\{} \\
\mbox{}\ \ \ \ \ \ \ \ \ \ \ \ \ \ \ \ \textcolor{ForestGreen}{int}\ sum\ \textcolor{BrickRed}{=}\ M\textcolor{BrickRed}{[}i\textcolor{BrickRed}{][}k\textcolor{BrickRed}{]}\ \textcolor{BrickRed}{+}\ M\textcolor{BrickRed}{[}k\textcolor{BrickRed}{][}j\textcolor{BrickRed}{];}\ \ \ \ \ \ \ \  \\
\mbox{}\ \ \ \ \ \ \ \ \ \ \ \ \ \ \ \ M\textcolor{BrickRed}{[}i\textcolor{BrickRed}{][}j\textcolor{BrickRed}{]}\ \textcolor{BrickRed}{=}\ \textcolor{BrickRed}{(}M\textcolor{BrickRed}{[}i\textcolor{BrickRed}{][}j\textcolor{BrickRed}{]}\ \textcolor{BrickRed}{\textgreater{}}\ sum\textcolor{BrickRed}{)}\ \textcolor{BrickRed}{?}\ sum\ \textcolor{BrickRed}{:}\ M\textcolor{BrickRed}{[}i\textcolor{BrickRed}{][}j\textcolor{BrickRed}{];} \\
\mbox{}\ \ \ \ \ \ \ \ \ \ \ \ \textcolor{Red}{\}} \\
\mbox{}\textcolor{Red}{\}}\ \ \ \ \ \ \ \ \ \ \  \\
\mbox{} \\
\mbox{}\textcolor{ForestGreen}{int}\ \textbf{\textcolor{Black}{main}}\ \textcolor{BrickRed}{(}\textcolor{ForestGreen}{int}\ argc\textcolor{BrickRed}{,}\ \textcolor{ForestGreen}{char}\ \textcolor{BrickRed}{*}argv\textcolor{BrickRed}{[])}\textcolor{Red}{\{} \\
\mbox{}\ \ \ \ \textbf{\textcolor{Blue}{struct}}\ \textcolor{TealBlue}{timespec}\ t$\_$antes\textcolor{BrickRed}{,}\ t$\_$despues\textcolor{BrickRed}{;} \\
\mbox{}\ \ \ \ \textcolor{ForestGreen}{int}\ dim\textcolor{BrickRed}{;}\ \ \ \ \ \ \ \ \ \ \ \ \textit{\textcolor{Brown}{//\ Dimensión\ de\ la\ matriz}} \\
\mbox{}\ \ \ \  \\
\mbox{}\ \ \ \ \textit{\textcolor{Brown}{//Lectura\ de\ los\ parametros\ de\ entrada}} \\
\mbox{}\ \ \ \ \textbf{\textcolor{Blue}{if}}\ \textcolor{BrickRed}{(}argc\ \textcolor{BrickRed}{!=}\textcolor{Purple}{2}\textcolor{BrickRed}{)}\textcolor{Red}{\{} \\
\mbox{}\ \ \ \ \ \ \ \ cerr\ \textcolor{BrickRed}{\textless{}\textless{}}\ \texttt{\textcolor{Red}{"{}Uso\ del\ programa:\ "{}}}\ \textcolor{BrickRed}{+}\ \textcolor{BrickRed}{(}string\textcolor{BrickRed}{)(}argv\textcolor{BrickRed}{[}\textcolor{Purple}{0}\textcolor{BrickRed}{])}\ \textcolor{BrickRed}{+}\ \texttt{\textcolor{Red}{"{}\ \textless{}número\ positivo\textgreater{}"{}}}\ \textcolor{BrickRed}{\textless{}\textless{}}\ endl\textcolor{BrickRed}{;}\ \  \\
\mbox{}\ \ \ \ \ \ \ \ \textbf{\textcolor{Blue}{return}}\ \textcolor{BrickRed}{-}\textcolor{Purple}{1}\textcolor{BrickRed}{;} \\
\mbox{}\ \ \ \ \textcolor{Red}{\}} \\
\mbox{}\ \ \ \  \\
\mbox{}\ \ \ \ dim\ \textcolor{BrickRed}{=}\ \textbf{\textcolor{Black}{atoi}}\textcolor{BrickRed}{(}argv\textcolor{BrickRed}{[}\textcolor{Purple}{1}\textcolor{BrickRed}{]);} \\
\mbox{}\ \ \ \ \textbf{\textcolor{Blue}{if}}\ \textcolor{BrickRed}{(}dim\textcolor{BrickRed}{\textless{}}\textcolor{Purple}{0}\textcolor{BrickRed}{)}\ \textbf{\textcolor{Blue}{return}}\ \textcolor{BrickRed}{-}\textcolor{Purple}{1}\textcolor{BrickRed}{;} \\
\mbox{}\ \ \ \ \textcolor{ForestGreen}{int}\ \textcolor{BrickRed}{**}\ M\ \textcolor{BrickRed}{=}\ \textbf{\textcolor{Black}{ReservaMatriz}}\textcolor{BrickRed}{(}dim\textcolor{BrickRed}{);} \\
\mbox{}\ \ \ \  \\
\mbox{}\ \ \ \ \textbf{\textcolor{Black}{RellenaMatriz}}\textcolor{BrickRed}{(}M\textcolor{BrickRed}{,}dim\textcolor{BrickRed}{);} \\
\mbox{}\ \ \ \  \\
\mbox{}\ \ \ \  \\
\mbox{}\ \ \ \ \textit{\textcolor{Brown}{//\ Empieza\ el\ algoritmo\ de\ floyd}} \\
\mbox{}\ \ \ \  \\
\mbox{}\ \ \ \ \textbf{\textcolor{Black}{clock$\_$gettime}}\textcolor{BrickRed}{(}CLOCK$\_$REALTIME\textcolor{BrickRed}{,\&}t$\_$antes\textcolor{BrickRed}{);} \\
\mbox{}\ \ \ \ \textbf{\textcolor{Black}{Floyd}}\ \textcolor{BrickRed}{(}M\textcolor{BrickRed}{,}dim\textcolor{BrickRed}{);} \\
\mbox{}\ \ \ \ \textbf{\textcolor{Black}{clock$\_$gettime}}\textcolor{BrickRed}{(}CLOCK$\_$REALTIME\textcolor{BrickRed}{,\&}t$\_$despues\textcolor{BrickRed}{);} \\
\mbox{}\ \ \ \  \\
\mbox{}\ \ \ \ cout\textcolor{BrickRed}{.}\textbf{\textcolor{Black}{precision}}\textcolor{BrickRed}{(}\textcolor{Purple}{3}\textcolor{BrickRed}{);} \\
\mbox{}\ \ \ \ cout\ \textcolor{BrickRed}{\textless{}\textless{}}\ \textcolor{BrickRed}{(}\textcolor{ForestGreen}{double}\textcolor{BrickRed}{)}\ \textcolor{BrickRed}{(}t$\_$despues\textcolor{BrickRed}{.}tv$\_$sec\textcolor{BrickRed}{-}t$\_$antes\textcolor{BrickRed}{.}tv$\_$sec\textcolor{BrickRed}{)+} \\
\mbox{}\ \ \ \ \ \ \ \ \textcolor{BrickRed}{(}\textcolor{ForestGreen}{double}\textcolor{BrickRed}{)}\ \textcolor{BrickRed}{((}t$\_$despues\textcolor{BrickRed}{.}tv$\_$nsec\textcolor{BrickRed}{-}t$\_$antes\textcolor{BrickRed}{.}tv$\_$nsec\textcolor{BrickRed}{)/(}\textcolor{Purple}{1}\textcolor{BrickRed}{.}e\textcolor{BrickRed}{+}\textcolor{Purple}{9}\textcolor{BrickRed}{))}\ \textcolor{BrickRed}{\textless{}\textless{}}\ endl\textcolor{BrickRed}{;} \\
\mbox{} \\
\mbox{}\ \ \ \ \textbf{\textcolor{Black}{LiberaMatriz}}\textcolor{BrickRed}{(}M\textcolor{BrickRed}{,}dim\textcolor{BrickRed}{);} \\
\mbox{}\ \ \ \  \\
\mbox{}\ \ \ \ \textbf{\textcolor{Blue}{return}}\ \textcolor{Purple}{0}\textcolor{BrickRed}{;} \\
\mbox{}\textcolor{Red}{\}}\ \ \  \\
\mbox{} \\
\mbox{}


\subsubsection{Torres de Hanoi}
% Generator: GNU source-highlight, by Lorenzo Bettini, http://www.gnu.org/software/src-highlite
\noindent
\mbox{}\textit{\textcolor{Brown}{/**}} \\
\mbox{}\textit{\textcolor{Brown}{\ *\ }}\textcolor{ForestGreen}{@file}\textit{\textcolor{Brown}{\ Resolución\ del\ problema\ de\ las\ Torres\ de\ Hanoi}} \\
\mbox{}\textit{\textcolor{Brown}{\ */}} \\
\mbox{} \\
\mbox{}\textbf{\textcolor{RoyalBlue}{\#include}}\ \texttt{\textcolor{Red}{\textless{}iostream\textgreater{}}} \\
\mbox{}\textbf{\textcolor{RoyalBlue}{\#include}}\ \texttt{\textcolor{Red}{\textless{}cstdlib\textgreater{}}} \\
\mbox{}\textbf{\textcolor{RoyalBlue}{\#include}}\ \texttt{\textcolor{Red}{\textless{}ctime\textgreater{}}} \\
\mbox{}\textbf{\textcolor{Blue}{using}}\ \textbf{\textcolor{Blue}{namespace}}\ std\textcolor{BrickRed}{;} \\
\mbox{} \\
\mbox{}\textit{\textcolor{Brown}{/**}} \\
\mbox{}\textit{\textcolor{Brown}{\ *\ }}\textcolor{ForestGreen}{@brief}\textit{\textcolor{Brown}{\ Resuelve\ el\ problema\ de\ las\ Torres\ de\ Hanoi}} \\
\mbox{}\textit{\textcolor{Brown}{\ *\ }}\textcolor{ForestGreen}{@param}\textit{\textcolor{Brown}{\ M:\ número\ de\ discos.\ M\ \textgreater{}\ 1.}} \\
\mbox{}\textit{\textcolor{Brown}{\ *\ }}\textcolor{ForestGreen}{@param}\textit{\textcolor{Brown}{\ i:\ número\ de\ columna\ en\ que\ están\ los\ discos.}} \\
\mbox{}\textit{\textcolor{Brown}{\ *\ i\ es\ un\ valor\ de\ \{1,\ 2,\ 3\}.\ i\ !=\ j.}} \\
\mbox{}\textit{\textcolor{Brown}{\ *\ }}\textcolor{ForestGreen}{@param}\textit{\textcolor{Brown}{\ j:\ número\ de\ columna\ a\ que\ se\ llevan\ los\ discos.}} \\
\mbox{}\textit{\textcolor{Brown}{\ *\ j\ es\ un\ valor\ de\ \{1,\ 2,\ 3\}.\ j\ !=\ i.}} \\
\mbox{}\textit{\textcolor{Brown}{\ *\ }} \\
\mbox{}\textit{\textcolor{Brown}{\ *\ Esta\ función\ imprime\ en\ la\ salida\ estándar\ la\ secuencia\ de}} \\
\mbox{}\textit{\textcolor{Brown}{\ *\ movimientos\ necesarios\ para\ desplazar\ los\ M\ discos\ de\ la}} \\
\mbox{}\textit{\textcolor{Brown}{\ *\ columna\ i\ a\ la\ j,\ observando\ la\ restricción\ de\ que\ ningún}} \\
\mbox{}\textit{\textcolor{Brown}{\ *\ disco\ se\ puede\ situar\ sobre\ otro\ de\ tamaño\ menor.\ Utiliza}} \\
\mbox{}\textit{\textcolor{Brown}{\ *\ una\ única\ columna\ auxiliar.}} \\
\mbox{}\textit{\textcolor{Brown}{\ */}} \\
\mbox{} \\
\mbox{}\textcolor{ForestGreen}{void}\ \textbf{\textcolor{Black}{hanoi}}\ \textcolor{BrickRed}{(}\textcolor{ForestGreen}{int}\ M\textcolor{BrickRed}{,}\ \textcolor{ForestGreen}{int}\ i\textcolor{BrickRed}{,}\ \textcolor{ForestGreen}{int}\ j\textcolor{BrickRed}{);} \\
\mbox{} \\
\mbox{}\textcolor{ForestGreen}{void}\ \textbf{\textcolor{Black}{hanoi}}\ \textcolor{BrickRed}{(}\textcolor{ForestGreen}{int}\ M\textcolor{BrickRed}{,}\ \textcolor{ForestGreen}{int}\ i\textcolor{BrickRed}{,}\ \textcolor{ForestGreen}{int}\ j\textcolor{BrickRed}{)}\textcolor{Red}{\{} \\
\mbox{}\ \ \ \ \textbf{\textcolor{Blue}{if}}\ \textcolor{BrickRed}{(}M\ \textcolor{BrickRed}{\textgreater{}}\ \textcolor{Purple}{0}\textcolor{BrickRed}{)}\textcolor{Red}{\{} \\
\mbox{}\ \ \ \ \ \ \ \ \textbf{\textcolor{Black}{hanoi}}\textcolor{BrickRed}{(}M\textcolor{BrickRed}{-}\textcolor{Purple}{1}\textcolor{BrickRed}{,}\ i\textcolor{BrickRed}{,}\ \textcolor{Purple}{6}\textcolor{BrickRed}{-}i\textcolor{BrickRed}{-}j\textcolor{BrickRed}{);} \\
\mbox{}\ \ \ \ \ \ \ \ \textit{\textcolor{Brown}{//cout\ \textless{}\textless{}\ i\ \textless{}\textless{}\ "{}\ -\textgreater{}\ "{}\ \textless{}\textless{}\ j\ \textless{}\textless{}\ endl;}} \\
\mbox{}\ \ \ \ \ \ \ \ \textbf{\textcolor{Black}{hanoi}}\ \textcolor{BrickRed}{(}M\textcolor{BrickRed}{-}\textcolor{Purple}{1}\textcolor{BrickRed}{,}\ \textcolor{Purple}{6}\textcolor{BrickRed}{-}i\textcolor{BrickRed}{-}j\textcolor{BrickRed}{,}\ j\textcolor{BrickRed}{);} \\
\mbox{}\ \ \ \ \textcolor{Red}{\}} \\
\mbox{}\textcolor{Red}{\}} \\
\mbox{} \\
\mbox{}\textcolor{ForestGreen}{int}\ \textbf{\textcolor{Black}{main}}\textcolor{BrickRed}{(}\textcolor{ForestGreen}{int}\ argc\textcolor{BrickRed}{,}\ \textcolor{ForestGreen}{char}\textcolor{BrickRed}{*}\ argv\textcolor{BrickRed}{[])}\textcolor{Red}{\{} \\
\mbox{}\ \ \ \ \textbf{\textcolor{Blue}{if}}\ \textcolor{BrickRed}{(}argc\ \textcolor{BrickRed}{!=}\textcolor{Purple}{2}\textcolor{BrickRed}{)}\textcolor{Red}{\{} \\
\mbox{}\ \ \ \ \ \ \ \ cerr\ \textcolor{BrickRed}{\textless{}\textless{}}\ \texttt{\textcolor{Red}{"{}Uso\ del\ programa:\ "{}}}\ \textcolor{BrickRed}{+}\ \textcolor{BrickRed}{(}string\textcolor{BrickRed}{)(}argv\textcolor{BrickRed}{[}\textcolor{Purple}{0}\textcolor{BrickRed}{])}\ \textcolor{BrickRed}{+}\ \texttt{\textcolor{Red}{"{}\ \textless{}número\ positivo\textgreater{}"{}}}\ \textcolor{BrickRed}{\textless{}\textless{}}\ endl\textcolor{BrickRed}{;}\ \  \\
\mbox{}\ \ \ \ \ \ \ \ \textbf{\textcolor{Blue}{return}}\ \textcolor{BrickRed}{-}\textcolor{Purple}{1}\textcolor{BrickRed}{;} \\
\mbox{}\ \ \ \ \textcolor{Red}{\}} \\
\mbox{}\ \ \ \ \textcolor{ForestGreen}{int}\ n\ \textcolor{BrickRed}{=}\ \textbf{\textcolor{Black}{atoi}}\textcolor{BrickRed}{(}argv\textcolor{BrickRed}{[}\textcolor{Purple}{1}\textcolor{BrickRed}{]);}\ \ \ \  \\
\mbox{}\ \ \ \ \textbf{\textcolor{Blue}{if}}\ \textcolor{BrickRed}{(}n\textcolor{BrickRed}{\textless{}}\textcolor{Purple}{0}\textcolor{BrickRed}{)}\ \textbf{\textcolor{Blue}{return}}\ \textcolor{BrickRed}{-}\textcolor{Purple}{1}\textcolor{BrickRed}{;} \\
\mbox{} \\
\mbox{}\ \ \ \ \textbf{\textcolor{Blue}{struct}}\ \textcolor{TealBlue}{timespec}\ t$\_$antes\textcolor{BrickRed}{,}\ t$\_$despues\textcolor{BrickRed}{;} \\
\mbox{}\ \ \ \  \\
\mbox{}\ \ \ \ \textbf{\textcolor{Black}{clock$\_$gettime}}\textcolor{BrickRed}{(}CLOCK$\_$REALTIME\textcolor{BrickRed}{,\&}t$\_$antes\textcolor{BrickRed}{);} \\
\mbox{}\ \ \ \ \textbf{\textcolor{Black}{hanoi}}\ \textcolor{BrickRed}{(}n\textcolor{BrickRed}{,}\textcolor{Purple}{1}\textcolor{BrickRed}{,}\textcolor{Purple}{2}\textcolor{BrickRed}{);} \\
\mbox{}\ \ \ \ \textbf{\textcolor{Black}{clock$\_$gettime}}\textcolor{BrickRed}{(}CLOCK$\_$REALTIME\textcolor{BrickRed}{,\&}t$\_$despues\textcolor{BrickRed}{);} \\
\mbox{}\ \ \ \  \\
\mbox{}\ \ \ \ cout\textcolor{BrickRed}{.}\textbf{\textcolor{Black}{precision}}\textcolor{BrickRed}{(}\textcolor{Purple}{3}\textcolor{BrickRed}{);} \\
\mbox{}\ \ \ \ cout\ \textcolor{BrickRed}{\textless{}\textless{}}\ \textcolor{BrickRed}{(}\textcolor{ForestGreen}{double}\textcolor{BrickRed}{)}\ \textcolor{BrickRed}{(}t$\_$despues\textcolor{BrickRed}{.}tv$\_$sec\textcolor{BrickRed}{-}t$\_$antes\textcolor{BrickRed}{.}tv$\_$sec\textcolor{BrickRed}{)+} \\
\mbox{}\ \ \ \ \ \ \ \ \textcolor{BrickRed}{(}\textcolor{ForestGreen}{double}\textcolor{BrickRed}{)}\ \textcolor{BrickRed}{((}t$\_$despues\textcolor{BrickRed}{.}tv$\_$nsec\textcolor{BrickRed}{-}t$\_$antes\textcolor{BrickRed}{.}tv$\_$nsec\textcolor{BrickRed}{)/(}\textcolor{Purple}{1}\textcolor{BrickRed}{.}e\textcolor{BrickRed}{+}\textcolor{Purple}{9}\textcolor{BrickRed}{))}\ \textcolor{BrickRed}{\textless{}\textless{}}\ endl\textcolor{BrickRed}{;} \\
\mbox{}\ \ \ \  \\
\mbox{}\ \ \ \ \textbf{\textcolor{Blue}{return}}\ \textcolor{Purple}{0}\textcolor{BrickRed}{;} \\
\mbox{}\textcolor{Red}{\}}


\end{allintypewriter}
}




\section {Conclusiones}
\subsection {Gráficas}
\imagen{../regressionPlots/burbuja_fit.jpg}{Ordenación por burbuja}

\imagen{../regressionPlots/fibonacci_fit.jpg}{Serie de Fibonacci}

\imagen{../regressionPlots/floyd_fit.jpg}{Algoritmo de Floyd}

\imagen{../regressionPlots/hanoi_fit.jpg}{Torres de Hanoi}

\imagen{../regressionPlots/heapsort_fit.jpg}{Ordenación por heapsort}

\imagen{../regressionPlots/insercion_fit.jpg}{Ordenación por inserción}

\imagen{../regressionPlots/mergesort_fit.jpg}{Ordenación por mergesort}

\imagen{../regressionPlots/quicksort_fit.jpg}{Ordenación por quicksort}

\imagen{../regressionPlots/seleccion_fit.jpg}{Ordenación por seleccion}

\newpage
\section{Apéndice: Tablas de datos}
% TABLA DE PRUEBA (Cambiar por datos de verdad...)
% generada mediante `./gengraf.sh "burbuja seleccion insercion" 3`
\begin{center}
    \begin{longtabu} to \linewidth{ l | *{3}{d{10}}}  % máx 10 decimales
\rowfont\bfseries Tamaño & \multicolumn{1}{l}{burbuja} & \multicolumn{1}{l}{seleccion} & \multicolumn{1}{l}{insercion} \\ \hline
    \endhead
    \endfoot
    \\ \hline
    \endlastfoot
10 & 0.0000011 & 0.0000010714 & 0.0000009218 \\
210 & 0.0001376 & 0.00009386 & 0.00006644 \\
410 & 0.000482 & 0.0002978 & 0.00019898 \\
610 & 0.0007078 & 0.0005778 & 0.000509 \\
810 & 0.0013222 & 0.000961 & 0.0009356 \\
1010 & 0.00246 & 0.0014 & 0.001422 \\
1210 & 0.003506 & 0.002008 & 0.002058 \\
1410 & 0.004836 & 0.00259 & 0.0025162 \\
1610 & 0.006208 & 0.003332 & 0.002768 \\
1810 & 0.00731 & 0.00404 & 0.003692 \\
2010 & 0.006584 & 0.004386 & 0.005536 \\
2210 & 0.00891 & 0.005908 & 0.006758 \\
2410 & 0.008302 & 0.006312 & 0.005614 \\
2610 & 0.008564 & 0.008096 & 0.007778 \\
2810 & 0.011416 & 0.00395 & 0.007554 \\
3010 & 0.011628 & 0.004918 & 0.008848 \\
3210 & 0.014066 & 0.009786 & 0.008754 \\
3410 & 0.01622 & 0.008532 & 0.009828 \\
3610 & 0.01578 & 0.010272 & 0.010922 \\
3810 & 0.01582 & 0.009466 & 0.011364 \\
4010 & 0.0197 & 0.00736 & 0.00873 \\
4210 & 0.02228 & 0.007608 & 0.012756 \\
4410 & 0.02414 & 0.00998 & 0.014812 \\
4610 & 0.0274 & 0.011016 & 0.01607 \\
4810 & 0.03172 & 0.01207 & 0.01814 \\
5010 & 0.0323 & 0.014922 & 0.01634 \\
5210 & 0.03652 & 0.014686 & 0.01834 \\
5410 & 0.03694 & 0.01282 & 0.0157 \\
5610 & 0.04406 & 0.01696 & 0.01948 \\
5810 & 0.05026 & 0.01514 & 0.02026 \\
6010 & 0.04842 & 0.01808 & 0.0217 \\
6210 & 0.05126 & 0.02174 & 0.02422 \\
6410 & 0.05456 & 0.02024 & 0.02578 \\
6610 & 0.0633 & 0.023 & 0.02776 \\
6810 & 0.06438 & 0.01962 & 0.02876 \\
7010 & 0.0664 & 0.02212 & 0.02902 \\
7210 & 0.07484 & 0.02212 & 0.02952 \\
7410 & 0.0778 & 0.02792 & 0.03208 \\
7610 & 0.0812 & 0.02454 & 0.03266 \\
7810 & 0.08816 & 0.029 & 0.03508 \\
8010 & 0.09084 & 0.0265 & 0.0358 \\
8210 & 0.09832 & 0.03044 & 0.04556 \\
8410 & 0.10698 & 0.03102 & 0.03926 \\
8610 & 0.1066 & 0.02998 & 0.04168 \\
8810 & 0.1135 & 0.03358 & 0.0441 \\
9010 & 0.12048 & 0.03942 & 0.04584 \\
9210 & 0.1262 & 0.03502 & 0.05034 \\
9410 & 0.1318 & 0.03802 & 0.04996 \\
9610 & 0.1382 & 0.04034 & 0.0519 \\
9810 & 0.1424 & 0.04118 & 0.0545 \\

    \end{longtabu}
\end{center}



%----------------------------------------------------------------------------------------
%	ABSTRACT AND KEYWORDS
%----------------------------------------------------------------------------------------

%\renewcommand{\abstractname}{Summary} % Uncomment to change the name of the abstract to something else

%\begin{abstract}
%Morbi tempor congue porta. Proin semper, leo vitae faucibus dictum, metus mauris lacinia lorem, ac congue leo felis eu turpis. Sed nec nunc pellentesque, gravida eros at, porttitor ipsum. Praesent consequat urna a lacus lobortis ultrices eget ac metus. In tempus hendrerit rhoncus. Mauris dignissim turpis id sollicitudin lacinia. Praesent libero tellus, fringilla nec ullamcorper at, ultrices id nulla. Phasellus placerat a tellus a malesuada.
%\end{abstract}

%\hspace*{3,6mm}\textit{Keywords:} lorem , ipsum , dolor , sit amet , lectus % Keywords

%\vspace{30pt} % Some vertical space between the abstract and first section

%----------------------------------------------------------------------------------------
%	ESSAY BODY
%----------------------------------------------------------------------------------------

% \section*{Introduction}
% 
% This statement requires citation \cite{Smith:2012qr}; this one does too \cite{Smith:2013jd}. Lorem ipsum dolor sit amet, consectetur adipiscing elit. Aenean dictum lacus sem, ut varius ante dignissim ac. Sed a mi quis lectus feugiat aliquam. Nunc sed vulputate velit. Sed commodo metus vel felis semper, quis rutrum odio vulputate. Donec a elit porttitor, facilisis nisl sit amet, dignissim arcu. Vivamus accumsan pellentesque nulla at euismod. Duis porta rutrum sem, eu facilisis mi varius sed. Suspendisse potenti. Mauris rhoncus neque nisi, ut laoreet augue pretium luctus. Vestibulum sit amet luctus sem, luctus ultrices leo. Aenean vitae sem leo.
% 
% Nullam semper quam at ante convallis posuere. Ut faucibus tellus ac massa luctus consectetur. Nulla pellentesque tortor et aliquam vehicula. Maecenas imperdiet euismod enim ut pharetra. Suspendisse pulvinar sapien vitae placerat pellentesque. Nulla facilisi. Aenean vitae nunc venenatis, vehicula neque in, congue ligula.
% 
% Pellentesque quis neque fringilla, varius ligula quis, malesuada dolor. Aenean malesuada urna porta, condimentum nisl sed, scelerisque nisi. Suspendisse ac orci quis massa porta dignissim. Morbi sollicitudin, felis eget tristique laoreet, ante lacus pretium lacus, nec ornare sem lorem a velit. Pellentesque eu erat congue, ullamcorper ante ut, tristique turpis. Nam sodales mi sed nisl tincidunt vestibulum. Interdum et malesuada fames ac ante ipsum primis in faucibus.
% 
% %------------------------------------------------
% 
% \section*{Section Name}
% 
% Cras gravida, est vel interdum euismod, tortor mi lobortis mi, quis adipiscing elit lacus ut orci. Phasellus nec fringilla nisi, ut vestibulum neque. Aenean non risus eu nunc accumsan condimentum at sed ipsum.
% \begin{wrapfigure}{l}{0.4\textwidth} % Inline image example
% \begin{center}
% \includegraphics[width=0.38\textwidth]{fish.png}
% \end{center}
% \caption{Fish}
% \end{wrapfigure}
% Aliquam fringilla non diam sed varius. Suspendisse tellus felis, hendrerit non bibendum ut, adipiscing vitae diam. Lorem ipsum dolor sit amet, consectetur adipiscing elit. Nulla lobortis purus eget nisl scelerisque, commodo rhoncus lacus porta. Vestibulum vitae turpis tincidunt, varius dolor in, dictum lectus. Aenean ac ornare augue, ac facilisis purus. Sed leo lorem, molestie sit amet fermentum id, suscipit ut sem. Vestibulum orci arcu, vehicula sed tortor id, ornare dapibus lorem. Praesent aliquet iaculis lacus nec fermentum. Morbi eleifend blandit dolor, pharetra hendrerit neque ornare vel. Nulla ornare, nisl eget imperdiet ornare, libero enim interdum mi, ut lobortis quam velit bibendum nibh.
% 
% Morbi tempor congue porta. Proin semper, leo vitae faucibus dictum, metus mauris lacinia lorem, ac congue leo felis eu turpis. Sed nec nunc pellentesque, gravida eros at, porttitor ipsum. Praesent consequat urna a lacus lobortis ultrices eget ac metus. In tempus hendrerit rhoncus. Mauris dignissim turpis id sollicitudin lacinia. Praesent libero tellus, fringilla nec ullamcorper at, ultrices id nulla. Phasellus placerat a tellus a malesuada.
% 
% %------------------------------------------------
% 
% \section*{Conclusion}
% 
% Fusce in nibh augue. Cum sociis natoque penatibus et magnis dis parturient montes, nascetur ridiculus mus. In dictum accumsan sapien, ut hendrerit nisi. Phasellus ut nulla mauris. Phasellus sagittis nec odio sed posuere. Vestibulum porttitor dolor quis suscipit bibendum. Mauris risus lectus, cursus vitae hendrerit posuere, congue ac est. Suspendisse commodo eu eros non cursus. Mauris ultrices venenatis dolor, sed aliquet odio tempor pellentesque. Duis ultricies, mauris id lobortis vulputate, tellus turpis eleifend elit, in gravida leo tortor ultricies est. Maecenas vitae ipsum at dui sodales condimentum a quis dui. Nam mi sapien, lobortis ac blandit eget, dignissim quis nunc.
% 
% \begin{enumerate}
% \item First numbered list item
% \item Second numbered list item
% \end{enumerate}
% 
% Donec luctus tincidunt mauris, non ultrices ligula aliquam id. Sed varius, magna a faucibus congue, arcu tellus pellentesque nisl, vel laoreet magna eros et magna. Vivamus lobortis elit eu dignissim ultrices. Fusce erat nulla, ornare at dolor quis, rhoncus venenatis velit. Donec sed elit mi. Sed semper tellus a convallis viverra. Maecenas mi lorem, placerat sit amet sem quis, adipiscing tincidunt turpis. Cras a urna et tellus dictum eleifend. Fusce dignissim lectus risus, in bibendum tortor lacinia interdum.

%----------------------------------------------------------------------------------------
%	BIBLIOGRAPHY
%----------------------------------------------------------------------------------------

% \bibliographystyle{unsrt}
% 
% \bibliography{sample}

%----------------------------------------------------------------------------------------



\end{document}